\documentclass[11pt,german,twocolumn]{scrartcl}
\usepackage[a4paper,vscale=0.94,hscale=0.9,includeheadfoot]{geometry}
%\usepackage[a5paper,vscale=0.94,hscale=0.9,includeheadfoot]{geometry}
\usepackage{kibitzer}
\usepackage{babel}
\usepackage{tabularx}
\usepackage{courier}
\usepackage{calc}
\usepackage{german}

\pagestyle{headings}

% \setlength{\topmargin}{-16mm}
% \setlength{\oddsidemargin}{-20mm}
% \setlength{\evensidemargin}{-20mm}
% \setlength{\textwidth}{200mm}   % -> Rand links 5mm breiter als rechts 
% \setlength{\textheight}{256mm}
\setlength{\columnsep}{10mm}
\setlength{\columnseprule}{0.4pt}

\handwidth22mm
\parindent0mm
\parskip2ex plus1ex minus1ex

%% Farbe
\usepackage{color,pxfonts}
\definecolor{ClColor}{rgb}{0.0,0.6,0.0}
\definecolor{DiColor}{rgb}{1.0,0.3,0.0}
\definecolor{HeColor}{rgb}{0.9,0.0,0.0}
\definecolor{SpColor}{rgb}{0.0,0.0,0.7}
\renewcommand{\Cl}{{\color{ClColor}{$\clubsuit$}}}
\renewcommand{\Di}{{\color{DiColor}{$\vardiamondsuit$}}}
\renewcommand{\He}{{\color{HeColor}{$\varheartsuit$}}}
\renewcommand{\Sp}{{\color{SpColor}{$\spadesuit$}}}

\def\pik{\,\Sp}
\def\coe{\,\He}
\def\kar{\,\Di}
\def\tre{\,\Cl}

\def\pi{\Sp}
\def\co{\He}
\def\ka{\Di}
\def\tr{\Cl}

\def\mi{\,\Cl /\Di}
\def\ma{\,\He /\Sp}
\def\good{$^+$}
\def\bad{$^-$}
\def\ra{$\rightarrow$}
\def\pl{$\uparrow$}
\def\uf{\textsf{\,UF}}
\def\of{\textsf{\,OF}}
\def\ufa{\textsf{UF}}
\def\aufa{\textsf{aUF}}
\def\ofa{\textsf{OF}}
\def\aof{\textsf{\,aOF}}
\def\aofa{\textsf{aOF}}
\def\SA{\textsf{\,S\kern-0.08emA}}
\def\kontra{\textsf{X}}
\def\sep{\,--\,}
\newcommand{\conv}[1]{\emph{#1}}
\def\bal{\textsc{Ausg}}
\def\unbal{\textsc{Unausg}}
\def\nat{\textsc{Nat}}
\def\pf{\textsc{PF}}
\def\maxi{\textsc{Max}}
\def\mini{\textsc{Min}}
\def\inv{\textsc{Einl}}
\def\nf{\textsc{NF}}
\def\rel{\textsc{Rel}}
\def\stp{\textsc{Stop}}
\def\hstp{\textsc{Hstop}}
\def\cstop{\co-\stp}
\def\pstop{\pi-\stp}
\def\tstop{\tr-\stp}
\def\kstop{\ka-\stp}
\def\chstop{\co-\hstp}
\def\phstop{\pi-\hstp}
\def\thstop{\tr-\hstp}
\def\khstop{\ka-\hstp}
\def\aw{\textsc{Aw}}
\def\eo{\textsc{E\"o}}
\def\bdsc{\begin{description}}
\def\edsc{\end{description}}
\newcommand{\bidex}[1]{\bcbidding{\textsl{West}}{\textsl{Ost}}{#1}}
\def\xfer{\textsc{Trf}}
\def\xferto{\xfer\,\ra\,}
\def\slamint{\textsc{Schl-Int}}

\setlength{\labelsep}{2ex}

\newcolumntype{Y}{>{\raggedright\arraybackslash}X}

\newcommand\bidins[1]%
{%
\begin{flushleft}
\begin{tabularx}{\columnwidth}{llX}%
#1
\end{tabularx}%
\end{flushleft}
}

\newcommand\bidseq[1]%
{%
\begin{flushleft}
\begin{tabularx}{\columnwidth}{llX}%
#1
\end{tabularx}%
\end{flushleft}
}

\newcommand\notebox[1]%
{%
\setlength{\fboxsep}{2ex}
\fbox{\parbox{\columnwidth - 4ex}{#1}}
}
  

\begin{document}

\setlength{\itemsep}{0ex plus0.2ex}

\onecolumn
{\centering\Huge\ 5er Oberfarben (R. Bartels und Th. Schmitt)}
\tableofcontents
\twocolumn

\section{"Ubersicht der Er"offnungen}

\subsection*{1er-Stufe}
\bidins{%
1\mi & 12\pl	& 3\pl er\mi\\[1ex]
1\ma & 12\pl	& 5\pl er\ma\\[1ex]
1\SA & 15-17	& \bal, 5er\of\ m"oglich
}

\subsection*{2er-Stufe}
\bidins{%
2\tre	& 6-10	& Weak Two in \kar, oder\\
	& 16\pl	& 6er-Farbe, \textbf{8} Spielstiche, oder\\
	& 22-23	& \bal, oder\\
	& 26-27	& \bal\\[1ex]
2\kar	& 6-10	& Weak Two in \coe, oder\\
	& 18\pl	& 6er-Farbe, \textbf{9} Spielstiche, oder\\
	& 24-25	& \bal, oder\\
	& 28\pl	& \bal\\[1ex]
2\coe	& 6-10	& Zweif"arber mit \coe\\[1ex]
2\pik	& 6-10	& Weak Two in \pik\\[1ex]
2\SA	& 20-21	& \bal
}

\subsection*{3er-Stufe}
\bidins{%
3\uf	& 5-10	& 7\pl{}er\uf, in 1. Hand nur in Nichtgefahr, in 3. Hand 6er-Farbe m"oglich\\[1ex]
3\of	& 5-10	& 7\pl{}er\of\\[1ex]
3\SA	& 	& Gambling in 1./2. Hand, in 3./4. Hand zum spielen
}

\subsection*{4er-Stufe}
\bidins{%
4\mi	& 	& stehendes 7er\ma\ mit einem Nebenwert\\[1ex]
4\of	&	& 7\good\pl er\of, zum Spielen\\[1ex]
4\SA	&	& 6/5\pl\uf
}

\section{Die 1\mi-Er"offnung}

Bei gleicher L"ange in den \uf\ wird 1\kar\ er"offnet, mit 3/3 in den \uf\
wird 1\tre\ er"offnet.
Ohne 5\pl er \ofa\ wird grunds"atzlich die l"angere \uf\ er"offnet.

Siehe auch \textit{Er"offnungsregel f"ur Zweif"arber mit 6/5 \uf/\of} auf
Seite \pageref{zfregel}.

\subsection{Antworten des Partners (nach 1\tre-Er"offnung)}
\bidins{%
1\kar	& 6-7	& \bal\ ohne 4er\of, 3-3-3-4 Verteilung mit 3er\kar\ m"oglich, oder\\
	& 6\pl	& \ka-Einf"arber, oder\\
	& 12\pl	& 5er\kar\ und 4er\of\ \conv{(Walsh-\ka)}\\[1ex]
1\of	& 6\pl	& 4\pl er\of\\[1ex]
1\SA	& 8-10	& \bal\ ohne 4er\of\\[1ex]
2\tre	& 10\good\pl & 5\pl er\tre\ \conv{(Inverted)}\\[1ex]
2\kar	& 2-5	& 5/5\of\\[1ex]
2\of	& 5-8	& gute 6\pl er\of\ \conv{(Weak Jump)}\\
        &       & \ra\ 2\SA\ Ogust \\[1ex]
2\SA    & 2-6   & 5\pl er\tre\ \conv{(Inverted Spezial)}\\
3\tre	& 7-9	& 5\pl er\tre\ \conv{(Inverted)}\\[1ex]
3\kar	& 12-15 & 5/5 in \tre\,+\kar\ \conv{(Fit Jump)}\\
3\coe	& 12-15 & 5/5 in \tre\,+\coe\ \conv{(Fit Jump)}\\
3\pik	& 12-15	& 5/5 in \tre\,+\pik\ \conv{(Fit Jump)}\\[1ex]
3\SA	& 13-15	& zum Spielen\\[1ex]
4\tre	&	& \conv{KCB}\\[1ex]
4\of	&	& zum Spielen
}

\notebox{%
Weiterreizung:
\begin{itemize}
\item Ein unn"otiger Sprung in einer neuen Farbe ist \conv{Splinter}.
\item Ein unn"otiger Doppelsprung ist \conv{Exclusion RKCB}.
\end{itemize}
}

Nach 1\kar{}-Er"offnung sind die Antworten entsprechend
(Ausnahme: [1\kar\sep2\tre] \ra\ref{inverted}).

\subsection{Bietsequenzen nach 1\mi-Er"offnung}

\subsubsection{Bietsequenzen nach [1\tre-1\kar]}

Wegen \conv{Walsh-\ka} zeigt \aw\ nach 1\tre-Er"offnung seine \of\
sofort, au"ser mit einer starken Hand und l"angeren
Karos. \mbox{[1\tre-1\kar]} lehnt eine 4er\of\ bei \aw\ also
meistens ab.
\eo\ reizt daher \SA\ weiter, wenn er \bal\ ist, eine etwaige 4er\of\
verschweigt er.  Ein \of-R"uckgebot von \eo\ zeigt \unbal\ mit langen
Treffs.

\bdsc
\item[1\tre\sep1\kar; ?]~
  \bdsc
  \item[1\coe] 5\pl er\tre\ und 4er\coe\ oder 4-4-1-4
  \item[1\pik] 5\pl er\tre\ und 4er\pik
  \item[1\SA] 12-14, eine oder beide \of\ m"oglich
    \bdsc
      \item[2\tre] Schlemm-Interesse in \tre\ oder \kar;
            \textbf{Frage nach Verteilung} (s.~u.)
      \item[2\kar] schwach; zum spielen
      \item[2\ma] \pf\ mit 5/4-Verteilung
      \item[2\SA] \nat
      \item[3\tre/\co/\pi] \conv{Splinter} mit 6er\kar
      \item[3\kar] \inv\ mit 6er\kar
    \edsc
  \item[2\tre] 12-14, 6er\tre
    \bdsc
      \item[2\coe] \cstop, kein \pstop.
        \ra~2\pik\ Frage nach \hstp\ \conv{(VFF)}
      \item[2\pik] zeigt \pstop, \ra~3\tre\ verneint \cstop, \ra\
        3\coe\ Frage nach \hstp\ \conv{(VFF)}
      \item[3\ma] \textsc{Splinter} mit \tre-Anschluss
      \item[4\ma] \textsc{Exclusion RCKB} auf \tre-Basis
    \edsc
  \item[2\SA] 18-19, eine oder beide \of\ m"oglich
    \bdsc
    \item[3\tre] \conv{Wolff Sign Off}, verlangt \rel\ 3\kar, worauf
      man passen kann oder Sign Off gibt
    \item[3\kar] Forcing
    \item[3\of] 12\pl\ FP, 5/4-Verteilung, Schlemm m"oglich (s.~u.)
      \bdsc
      \item[3\SA] \nat, kein Fit
      \item[4\of] Fit in der Oberfarbe
      \item[Rest] Steps \conv{RKCB} auf \ka-Basis
      \edsc
    \edsc
  \edsc
\edsc

\bdsc
\item[1\tre\sep1\kar; 1\SA\sep2\tre; ?]~

  Hat \aw\ Schlemm-Interesse in \uf, so kann er die genaue
  Verteilung der \bal\ Hand von \eo\ erfragen.
  \bdsc
  \item[2\kar] 3er\kar, \textbf{nicht 4-3-3-3}
    \bdsc
      \item[2\coe] \rel, Frage nach Verteilung
        \bdsc
        \item[2\pik] 5er \tr\ (x-x-3-5)
        \item[2\SA] 4er \tr\ (x-x-3-4)
        \edsc
      \item[3\mi] \conv{RKCB}
    \edsc
  \item[2\coe] 3-4-2-4
  \item[2\pik] 4-3-2-4
  \item[2\SA]  beliebige 4-3-3-3-Verteilung
    \bdsc
      \item[3\tre] \rel, Frage nach 4er-Farbe
        \bdsc
        \item[3\kar] \textbf{4er \tr}
        \item[3\coe] 4er \co
        \item[3\pik] 4er \pi
        \edsc
    \edsc
  \item[3\tre] 3-3-2-5 (\ra~3\kar\ = \conv{RKCB})
  \item[3\kar] 4-4-2-3
  \edsc
\edsc

\bdsc
  \item[1\tre\sep1\kar; 2\SA\sep{}?]~

    \aw\ kann nun 3\of\ reizen und damit Schlemm-Interesse in der \of\
    zeigen.
    \bdsc
      \item[3\coe] 12\pl\ FP, 5/4-Verteilung, Schlemm m"oglich
        \bdsc
          \item[3\pik] 1 oder 4 Key Cards auf \ka-Basis
          \item[3\SA] \nat, kein Fit
          \item[4\tre] 0 oder 3 Key Cards auf \ka-Basis
          \item[4\kar] 2 Key Cards auf \ka-Basis
          \item[4\coe] \co-Fit
          \item[4\pik] 2 Key Cards mit \ka-Dame
        \edsc
      \item[3\pik] 12\pl\ FP, 5/4-Verteilung, Schlemm m"oglich
        \bdsc
          \item[3\SA] \nat, kein Fit
          \item[4\tre] 1 oder 4 Key Cards auf \ka-Basis
          \item[4\kar] 0 oder 3 Key Cards auf \ka-Basis
          \item[4\coe] 2 Key Cards auf \ka-Basis (mit/ohne Q)
          \item[4\pik] \pi-Fit
        \edsc
    \edsc
\edsc

\subsubsection{Bietsequenzen nach [1\tre-1\of]}

\bdsc
\item[1\tre\sep1\coe; ?]~
  \bdsc
  \item[1\pik] 12\pl\ FP, \nat

    Wiederholt der Antwortende seine Oberfarbe auf niedrigster Stufe
    nachdem der Er"offner zwei Farben gereizt hat, so zeigt dies eine
    einladende Hand mit 9-11 FP und 6er-Farbe.

    Wiederholt er seine Oberfarbe im Sprung, so zeigt dies eine
    partieforcierende Hand mit 6er-Farbe:
    \bdsc
    \item[2\coe] 9-11 FP, 6er\coe, \inv
    \item[3\coe] 6er\coe, \pf
    \edsc
  \item[1\SA] 12-14 FP \bal, kein 4er\pik\ (\ra \ref{1sarebid})
    %\bdsc
    %\item[2\tre] \conv{Checkback Stayman}
    %\item[2\kar] 4er\coe, 5\pl\kar, zum spielen
    %\item[2\coe] 5er\coe, schwach, zum spielen
    %\item[2\pik] 5/4 \coe\,+\pik, \pf
    %\item[2\SA] \nat
    %\item[3\mi] \nat, \pf
    %\item[3\coe] 6er\coe, Schlemm-Interesse\footnotemark[1]
    %\edsc
  \edsc

\item[1\tre\sep1\pik; ?]~
  \bdsc
  \item[1\SA] 12-14 FP \bal\ (\ra \ref{1sarebid})
  \item[2\tre] 12-16\bad, 5\pl{}er-Farbe (s.~u.)
  \item[2\SA] 18-19 FP, \bal; siehe \conv{Wolff} etc.
  \item[3\kar/\co] \conv{Splinter}: 4er\pik, Single \ka/\co
  \item[3\SA] 18-19 FP, stehendes 6er\tre, K"urze in \pik, Deckung in den
    ungereizten Farben, 8-8$\frac{1}{2}$ Stiche
  \item[4\tre] \conv{Fit Jump}: 18\good\pl\ FP, 4er\pik, 5\good\pl\tre
  \item[4\kar/\co] \conv{Exclusion RCKB}: 18\good\pl\ FP, 4er\pik
  \edsc

\item[1\tre\sep1\pik; 2\tre\sep{}?] 12-16\bad, 5\pl{}er-Farbe
  \bdsc
  \item[2\kar] \conv{Dritte Farbe Forcing}
    \bdsc
    \item[2\coe] Frage nach \hstp\ (\conv{VFF}), kein 3er\pik
    \item[2\pik] 3er\pik, \mini\ f"ur 2\tre-R"uckgebot\\
      \ra~3\tre\ = Forcing mit \tre
    \item[2\SA] \cstop, kein 3er\pik, \mini
    \item[3\pik] 3er\pik, \maxi\ f"ur 2\tre-R"uckgebot
    \edsc
  \item[3\tre] \inv
  \edsc
\edsc

\subsubsection{Bietsequenzen nach 1\SA-R"uckgebot} \label{1sarebid}

Neben den nat"urlichen Antworten benutzen wir die beiden Transfergebote 2\tre\
und 2\SA, um m"oglichst viele Verteilungen zeigen zu k"onnen. Das vorher
benutzte System mit Checkback-Stayman hatte eine L"ucke, wenn der Antwortende
eine starke Hand mit 4er-\ofa\ und 5er-\ufa\ hatte, die Er"offungsfarbe war.
Nicht exakt reizen lassen sich jetzt nur noch mindestens einladende Zweif"arber
mit 5er-\ofa\ und 4er-\ufa\ sowie genau einladende Zweif"arber mit 4er-\ofa\
und 5er-\tre.

Wir unterscheiden folgende Handtypen:
\bdsc
\item[A]
  \bdsc
  \item[1] einladend, 4/5\pl\ \ofa/\ka
  \item[2] schwach, 5er-\pi, evtl. 4\pl\ \co
  \item[3] \pf, 5/5
  \item[4] \slamint, 6er-\ofa
  \item[5] starker \ofa-Einf"arber mit Single
  \edsc
\item[B]
  \bdsc
  \item[1] schwach, 4/5\pl\ \ofa/\ka
  \item[2] mindestens \inv, 5er-\ofa, evtl. 4er-\ufa
  \item[3] genau \inv
    \bdsc
    \item[a] \bal, 4er-\ofa
    \item[b] 5/4-Zweif"arber mit beiden \ofa
    \item[c] 5/5-Zweif"arber
    \item[d] Einf"arber in einer \ofa
    \edsc
  \item[4] starker \ofa-Einf"arber mit Chicane
  \edsc
\item[C]
  \bdsc
  \item[1] schwach, 4/5\pl\ \ofa/\tr
  \item[2] stark:
    \bdsc
    \item[a] 4/5 \ofa/\tr\ nach 1\tre-Er"offung
    \item[b] 4/5 \ofa/\ka\ nach 1\kar-Er"offung
    \item[c] 5/4 \pi/\co\ nach 1\pik-Antwort
    \edsc
  \edsc
\edsc


\bdsc
\item[1\tre\sep1\pik; 1\SA] 12-14 FP \bal
  \bdsc
  \item[2\tre; 2\kar] \xfer, verlangt 2\kar-\rel\ (B)
    \bdsc
    \item[pass] schwach (B2)
    \item[2\coe] 5/4 \pi/\co, \inv\ (B3b)
    \item[2\pik] 5er-\pi, mind. \inv\ (B2)
    \item[2\SA] \bal, \inv, 4er-\pi\ (B3a)
    \item[3\ufa/\co] 5/5, \inv\ (B3c)
    \item[3\pik] 6er-\pi, \inv\ (B3d)
    \item[3\SA] \nat, 5er-\pi\ (B2)
    \item[4\ufa/\co] 6\good er-\pi, Chicane \ufa/\coe\ (Autosplinter, B4)
    \edsc
  \item[2\kar] 4er\pik, 5\pl{}\kar, genau \inv\ (sonst Walsh, A1)
  \item[2\coe] 5/4 \pik\,+\coe, zum Spielen oder ausbessern (A2)
  \item[2\pik] 5er\pik, schwach (A2)
  \item[2\SA; 3\tre] \xfer, verlangt 3\tre-\rel\ (C)
    \bdsc
    \item[pass] schwach, 4/5\pl\ \pi/\tr\ (C1)
    \item[3\kar] stark, 4/5 \pi/\tr\ (nach 1\tre-Er"offnung, C2a) \\
                 stark, 4/5 \pi/\ka\ (nach 1\kar-Er"offnung, C2b)
    \item[3\coe] stark, 5/4 \pi/\co\ (C2c)
    \edsc
  \item[3\ufa/\co] 5/5 \pi+\ufa/\co, \pf\ (A3)
  \item[3\pik] 6er\pik, leichtes \slamint\footnotemark[1] (A4)
  \item[4\ufa/\co] 6\good er-\pi, Single \ufa/\co\ (Autosplinter, A5)
  \edsc
\edsc

\footnotetext[1]{Einladende H"ande mit 6er-Farbe "uber 2\tre\ reizen.}

\subsubsection{Bietsequenzen nach \conv{Reverse}}

\bdsc
  \item[1\tre\sep1\pik; 2\kar\sep{}?]~

    \conv{Reverse} zeigt 16\good\pl\ und ist selbstforcierend f"ur
    \eo.
    \bdsc
    \item[2\coe] \conv{VFF}, sagt hier aber nichts "uber die St"arke
      aus, da 2\kar\ selbsforcierend war.  Verneint in diesem Fall
      \textbf{5er}\pik, da man \pi\ h"atte wiederholen k"onnen ohne
      dass \eo\ passen darf.
      \bdsc
        \item[2\SA] \mini,\cstop, \nf
        \item[3\tre] \mini, \nf
        \item[3\kar] \maxi, \pf
        \item[3\SA] \maxi, \cstop
      \edsc
    \item[2\pik] 5er-Farbe, schwach oder stark
      \bdsc
        \item[2\SA] \mini, \cstop, kein 3er\pik
        \item[3\tre] \mini, kein 3er\pik
        \item[3\kar] \maxi, kein 3er\pik
        \item[3\coe] \conv{VFF}, Frage nach \cstop
        \item[3\pik] \mini\ mit 3er\pik
        \item[3\SA] \maxi, \cstop, kein 3er\pik
      \edsc
    \item[2\SA; 3\tre] \conv{Ingberman}\footnotemark[2], verlangt 3\tre-\rel
      \bdsc
      \item[pass/3\kar] zum Spielen
      \item[3\SA] 8-9 FP \nat
      \item[4\SA] 12\pl\ FP, quantitative Schlemmeinladung
      \edsc
      Mit Zusatzst"arke (18\good\pl) muss \eo\ das 3\tre\ \rel\
      "uberspringen.
    \item[3\uf] \pf
    \item[3\SA] 10-11 FP, \cstop
    \edsc
\edsc
\footnotetext[2]{frz. \conv{Moderateur}}

\subsubsection{Bietsequenzen nach \conv{Inverted}} \label{inverted}

Nach [1\tre-2\tre] bzw. [1\kar-2\kar] gibt es kein Reverse.  \eo\
zeigt mit 2\SA\ eine ausgeglichene, passbare Hand.  3\tre/\kar\ ist
ebenfalls passbar.  Alle anderen Gebote des Er"offners zeigen Werte
(Stopper) und sind \pf.

\bdsc
  \item[1\tre\sep2\tre; ?]~
    \bdsc
      \item[2\kar] 14\good\pl, \kstop
        \bdsc
        \item[2\coe] \cstop\ (h"ochstens \phstop)
          \bdsc
          \item[2\pik] fragt nach \phstop\ (siehe \conv{VFF})
          \edsc
        \item[2\pik] \pstop\ (h"ochstens \chstop)
        \item[2\SA] \stp\ in \co\ und \pi.
        \edsc
      \item[2\coe-2\pik] 14\good\pl, \stp\ in der Farbe
    \edsc
\edsc

\bdsc
  \item[1\kar\sep2\tre; ?]~
    \bdsc
    \item[2\kar] 12-13 FP \bal\ oder \nat\ (kann 3er\kar\ sein)
    \item[2\SA] 14 FP \bal
    \edsc
\edsc

\subsubsection*{\label{zfregel}Er"offnungsregel f"ur Zweif"arber mit 6/5 \uf/\of}

\bdsc
\setlength{\labelsep}{1ex}
\item[4\pl{} Verlierer:] \of\ er"offnen und \uf\ billig nachreizen
\item[3-4 Verlierer:] \uf\ er"offnen und anschlie"send \conv{Reverse}
  reizen
\item[0-3\bad\ Verlierer:] \pf\ er"offnen
\edsc

\section{Die 1\of-Er"offnungen}

Die Fortsetzung nach 1\of-Er"offnungen folgt folgenden Prinzipien:
\begin{itemize}
\setlength{\itemsep}{0.5ex}
\item Schwache bis einladende H"ande mit 4\pl{}er-Anschlu"s werden durch
  \conv{Bergen Raises} gezeigt.
\item Mit \pf\ und gutem Trumpfanschluss reizen wir
  \conv{Splinter} oder \conv{Stenberg 2\SA}.
\item Die restlichen starken Varianten werden durch verz"ogertes
  Reizen der Trumpfunterst"utzung gezeigt (Farbwechsel).
\end{itemize}

\subsection{Antworten auf 1\of-Er"offnung}

\bidins{%
  1\SA & 6-10 & kein Anschluss, keine weitere \ofa\\[1ex]
  2\tre & 10\pl & \textbf{k"unstlich}, selbstforcierend\\[1ex]
  2\kar & 10\pl & 5\pl\kar\\[1ex]
  2\of & 6-10 & 3er-Anschluss\\[1ex]
  2\SA & 12\good\pl & 4\pl{}er-Anschluss \conv{(Stenberg)}, \pf\ (siehe
    \ref{stenberg}, S.~\pageref{stenberg})
    (nach vormaligem Pass: 11-12, Double-Anschluss und 3er\aof)\\[1ex]
  3\tre & 9\good-11 & 4\pl{}er-Anschluss \conv{(Bergen Raise)}\\
  3\kar & 7-9\bad & 4\pl{}er-Anschluss \conv{(Bergen Raise)}\\
  3\of & 0-6 & 4\pl{}er-Anschluss \conv{(Bergen Raise)}\\[1ex]
  3\aof & 11-14 & beliebiges Chicane \conv{(Splinter)}, s.~u.\\
  3\SA & 11-14 & Single in \aofa\ \conv{(Splinter)}, s.~u.\\
  4\tre/\ka & 11-14 & \tr/\ka-Single \conv{(Splinter)}\\[1ex]
  4\of && zum spielen\\[1ex]
  4\aof && \conv{Exclusion RKCB}\\[3ex]
  \multicolumn{3}{l}{Nach 1\coe-Er"offnung:}\\[1ex]
  2\pik & 5-8 & 6\pl\pik\ \conv{(Weak Jump)}\\[2ex]
  \multicolumn{3}{l}{Nach 1\pik-Er"offnung:}\\[1ex]
  2\coe & 10\good\pl & 5\pl\coe
}

\notebox{%
\conv{Splinter}-Gebote zeigen immer einen mindestens guten
3er-Anschluss in der Trumpffarbe sowie eine K"urze.  Die Werte f"ur
ein Vollspiel werden versprochen, aber die Punktspanne ist nach oben
limitiert.  Mit st"arkeren H"anden sollte \conv{Stenberg}
gereizt werden.
}

\subsubsection{Spezielle \conv{Splinter}-Gebote nach 1\of-Er"offnung}
[1\of-3\SA;] zeigt ein Single in der anderen \of.

Chicane-\conv{Splinter} werden mit einem Sprung in die
andere \of\ auf 3er-Stufe gezeigt:
\bdsc
\item[1\coe\sep3\pik; ?] beliebiges Chicane
  \bdsc
  \item 3\SA\ \rel, Frage nach dem Chicane
    \bdsc
    \item[4\tre] \tr-Chicane
    \item[4\kar] \ka-Chicane
    \item[4\coe] \pi-Chicane
    \edsc
  \edsc
\item[1\pik\sep3\coe; ?] beliebiges Chicane
  \bdsc
  \item 3\pik\ \rel, Frage nach dem Chicane
    \bdsc
    \item[3\SA] \co-Chicane
    \item[4\tre] \tr-Chicane
    \item[4\kar] \ka-Chicane
    \edsc
  \edsc
\edsc

\subsection{Bietsequenzen nach 1\of-Er"offnung}

\bdsc
\item[1\coe\sep1\pik; ?]~
  \bdsc
  \item[1\SA] 12-14 FP \bal
    \bdsc
    \item[2\tre] \conv{Checkback Stayman}
    \item[2\kar] 4er\pik, 5\pl\kar, zum spielen
    \item[3\tre] \nat, forcing
      \bdsc
      \item[3\kar] Frage nach \hstp, zeigt \khstop, verneint 3er\pik\
        (siehe \conv{VFF})
      \item[3\coe] 2-5-3-3, verneint \khstop
      \item[3\pik] 3er\pik, \maxi
      \item[3\SA] \kstop, verneint 3er\pik
      \item[4\pik] 3er\pik, \mini
      \edsc
    \edsc
  \item[2\tre/\ka] 12-18, 54\pl-Verteilung
    \bdsc
    \item[2\coe] Ausbessern, \coe-Double
    \item[3\tre/\ka] \inv
    \edsc
    \textbf{Nach \conv{1~"uber~1} ist die Hebung von \eo's \emph{zweiter} Farbe
    auf die 3er-Stufe lediglich einladend.}
  \item[2\SA] 18-19 FP \bal
  \item[3\tre/\ka] 18\good\pl\ FP, 5\pl\coe\ und 4\pl\uf\\
    \ra~3\coe\ = st"arker als 4\coe\ (\emph{Principle of Fast Arrival});
    \ra~3\pik\ ist dann \conv{Cue Bid}, aber keine K"urze
    
    \textbf{Ein \conv{Cue Bid} in Partners erster Farbe zeigt \emph{nie} eine
    K"urze.}
  \edsc

\item[1\pik\sep2\tre; ?]~
\bdsc
\item[2\kar] 12-18 FP, 54\pl
  \bdsc
  \item[2\pik] \inv\ mit 3er\pik
  \item[2\SA] \nat\ \nf
  \item[3\tre] \nat\ \nf
  \item[3\kar] \pf
    \bdsc
    \item[3\coe] Frage nach \hstp\ \conv{(VFF)}
    \item[3\pik] kein \chstop, verspricht kein 6er\pik
    \item[3\SA] \cstop
    \edsc
  \item[3\pik] Schlemm-Interesse mit 3er\pik
  \edsc
  \textbf{Nach \conv{2~"uber~1} ist die Hebung von \eo's
    \emph{zweiter} Farbe auf die 3er-Stufe \pf.}
\item[2\coe]~
  \bdsc
  \item[2\SA] \nat
    \bdsc
    \item[3\kar] 5/5 in den \ofa, \pf
    \item[3\coe] 5/5 in den \ofa, \nf
    \edsc
  \item[3\coe] \pf
  \edsc
\item[2\pik] 12-14 FP, kann 5er\pik\ sein!
  \bdsc
  \item[3\tre] 6er\tre, \nf
  \item[3\kar/\co] Werte (siehe \conv{DFF}), \pf
  \item[3\pik] \inv, 3er\pik
  \edsc
\item[2\SA] 18-19 FP \bal
\item[3\tre] 16\pl\ FP, 54\pl\ \pik\,+\tre
  \bdsc
  \item[3\kar] \kar-Werte, \ra~3\coe\ = Frage nach \hstp
  \item[3\coe] \coe-Werte
  \edsc
\edsc
\item[1\pik\sep2\coe; ?]~
  \bdsc
  \item[3\coe] 16\pl\ FP, 5/3\pl\ \pi\,+\coe
  \item[4\coe] 12-14 FP, 4er-Anschluss (3er-Anschluss und \mini\
    "uber 2\pik)
  \edsc
\edsc

\bcbiddingpair{West}{Ost}
{
  1\pik & 2\kar\\
  3\tre\al{a} & 3\kar\al{b}\\
  3\SA\al{c}  & 4\SA\al{d}
}
{
  \Meaning{a}{5/4 in \pi\,+\tre}
  \Meaning{b}{forcing}
  \Meaning{c}{\cstop}
  \Meaning{d}{quantitativ}
}

\subsection{Weiterreizung nach [1\of\sep2\SA] \conv{(Stenberg)}\label{stenberg}}

2\SA\ auf eine 1\of-Er"offnung verspricht 4er-Anschluss und
Vollspielwerte, oder besser.

Der Er"offner reizt daraufhin eine K"urze, falls vorhanden. Hat er
keine K"urze, so reizt er mit 11-13 4\of, mit 14-15 3\SA\ und mit
16\pl\ 3\of\ (\conv{principle of fast arrival}, je h"oher wir reizen,
desto \emph{schw"acher} sind wir).

Hat \eo\ eine K"urze gezeigt, so fragt die n"achste Stufe (\rel) nach
Art der K"urze und nach Keycards.  Die erste Antwortstufe zeigt ein
Chicane, weiteres \rel\ ist \conv{RKCB}.  Alle anderen Antwortstufen
zeigen ein Single und beantworten gleichzeitig \conv{RKCB}.

Hat \eo\ keine K"urze gezeigt, so ist die n"achste Stufe \conv{RKCB}.

\bdsc
\item[1\of\sep2\SA; ?]~
  \bdsc
  \item[3\uf/\aof] K"urze in der gereizten Farbe, \ra~\rel\ = Frage
    nach Art der K"urze sowie \conv{RKCB},~\ra
    \bdsc
    \item[n"achste Stufe:] K"urze ist ein Chicane, \ra~\rel\ = \conv{RKCB}
    \item[Rest:] direkte 1430-Antworten auf
      \conv{RKCB}, K"urze ist ein Single
    \edsc
  \item[3\of] 16\pl, keine K"urze, \ra~\rel\ = \conv{RKCB}
  \item[3\SA] 14-15, keine K"urze, \ra~4\tre\ = \conv{RKCB}
  \item[4\of] 11-13, keine K"urze, \ra~\rel\ = \conv{RKCB}
  \edsc
\item[1\coe\sep2\SA; 3\tre\sep3\kar; ?]~
  
  (\eo\ hat eine K"urze in \tr, \aw\ fragt nach Keycards.)~\ra
  \bdsc
  \item[3\coe] Chicane in \tre\\
    \ra~3\pik\ = \conv{RKCB}
  \item[3\pik] \tre-Single, eine oder vier Keycards
  \item[3\SA] \tre-Single, keine oder drei Keycards
  \item[4\tre] \tre-Single, zwei Keycards ohne \co-Dame
  \item[4\kar] \tre-Single, zwei Keycards mit \co-Dame
  \edsc
\edsc

\subsubsection{Beispielreizungen zu \conv{Stenberg}}

\westhand{432}{AB10852}{A3}{K2}
\easthand{KB6}{K976}{K94}{A87}
\centerline{\showEWgame}
\bcbiddingpair{West}{Ost}
{
  1\coe & 2\SA\\
  4\coe\al{a} & pass
}
{
  \Meaning{a}{Minimum (11-13)}
}

\vfill
\westhand{4}{AB10852}{A1042}{K2}
\easthand{876}{KD76}{KD3}{A87}
\centerline{\showEWgame}
\bcbiddingpair{West}{Ost}
{
  1\coe & 2\SA\\
  3\pik\al{a} & 3\SA\al{b}\\
  4\pik\al{c} & 6\coe
}
{
  \Meaning{a}{\pi-K"urze (Single oder Chicane)}
  \Meaning{b}{Frage}
  \Meaning{c}{\pi-Single, zwei Keycards ohne Trumpf-Dame}
}

\vfill
\westhand{KB1042}{DB753}{2}{K3}
\easthand{AD53}{AK}{653}{A542}
\centerline{\showEWgame}
\bcbiddingpair{West}{Ost}
{
  1\pik & 2\SA\\
  3\kar\al{a} & 3\coe\al{b}\\
  3\SA{c} & 6\pik
}
{
  \Meaning{a}{\ka-K"urze (Single oder Chicane)}
  \Meaning{b}{Frage}
  \Meaning{c}{\ka-Single, eine oder vier Keycards}
}

\vfill
\westhand{K752}{AD542}{-}{K742}
\easthand{A8}{KB76}{D62}{ADB3}
\centerline{\showEWgame}
\bcbiddingpair{West}{Ost}
{
  1\coe & 2\SA\\
  3\kar\al{a} & 3\coe\al{b}\\
  3\pik\al{c} & 3\SA\al{d}\\
  4\tre\al{e} & 4\kar\al{f}\\
  4\pik\al{g} & 4\SA\al{h}\\
  5\tre\al{i} & 7\coe
}
{
  \Meaning{a}{\ka-K"urze (Single oder Chicane)}
  \Meaning{b}{Frage}
  \Meaning{c}{\kar-Chicane}
  \Meaning{d}{\conv{RKCB}}
  \Meaning{e}{eine oder vier Keycards}
  \Meaning{f}{Frage nach \co-Dame}
  \Meaning{g}{\co-Dame und \pi-K"onig vorhanden}
  \Meaning{h}{Frage nach weiteren K"onigen}
  \Meaning{i}{\tr-K"onig vorhanden}
}

\section{Die 1\SA-Er"offnung}

Die Er"offnung 1\SA\ verspricht 15-17~FP und eine ausgeglichene
Verteilung.  5332-Verteilungen mit einer 5er-\ofa\ sind m"oglich.

\subsection{Antworten auf 1\SA-Er"offnung}
\bidins{%
2\tre & ab 0 & \conv{Stayman}, verspricht 4er-\ofa\\
2\kar/\co & ab 0 & \xferto\coe/\pi, verspricht
5\pl{}er-Farbe\\
2\pik & ab 0 & \xferto\ufa, verspricht 6\pl{}er-\ufa\ oder
55\uf.\\
2\SA & 8-9 & \nat, \inv\\
3\,$x$ && 6\pl{}er-Farbe, \slamint{}\\
3\SA && zum spielen\\
4\tre && \conv{Gerber}-Assfrage (Antworten: 0 oder 4, 1, 2, 3)\\
4\kar && 55\pl\ in den \ofa, kein \slamint{}\\
4\SA && quantitativ\\
5\SA && quantitativ (Einladung zum Gro"sschlemm)
}

\subsection{Weiterreizung nach 1\SA-Er"offnung}

\subsubsection{Reizung von \ofa-Zweif"arbern ("Ubersicht)}

Folgende Tabelle gibt eine "Ubersicht, wie beim Reizen von
\ofa-Zweif"arbern (5/4\pl-Verteilung) zu verfahren ist -- jeweils mit
schwachen, einladenden und starken H"anden.  Details siehe unten.

\begin{tabular}[t]{|l|c|c|}
\hline
\textbf{St"arke} & \textbf{5/4} & \textbf{5/5}\\
\hline
\hline
schwach & \conv{Stayman} & \conv{Transfer}\\
\hline
\inv & \multicolumn{2}{c|}{\conv{Transfer}}\\
\hline
\pf & \conv{Stayman}, dann \conv{Smolen} & 4\kar\\
\hline
\pf\ (stark) & \conv{Stayman}, dann \conv{Smolen} & \conv{Transfer}\\
\hline
Schlemm & \multicolumn{2}{c|}{\conv{Stayman}, dann \conv{Smolen}}\\
\hline
\end{tabular}

\subsubsection{Schwache und einladende \conv{Stayman}-Sequenzen}

Mit schwachen (0-7~FP) H"anden ist \conv{Stayman} nur erlaubt, wenn
man jede Antwort von \eo\ (2\kar, 2\coe, 2\pik) passen kann -- also
mit 4-4-4-1, 4-4-5-0, 3-4-5-1 und 4-3-5-1 oder mit 5/4 in den \ofa.

\bdsc
\item[1\SA\sep2\tre; 2\kar\sep?]~
\bdsc
\item[2\coe] 4/4 oder 5/4 in \coe/\pi, mit 3/3~\ofa\ muss \eo\ passen.
\item[2\pik] 5/4~\pik/\co, \nf
\edsc
\edsc

Mit einladenden H"anden reizen wir eine 5/4-Verteilung in den \ofa\
"uber \conv{Transfer}.

\subsubsection{Starke \conv{Stayman}-Sequenzen}

\bdsc
\item[1\SA\sep2\tre; ?]~
\bdsc
\item[2\kar] keine 4er-\ofa
\bdsc
\item[3\coe] \pf\pl, \xferto\pi\ \conv{(Smolen Transfer)};
  verspricht 4er\coe\ und 5er\pik
  \begin{itemize}
  \item mit 5er\coe\ besteht starkes \slamint{}
  \item mildes \slamint{} mit 5/5~\ofa\ "uber \conv{Transfer}
  \item ohne \slamint{} mit 5/5~\ofa\ sofort 4\kar\ reizen
  \end{itemize}
\item[3\pik] \pf\pl, \xferto\co\ \conv{(Smolen Transfer)};
  verspricht 5er\pik\ und 4er\coe\ (siehe oben)
\edsc
\item[2\coe] 4er\coe\ (4er\pik\ m"oglich)
  \bdsc
  \item[2\pik] \slamint{} in \co; keine K"urze
    \bdsc
    \item[2\SA] negativ
    \item[3\tre/\ka] \conv{Cue Bid}
    \edsc
  \item[3\pik-4\kar] \conv{Splinter}
  \item[4\SA] quantitativ mit 4er\pik
  \edsc
\item[2\pik] 4er\pik
  \bdsc
  \item[3\coe] \slamint{} in \pi; keine K"urze
    \bdsc
    \item[3\pik] positiv
    \item[3\SA] negativ
    \item[4\tre/\ka] \conv{Cue Bid}
    \edsc
  \edsc
\edsc
\edsc

\section{Verhalten nach Zwischenreizung durch die Gegner}

\subsection{Allgemeines}
\begin{itemize}
\item Das Reizen einer neuen Farbe auf der 1er-Stufe ist weiterhin
forcierend f"ur eine Runde.
%
\item Das Reizen einer neuen Farbe auf der 2er-Stufe ist nicht forcierend
falls die Zwischenreizung gest"ort hat, sonst forcierend:
\begin{description}
\item[1\coe\sep(1\pik)\sep2\kar] ist forcierend da man auch
ohne die Zwischenreizung 2\kar\ gereizt h"atte.
\item[1\tre\sep(1\pik)\sep2\coe] ist nicht forcierend da man ohne die
  Zwischenreizung 1\coe\ gereizt h"atte.
\end{description}
%
\item Eine neue Farbe auf der 3er-Stufe ist immer forcierend
  (z.~B. 1\pik\sep(2\kar)\sep3\tre).
\item Nach \conv{Informationskontra} ist eine neue Farbe auf der 2er-Stufe
  nicht forcierend.
\item Nach \conv{Weak Jumps} ist das Reizen einer neuen Farbe immer
  forcierend.
\item \conv{Kontra} ist negativ bis 3\coe\ oder zeigt eine beliebige
  starke Hand f"ur die es in der augenblicklichen Situation kein Gebot
  gab.
\item Der \conv{"Uberruf}
  \begin{itemize}
    \item nach \ufa-Er"offnung fragt nach Stopper.
    \item nach \ofa-Er"offnung zeigt eine partieforcierende Hand mit
      Anschlu"s in der \ofa\ (siehe unten).
    \end{itemize}
\item 1\SA\ und 3\SA\ sind nat"urlich.
\end{itemize}

\subsection{Nach \of-Er"offnung}
\begin{itemize}
\item Alle \ofa-Hebungen sind schwach.  Falls \conv{Bergen Raises}
  noch m"oglich sind, werden diese angewendet.
\item {[}1\of\sep($x$)\sep2\SA{]} zeigt
  \begin{itemize}
  \item wenn \conv{Bergen Raises} noch m"oglich sind: eine einladende
    Hebung mit genau 3er-Anschluss,
  \item sonst: eine einladende Hebung mit 3er- oder 4er-Anschluss.
  \end{itemize}
\item {[}1\of\sep(\kontra)\sep2\SA{]} zeigt
  \begin{itemize}
    \item eine einladende Hebung mit genau 3er-Anschluss oder
    \item eine partieforcierende Hebung.
  \end{itemize}
Die anderen F"alle k"onnen mittels \conv{Bergen} gezeigt werden.
\item Der \conv{"Uberruf} der gegnerischen Farbe zeigt eine
  partieforcierende Hand mit Anschluss in der \of\
  ([1\of\sep(1\,$x$)\sep2\,$x$] oder [1\of\sep(2\,$x$)\sep3\,$x$]).
\end{itemize}

\subsubsection{Beispiele}
\begin{description}
\item[1\pik\sep(2\kar)\sep2\pik] ist m"oglich ab 0 Punkten mit
  3er-Anschluss.
\item[1\coe\sep(1\pik)\sep3\kar] ist weiterhin \conv{Bergen} und zeigt
  7-9 Punkte mit 4er-Anschluss.
\item[1\coe\sep(2\tre)\sep3\kar] ist nat"urlich und forcierend
  (s.~o.), zeigt keinen Coeur-Anschluss (\conv{Bergen Raises} sind
  nicht mehr m"oglich, da das 3\tre-Gebot nun eine andere Bedeutung
  hat).
\item[1\pik\sep(X)\sep2\SA] zeigt 3er\pik-Anschluss, einladend oder
  eine Hand ab 12 Punkten mit 4er- oder besserem Anschluss (die w"are
  zu stark f"ur \conv{Bergen}).
\item[1\pik\sep(2\kar)\sep2\SA] zeigt 3er- oder 4er\pik-Anschluss und
  ist einladend (3\kar\ w"are partieforcierend; \conv{Bergen} kann
  nicht mehr gereizt werden um die einladende Hand mit 4er\pik{}
  zu zeigen).
\item[1\coe\sep(1\pik)\sep2\pik] zeigt eine partieforcierende Hand mit
  Coeur-Anschluss.
\end{description}

\subsection{Nach \uf-Er"offnung}
\begin{itemize}
\item Es gilt weiterhin \conv{Inverted Minors}.
\item Der \conv{"Uberruf} der gegnerischen Farbe fragt nach
  \SA-Stopper ([1\uf\sep(1\,$x$)\sep2\,$x$] oder
  [1\uf\sep(2\,$x$)\sep3\,$x$]).
\item {[}1\uf\sep(1\,$x$)\sep2\SA] ist \conv{Inverted} (2-5~FP mit
  5\pl{}er Anschluss in der er"offneten Unterfarbe; wie in einer
  ungest"orten Reizung).
\end{itemize}

\subsubsection{Beispiele}
\begin{description}
\item[1\kar\sep(\kontra)\sep2\kar] zeigt 10\pl\ Punkte und 5\pl{}er
  Karo-Anschluss.
\item[1\tre\sep(1\pik)\sep2\SA] zeigt 2-5 Punkte und 5\pl{}er
  Treff-Anschluss.
\item[1\kar\sep(2\tre)\sep3\tre] ist partieforcierend und fragt nach
  Treff-Stopper.
\end{description}

\subsection{Weitere Bietsequenzen}

\bdsc
\item[1\tre\sep(1\kar)\sep?]~
\bdsc
\item[\kontra] beide \of\ \conv{(Negativkontra)}
\item[1\coe/\pi] mindestens 4er-L"ange
\item[2\kar] \pf\ und Frage nach \ka-Stopper
\item[3\kar] Karo-\conv{Splinter}
\edsc
\item[1\tre\sep(1\coe)\sep?]~
\bdsc
\item[\kontra] genau 4er\pik\ oder starke mit \ka-L"ange
\item[1\pik] mindestens 5er\pik
\item[2\kar] \nf
\item[2\coe] \pf\ und Frage nach \co-Stopper
\item[3\coe] Coeur-\conv{Splinter}
\edsc
\item[1\tre\sep(1\pik)\sep?]~
\bdsc
\item[\kontra] genau 4er\coe\ oder starke mit \ka- oder \co-L"ange
\item[2\kar] \nf
\item[2\coe] mindestens 5er\coe, \nf
\item[2\pik] \pf\ und Frage nach \pi-Stopper
\item[3\pik] Pik-\conv{Splinter}
\edsc
\edsc

Die Reizung nach 1\kar-Er"offnung ist analog.

\bdsc
\item[1\coe\sep(1\pik)\sep?]~
\bdsc
\item[\kontra] beide \ufa\ \conv{(Negativkontra)}
\item[2\tre/\ka] 10\pl\ FP, 4\pl{}er-L"ange, forcierend
\item[2\coe] ab 0 FP, 3er-Anschluss
\item[2\pik] \pf\ mit 3\pl{}er-Anschluss
\item[2\SA] \inv\ mit \emph{genau} 3er-Anschluss
\item[3\tre/\ka] weiterhin \conv{Bergen Raises}
\item[3\pik] beliebiger Single-\conv{Splinter} (wie in ungest"orter
  Reizung)
\edsc
\item[1\coe\sep(2\kar)\sep?]~
\bdsc
\item[\kontra] \conv{Negativ} oder stark mit \pik-L"ange
\item[2\pik] \nf\ (denn das 1\pik-Gebot war nicht mehr frei)
\item[2\SA] \inv\ mit 3er- oder 4er-Anschluss
\item[3\tre] \nat, forcing
\item[3\kar] \pf\ mit 3\pl{}er-Anschluss
\edsc
\edsc

Alle "ubrigen Gebote unterscheiden sich nicht von einer ungest"orten
Reizung (\conv{Fit Jumps}, \conv{Inverted Minors}, \conv{Weak Jumps}).

\subsection{Der Gegner reizt einen Zweif"arber \conv{(Unusual over
    Unusual)}}

Reizt der Gegner "uber unsere 1\,$x$-Er"offnung einen Zweif"arber, bei
dem beide Farben bekannt sein m"ussen, so verwenden wir die Konvention
\conv{Unusual over Unusual}:
\begin{itemize}
\item \conv{Kontra} zeigt Straf-Bereitschaft f"ur mindestens eine der
  beiden gegnerischen Farben.
\item Die niedrigere bzw. h"ohere der gegnerischen Farben steht
  f"ur die niedrigere bzw. h"ohere unserer Farben.
\item Direktes Heben der er"offneten Farbe ist rein kompetitiv.
\item Indirektes Heben der er"offneten Farbe (durch Reizen der
  entsprechenden gegnerischen Farbe) ist mindestens einladend.
\item Direktes Reizen der freien Farbe zeigt 5er-L"ange und ist
  forcierend.
\item Indirektes Reizen der freien Farbe zeigt 5er-L"ange und ist nur
  einladend.
\end{itemize}

\minisec{Beispiel:}
\bdsc
\item[1\coe\sep(2\SA*)\sep?] (2\SA\ = beide \ufa)
\bdsc
\item[3\tre] \inv\pl\ mit \coe-Anschluss
\item[3\kar] \inv\ mit \pik-L"ange
\item[3\coe] kompetitiv (ersetzt 2\coe-Gebot)
\item[3\pik] \pf\ mit \pik-L"ange
\edsc
\edsc

\subsection{Beide Gegner reizen eine Farbe}

Reizen beide Gegner eine Farbe, so ist der \conv{"Uberruf} einer der
Gegnerfarben \emph{nicht} die Frage nach Stopper sondern \emph{zeigt}
einen Stopper in der "uberrufenen Farbe:

\bidex
{
  1\coe & 1\pik & 2\tre & 2\kar\\
  -- & -- & 2\pik\al{a} & --\\
  2\SA\al{b}
}
{
  \Meaning{a}{zeigt \pstop}
  \Meaning{b}{\nat, zeigt \kstop\ (3\kar\ w"are Frage nach \khstop)}
}

\bidex
{
  1\kar & 1\pik & \kontra & 2\tre\\
  2\pik\al{a} & -- & 2\SA\al{b}
}
{
  \Meaning{a}{zeigt \pstop}
  \Meaning{b}{\nat, zeigt \tstop}
}

\subsection{Nach \SA-Er"offnung}

\section{Die 2\tre-Er"offnung}

\section{Die 2\kar-Er"offnung}

\section{Die 2\coe-Er"offnung}

\section{Die 2\pik-Er"offnung}

\section{Die 2\SA-Er"offnung}

\section{Sperransagen auf 3er-Stufe}

\section{Die 3\SA-Er"offnung}

\section{Die 4\ufa-Er"offnung}

\section{Die Gegenreizung}

\subsection{Informationskontra}

Besonderheit bei Antworten: Der "Uberruf der er"offneten Farbe ist entweder
stark (11\pl\ FP) oder zeigt eine Hand mit beiden \ofa\ und 8-10 FP. Der Sprung
in eine \ofa\ (2er-Stufe) zeigt 8-10 FP und eine 4er-Farbe. Der Doppelsprung
(3er-Stufe) verspricht die gleiche Punktzahl, aber eine 5er-Farbe.

Nach [(1\ofa)\sep\kontra\sep(2\ofa)] ist 2\SA\ Lebensohl, ebenfalls nach Weak Two-
Er"offnung: [(2\ofa)\sep\kontra\sep(pass)].

\subsection{Farbgegenreizung}

Antworten:

\begin{itemize}
\item Alle Hebungen sind schwach.
\item Neue Farbe ist \nf\ (8-12 FP).
\item Neue Farbe im Sprung ist Fit-Jump.
\item Der "Uberruf zeigt starke H"ande:
  \begin{itemize}
  \item eine mindestens \inv\ Hand mit Fit oder
  \item einen Einf"arber mit mehr als 12 FP oder
  \item eine starke \SA-Hand ohne \stp.
  \end{itemize}
\end{itemize}

\subsection{Crash gegen starke 1\tre-Er"offnung}

\bdsc
\item[(1\tre)] 16\pl\ FP, bel. Vert.
\bdsc
\item[\kontra] \underline{C}olour - gleichfarbige (\tr/\pi\ oder \ka/\co)
\item[1\kar] \underline{Ra}nk - gleichrangige (\tr/\ka\ oder \co/\pi)
\item[1\of] \nat
\item[1\SA] \underline{Sh}ape - gleichf"ormige (\tr/\co\ oder \ka/\pi)
\item[2\uf] \nat
\edsc
\edsc

\subsection{Die 1\SA-Gegenreizung}

Die 1\SA-Gegenreizung in zweiter Position zeigt 15-18 FP und \stp. Weiterreizung wie nach
1\SA-Er"offnung, d.h. Stayman und Transfers wenn der rechte Gegner passt,
ansonsten Lebensohl.

Hat der Gegner eine \ofa\ er"offnet, so ist der Transfer \emph{in} diese \ofa\
Stayman. 2\tre\ ist dann ebenfalls ein Transfer.

\subsection{Weak Jumps}

Weak Jumps zeigen 6-10 FP und eine 6\pl{}er-Farbe, manchmal auch blo"s eine
5er-Farbe. 2\SA\ vom \eo\ fragt nach Qualit"at und St"arke, siehe \ref{ogust},
S.~\pageref{ogust}.

\subsection{Michaels Pr"azis}

\bdsc
\item[(1\ufa)\sep2\ka] 5/5\pl\ in \ofa
\item[(1\ufa)\sep2\SA] 5/5\pl\ in \co/\aufa
\item[(1\ofa)\sep2\ofa] 5/5\pl\ in \aofa/\tr
\item[(1\ofa)\sep2\SA] 5/5\pl\ in \tr/\ka
\item[(1\ofa)\sep3\tre] 5/5\pl\ in \aofa/\ka
\edsc

Nach \ofa-Er"offung gibt somit keinen schwachen Sprung in \tr. Eine
vorbereitende 1\tre-Er"offnung wird als echt behandelt.

\subsection{Gegenreizung gegen 1\SA: Multi-Landy}

\bdsc
\item[(1\SA)\sep?] ~
 \bdsc
 \item[\kontra] Gegen starken \SA: 5\pl{}er-\ufa\ und 4er \ofa,
  gegen schwachen \SA: mindestens gleiche St"arke
 \item[2\tre] 5\pl/4\pl\ in \ofa
  \bdsc
  \item[2\kar] zeigt gleiche L"ange in \ofa
  \edsc
 \item[2\kar] Einf"arber in einer \ofa
 \item[2\ofa] 5\pl{}er \ofa\ und 4er \ufa
 \item[2\SA] 5/5\pl\ in \ufa
 \item[2\ufa] Einf"arber in einer \ufa
 \edsc
\edsc

\section{Konventionen zum System}

\subsection{Long Suit Trial Bids}

Nach [1\ofa\sep2\ofa;] ist 2\SA\ ein allgemeines Trial Bid was zu 3 oder 4 in
\ofa\ einl"adt. \eo\ muss nicht \bal\ sein, er hat lediglich keine
unterst"utzungbed"urftige Farbe. \aw\ soll mit Minimum Sign Off geben und mit
Maximum das Vollspiel ansagen. Mit massierten Werten in einer Farbe reizt der
\aw\ diese Farbe auf 3er-Stufe. Bei St"orung durch die Gegner ersetzt die
niedrigste freie Farbe, sofern es noch eine gibt, das 2\SA-Trial Bid.

\subsubsection{Beispielreizungen zu Trial Bids}

\bcbiddingpair{West}{Ost}
{
  1\tre & 1\coe \\
  1\pik & 2\pik \\
  2\SA\al{a}
}
{
  \Meaning{a}{allgemeines Trial Bid}
}

\bcbiddingpair{West}{Ost}
{
  1\tre & 1\pik \\
  2\pik & 2\SA\al{a}
}
{
  \Meaning{a}{allgemeines Trial Bid, auch als Schlemmvorbereiung nutzbar}
}

\bidex
{
  1\pik & 2\tre & 2\pik & 3\tre \\
  pass\al{a} \\
  \kontra\al{b} \\
  3\kar\al{c} \\
  3\coe\al{d} \\
  3\pik\al{e}
}
{
  \Meaning{a}{5er-\pik, Minimum}
  \Meaning{b}{Strafe}
  \Meaning{c}{allgemeines Trial Bid (n"achste Stufe ersetzt 2\SA)}
  \Meaning{d}{normales Trial Bid}
  \Meaning{e}{kompetitiv}
}

\bidex
{
  1\pik & 2\coe & 2\pik & 3\coe \\
  \kontra\al{a}
}
{
  \Meaning{a}{allgemeines Trial Bid (Competitive \kontra/Full Value \kontra; Gegnerfarbe direkt unter uns)}
}

\subsection{Checkback-Stayman}

Der \aw\ benutzt diese Konvention mit einer 5er \ofa\ und einer mindestens
einladenden Hand, wenn der \eo\ 1\SA\ zur"uckgeboten hat; der \aw\ m"ochte
wissen, ob der \eo\ einen 3er-Anschluss hat.

\bdsc
\item[1\ufa\sep1\pik; 1\SA\sep2\tre] Checkback-Stayman
 \bdsc
 \item[2\kar] 12-13\bad\ FP, kein 3er-\pi, kein 4er-\co
  \bdsc
  \item[2\coe] 10/11 FP, 5/5 in \co\ und \pi, \nf
  \item[2\pik] 9-11 FP, 6er-\pi, \nf
  \item[2\SA] 11/12 FP, 5er-\pi, \nf
  \item[3\coe] 5/5 in \co\ und \pi, \slamint
  \edsc
 \item[2\coe] 12-13\bad\ FP, 4er-\co, 3er-\pi\ m"oglich
 \item[2\pik] 12-13\bad\ FP, 3er-\pi
 \item[2\SA]  13\good-14 FP, kein 3er-\pi, kein 4er-\co
 \item[3\ufa] 13\good-14 FP, 3er-\pi, gute \ufa
 \item[3\pik] 13\good-14 FP, 3er-\pi
 \edsc
\edsc

\subsection{Dritte Farbe Forcing (DFF)}

Nach Farbwiederholung des \eo\ ist die dritte, vom \aw\ gereizte Farbe DFF.
Nach \ofa-Er"offnung soll der \eo\ vorrangig 3er-Anschluss in der \aofa\ des
\aw\ zeigen, je nach St"arke billig oder im Sprung. Die dritte Farbe zeigt aber
auch Werte in der Absicht einen \SA-Kontrakt anzusteuern. Die dritte Farbe auf
der 2er-Stufe gereizt zeigt eine mindestens einladende Hand, auf der 3er-Stufe
eine partieforcierende.

\bcbiddingpair{West}{Ost}
{
  2\tre\al{a} & 2\kar\al{b} \\
  2\coe & 2\tre \\
  3\coe & 3\pik\al{c}
}
{
  \Meaning{a}{Semiforcing}
  \Meaning{b}{Relais}
  \Meaning{c}{DFF, zeigt \pi-Werte}
}

\subsection{Vierte Farbe Forcing (VFF)}

Die Vierte Farbe auf der 2er-Stufe gereizt ist mindestens einladend, auf der
3er-Stufe partieforcierend. Priorit"aten des Antwortenden:

1. 3er-Anschluss in der \ofa\ des Partners zeigen

\bcbiddingpair{West}{Ost}
{
  1\coe & 1\pik \\
  2\tre & 2\kar\al{a} \\
  2\pik\al{b} \\
  3\pik\al{c}
}
{
  \Meaning{a}{VFF}
  \Meaning{b}{3er-\pi, Minimum}
  \Meaning{c}{3er-\pi, Zusatzwerte}
}

2. \SA\ mit Stopper in der vierten Farbe reizen

\bcbiddingpair{West}{Ost}
{
  1\coe & 1\pik \\
  2\tre & 2\kar\al{a} \\
  2\SA\al{b} \\
  3\SA\al{c} \\
  4\SA!\al{d}
}
{
  \Meaning{a}{VFF}
  \Meaning{b}{12-14 FP, kein 3er-\pi, \ka-\stp}
  \Meaning{c}{15-16 FP, kein 3er-\pi, \ka-\stp}
  \Meaning{d}{\nat, 17-18 FP, kein 3er-\pi, \ka-\stp}
}

3. Verteilung zeigen

Nach [1\tre\sep1\kar; 1\coe\sep1\pik;] darf der \eo\ mit
\hand{KD94}{AB42}{2}{AKB2} nicht in 4\pik\ springen, da 1\pik\ VFF war und nicht
nat"urlich gewesen sein muss. (Wenn 1\pik\ nat"urlich war, zeigt dies eine
Er"offnung wegen Walsh.) In dieser Situation ist das richtige Gebot 3\coe!

Ebenso zeigt man eine starke Hand mit Trumpunterst"utzung, wenn man die vierte
Farbe reizt und anschlie"send die Farbe des Partners hebt. Wird die vierte
Farbe im Sprung gereizt, ist dies nat"urlich und zeigt einen partieforcierenden
5/5-Zweif"arber.

Manchmal fragt die vierte Farbe auch nur nach Halbstopper, n"amlich dann, wenn
der Partner einen Vollstopper in der vierten Farbe bereits verneint hat.
Halbstopper sind: Kx, Dxx, Bxx, 10xxx.

\subsubsection{Beispielreizungen zu Vierte Farbe Forcing}

\bcbiddingpair{West}{Ost}
{
  1\tre & 1\kar \\
  2\tre\al{a} & 2\pik\al{b} \\
  3\tre\al{c} & 3\coe\al{d}
}
{
  \Meaning{a}{zeigt nach 1\kar\ 6er-Farbe}
  \Meaning{b}{DFF, Werte in \pi}
  \Meaning{c}{kein \co-\stp\ (sonst 2\SA)}
  \Meaning{d}{fragt nach \co-\hstp}
}
%
\bcbiddingpair{West}{Ost}
{
  1\kar & 1\coe \\
  2\kar & 3\tre\al{a} \\
  3\pik\al{b}
}
{
  \Meaning{a}{DFF, \tr-Werte, \pf, fragt \co-Anschluss}
  \Meaning{b}{VFF, kein 3er-\co, fragt nach \pi-\hstp}
}
%
\bcbiddingpair{West}{Ost}
{
  1\kar & 2\tre \\
  3\kar & 3\coe\al{a} \\
  3\pik\al{b}
}
{
  \Meaning{a}{DFF, \co-Werte, \pf, kein \pi-\stp\ (sonst 3\SA)}
  \Meaning{b}{VFF, fragt nach \pi-\hstp}
}
%
\bcbiddingpair{West}{Ost}
{
  1\tre & 2\tre\al{a} \\
  2\kar\al{b} & 3\coe\al{c} \\
  2\pik\al{d}
}
{
  \Meaning{a}{Inverted}
  \Meaning{b}{14\good\pl\ FP, \ka-Werte}
  \Meaning{c}{\co-Werte, kein \pi-\stp\ (sonst 2/3\SA)}
  \Meaning{d}{VFF, fragt nach \pi-\hstp}
}
%
\bcbiddingpair{West}{Ost}
{
  1\pik & 2\tre \\
  2\coe & 3\kar\al{a} \\
  3\coe\al{b} & 3\SA\al{c} \\
  pass\al{d}
}
{
  \Meaning{a}{VFF, fragt nach \ka-\stp}
  \Meaning{b}{kein \ka-\stp, muss kein 5er-\co\ sein}
  \Meaning{c}{zeigt \ka-\hstp}
  \Meaning{d}{ebenfalls \ka-\hstp}
}

Besonderheit:

\bcbiddingpair{West}{Ost}
{
  1\pik & 2\kar \\
  2\coe & 2\SA \\
  3\coe\al{a} \\
  3\tre\al{b}
}
{
  \Meaning{a}{\nf!, \ofa-Zweif"arber}
  \Meaning{b}{VFF, \ofa-Zweif"arber}
}

\subsection{Lebensohl}

Wir spielen Lebensohl in folgenden Situationen:
%
\begin{itemize}
\item nach \SA-Er"offnung des Partners und Farbgegenreizung der Gegner
 (Lebensohl \emph{mit} Stopper)
\item nach Weak Two-Er"offung des Gegners und Informationskontra des Partners
\item nach \ofa-Er"offnung des Gegners, Informationskontra des Partners und
 \ofa-Hebung des Partners des \eo\ (siehe auch Kompetitive 2\SA).
\end{itemize}

\subsection{Kontras}

\subsubsection{Informationskontra}

Siehe Gegenreizung.

\subsubsection{Negativkontra}

Siehe Verhalten nach Zwischenreizung der Gegner.

\subsubsection{Antwortkontra (Responsive \kontra)}

Nach Informationskontra oder Farbgegenreizung vom Partner und
\emph{Hebung der Er"offnerfarbe} zeigt das Antwortkontra die
nicht gereizten Farben (siehe auch Scrambling 2\SA).

\bidex
{
  1\kar & \kontra & 2\kar & \kontra\al{a} \\
  1\coe & \kontra & 2\coe & \kontra\al{b} \\
  1\coe & 1\pik & 2\coe & \kontra\al{c} \\
  1\tre & 1\coe & 2\pik & \kontra\al{d} \\
}
{
  \Meaning{a}{beide \ofa\ zu viert}
  \Meaning{b}{beide \ufa\ zu viert, kein 4er-\pi}
  \Meaning{c}{beide \ufa\ zu viert, kein 3er-\pi}
  \Meaning{d}{Snapdragon \kontra, zeigt 9/10\pl\ FP, 5er-\ka\ und Double-Figur in \co}
}

\subsubsection{Competitive \kontra}

Das Competitive \kontra benutzen wir, um noch weitere Information bez"uglich der St"arke
einer Hand mitzuteilen.

\bidex
{
  1\tre & 1\coe & 1\pik & 2\coe \\
  pass & pass & \kontra\al{a}
}
{
  \Meaning{a}{Zusatzst"arke, ab etwa 10\pl\ FP}
}

\bidex
{
  1\pik & 2\coe & 2\pik & 3\coe \\
  \kontra\al{a}
}
{
  \Meaning{a}{Full Value (siehe auch Trial Bids)}
}

\bidex
{
  1\pik & 2\tre & 2\pik & 3\tre \\
  \kontra\al{a}
}
{
  \Meaning{a}{Strafe (siehe auch Trial Bids)}
}

\bidex
{
  1\pik & 2\coe & 2\tre & pass \\
  pass & \kontra\al{a}
}
{
  \Meaning{a}{Wiederbelebung, zeigt gute Hand und Toleranz f"ur die nicht gereizten Farben}
}

\subsubsection{Ausspielkontra (Lightner \kontra)}

Das Kontra auf einen Endkontrakt verlangt normalerweise die vom Dummy zuerst gereizte Farbe.
Nach 1\SA\sep 3\SA\ verlangt Kontra das Ausspiel der k"urzeren Oberfarbe.

\subsubsection{Trump Support \kontra}

Diese Konvention benutzt der Er"offner, wenn der Partner eine Oberfarbe gereizt
und der n"achste Gegner gesprochen hat. Er zeigt mit \kontra\ bzw.
\kontra\kontra\ 3er-Anschluss in der Oberfarbe des Partners.

\bidex
{
  1\kar & pass & 1\pik & 2\tre \\
  2\pik\al{a} \\
  \kontra\al{b}
}
{
  \Meaning{a}{4er-Anschluss}
  \Meaning{b}{3er-Anschluss}
}

\bidex
{
  1\kar & pass & 1\pik & \kontra \\
  2\pik\al{a} \\
  \kontra\kontra\al{b}
}
{
  \Meaning{a}{4er-Anschluss}
  \Meaning{b}{3er-Anschluss}
}

\subsection{1\SA\ als Zweif"arber}

Diese Konvention benutzen wir in vierter Position nach Er"offnung, pass vom
Partner und einem Farbgebot auf der 1er-Stufe des rechten Gegners, um die
beiden nicht gereizten Farben zu zeigen.

S"ud \\
\dealerW
\vhand{DB108}{42}{53}{KD1094}
\begin{bidding}
  1\kar & pass & 1\coe & 1\SA \\
\end{bidding}

S"ud \\
\vhand{2}{AD105}{KB952}{643}
\begin{bidding}
  1\tre & pass & 1\pik & 1\SA \\
\end{bidding}

\subsection{2\SA\ als Zweif"arber}

In zweiter Postition ist 2\SA\ Michaels Pr"azis, in vierter Position wie 1\SA\
ein Zweif"arber, nur mit besserer Verteilung.

S"ud \\
\vhand{AB1087}{4}{73}{KD1093}
\begin{bidding}
  1\kar & pass & 1\coe & 2\SA \\
\end{bidding}

Haben die Gegner eine Oberfarbe er"offnet und gehoben, zeigt 2\SA\ einen
unbestimmten Zweif"arber. Die Reizung (1\kar)\sep pass\sep(1\pik)\sep2\pik\
zeigt eine echte Farbe.

\subsection{Nebul"ose 2\SA\ (Scrambling 2\SA)}

Haben die Gegner eine Farbe er"offnet \emph{und gehoben} und der Partner
gibt ein Informationskontra, sowohl in der direkten als auch in der Pass
Out-Position, so ist 2\SA\ nicht echt, sondern bedeutet, dass man kein klares
Gebot hat.

S"ud \\
\vhand{9754}{K52}{A64}{D54}
\begin{bidding}
  1\pik & pass & 2\pik & pass \\
  pass & \kontra & pass & 2\SA
\end{bidding}

Partner kann 1-5-4-3, 1-4-5-3 oder 1-4-3-5 verteilt sein,
wir wollen nicht im 3-3-Fit landen.

S"ud \\
\vhand{B65}{D3}{K1053}{K763}
\begin{bidding}
  1\pik & pass & 2\pik & pass \\
  pass & \kontra & pass & 2\SA
\end{bidding}

Partner wird entweder seine 5er-Farbe reizen oder seine 4er-Farben von unten
nach oben.

\subsection{Kompetitive 2\SA}

Hat der rechte Gegner ein Gebot auf der 2er-Stufe abgegeben, sind in
kompetitiven Situationen die 2\SA-Ansagen nicht nat"urlich, sondern zeigen den
Wunsch, in der 3er-Stufe zu spielen. Der Partner muss 3\tre\ bieten (siehe auch
Lebensohl), wonach der 2\SA-Reizer seine Farbe zeigt.

S"ud \\
\vhand{32}{AK865}{5}{KB1063}
\begin{bidding}
  & & & 1\coe \\
  pass & 1\SA & 2\pik & 2\SA \\
  pass & 3\tre & pass & pass
\end{bidding}

Ein direktes 3\tre-Gebot von S"ud h"atte eine st"arkere Er"offung gezeigt.

S"ud \\
\vhand{K65}{A74}{D1093}{873}
\begin{bidding}
  1\coe & 2\kar & 2\coe & 3\kar
\end{bidding}

Hier direkt 3\kar, konstruktiv.

S"ud \\
\vhand{865}{A74}{D1093}{873}
\begin{bidding}
  1\coe & 2\kar & 2\coe & 2\SA \\
  pass & 3\tre & pass & 3\kar
\end{bidding}

Mit dieser schwachen Hand zuerst 2\SA\ und sp"ater 3\kar.

S"ud \\
\vhand{5}{AKD63}{AKB74}{52}
\begin{bidding}
  & & & 1\coe \\
  1\pik & pass & 2\pik & 3\kar
\end{bidding}

Stark genug, um 3\kar\ zu reizen. Mit \co\ A 9 7 6 4 und der gleichen
\ka-Haltung w"urde man zuerst 2\SA\ reizen.

S"ud \\
\vhand{63}{AKB8643}{K73}{9}
\begin{bidding}
  & & & 1\coe \\
  1\pik & 1\SA & 2\pik & 2\SA \\
  pass & 3\tre & pass & 3\coe
\end{bidding}

Nicht immer muss der Partner des 2\SA-Reizers 3\tre\ sagen, dann n"amlich, wenn
er eine st"arkere Hand hat und die Gefahr besteht, dass der 2\SA-Reizer 3\tre\
passt:

S"ud \\
\vhand{A963}{A5}{762}{AB85}
\begin{bidding}
  & 1\kar & 1\coe & \kontra \\
  2\coe & 2\SA & pass & 3\coe
\end{bidding}

\subsection{Schlemmkonventionen}

\subsubsection{Mixed Cue Bids}

Kontrollgebote zeigen Erst- oder Zweitrundenkontrolle, also A/K/Single/Chicane.
Ein erstes Kontrollgebot auf der 5er-Stufe zeigt allerdings
Erstrundenkontrolle. Nicht K"urze in Partners erster Farbe als \emph{erstes
Kontrollgebot} reizen!

\westhand{2}{AD932}{AK92}{DB3}
\easthand{DB543}{B74}{DB3}{AK}
\centerline{\showEWgame}
\bcbiddingpair{West}{Ost}
{
  1\coe & 1\pik \\
  3\kar & 3\coe\al{a} \\
  4\kar\al{b} & 5\tre\al{c} \\
  5\kar\al{d} & 6\tre\al{e} \\
  6\coe\al{f}
}
{
  \Meaning{a}{st"arker als 4\coe}
  \Meaning{b}{\ka-Kontrolle}
  \Meaning{c}{keine \pi-Kontrolle, \tr-Erstrundenk.}
  \Meaning{d}{\pi-K"urze (sonst Sign off), \ka-Erstrundenk.}
  \Meaning{e}{\tr-Erst- und Zweitrundenkontrolle}
  \Meaning{f}{\pi-Single}
}

\subsubsection{Assfragen}

\paragraph{Gerber}

\bdsc
\item[1\SA\sep4\tre] Gerber
 \bdsc
 \item[4\kar/\coe/\pik/\SA] 0/4, 1, 2, 3 Asse
  \bdsc
  \item[4\SA] zum Spielen
  \item[\rel] Frage nach platzierten K"onigen
  \edsc
 \edsc
\edsc

\paragraph{(Roman) Keycard-Assfrage (KCB)}

4\SA\ auf Oberfarben-, 4\tre\ auf \tr- und 4\kar\ auf \ka-Basis. Antworten:
1/4, 3/0, 2 ohne Q, 2 mit Q. Anschlie"send rollend nach Trumpf-Dame und
gleichzeitig nach platzierten K"onigen. Hat man selbst die Trumpf-Dame, so ist
das "ubern"achste Gebot unter Auslassung der Trumpffarbe die Frage nach
platzierten K"onigen. Sind alle K"onige an Bord, kann man noch nach platzierten
Damen fragen.

Antwortschema: Hat der Antwortende die Trumpf-Dame nicht, geht er auf die
Trumpffarbe zur"uck. Hat er die Trumpf-Dame, aber keinen Nebenk"onig, so reizt
er 6 in der Trumpffarbe (das n"achste Gebot des Partners ist jetzt die Frage
nach platzierten Damen). Hat der die Trumpf-Dame und einen oder mehrere
Nebenfarbk"onige, so reizt er die niedrigste Farbe, in der er einen K"onig hat
und verneint damit gleichzeitig den K"onig in einer Farbe, die er h"atte
billiger reizen k"onnen. Das niedrigste \SA-Gebot zeigt den K"onig in der
Fragefarbe.

Man kann nun mit dem n"achsten Gebot nach weiteren K"onigen fragen, mit dem
"ubern"achsten nach platzierten Damen. Das Antwortschema bleibt sich gleich.

\westhand{A742}{76}{A2}{AKD52}
\easthand{KDB65}{AK}{KD3}{763}
\centerline{\showEWgame}
\bcbiddingpair{West}{Ost}
{
  1\tre & 1\pik \\
  4\tre\al{a} & 4\SA\al{b} \\
  5\kar\al{c} & 5\SA\al{d} \\
  6\tre\al{e} & 6\coe\al{f} \\
  7\tre\al{g} & 7\SA
}
{
  \Meaning{a}{\tr-Fit, gute \tr-Farbe}
  \Meaning{b}{KCB auf \pi-Basis}
  \Meaning{c}{3/0}
  \Meaning{d}{platziere K"onige?}
  \Meaning{e}{\tr-K"onig}
  \Meaning{f}{platziere Damen? (6\kar\ weitere K"onige?)}
  \Meaning{g}{\tr-Dame}
}

\westhand{A2}{K754}{AKD43}{52}
\easthand{K532}{ADB}{7632}{A3}
\centerline{\showEWgame}
\bcbiddingpair{West}{Ost}
{
  1\kar & 1\pik \\
  2\coe & 4\kar\al{a} \\
  4\pik\al{b} & 4\SA\al{c} \\
  5\coe\al{d} & 7\kar\al{e}
}
{
  \Meaning{a}{KCB}
  \Meaning{b}{3/0}
  \Meaning{c}{\ka-Dame?}
  \Meaning{d}{\ka-Dame und \co-K"onig, ohne \tr-K"onig}
  \Meaning{e}{bei 3er \pi\ und Single \tr\ leider chancenlos}
}

\paragraph{Preempt Keycard-Assfrage (PKCA)}

\paragraph{Exclusion Keycard-Assfrage (Voidwood)}

\subsubsection{DOPI-ROPI}

\subsubsection{Josephine}

\section{Ausspiele und Markierungen}

\subsection{Ausspiele gegen Farbkontrakte}

\subsection{Ausspiele gegen \SA-Kontrakte}

\subsection{Markierungen}

\begin{appendix}

\section{Glossar}

\begin{flushleft}
\begin{tabularx}{\columnwidth}{lX}%
$n$\good, $n$\bad & $n$ gute/schlechte Punkte/Karten \\
$n$\pl & mindestens $n$ Punkte/Karten \\
\ufa, \ofa & Unterfarbe(n), Oberfarbe(n) \\
\aufa, \aofa & andere \ufa, \ofa\\
\bal & ausgeglichen (4333, 4432, 5332)\\
\unbal & nicht ausgeglichen \\
\aw & Antwortender \\
\inv & einladend \\
\eo & Er"offner \\
\maxi & Maximum \\
\mini & Minimum \\
\nat & nat"urlich \\
\nf & nicht forcierend \\
\pf & Partieforcing \\
\rel & Relais \\
\stp & Stopper (A, Kx, Dxx, Bxxx) \\
\hstp & Halbstopper (K, Dx, Bxx, 10xxx) \\
\slamint & Schlemminteresse \\
\xfer & Transfer \\
\end{tabularx}%
\end{flushleft}

\section{"Anderungen am System}

Folgende "Anderungen sind m"oglich:
%
\begin{itemize}
\item 1\ofa\sep2\tre: nat"urlich
\item Unusual over unusual: "Uberruf der Gegnerfarben immer st"arker als direkte Reizung
\end{itemize}

\subsection{Checkback Stayman}

Checkback Stayman wurde durch den 2\tre/2\SA-Transfer (\ra\ref{1sarebid})
ersetzt. (Mai 2005)

\bdsc
\item[1\tre\sep1\pik; 1\SA] 12-14 FP \bal
  \bdsc
  \item[2\tre] \conv{Checkback Stayman}
  \item[2\kar] 4er\pik, 5\pl{}\kar, zum spielen
  \item[2\coe] 5/4 \pik\,+\coe, zum spielen oder ausbessern
  \item[2\pik] 5er\pik, schwach
  \item[2\SA] \nat
  \item[3\mi] \nat, \pf
  \item[3\coe] 5/4 \pik\,+\coe, \pf
  \item[3\pik] 6er\pik, Schlemm-Interesse\footnotemark[1]
  \edsc
\edsc

\section{Spieler}

Thomas Schmitt,
Regine Bartels,
Wendelin Albert,
Christoph Berg,
Frederic Boldt,
Helmut Horacek,
Frank Luithle.

\section{Layout}

4er \pi\ oder 4er\pik\ oder 4er-\pi?

4er \ofa\ oder 4er\of\ oder 4er-\ofa?

1\tre, 1\SA\ (nicht 1 \tr)


\end{appendix}

\end{document}

%%%%%%%%%%%%%%%%%%%%%%%%%%%%%%%%%%%%%%%%%%%%%%%%%%%%%%%%%%%%%%%%%%%%%%%%%%%%%%%%%%%%%%%%%5

\bcbiddingpair{West}{Ost}
{
}
{
  \Meaning{a}{}
  \Meaning{b}{}
  \Meaning{c}{}
  \Meaning{d}{}
}

\dealer{S, N--S\\\mbox{}}
\comment{Board 3\\\mbox{}}

\dealerS
\northhand{DB6}{D52}{B10432}{B5}
\southhand{A107}{A763}{KD7}{AK8}
\westhand{K9853}{B8}{65}{6432}
\easthand{42}{K1094}{A98}{D1097}

Erst nur die Nord-S\"ud-H\"ande mit Reizung zu viert \ldots

\playproblemNS
{
2\,SA & --- & 3\,SA & --- \\
---   & --- & --- 
}

\ldots oder zu zweit:

\playproblemNSpair
{
2\,SA & 3\,SA \\
---   & --- 
}

dann die ganze Hand

\medskip

\centerline{\showgame}

\newpage

Oder vielleicht doch lieber als Gegenspielproblem?

\bidders{Fritz}{Franz}{Fuzzy}{Alf}

\playproblemNW
{
2\,SA & --- & 3\,SA & --- \\
---   & --- & --- 
}

bzw.

\playproblemNWpair
{
2\,SA & 3\,SA \\
---   & --- 
}

oder andersherum

\medskip

\playproblemNE
{
2\,SA & --- & 3\,SA & --- \\
---   & --- & --- 
}

bzw.

\medskip

\playproblemNEpair
{
2\,SA & 3\,SA \\
---   & --- 
}

\newpage

\dealer{}
\comment{}

Oder haben die Gegner ein Reizproblem?

\medskip

\playproblemEW
{
2\,SA & --- & 3\,SA & --- \\
---   & --- & --- 
}

bzw.

\medskip

\playproblemEWpair
{
2\,SA & 3\,SA \\
---   & --- 
}

Ein Reizproblem zu viert:

\biddingproblem{1.}{ADB96}{KD52}{43}{82}
{
1\,\pik & 2\,\kar & 2\,\pik & 3\,\kar \\
?
}

und eins zu zweit (ohne Nummer):

\biddingproblempair{}{ADB96}{KD52}{43}{82}
{
1\,\pik & 2\,\pik \\
?
}

Wie soll man

\onesuitNS{ABxx}{Kxx}

auf 3 Stiche spielen?

Gewinnt Ihre Spielweise auch bei

\onesuit{ABxx}{Kxx}{xxxx}{Dx}

Dann haben Sie es richtig gemacht!

K\"onnen Sie es auch auf Ost-West?

\onesuitEW{KBxx}{Axx}

Auch Farben werden mit dem richtigen Abstand zwischen den einzelnen
Karten gesetzt: Wie spielen Sie von \suit{AD10xx} aus?

\hanghand{AKD}{AKD}{AKD10}{AKD} An diesem Beispiel kann man gut
den Unterschied zwischen den Wissenschaftlern und den Zockern
im Bridge erkl\"aren: Ein echter Zocker w\"urde sicher
mit 7\,SA er\"offnen; der Wissenschaftler wendet lieber die ber\"uchtigte 
Hofstein-Konvention an: Die Er\"offnung 5\,SA fragt unmittelbar nach
Buben, wobei der Partner die Farbe seines niedrigsten Buben bietet
(ohne Buben bietet er Ohne :-). Falls der Partner einen Buben zeigt,
den man nicht gebrauchen kann, wie hier \tre-Buben, dann reizt man
die Farbe, in der man den Buben braucht. Hat der Partner den gesuchten
Buben, bietet er 7\,\tre, ansonsten wieder 6\,SA. Damit ist dieses
h\"aufig autretende Bietproblem ein f\"ur alle Mal gel\"ost.

\comment{Problem 1}
\setlength{\handwidth}{27mm}
\northhand{-}{B1098765}{DB10}{AKD}
\southhand{A765432}{AKD}{AK}{2}
\hangNSgame In einem Paarturnier haben Sie und Ihr Partner schon
fr\"uh erkannt, da\ss{} diese beiden H\"ande Schlemm produzieren werden.
Nach etwa 22 Bietrunden konnten Sie mit Ihrem genialen System
herausfinden, da\ss{} im Prinzip 13~Stiche von oben da sind. Gierig
wie Sie nun einmal sind, haben Sie dann 7\,SA geboten, anstelle sich mit
7\,\coe\ zu begn\"ugen. Jetzt brauchen Sie den Kontrakt blo\ss{}
noch zu erf\"ullen. Wie spielen Sie diese Hand in 7\,SA nach 
dem Ausspiel von \tre\,B? 

\newpage
\comment{}

\dots\ und hier ein Auszug aus dem letzten \"Uberstich (hoffentlich mit
freundlicher Genehmigung des Herausgebers :-):

\westhand{2}  {AB42}{KB10953}{B8}
\easthand{B75}{D10} {A42}    {AK752}

\bcproblem{Hand 1}{N}{NS}
{Nord er\"offnet 2\Di\ Flannery (5\pl er \He, 4er \Sp, 11-15).
S\"ud bietet 2\Sp\ falls m\"oglich.}
{5\Di=20, 4\Di=13, 3\Cl=8, \\ 2\Sp(S\"ud)=4}

Es ist ziemlich schwierig, in dieser Hand \"uberhaupt in die Reizung zu kommen.
Michaels Vater bezweifelt, da\ss{} 5\Di\ der Topkontrakt sei. Zugegebenerweise
ist dieser in erster Linie auf den Karo--Schnitt (vorausgesetzt Nord h\"alt
nicht gleich ein 4er Karo) angewiesen. Doch die Tatsache, da\ss{} Nord er\"offnet hat plus
die Zusatzchance, da\ss{} Coeur--Impa\ss{} und die Tr\"umpfe 2--2 stehen, macht f\"ur
mich 5\Di\ zu einer guten Wette.

\bcbidding{Fu\ss{}}{Gro}
{
           & 2\Di & p          & 2\Sp \\
p          & p    & X\al{a}    & p    \\
3\Di\al{b} & p    & p          & p    
}
{
\Meaning{a}{Take--out}
\Meaning{b}{Zusatzst\"arke! 2SA w\"are Lebensohl}
}

\bcbidding{Frans}{Alain}
{
           & 2\Di & p          & 2\Sp \\
3\Di       & p    & 3\Sp\al{a} & p    \\
4\Di       & p    & 5\Cl\al{b} & p    \\
5\Di       & p    & p          & p    
}
{
\Meaning{a}{Frage nach \Sp--Stopper}
\Meaning{b}{Cue--bid}
}

\vspace{0.5cm}

\underline{\bf Running Score:} 

\vspace{0.5cm}

Gro -- Fu\ss{} 13, Alain -- Frans 20

\newpage

\westhand{1063}{AD4} {A84}  {A742}
\easthand{B}   {K103}{KD752}{KB53}

\bcproblem{Hand 2}{W}{NS}{}
{5\Cl=20, 5\Di=14, 6\Cl=8, \\ 6\Di=6, 3SA=4}

In dieser Hand gilt es, die Pik--Schw\"ache zu lokalisieren, um 3SA vermeiden
zu k\"onnen. Bei der Wahl der Unterfarbe ist der stabilere 4--4--Fit in Treff
dem 5--3--Fit in Karo vorzuziehen. Hat man dies alles herausgefunden, stellt
sich noch die Frage, ob man sich mit Partie begn\"ugen oder den Kleinschlemm
ansteuern soll. Insgesamt eine nicht ganz leichte Aufgabe$\dots$

\bcbiddingpair{Fu\ss{}}{Gro}
{
1SA        & 2\Sp\al{a} \\
3\Cl\al{b} & 3\He\al{c} \\
4\Cl       & 5\Cl       \\
6\Cl\al{d} & p 
}
{
\Meaning{a}{Frage nach Min/Max}
\Meaning{b}{Maximum}
\Meaning{c}{1--3--(45) Verteilung, \pf}
\Meaning{d}{keine Werte in \Sp\ und die \"ubrigen Asse. Etwas optimistisch.}
}

\bcbiddingpair{Frans}{Alain}
{
1\Cl      & 1\Di\al{a} \\
1SA\al{b} & 3\Cl       \\
3SA       & p 
}
{
\Meaning{a}{Walsh = keine 4er OF \\ oder \pf}
\Meaning{b}{12-14 P., 4er OF(n) \\ m\"oglich}
}

\vspace{0.5cm}

Frans und Alain scheitern leider schon an der ersten H\"urde und landen im
falschen Vollspiel. Gro und Fu\ss{} vermeiden glorreich 3SA, landen dann aber im
Kleinschlemm, der etwas gezogen ist.

\vspace{0.25cm}

\underline{\bf Running Score:}

\vspace{0.5cm}

Gro -- Fu\ss{} 21, Alain -- Frans 24

\newpage

\onecolumn{
\Huge
\setgamesize{\Huge}
\setlength{\handwidth}{55mm}	% The width of a hand in a game display
\setlength{\bidwidth}{100mm}	% The width of the table with the bidding sequence
\setlength{\cardskip}{2pt}	% The distance between subsequent playing cards

%\centerline{\showgame}

\playproblemNS
{
2\,SA & --- & 3\,SA & --- \\
---   & --- & --- 
}
}
\end{document}

%%%%%%%%%%%%%%%%%%%%%%%%%%%%%%%%%%
%that's all, hope you enjoy
%%%%%%%%%%%%%%%%%%%%%%%%%%%%%%%%%%

