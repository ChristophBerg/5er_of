\documentclass[10pt,german,twocolumn,DIV15]{scrartcl}
\usepackage{kibitzer}
\usepackage{babel}
\usepackage{courier}

\pagestyle{headings}

% \setlength{\topmargin}{-16mm}
% \setlength{\oddsidemargin}{-20mm}
% \setlength{\evensidemargin}{-20mm}
% \setlength{\textwidth}{200mm}   % -> Rand links 5mm breiter als rechts 
% \setlength{\textheight}{256mm}
\setlength{\columnsep}{8mm}
% \setlength{\columnseprule}{0.4pt}

\handwidth22mm
\parindent0mm
\parskip2ex plus1ex minus1ex

\def\pik{\,\Sp}
\def\coe{\,\He}
\def\kar{\,\Di}
\def\tre{\,\Cl}

\def\pi{\Sp}
\def\co{\He}
\def\ka{\Di}
\def\tr{\Cl}

\def\mi{\,$\clubsuit /\diamondsuit$} 
\def\ma{\,$\heartsuit /\spadesuit$}
\def\good{$^+$}
\def\bad{$^-$}
\def\ra{$\rightarrow$}
\def\pl{$\uparrow$}
\def\uf{\textsf{\,UF}}
\def\of{\textsf{\,OF}}
\def\ufa{\textsf{UF}}
\def\ofa{\textsf{OF}}
\def\aof{\textsf{\,aOF}}
\def\SA{\textsf{\,SA}}
\def\kontra{\textsf{X}}
\def\sep{\,--\,}
\newcommand{\conv}[1]{\emph{#1}}
\def\bal{\textsc{Ausg}}
\def\unbal{\textsc{Unausg}}
\def\nat{\textsc{Nat}}
\def\pf{\textsc{PF}}
\def\maxi{\textsc{Max}}
\def\mini{\textsc{Min}}
\def\inv{\textsc{Einl}}
\def\nf{\textsc{NF}}
\def\rel{\textsc{Rel}}
\def\stp{\textsc{Stop}}
\def\hstp{\textsc{Hstop}}
\def\cstop{\coe-\stp}
\def\pstop{\pik-\stp}
\def\tstop{\tre-\stp}
\def\kstop{\kar-\stp}
\def\chstop{\coe-\hstp}
\def\phstop{\pik-\hstp}
\def\thstop{\tre-\hstp}
\def\khstop{\kar-\hstp}
\def\aw{\textsc{Aw}}
\def\eo{\textsc{E\"o}}
\def\bdsc{\begin{description}}
\def\edsc{\end{description}}

\setlength{\labelsep}{2ex}

\newcommand\bidins[1]%
{%
\begin{tabular}[t]{llp{32ex}}%
#1
\end{tabular}%
}

\newcommand\bidseq[1]%
{%
\begin{flushleft}
\begin{tabular}[t]{llp{32ex}}%
#1
\end{tabular}%
\end{flushleft}
}

\newcommand\notebox[1]%
{%
\setlength{\fboxsep}{2ex}
\fbox{\parbox{0.43\textwidth}{#1}}
}
  

\begin{document}

\setlength{\itemsep}{0ex plus0.2ex}

\onecolumn
{\centering\Huge\ 5er Oberfarben (R. Bartels und Th. Schmitt)}
\tableofcontents
\twocolumn

\section{"Ubersicht der Er"offnungen}

\subsection*{1er-Stufe}
\bidins{%
1\mi & 12\pl	& 3\pl \mi\\[1ex]
1\ma & 12\pl	& 5\pl \ma\\[1ex]
1\SA & 15-17	& \bal, 5er\of\ m"oglich\\
}

\subsection*{2er-Stufe}
\bidins{%
2\tre	& 6-10	& Weak Two in \kar, oder\\
	& 16\pl	& 6er-Farbe, \textbf{8} Spielstiche, oder\\
	& 22-23	& \bal, oder\\
	& 26-27	& \bal\\[1ex]
2\kar	& 6-10	& Weak Two in \coe, oder\\
	& 18\pl	& 6er-Farbe, \textbf{9} Spielstiche, oder\\
	& 24-25	& \bal, oder\\
	& 28\pl	& \bal\\[1ex]
2\coe	& 6-10	& Zweif"arber mit \coe\\[1ex]
2\pik	& 6-10	& Weak Two in \pik\\[1ex]
2\SA	& 20-21	& \bal\\
}

\subsection*{3er-Stufe}
\bidins{%
3\uf	& 5-10	& 7\pl{}\uf, in 1. Hand nur in Nichtgefahr, in 3. Hand 6er-Farbe m"oglich\\[1ex]
3\of	& 5-10	& 7\pl{}\of\\[1ex]
3\SA	& 	& Gambling in 1./2. Hand, in 3./4. Hand zum spielen\\
}

\subsection*{4er-Stufe}
\bidins{%
4\mi	& 	& stehende 7\ma\ mit einem Nebenwert\\[1ex]
4\of	&	& 7\good\pl\of, zum spielen\\[1ex]
4\SA	&	& 65\pl\uf\\
}

\section{Die 1\mi-Er"offnung}

Ohne 5\pl\of\ wird grunds"atzlich die l"angere \uf\ er"offnet.  Bei
gleicher L"ange in den \uf\ wird 1\kar\ er"offnet, mit 33 in den \uf\
wird 1\tre\ er"offnet.

Siehe auch \textit{Er"offnungsregel f"ur Zweif"arber mit 6/5 \uf/\of} auf
Seite \pageref{zfregel}.

\subsection{Antworten des Partners (nach 1\tre-Er"offnung)}
\bidins{%
1\kar	& 6-7	& \bal\ ohne 4er\of, 3-3-3-4 Verteilung mit 3er\kar\ m"oglich, oder\\
	& 6\pl	& \kar-Einf"arber, oder\\
	& 12\pl	& 5er\kar\ und 4er\of\ \conv{(Walsh-\ka)}\\[1ex]
1\of	& 6\pl	& 4\pl\of\\[1ex]
1\SA	& 8-10	& \bal\ ohne 4er\of\\[1ex]
2\tre	& 10\good\pl & 5\pl\tre\ \conv{(Inverted)}\\[1ex]
2\kar	& 2-5	& 55\of\\[1ex]
2\of	& 5-8	& gute 6\pl\of\ \conv{(Weak Jump)}\\[1ex]
2\SA    & 2-6   & 5\pl\tre\ \conv{(Inverted Spezial)}\\[1ex]
3\tre	& 7-9	& 5\pl\tre\ \conv{(Inverted)}\\[1ex]
3\kar	& 12-15 & 5/5 in \tre\,+\kar\ \conv{(Fit Jump)}\\[1ex]
3\coe	& 12-15 & 5/5 in \tre\,+\coe\ \conv{(Fit Jump)}\\[1ex]
3\pik	& 12-15	& 5/5 in \tre\,+\pik\ \conv{(Fit Jump)}\\[1ex]
3\SA	& 13-15	& zum spielen\\[1ex]
4\tre	&	& \conv{KCB}\\[1ex]
4\of	&	& zum spielen\\
}

\notebox{%
Ein unn"otiger Sprung in einer neuen Farbe ist
  \textsc{Splinter}.\\
Ein unn"otiger Doppelsprung ist
  \textsc{Exclusion \mbox{RKCB}}.
}

\subsection{Bietsequenzen nach 1\mi-Er"offnung}

\subsubsection{Bietsequenzen nach [1\tre-1\kar]}

Wegen \conv{Walsh-\ka} zeigt \aw\ nach 1\tre-Er"offnung seine \of\
sofort, au"ser mit einer starken Hand und noch l"angeren
Karos. \mbox{[1\tre-1\kar]} lehnt eine 4er\of\ bei \aw\ also
meistens ab.

\eo\ reizt daher \SA\ weiter, wenn er \bal\ ist, eine etwaige 4er\of\
verschweigt er.  Ein \of-R"uckgebot von \eo\ zeigt \unbal\ mit langen
Treffs.

\bdsc
\item[1\tre\ -- 1\kar;] ?
  \bdsc
  \item[1\coe] 5\pl\tre\ und 4er\coe\ oder 4-4-1-4
  \item[1\pik] 5\pl\tre\ und 4er\pik
  \item[1\SA] 12-14, eine oder beide \of\ m"oglich
    \bdsc
      \item[2\tre] Schlemm-Interesse in \tre\ oder \kar;
            \textbf{Frage nach Verteilung} (s.~u.)
      \item[2\kar] schwach; zum spielen
      \item[2\ma] \pf\ mit 5/4-Verteilung
      \item[2\SA] \nat
      \item[3\tre/\coe/\pik] \conv{Splinter} mit 6er\kar
      \item[3\kar] \inv\ mit 6er\kar
    \edsc
  \item[2\tre] 12-14, 6er\tre
    \bdsc
      \item[2\coe] \cstop, kein \pstop.
        \ra~2\pik\ Frage nach \hstp\ \conv{(VFF)}
      \item[2\pik] zeigt \pstop, \ra~3\tre\ verneint \cstop, \ra\
        3\coe\ Frage nach \hstp\ \conv{(VFF)}
      \item[3\ma] \textsc{Splinter} mit \tre-Anschluss
      \item[4\ma] \textsc{Exclusion RCKB} auf \tre-Basis
    \edsc
  \item[2\SA] 18-19, eine oder beide \of\ m"oglich
    \bdsc
    \item[3\tre] \conv{Wolff Sign Off}, verlangt \rel\ 3\kar, worauf
      man passen kann oder Sign Off gibt
    \item[3\kar] Forcing
    \item[3\of] 12\pl\ FP, 5/4-Verteilung, Schlemm m"oglich (s.~u.)
      \bdsc
      \item[3\SA] \nat, kein Fit
      \item[4\of] Fit in der Oberfarbe
      \item[Rest] Steps \conv{RKCB} auf \kar-Basis
      \edsc
    \edsc
  \edsc
\edsc

\bdsc
\item[1\tre\ -- 1\kar; 1\SA\ -- 2\tre;] ?

  Hat \aw\ Schlemm-Interesse in \uf, so kann er die genaue
  Verteilung der \bal\ Hand von \eo\ erfragen.
  \bdsc
  \item[2\kar] 3er\kar, \textbf{nicht 4-3-3-3}
    \bdsc
      \item[2\coe] \rel, Frage nach Verteilung\\
        \ra~2\pik\ = 5er\tre\\
        \ra~2\SA\ = 4er\tre
      \item[3\mi] \conv{RKCB}
    \edsc
  \item[2\coe] 3-4-2-4
  \item[2\pik] 4-3-2-4
  \item[2\SA]  beliebige 4333
    \bdsc
      \item[3\tre] \rel, Frage der 4er-Farbe\\
        \ra~3\kar\ = \textbf{4er\tre}\\
        \ra~3\coe\ = 4er\coe\\
        \ra~3\pik\ = 4er\pik
    \edsc
  \item[3\tre] 3-3-2-5 (\ra~3\kar\ = \conv{RKCB})
  \item[3\kar] 4-4-2-3
  \edsc
\edsc

\bdsc
  \item[1\tre\ -- 1\kar; 2\SA\ --] ?

    \aw\ kann nun 3\of\ reizen und damit Schlemm-Interesse in der \of\
    zeigen.
    \bdsc
      \item[3\coe] 12\pl\ FP, 5/4-Verteilung, Schlemm m"oglich
        \bdsc
          \item[3\pik] 1 oder 4 Key Cards auf \kar-Basis
          \item[3\SA] \nat, kein Fit
          \item[4\tre] 0 oder 3 Key Cards auf \kar-Basis
          \item[4\kar] 2 Key Cards auf \kar-Basis
          \item[4\coe] Fit in der \of
          \item[4\pik] 2 Key Cards mit Trumpf-Dame (\kar-Dame)
        \edsc
      \item[3\pik] 12\pl\ FP, 5/4-Verteilung, Schlemm m"oglich
        \bdsc
          \item[3\SA] \nat, kein Fit
          \item[4\tre] 1 oder 4 Key Cards auf \kar-Basis
          \item[4\kar] 0 oder 3 Key Cards auf \kar-Basis
          \item[4\coe] 2 Key Cards
          \item[4\pik] Fit in der \of
        \edsc
    \edsc
\edsc

\subsubsection{Bietsequenzen nach [1\tre-1\of]}

\footnotetext[1]{Einladende H"ande mit 6er-Farbe "uber
  \conv{Checkback} reizen.}

\bdsc
\item[1\tre\ -- 1\coe;] ?
  \bdsc
  \item[1\pik] 12\pl\ FP, \nat

    Wiederholt der Antwortende seine Oberfarbe auf niedrigster Stufe
    nachdem der Er"offner zwei Farben gereizt hat, so zeigt dies eine
    einladende Hand mit 9-11 FP und 6er-Farbe.

    Wiederholt er seine Oberfarbe im Sprung, so zeigt dies eine
    partieforcierende Hand mit 6er-Farbe:
    \bdsc
    \item[2\coe] 9-11 FP, 6er\coe, \inv
    \item[3\coe] 6er\coe, \pf
    \edsc
  \item[1\SA] 12-14 FP \bal, kein 4er\pik
    \bdsc
    \item[2\tre] \conv{Checkback Stayman}
    \item[2\kar] 4er\coe, 5\pl\kar, zum spielen
    \item[2\coe] 5er\coe, schwach, zum spielen
    \item[2\pik] 5/4 \coe\,+\pik, \pf
    \item[2\SA] \nat
    \item[3\mi] \nat, \pf
    \item[3\coe] 6er\coe, Schlemm-Interesse\footnotemark[1]
    \edsc
  \edsc

\item[1\tre\ -- 1\pik;] ?
  \bdsc
  \item[1\SA] 12-14 FP \bal\ (s.~u.)
  \item[2\tre] 12-16\bad, 5\pl{}er-Farbe (s.~u.)
  \item[2\SA] 18-19 FP, \bal; siehe \conv{Wolff} etc.
  \item[3\kar/\co] 4er\pik, Single \ka/\co, \conv{Splinter}
  \item[3\SA] 18-19 FP, stehendes 6er\tre, K"urze in \pik, Deckung in den
    ungereizten Farben, 8-8$\frac{1}{2}$ Stiche
  \item[4\tre] 18\good\pl, 4er\pik, 5\good\pl\tre
  \item[4\kar/\co] 18\good\pl, 4er\pik, \conv{Exclusion RCKB}
  \edsc
\item[1\tre\ -- 1\pik; 1\SA\ --] ?

  12-14 FP \bal
  \bdsc
  \item[2\tre] \conv{Checkback Stayman}
  \item[2\kar] 4er\pik, 5\pl{}\kar, zum spielen
  \item[2\coe] 5/4 \pik\,+\coe, zum spielen oder ausbessern
  \item[2\pik] 5er\pik, schwach
  \item[2\SA] \nat
  \item[3\mi] \nat, \pf
  \item[3\coe] 5/4 \pik\,+\coe, \pf
  \item[3\pik] 6er\pik, Schlemm-Interesse\footnotemark[1]
  \edsc
\item[1\tre\ -- 1\pik; 2\tre\ --] ?

  12-16\bad, 5\pl{}er-Farbe
  \bdsc
  \item[2\kar] \conv{Dritte Farbe Forcing}
    \bdsc
    \item[2\coe] Frage nach \hstp\ (\conv{VFF}), kein 3er\pik
    \item[2\pik] 3er\pik, \mini\ f"ur 2\tre-R"uckgebot\\
      \ra~3\tre\ = Forcing mit \tre
    \item[2\SA] \cstop, kein 3er\pik, \mini
    \item[3\pik] 3er\pik, \maxi\ f"ur 2\tre-R"uckgebot
    \edsc
  \item[3\tre] \inv
  \edsc
\edsc

\footnotetext[1]{Einladende H"ande mit 6er-Farbe "uber
  \conv{Checkback} reizen.}

\subsubsection{Bietsequenzen nach \conv{Reverse}}

\bdsc
  \item[1\tre\ -- 1\pik; 2\kar\ --] ?

    \conv{Reverse} zeigt 16\good\pl\ und ist selbstforcierend f"ur
    \eo.
    \bdsc
    \item[2\coe] \conv{VFF}, sagt hier aber nichts "uber die St"arke
      aus, da 2\kar\ selbsforcierend war.  Verneint in diesem Fall
      \textbf{5er}\pik, da man \pik\ h"atte wiederholen k"onnen ohne
      dass \eo\ passen darf.
      \bdsc
        \item[2\SA] \mini,\cstop, \nf
        \item[3\tre] \mini, \nf
        \item[3\kar] \maxi, \pf
        \item[3\SA] \maxi, \cstop
      \edsc
    \item[2\pik] 5er-Farbe, schwach oder stark
      \bdsc
        \item[2\SA] \mini, \cstop, kein 3er\pik
        \item[3\tre] \mini, kein 3er\pik
        \item[3\kar] \maxi, kein 3er\pik
        \item[3\coe] \conv{VFF}, Frage nach \cstop
        \item[3\pik] \mini\ mit 3er\pik
        \item[3\SA] \maxi, \cstop, kein 3er\pik
      \edsc
    \item[2\SA] \conv{Ingberman (Moderateur)}, verlangt \ra~3\tre\ \rel, danach:
      \bdsc
      \item[pass/3\kar] zum Spielen
      \item[3\SA] 8-9 FP \nat
      \item[4\SA] 12\pl\ FP, quantitative Schlemmeinladung
      \edsc
      Mit Zusatzst"arke (18\good\pl) muss \eo\ das 3\tre\ \rel\
      "uberspringen.
    \item[3\uf] \pf
    \item[3\SA] 10-11 FP, \cstop
    \edsc
\edsc

\subsubsection{Bietsequenzen nach \conv{Inverted}}

Nach [1\tre-2\tre] bzw. [1\kar-2\kar] gibt es kein Reverse.  \eo\
zeigt mit 2\SA\ eine ausgeglichene, passbare Hand.  3\tre/\kar\ ist
ebenfalls passbar.  Alle anderen Gebote des Er"offners zeigen Werte
(Stopper) und sind \pf.

\bdsc
  \item[1\tre\ -- 2\tre;] ?
    \bdsc
      \item[2\kar] 14\good\pl, \kstop
        \bdsc
        \item[2\coe] \cstop\ (h"ochstens \phstop), \ra~2\pik\ fragt
          nach \phstop\ (siehe \conv{VFF})
        \item[2\pik] \pstop\ (h"ochstens \chstop)
        \item[2\SA] \stp\ in \co\ und \pi.
        \edsc
      \item[2\coe-2\pik] 14\good\pl, \stp\ in der Farbe
    \edsc
\edsc

\subsubsection*{Sonstiges}

Nach [1\tre-2\of] \conv{(Weak Jump)} kann \eo\ mit 2\SA\
\conv{(Ogust)} nach St"arke und Farbqualit"at fragen (wie nach
\conv{Weak Two} Er"offnung, siehe \ref{ogust}, S.~\pageref{ogust}).

Nach 1\kar{}-Er"offnung sind die Antworten entsprechend.

Ausnahme:
\bdsc
  \item[1\kar\ -- 2\tre;] ?
    \bdsc
    \item[2\kar] 12-13 FP \bal\ oder \nat
    \item[2\SA] 14 FP \bal
    \edsc
\edsc
        

\subsubsection*{\label{zfregel}Er"offnungsregel f"ur Zweif"arber mit 6/5 \uf/\of}

\bdsc
\setlength{\labelsep}{1ex}
\item[4\pl{} Verlierer:] \of\ er"offnen und \uf\ billig nachreizen
\item[3-4 Verlierer:] \uf\ er"offnen und anschliessend \conv{Reverse}
  reizen
\item[0-2 Verlierer:] \pf\ er"offnen
\edsc

\newpage
\section{Die 1\of-Er"offnungen}

Die Fortsetzung nach 1\of-Er"offnungen folgt folgenden Prinzipien:
\begin{itemize}
\setlength{\itemsep}{0.5ex}
\item Schwache bis einladende H"ande mit 4\pl{}er-Anschlu"s werden durch
  \conv{Bergen Raises} gezeigt.
\item Mit \pf\ und gutem Trumpfanschluss reizen wir
  \conv{Splinter} oder \conv{Stenberg 2\SA}.
\item Die restlichen starken Varianten werden durch verz"ogertes
  Reizen der Trumpfunterst"utzung gezeigt (Farbwechsel).
\end{itemize}

\subsection{Die 1\coe-Er"offnung}

\bdsc
  \item[1\pik] 6\pl, 4\pl\pik
  \item[1\SA] 6-10, kein 3er\coe, kein 4er\pik
  \item[2\tre] 10\pl, \textbf{k"unstlich}, selbstforcierend
  \item[2\kar] 10\pl, 5\pl\kar
  \item[2\coe] 6-10\bad, 3er\coe
  \item[2\pik] 5-8, 6\pl\pik\ \conv{(Weak Jump)}
  \item[2\SA] 12\good\pl, 4er\coe\ \conv{(Stenberg)}, \pf\ (siehe
    \ref{stenberg}, S.~\pageref{stenberg})\\
    (nach vormaligem Pass: 11-12, Double-\coe\ und 3er\pik)
  \item[3\tre] 9\good-11, 4\pl\coe\ \conv{(Bergen Raise)}
  \item[3\kar] 7-9\bad, 4\pl\coe\ \conv{(Bergen Raise)}
  \item[3\coe] 0-6, 4\pl\coe\ \conv{(Bergen Raise)}
  \item[3\pik] 11-14, 4\pl\coe\ (oder 3\good\coe), beliebiges Chicane\\
    \ra~3\SA\ fragt nach dem Chicane (siehe S.~\pageref{chicaneask})
  \item[3\SA] 11-14, 4\pl\coe\ (oder 3\good\coe), \pi-Single \conv{(Splinter)}
  \item[4\tre/\ka] 11-14, 4\pl\coe\ (oder 3\good\coe), \tr/\ka-Single
    \conv{(Splinter)}
  \item[4\coe] zum spielen
  \item[4\pik] \conv{Exclusion RKCB} auf \coe-Basis
\edsc

\subsection{Die 1\pik-Er"offnung}

\bdsc
  \item[1\SA] 6-10, kein 3er\pik
  \item[2\tre] 10\good\pl, \textbf{k"unstlich}, selbstforcierend
  \item[2\kar] 10\good\pl, 5\pl\kar
  \item[2\coe] 10\good\pl, 5\pl\coe
  \item[2\pik] 6-10\bad, 3er\pik
  \item[2\SA] 12\good\pl, 4er\pik\ \conv{(Stenberg)}, \pf\ (siehe
    \ref{stenberg}, S.~\pageref{stenberg})\\
    (nach vormaligem Pass: 11-12, Double-\pik\ und 3er\coe)
  \item[3\tre] 9\good-11, 4\pl\pik\ \conv{(Bergen Raise)}
  \item[3\kar] 7-9\bad, 4\pl\pik\ \conv{(Bergen Raise)}
  \item[3\coe] 11-14, 4\pl\pik\ (oder 3\good\pik), beliebiges Chicane\\
    \ra~3\pik\ fragt nach dem Chicane (siehe S.~\pageref{chicaneask})
  \item[3\pik] 0-6, 4\pl\pik\ \conv{(Bergen Raise)}
  \item[3\SA] 11-14, 4\pl\pik\ (oder 3\good\pik), \co-Single \conv{(Splinter)}
  \item[4\tre/\ka] 11-14, 4\pl\pik\ (oder 3\good\pik), \tr/\ka-Single
    \conv{(Splinter)}
  \item[4\coe] zum spielen
  \item[4\pik] \conv{Exclusion RKCB} auf \coe-Basis
\edsc

\subsection{Bietsequenzen nach 1\of-Er"offnung}

\bdsc
\item[1\coe\ -- 1\pik;] ?
  \bdsc
  \item[1\SA] 12-14 \bal
    \bdsc
    \item[2\tre] \conv{Checkback Stayman}
    \item[2\kar] 4er\pik, 5\pl\kar, zum spielen
    \item[3\tre] \nat, forcing
      \bdsc
      \item[3\kar] Frage nach \hstp, zeigt \khstop, verneint 3er\pik\
        (siehe \conv{VFF})
      \item[3\coe] 2-5-3-3, verneint \khstop
      \item[3\pik] 3er\pik, \maxi
      \item[3\SA] \kstop, verneint 3er\pik
      \item[4\pik] 3er\pik, \mini
      \edsc
    \edsc
  \item[2\tre/\ka] 12-18, 54\pl-Verteilung
    \bdsc
    \item[2\coe] Ausbessern, \coe-Double
    \item[3\tre/\ka] \inv
    \edsc
    \textbf{Nach \conv{1~"uber~1} ist die Hebung von \eo's \emph{zweiter} Farbe
    auf die 3er-Stufe lediglich einladend.}
  \item[2\SA] 18-19 FP \bal
  \item[3\tre/\ka] 18\good\pl FP, 5\pl\coe\ und 4\pl\uf\\
    \ra~3\coe\ = st"arker als 4\coe\ (\emph{Principle of Fast Arrival});
    \ra~3\pik\ ist dann \conv{Cue Bid}, aber keine K"urze
    
    \textbf{Ein \conv{Cue Bid} in Partners erster Farbe zeigt \emph{nie} eine
    K"urze.}
  \edsc
\pagebreak
\item[1\pik\ -- 2\tre;] ?
\bdsc
\item[2\kar] 12-18 FP, 54\pl
  \bdsc
  \item[2\pik] \inv\ mit 3er\pik
  \item[2\SA] \nat\ \nf
  \item[3\tre] \nat\ \nf
  \item[3\kar] \pf
    \bdsc
    \item[3\coe] Frage nach \hstp\ \conv{(VFF)}
    \item[3\pik] kein \chstop, verspricht kein 6er\pik
    \item[3\SA] \cstop
    \edsc
  \item[3\pik] Schlemm-Interesse mit 3er\pik
  \edsc
  \textbf{Nach \conv{2~"uber~1} ist die Hebung von \eo's
    \emph{zweiter} Farbe auf die 3er-Stufe \pf.}
\item[2\coe]~
  \bdsc
  \item[2\SA] \nat
    \bdsc
    \item[3\kar] 5/5 in den \of, \pf
    \item[3\coe] 5/5 in den \of, \nf
    \edsc
  \item[3\coe] \pf
  \edsc
\item[2\pik] 12-14 FP, kann 5er\pik\ sein!
  \bdsc
  \item[3\tre] 6er\tre, \nf
  \item[3\kar/\co] Werte (siehe \conv{DFF}), \pf
  \item[3\pik] \inv, 3er\pik
  \edsc
\item[2\SA] 18-19 FP \bal
\item[3\tre] 16\pl\ FP, 54\pl\ \pik\,+\tre
  \bdsc
  \item[3\kar] \kar-Werte, \ra~3\coe\ = Frage nach \hstp
  \item[3\coe] \coe-Werte
  \edsc
\edsc
\item[1\pik\ -- 2\coe;] ?
  \bdsc
  \item[3\coe] 16\pl\ FP, 53\pl\ \pi\,+\coe
  \item[4\coe] 12-14 FP, 4er-Anschluss (3er-Anschluss und \mini\
    "uber 2\pik)
  \edsc
\edsc

\bcbiddingpair{West}{Ost}
{
  1\pik & 2\kar\\
  3\tre\al{a} & 3\kar\al{b}\\
  3\SA\al{c}  & 4\SA\al{d}
}
{
  \Meaning{a}{5/4 in \pi\,+\tre}
  \Meaning{b}{forcing}
  \Meaning{c}{\cstop}
  \Meaning{d}{quantitativ}
}

\subsubsection{Spezielle \conv{Splinter}-Gebote nach 1\of-Er"offnung}
[1\of-3\SA] zeigt ein Single in der anderen \of.

Chicane-\conv{Splinter} werden mit einem einfachen Sprung in die
andere \of\ gezeigt:
\bdsc
\item[{[}1\coe\ -- 3\pik{]}] beliebiges Chicane, \ra\ 3\SA\ \rel, Frage nach
  dem Chicane
  \bdsc
    \item[4\tre] \tr-Chicane
    \item[4\kar] \ka-Chicane
    \item[4\coe] \pi-Chicane
  \edsc
\item[{[}1\pik\ -- 3\coe{]}] beliebiges Chicane, \ra\ 3\pik\ \rel, Frage nach
  dem Chicane
  \bdsc
    \item[3\SA] \co-Chicane
    \item[4\tre] \tr-Chicane
    \item[4\kar] \ka-Chicane
  \edsc
\edsc

\subsection{Weiterreizung nach [1\of\ -- 2\SA] \conv{(Stenberg)}}

2\SA\ auf eine 1\of-Er"offnung verspricht 4er-Anschluss und
Vollspielwerte, oder besser.

Der Er"offner reizt daraufhin eine K"urze, falls vorhanden. Hat er
keine K"urze, so reizt er mit 11-13 4\of, mit 14-15 3\SA\ und mit
16\pl\ 3\of\ (\conv{principle of fast arrival}, je h"oher wir reizen,
desto \emph{schw"acher} sind wir).

Hat \eo\ eine K"urze gezeigt, so fragt die n"achste Stufe (\rel) nach
Art der K"urze und nach Keycards.  Die erste Antwortstufe zeigt ein
Chicane, weiteres \rel\ ist \conv{RKCB}.  Alle anderen Antwortstufen
zeigen ein Single und beantworten gleichzeitig \conv{RKCB}.

Hat \eo\ keine K"urze gezeigt, so ist die n"achste Stufe \conv{RKCB}.

\bdsc
\item[1\of\ -- 2\SA;] ?
  \bdsc
  \item[3\uf/\aof] K"urze in der gereizten Farbe, \ra~\rel\ = Frage
    nach Art der K"urze sowie \conv{RKCB},~\ra
    \bdsc
    \item[n"achste Stufe:] K"urze ist ein Chicane, \ra~\rel\ = \conv{RKCB}
    \item[Rest:] direkte 1430-Antworten auf
      \conv{RKCB}, K"urze ist ein Single
    \edsc
  \item[3\of] 16\pl, keine K"urze, \ra~\rel\ = \conv{RKCB}
  \item[3\SA] 14-15, keine K"urze, \ra~4\tre\ = \conv{RKCB}
  \item[4\of] 11-13, keine K"urze, \ra~\rel\ = \conv{RKCB}
  \edsc
\item[1\coe\ -- 2\SA; 3\tre\ -- 3\kar;] ?
  
  (\eo\ hat eine K"urze in \tr, \aw\ fragt nach Keycards.)~\ra
  \bdsc
  \item[3\coe] Chicane in \tre\\
    \ra~3\pik\ = \conv{RKCB}
  \item[3\pik] \tre-Single, eine oder vier Keycards
  \item[3\SA] \tre-Single, keine oder drei Keycards
  \item[4\tre] \tre-Single, zwei Keycards ohne \co-Dame
  \item[4\kar] \tre-Single, zwei Keycards mit \co-Dame
  \edsc
\edsc

\subsubsection{Beispielreizungen zu \conv{Stenberg}}

\westhand{432}{AB10852}{A3}{K2}
\easthand{KB6}{K976}{K94}{A87}
\centerline{\showEWgame}
\bcbiddingpair{West}{Ost}
{
  1\coe & 2\SA\\
  4\coe\al{a} & pass
}
{
  \Meaning{a}{Minimum (11-13)}
}

\vfill
\westhand{4}{AB10852}{A1042}{K2}
\easthand{876}{KD76}{KD3}{A87}
\centerline{\showEWgame}
\bcbiddingpair{West}{Ost}
{
  1\coe & 2\SA\\
  3\pik\al{a} & 3\SA\al{b}\\
  4\pik\al{c} & 6\coe
}
{
  \Meaning{a}{\pi-K"urze (Single oder Chicane)}
  \Meaning{b}{Frage}
  \Meaning{c}{\pi-Single, zwei Keycards ohne Trumpf-Dame}
}

\vfill
\westhand{KB1042}{DB753}{2}{K3}
\easthand{AD53}{AK}{653}{A542}
\centerline{\showEWgame}
\bcbiddingpair{West}{Ost}
{
  1\pik & 2\SA\\
  3\kar\al{a} & 3\coe\al{b}\\
  3\SA{c} & 6\pik
}
{
  \Meaning{a}{\ka-K"urze (Single oder Chicane)}
  \Meaning{b}{Frage}
  \Meaning{c}{\ka-Single, eine oder vier Keycards}
}

\vfill
\westhand{K752}{AD542}{-}{K742}
\easthand{A8}{KB76}{D62}{ADB3}
\centerline{\showEWgame}
\bcbiddingpair{West}{Ost}
{
  1\coe & 2\SA\\
  3\kar\al{a} & 3\coe\al{b}\\
  3\pik\al{c} & 3\SA\al{d}\\
  4\tre\al{e} & 4\kar\al{f}\\
  4\pik\al{g} & 4\SA\al{h}\\
  5\tre\al{i} & 7\coe
}
{
  \Meaning{a}{\ka-K"urze (Single oder Chicane)}
  \Meaning{b}{Frage}
  \Meaning{c}{\kar-Chicane}
  \Meaning{d}{\conv{RKCB}}
  \Meaning{e}{eine oder vier Keycards}
  \Meaning{f}{Frage nach \co-Dame}
  \Meaning{g}{\co-Dame und \pi-K"onig vorhanden}
  \Meaning{h}{Frage nach weiteren K"onigen}
  \Meaning{i}{\tr-K"onig vorhanden}
}
\raggedbottom

\section{Verhalten nach Zwischenreizung durch die Gegner}

\subsection{Allgemeines}
\begin{itemize}
\item Das Reizen einer neuen Farbe auf der 1er-Stufe ist weiterhin
forcierend f"ur eine Runde.
%
\item Das Reizen einer neuen Farbe auf der 2er-Stufe ist nicht forcierend
falls die Zwischenreizung gest"ort hat, sonst forcierend:
\begin{description}
\item[1\coe\sep(1\pik)\sep2\kar] ist forcierend da man auch
ohne die Zwischenreizung 2\kar\ gereizt h"atte.
\item[1\tre\sep(1\pik)\sep2\coe] ist nicht forcierend da man ohne die
  Zwischenreizung 1\coe\ gereizt h"atte.
\end{description}
%
\item Eine neue Farbe auf der 3er-Stufe ist immer forcierend
  (z.~B. 1\pik\sep(2\kar)\sep3\tre).
\item Nach \conv{Informationskontra} ist eine neue Farbe auf der 2er-Stufe
  nicht forcierend.
\item Nach \conv{Weak Jumps} ist das Reizen einer neuen Farbe immer
  forcierend.
\item \conv{Kontra} ist negativ bis 3\coe\ oder zeigt eine beliebige
  starke Hand f"ur die es in der augenblicklichen Situation kein Gebot
  gab.
\item Der \conv{"Uberruf}
  \begin{itemize}
    \item nach \ufa-Er"offnung fragt nach Stopper.
    \item nach \ofa-Er"offnung zeigt eine partieforcierende Hand mit
      Anschlu"s in der \ofa\ (siehe unten).
    \end{itemize}
\item 1\SA\ und 3\SA\ sind nat"urlich.
\end{itemize}

\subsection{Nach \of-Er"offnung}
\begin{itemize}
\item Alle \ofa-Hebungen sind schwach.  Falls \conv{Bergen Raises}
  noch m"oglich sind, werden diese angewendet.
\item {[}1\of\sep($x$)\sep2\SA{]} zeigt
  \begin{itemize}
  \item wenn \conv{Bergen Raises} noch m"oglich sind: eine einladende
    Hebung mit genau 3er-Anschluss,
  \item sonst: eine einladende Hebung mit 3er- oder 4er-Anschluss.
  \end{itemize}
\item {[}1\of\sep(\kontra)\sep2\SA{]} zeigt
  \begin{itemize}
    \item eine einladende Hebung mit genau 3er-Anschluss oder
    \item eine partieforcierende Hebung.
  \end{itemize}
Die anderen F"alle k"onnen mittels \conv{Bergen} gezeigt werden.
\item Der \conv{"Uberruf} der gegnerischen Farbe zeigt eine
  partieforcierende Hand mit Anschluss in der \of\
  ([1\of\sep(1\,$x$)\sep2\,$x$] oder [1\of\sep(2\,$x$)\sep3\,$x$]).
\end{itemize}

\subsubsection{Beispiele}
\begin{description}
\item[1\pik\sep(2\kar)\sep2\pik] ist m"oglich ab 0 Punkten mit
  3er-Anschluss.
\item[1\coe\sep(1\pik)\sep3\kar] ist weiterhin \conv{Bergen} und zeigt
  7-9 Punkte mit 4er-Anschluss.
\item[1\coe\sep(2\tre)\sep3\kar] ist nat"urlich und forcierend
  (s.~o.), zeigt keinen Coeur-Anschluss (\conv{Bergen Raises} sind
  nicht mehr m"oglich, da das 3\tre-Gebot nun eine andere Bedeutung
  hat).
\item[1\pik\sep(X)\sep2\SA] zeigt 3er\pik-Anschluss, einladend oder
  eine Hand ab 12 Punkten mit 4er- oder besserem Anschluss (die w"are
  zu stark f"ur \conv{Bergen}).
\item[1\pik\sep(2\kar)\sep2\SA] zeigt 3er- oder 4er\pik-Anschluss und
  ist einladend (3\kar\ w"are partieforcierend; \conv{Bergen} kann
  nicht mehr gereizt werden um die einladende Hand mit 4er\pik{}
  zu zeigen).
\item[1\coe\sep(1\pik)\sep2\pik] zeigt eine partieforcierende Hand mit
  Coeur-Anschluss.
\end{description}

\subsection{Nach \uf-Er"offnung}
\begin{itemize}
\item Es gilt weiterhin \conv{Inverted Minors}.
\item Der \conv{"Uberruf} der gegnerischen Farbe fragt nach
  \SA-Stopper ([1\uf\sep(1\,$x$)\sep2\,$x$] oder
  [1\uf\sep(2\,$x$)\sep3\,$x$]).
\item {[}1\uf\sep(1\,$x$)\sep2\SA] ist \conv{Inverted} (2-5~FP mit
  5\pl{}er Anschluss in der er"offneten Unterfarbe; wie in einer
  ungest"orten Reizung).
\end{itemize}

\subsubsection{Beispiele}
\begin{description}
\item[1\kar\sep(\kontra)\sep2\kar] zeigt 10\pl\ Punkte und 5\pl{}er
  Karo-Anschluss.
\item[1\tre\sep(1\pik)\sep2\SA] zeigt 2-5 Punkte und 5\pl{}er
  Treff-Anschluss.
\item[1\kar\sep(2\tre)\sep3\tre] ist partieforcierend und fragt nach
  Treff-Stopper.
\end{description}

\subsection{Weitere Bietsequenzen}

\bdsc
\item[1\tre\sep(1\kar)\sep?]~
\bdsc
\item[\kontra] beide \of
\item[1\coe/\pi] mindestens 4er-L"ange
\item[2\kar] \pf\ und Frage nach \ka-Stopper
\item[3\kar] Karo-\conv{Splinter}
\edsc
\item[1\tre\sep(1\coe)\sep?]~
\bdsc
\item[\kontra] genau 4er\pik\ oder eine starke Hand mit \ka-L"ange
\item[1\pik] mindestens 5er\pik
\item[2\kar] \nf
\item[2\coe] \pf\ und Frage nach \co-Stopper
\item[3\coe] Coeur-\conv{Splinter}
\edsc
\item[1\tre\sep(1\pik)\sep?]~
\bdsc
\item[\kontra] genau 4er\coe\ oder eine starke Hand mit \ka- oder \co-L"ange
\item[2\kar] \nf
\item[2\coe] mindestens 5er\coe, \nf
\item[2\pik] \pf\ und Frage nach \pi-Stopper
\item[3\pik] Pik-\conv{Splinter}
\edsc
\edsc

Alle "ubrigen Gebote unterscheiden sich nicht von einer ungest"orten
Reizung (\conv{Fit Jumps}, \conv{Inverted Minors}, \conv{Weak Jumps}).  Die
Reizung nach 1\kar-Er"offnung ist analog.

\end{document}

\bcbiddingpair{West}{Ost}
{
}
{
  \Meaning{a}{}
  \Meaning{b}{}
  \Meaning{c}{}
  \Meaning{d}{}
}

\dealer{S, N--S\\\mbox{}}
\comment{Board 3\\\mbox{}}

\dealerS
\northhand{DB6}{D52}{B10432}{B5}
\southhand{A107}{A763}{KD7}{AK8}
\westhand{K9853}{B8}{65}{6432}
\easthand{42}{K1094}{A98}{D1097}

Erst nur die Nord-S\"ud-H\"ande mit Reizung zu viert \ldots

\playproblemNS
{
2\,SA & --- & 3\,SA & --- \\
---   & --- & --- 
}

\ldots oder zu zweit:

\playproblemNSpair
{
2\,SA & 3\,SA \\
---   & --- 
}

dann die ganze Hand

\medskip

\centerline{\showgame}

\newpage

Oder vielleicht doch lieber als Gegenspielproblem?

\bidders{Fritz}{Franz}{Fuzzy}{Alf}

\playproblemNW
{
2\,SA & --- & 3\,SA & --- \\
---   & --- & --- 
}

bzw.

\playproblemNWpair
{
2\,SA & 3\,SA \\
---   & --- 
}

oder andersherum

\medskip

\playproblemNE
{
2\,SA & --- & 3\,SA & --- \\
---   & --- & --- 
}

bzw.

\medskip

\playproblemNEpair
{
2\,SA & 3\,SA \\
---   & --- 
}

\newpage

\dealer{}
\comment{}

Oder haben die Gegner ein Reizproblem?

\medskip

\playproblemEW
{
2\,SA & --- & 3\,SA & --- \\
---   & --- & --- 
}

bzw.

\medskip

\playproblemEWpair
{
2\,SA & 3\,SA \\
---   & --- 
}

Ein Reizproblem zu viert:

\biddingproblem{1.}{ADB96}{KD52}{43}{82}
{
1\,\pik & 2\,\kar & 2\,\pik & 3\,\kar \\
?
}

und eins zu zweit (ohne Nummer):

\biddingproblempair{}{ADB96}{KD52}{43}{82}
{
1\,\pik & 2\,\pik \\
?
}

Wie soll man

\onesuitNS{ABxx}{Kxx}

auf 3 Stiche spielen?

Gewinnt Ihre Spielweise auch bei

\onesuit{ABxx}{Kxx}{xxxx}{Dx}

Dann haben Sie es richtig gemacht!

K\"onnen Sie es auch auf Ost-West?

\onesuitEW{KBxx}{Axx}

Auch Farben werden mit dem richtigen Abstand zwischen den einzelnen
Karten gesetzt: Wie spielen Sie von \suit{AD10xx} aus?

\hanghand{AKD}{AKD}{AKD10}{AKD} An diesem Beispiel kann man gut
den Unterschied zwischen den Wissenschaftlern und den Zockern
im Bridge erkl\"aren: Ein echter Zocker w\"urde sicher
mit 7\,SA er\"offnen; der Wissenschaftler wendet lieber die ber\"uchtigte 
Hofstein-Konvention an: Die Er\"offnung 5\,SA fragt unmittelbar nach
Buben, wobei der Partner die Farbe seines niedrigsten Buben bietet
(ohne Buben bietet er Ohne :-). Falls der Partner einen Buben zeigt,
den man nicht gebrauchen kann, wie hier \tre-Buben, dann reizt man
die Farbe, in der man den Buben braucht. Hat der Partner den gesuchten
Buben, bietet er 7\,\tre, ansonsten wieder 6\,SA. Damit ist dieses
h\"aufig autretende Bietproblem ein f\"ur alle Mal gel\"ost.

\comment{Problem 1}
\setlength{\handwidth}{27mm}
\northhand{-}{B1098765}{DB10}{AKD}
\southhand{A765432}{AKD}{AK}{2}
\hangNSgame In einem Paarturnier haben Sie und Ihr Partner schon
fr\"uh erkannt, da\ss{} diese beiden H\"ande Schlemm produzieren werden.
Nach etwa 22 Bietrunden konnten Sie mit Ihrem genialen System
herausfinden, da\ss{} im Prinzip 13~Stiche von oben da sind. Gierig
wie Sie nun einmal sind, haben Sie dann 7\,SA geboten, anstelle sich mit
7\,\coe\ zu begn\"ugen. Jetzt brauchen Sie den Kontrakt blo\ss{}
noch zu erf\"ullen. Wie spielen Sie diese Hand in 7\,SA nach 
dem Ausspiel von \tre\,B? 

\newpage
\comment{}

\dots\ und hier ein Auszug aus dem letzten \"Uberstich (hoffentlich mit
freundlicher Genehmigung des Herausgebers :-):

\westhand{2}  {AB42}{KB10953}{B8}
\easthand{B75}{D10} {A42}    {AK752}

\bcproblem{Hand 1}{N}{NS}
{Nord er\"offnet 2\Di\ Flannery (5\pl er \He, 4er \Sp, 11-15).
S\"ud bietet 2\Sp\ falls m\"oglich.}
{5\Di=20, 4\Di=13, 3\Cl=8, \\ 2\Sp(S\"ud)=4}

Es ist ziemlich schwierig, in dieser Hand \"uberhaupt in die Reizung zu kommen.
Michaels Vater bezweifelt, da\ss{} 5\Di\ der Topkontrakt sei. Zugegebenerweise
ist dieser in erster Linie auf den Karo--Schnitt (vorausgesetzt Nord h\"alt
nicht gleich ein 4er Karo) angewiesen. Doch die Tatsache, da\ss{} Nord er\"offnet hat plus
die Zusatzchance, da\ss{} Coeur--Impa\ss{} und die Tr\"umpfe 2--2 stehen, macht f\"ur
mich 5\Di\ zu einer guten Wette.

\bcbidding{Fu\ss{}}{Gro}
{
           & 2\Di & p          & 2\Sp \\
p          & p    & X\al{a}    & p    \\
3\Di\al{b} & p    & p          & p    
}
{
\Meaning{a}{Take--out}
\Meaning{b}{Zusatzst\"arke! 2SA w\"are Lebensohl}
}

\bcbidding{Frans}{Alain}
{
           & 2\Di & p          & 2\Sp \\
3\Di       & p    & 3\Sp\al{a} & p    \\
4\Di       & p    & 5\Cl\al{b} & p    \\
5\Di       & p    & p          & p    
}
{
\Meaning{a}{Frage nach \Sp--Stopper}
\Meaning{b}{Cue--bid}
}

\vspace{0.5cm}

\underline{\bf Running Score:} 

\vspace{0.5cm}

Gro -- Fu\ss{} 13, Alain -- Frans 20

\newpage

\westhand{1063}{AD4} {A84}  {A742}
\easthand{B}   {K103}{KD752}{KB53}

\bcproblem{Hand 2}{W}{NS}{}
{5\Cl=20, 5\Di=14, 6\Cl=8, \\ 6\Di=6, 3SA=4}

In dieser Hand gilt es, die Pik--Schw\"ache zu lokalisieren, um 3SA vermeiden
zu k\"onnen. Bei der Wahl der Unterfarbe ist der stabilere 4--4--Fit in Treff
dem 5--3--Fit in Karo vorzuziehen. Hat man dies alles herausgefunden, stellt
sich noch die Frage, ob man sich mit Partie begn\"ugen oder den Kleinschlemm
ansteuern soll. Insgesamt eine nicht ganz leichte Aufgabe$\dots$

\bcbiddingpair{Fu\ss{}}{Gro}
{
1SA        & 2\Sp\al{a} \\
3\Cl\al{b} & 3\He\al{c} \\
4\Cl       & 5\Cl       \\
6\Cl\al{d} & p 
}
{
\Meaning{a}{Frage nach Min/Max}
\Meaning{b}{Maximum}
\Meaning{c}{1--3--(45) Verteilung, \pf}
\Meaning{d}{keine Werte in \Sp\ und die \"ubrigen Asse. Etwas optimistisch.}
}

\bcbiddingpair{Frans}{Alain}
{
1\Cl      & 1\Di\al{a} \\
1SA\al{b} & 3\Cl       \\
3SA       & p 
}
{
\Meaning{a}{Walsh = keine 4er OF \\ oder \pf}
\Meaning{b}{12-14 P., 4er OF(n) \\ m\"oglich}
}

\vspace{0.5cm}

Frans und Alain scheitern leider schon an der ersten H\"urde und landen im
falschen Vollspiel. Gro und Fu\ss{} vermeiden glorreich 3SA, landen dann aber im
Kleinschlemm, der etwas gezogen ist.

\vspace{0.25cm}

\underline{\bf Running Score:}

\vspace{0.5cm}

Gro -- Fu\ss{} 21, Alain -- Frans 24

\newpage

\onecolumn{
\Huge
\setgamesize{\Huge}
\setlength{\handwidth}{55mm}	% The width of a hand in a game display
\setlength{\bidwidth}{100mm}	% The width of the table with the bidding sequence
\setlength{\cardskip}{2pt}	% The distance between subsequent playing cards

%\centerline{\showgame}

\playproblemNS
{
2\,SA & --- & 3\,SA & --- \\
---   & --- & --- 
}
}
\end{document}

%%%%%%%%%%%%%%%%%%%%%%%%%%%%%%%%%%
%that's all, hope you enjoy
%%%%%%%%%%%%%%%%%%%%%%%%%%%%%%%%%%

