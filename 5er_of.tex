% Format-spezifische Sachen nach layout.tex...
% DIN A4, zweispaltig
\documentclass[11pt,german,twocolumn,twoside]{scrartcl}
\usepackage[a4paper,vscale=0.94,includehead,
  %rmargin=10mm,lmargin=15mm,  % zum Lochen
  hmargin=10mm,                % zum Abheften
]{geometry}

% DIN A5, zweispaltig
%\documentclass[8pt,german,twocolumn,twoside]{scrartcl}
%\usepackage[a5paper,vscale=0.94,includehead,hmargin=5mm]{geometry}

% DIN A6, einspaltig
%\documentclass[8pt,german,twoside]{scrartcl}
%\usepackage[a6paper,vmargin=1mm,includehead,hmargin=1mm]{geometry}
%\usepackage[normalmargins]{savetrees}

% Fonts
%\usepackage{times}

% Pakete, die wir hier schon laden m"ussen
\usepackage{color}
\usepackage{kibitzer}

% Farben definieren
\definecolor{ClColor}{rgb}{0.0,0.6,0.0}
\definecolor{DiColor}{rgb}{1.0,0.3,0.0}
\definecolor{HeColor}{rgb}{0.9,0.0,0.0}
\definecolor{SpColor}{rgb}{0.0,0.0,0.7}

% Farben wie in der Biet-Box
%\renewcommand{\Cl}{{\color{ClColor}{\clubs}}}
%\renewcommand{\Di}{{\color{DiColor}{\sdiamonds}}}
%\renewcommand{\He}{{\color{HeColor}{\shearts}}}
%\renewcommand{\Sp}{{\color{SpColor}{\spades}}}
%
% Schwarz-Rot
%\renewcommand{\Cl}{\clubs}
%\renewcommand{\Di}{{\color{HeColor}{\sdiamonds}}}
%\renewcommand{\He}{{\color{HeColor}{\shearts}}}
%\renewcommand{\Sp}{\spades}
%
% mit pxfonts (haesslich)
%\usepackage{pxfonts}
%\renewcommand{\Cl}{{\color{ClColor}{\relsize{2}$\clubsuit$}}}
%\renewcommand{\Di}{{\color{DiColor}{\relsize{2}$\vardiamondsuit$}}}
%\renewcommand{\He}{{\color{HeColor}{\relsize{1}$\varheartsuit$}}}
%\renewcommand{\Sp}{{\color{SpColor}{\relsize{2}$\spadesuit$}}}


% ... der Rest ist für alle Papiergr"o"sen etc. gleich

\usepackage{babel}
\usepackage{fancyhdr}
\usepackage{tabularx}
\usepackage{calc}
\usepackage{suitsymbols}
\usepackage{makeidx}
\usepackage{xspace}

% L"angen
\setlength{\columnsep}{10mm}
\setlength{\columnseprule}{0.4pt}
%\setlength{\labelsep}{1.7ex}
\setlength{\itemsep}{0ex plus0.2ex}

\handwidth7.5em
\cardskip0.15ex
\parindent0mm
\parskip2ex plus1ex minus1ex
\renewcommand{\arraystretch}{1.07}

% Farben ohne Space davor
\def\pi{\Sp\xspace}
\def\co{\He\xspace}
\def\ka{\Di\xspace}
\def\tr{\Cl\xspace}
\def\sa{\nobreak\textsf{S\kern-0.08emA}\xspace}
\def\nt{\nobreak\textsf{N\kern-0.08emT}\xspace}
% 2 Farben
\def\ofa{\nobreak\textsf{OF}\xspace}
\def\aofa{\nobreak\textsf{aOF}\xspace}
\def\ufa{\nobreak\textsf{UF}\xspace}
\def\aufa{\nobreak\textsf{aUF}\xspace}

% Farben mit Space davor
\def\pik{\nobreak\hspace{\cardskip}\Sp\xspace}
\def\coe{\nobreak\hspace{\cardskip}\He\xspace}
\def\kar{\nobreak\hspace{\cardskip}\Di\xspace}
\def\tre{\nobreak\hspace{\cardskip}\Cl\xspace}
\def\SA{\nobreak\hspace{\cardskip}\sa}
\def\NT{\nobreak\hspace{\cardskip}\nt}
% 2 Farben
\def\of{\nobreak\hspace{\cardskip}\textsf{OF}\xspace}
\def\aof{\nobreak\hspace{\cardskip}\textsf{aOF}\xspace}
\def\uf{\nobreak\hspace{\cardskip}\textsf{UF}\xspace}
\def\mi{\hspace{\cardskip}\Cl{}/\Di\xspace}
\def\ma{\hspace{\cardskip}\He{}/\Sp\xspace}

% diverse Makros
\def\good{$^+$\xspace}
\def\bad{$^-$\xspace}
\def\ra{$\rightarrow$\xspace}
\def\pl{$\uparrow$\xspace}
\def\any{$x$\xspace}
\def\anybid{\nobreak\hspace{\cardskip}\any}
\def\kontra{\textsf{X}\xspace}
\def\rekontra{\textsf{X\kern-0.08emX}\xspace}
\def\sep{\,--\,}
\newcommand{\conv}[1]{\emph{#1}}
\def\bal{\textsc{Ausg}\xspace}
\def\unbal{\textsc{Unausg}\xspace}
\def\nat{\textsc{Nat}\xspace}
\def\pf{\textsc{PF}\xspace}
\def\maxi{\textsc{Max}\xspace}
\def\mini{\textsc{Min}\xspace}
\def\inv{\textsc{Einl}\xspace}
\def\nf{\textsc{NF}\xspace}
\def\rel{\textsc{Rel}\xspace}
\def\stp{\textsc{Stop}\xspace}
\def\hstp{\textsc{Hstop}\xspace}
\def\cstop{\co-\stp}
\def\pstop{\pi-\stp}
\def\tstop{\tr-\stp}
\def\kstop{\ka-\stp}
\def\chstop{\co-\hstp}
\def\phstop{\pi-\hstp}
\def\thstop{\tr-\hstp}
\def\khstop{\ka-\hstp}
\def\aw{\textsc{Aw}\xspace}
\def\eo{\textsc{E\"o}\xspace}
\def\xfer{\textsc{Trf}\xspace}
\def\xferto{\xfer{}~\ra~}
\def\pup{\textsc{Pup}\xspace}
\def\pupto{\pup{}~\ra~}
\def\leadto{\nobreak\hspace{\cardskip}$\leadsto$\hspace{\cardskip}}
\def\slamint{\textsc{Schl-Int}\xspace}

% Descriptions f"ur Bietb"aume
\def\bdsc{\begin{description}}
\def\edsc{\end{description}}

% compactitem
\usepackage{paralist}
\setlength{\plitemsep}{0.6ex}
\setlength{\pltopsep}{0.8ex}
\setlength{\plpartopsep}{0ex}
\defaultleftmargin{2.5ex}{2.3ex}{2.1ex}{1.9ex}

% neue Spalte f"ur tabularx
\newcolumntype{Y}{>{\raggedright\arraybackslash}X}

% Reizung mit 4 H"anden
\newcommand{\bidex}[1]{\bcbidding{\textsl{West}}{\textsl{Ost}}{#1}}

% Dreispaltige Tabelle f"ur "Ubersichten
\newcommand\bidins[1]%
{%
\begin{flushleft}
\begin{tabularx}{\columnwidth}{llY}%
#1
\end{tabularx}%
\end{flushleft}
}
% ... mit fester Breite
\newcommand\bidinsfixed[1]{%
\begin{tabularx}{\columnwidth}{p{3em}p{2.5em}Y}%
#1
\end{tabularx}%
}

% Hand links, vertikale Linie, Text rechts
\newcommand{\handwithdesc}[5]{%
\handwidth6em
\vhand{#1}{#2}{#3}{#4}\hspace{1.5ex}\vrule\hspace{1ex}
\begin{minipage}[t]{\columnwidth-\handwidth-2.5ex-0.6pt}
#5
\end{minipage}
\smallskip
\handwidth7.5em
}

% W/O-Beispielhand mit Bietsequenz darunter
\newcommand{\exhand}[9]{%
\begin{minipage}{\columnwidth}
\westhand{#1}{#2}{#3}{#4}
\easthand{#5}{#6}{#7}{#8}
\centerline{\showEWgame}
\bigskip
{\smaller\begin{tabularx}{\columnwidth}{rYrY}
\multicolumn{2}{l}{West} & \multicolumn{2}{l}{Ost}\\
\hline
#9
\end{tabularx}}
\end{minipage}
}

% W/O-Bietsequenz
\newcommand{\woreizung}[1]{%
\begin{minipage}{\columnwidth}
{\smaller\begin{tabularx}{\columnwidth}{rYrY}
\multicolumn{2}{l}{West} & \multicolumn{2}{l}{Ost}\\
\hline
#1
\end{tabularx}}
\end{minipage}
}

\newenvironment{reizung}[1][t]%
{%
  \begin{tabular*}{\bidwidth}[#1]{@{}*{3}{l@{\extracolsep{0pt plus 1fil}}}l@{}}%
    West & Nord & Ost & S\"ud \\
    \hline
}
%    what's to be done at the end of the environment
{%
  \end{tabular*}%
}

% Reizung mit 4 H"anden
\newcommand{\reizungmittext}[3][\bidwidth]{%
\mbox{\smaller
\begin{minipage}{#1}
\begin{reizung}
#2
\end{reizung}
\end{minipage}%
\hspace{1.5ex}\vrule\hspace{1.5ex}%
\begin{minipage}{\columnwidth-#1-3ex-0.6pt}%
#3
\end{minipage}
}
}

% \newcommand\bidseq[1]%
% {%
% \begin{flushleft}
% \begin{tabularx}{\columnwidth}{llY}%
% #1
% \end{tabularx}%
% \end{flushleft}
% }

% Kasten mit Rahmen
\setlength{\fboxsep}{1.5ex}
\newcommand\notebox[1]%
{%
\fbox{\parbox{\columnwidth - 3ex}{#1}}%
}

% Index
\makeindex
\newcommand{\Index}[1]{#1\index{#1}}

% Pagestyle
\renewcommand{\sectionmark}[1]{\markleft{#1}}
\pagestyle{fancy}
\fancyhead[RE,LO]{\nouppercase{\emph{\leftmark}}}
\fancyhead[RO,LE]{\thepage}
\fancyfoot{}
\renewcommand{\headrulewidth}{0pt}


\begin{document}

\begin{center}
\textsf{{\relsize{5}\co{}~~5er-Oberfarben~~\pi\\[1ex]}
\tr{}~~Regine Bartels und Thomas Schmitt~~\ka\\[1ex]
\small \today}
\end{center}
\tableofcontents

%
%%%%%%%%%%%%%%%%%%%%%%%%%%%%%% Eroeffnungen %%%%%%%%%%%%%%%%%%%%%%%%%%%%%%
%
\newpage
\section{"Ubersicht der Er"offnungen}

\subsection*{1er-Stufe}
\bidinsfixed{%
1\mi & 12\pl	& 3\pl{}er-\mi\\[1ex]
1\ma & 12\pl	& 5\pl{}er-\ma\\[1ex]
1\SA & 15-17	& \bal, 5er-\ofa m"oglich
}

\subsection*{2er-Stufe}
\bidinsfixed{%
2\tre	& 6-10	& Weak Two in \ka, oder\\
	& 16\pl	& 6er-Farbe, 8 Spielstiche, oder\\
	& 22-23	& \bal, oder\\
	& 26-27	& \bal\\[1ex]
2\kar	& 6-10	& Weak Two in \co, oder\\
	& 18\pl	& 6er-Farbe, 9 Spielstiche, oder\\
	& 24-25	& \bal, oder\\
	& 28\pl	& \bal\\[1ex]
2\coe	& 6-10	& Zweif"arber mit \co\\[1ex]
2\pik	& 6-10	& Weak Two in \pi\\[1ex]
2\SA	& 20-21	& \bal, 5er-\ofa m"oglich
}

\subsection*{3er-Stufe}
\bidinsfixed{%
3\uf	& 5-10	& 7\pl{}er-\ufa (in 1. Hand in Nichtgefahr und 3. Hand 6er-Farbe m"oglich)\\[1ex]
3\of	& 5-10	& 7\pl{}er-\ofa\\[1ex]
3\SA	& 	& Gambling in 1./2. Hand, in 3./4. Hand zum Spielen
}

\subsection*{4er-Stufe}
\bidinsfixed{%
4\mi	& 	& stehendes 7er-\ma mit einem Nebenwert\\[1ex]
4\of	&	& 7\good{}\pl{}er-\ofa, zum Spielen\\[1ex]
4\SA	&	& 6/5\pl\uf
}

%
%%%%%%%%%%%%%%%%%%%%%%%%%%%%% 1 UF-Eroeffnung %%%%%%%%%%%%%%%%%%%%%%%%%%%%
%
\newpage
\section{Die 1\tre/\ka-Er"offnungen}

Bei gleicher L"ange in den \uf wird 1\kar er"offnet, mit 3/3 in den \uf
wird 1\tre er"offnet.
Ohne 5\pl{}er-\ofa wird grunds"atzlich die l"angere \uf er"offnet.

Siehe auch \textit{Er"offnungsregel f"ur Zweif"arber mit 6/5 \ufa/\ofa{}} auf
Seite \pageref{zfregel}.

\subsection{Antworten auf 1\tre-Er"offnung} \label{1treff}
\bidins{%
1\kar	& 6-7	& \bal ohne 4er-\ofa, 3-3-3-4 m"oglich, oder\\
	& 6\pl	& 7\pl{}er-\ka, oder\\
	& 12\pl	& 5\pl{}er-\ka und 4\pl{}er-\ofa \conv{(Walsh-\ka)}\\[1ex]
1\of	& 6\pl	& 4\pl{}er-\ofa\\[1ex]
1\SA	& 8-10	& \bal ohne 4er-\ofa\\[1ex]
2\tre	& 10\good{}\pl & 5\pl{}er-\tr \conv{(Inverted)}\\[1ex]
2\kar	& 2-5	& 5/5 \ofa\\[1ex]
2\of	& 5-8	& gute 6er-\ofa \conv{(Weak Jump)}\\
        &       & \ra 2\SA = \conv{\Index{Ogust}\footnote{\mini/\maxi entsprechend angepasst}} \\[1ex]
2\SA    & 2-6   & 5\pl{}er-\tr \conv{(Inverted Spezial)}\\
3\tre	& 7-9	& 5\pl{}er-\tr \conv{(Inverted)}\\[1ex]
3\kar/\co/\pi & 5-8 & gutes 7\pl{}er-\ka/\co/\pi \conv{(Weak Jump)}\\[1ex]
3\SA	& 13-15	& zum Spielen\\[1ex]
4\tre	&	& \conv{KCB}\\[1ex]
4\of	&	& zum Spielen
}

\notebox{\textbf{Weiterreizung:}
Ein unn"otiger Sprung in einer neuen Farbe ist \conv{\Index{Splinter}}.
Ein unn"otiger Doppelsprung in einer neuen Farbe ist \conv{Exclusion KCB}.}

Nach 1\kar{}-Er"offnung sind die Antworten entsprechend
(Ausnahme: [1\kar{}\sep2\tre{}] \ra\ref{inverted}).

%
%%%%%%%%%%%%%%%%%%%%%%%%%%%%% Walsh-Sequenzen %%%%%%%%%%%%%%%%%%%%%%%%%%%%
%
\subsection{Die \conv{Walsh}-Antwort auf 1\tre-Er"offnung}

Der Antwortende zeigt nach 1\tre-Er"offnung immer sofort eine
4er-\ofa, es sei denn
\begin{compactitem}
\item er h"alt ein 7\pl{}er-\ka, oder
\item die Karos sind l"anger als die \ofa und der Antwortende ist
  bereit, bei der n"achsten Gelegenheit die \ofa revers zu reizen
  (also ab 12 Punkten).
\end{compactitem}
In allen anderen F"allen zeigt der Antwortende immer \emph{sofort}
seine rangniedrigste 4er-\ofa.

Der Er"offner bietet nach 1\kar-Antwort mit ausgeglichener Hand auch
dann 1\SA zur"uck, wenn er eine oder beide \ofa h"alt -- denn der
Antwortende hat entweder keine 4er-\ofa oder wird diese nun reizen,
was gleichzeitig ein Partieforcing etabliert.

Ein \ofa-R"uckgebot des Er"offners nach [1\tre-1\kar;] zeigt
immer eine unausgeglichene Hand mit langen Treffs.

Diese Gebote sollten folgenderma"sen alertiert werden:
\begin{compactdesc}
\item[1\tr-1\ka;] \emph{"`keine 4er-\ofa oder stark genug, um Revers zu reizen"'}
\item[1\tr-1\ka;~1\SA] \emph{"`kann jede 4er-\ofa enthalten"'}
\item[1\tr-1\ka;~1\ofa] \emph{"`unausgeglichen"'}
\end{compactdesc}

\minisec{Bietsequenzen nach [1\tre{}\sep1\kar{}]}

\bdsc
\item[1\tre{}\sep1\kar; ?]~
  \bdsc
  \item[1\coe] 5\pl{}er-\tr und 4er-\co oder 4-4-1-4
  \item[1\pik] 5\pl{}er-\tr und 4er-\pi
  \item[1\SA] 12-14, eine oder beide \of m"oglich
    \bdsc
      \item[2\tre] \slamint in \tre oder \kar;
            \textbf{Frage nach Verteilung} (s.~u.)
      \item[2\kar] schwach; zum Spielen
      \item[2\ma] \pf mit 5/4-Verteilung
      \item[2\SA] \nat
      \item[3\tre/\co/\pi] \conv{\Index{Splinter}} mit 6er-\ka
      \item[3\kar] \inv mit 6er-\ka
    \edsc
  \item[2\tre] 12-14, 6er-\tr
    \bdsc
      \item[2\coe] \cstop, kein \pstop
        \bdsc
        \item[2\pik] Frage nach \phstop \conv{(VFF)}
        \edsc
      \item[2\pik] zeigt \pstop
        \bdsc
        \item[3\tre] verneint \cstop
        \item[3\coe] Frage nach \chstop \conv{(VFF)}
        \edsc
      \item[3\ma] \conv{Splinter} mit \tr-Anschluss
      \item[4\ma] \conv{EKCB} auf \tr-Basis
    \edsc
  \item[2\SA] 18-19 FP, eine oder beide \ofa m"oglich
    \bdsc
    \item[3\tre] \pupto3\kar, worauf man passen kann oder Sign Off gibt
    (\conv{Wolff Sign Off} \ra \ref{wolff})
    \item[3\kar] Forcing
    \item[3\of] 12\pl FP, 5/4-Verteilung, Schlemm m"oglich (s.~u.)
    \edsc
  \edsc
\edsc

\minisec{Verteilungsfrage nach 1\SA-R"uckgebot}

\bdsc
\item[1\tre{}\sep1\kar; 1\SA{}\sep2\tre; ?]~

  Hat der Antwortende Schlemm-Interesse in \ufa, so kann er die genaue
  Verteilung der ausgeglichenen Hand des Er"offners erfragen.

  \bdsc
  \item[2\kar] 3er-\ka, aber nicht 4333
    \bdsc
      \item[2\coe] \rel, Frage nach Verteilung
        \bdsc
        \item[2\pik] 5er-\tr (x-x-3-5)
        \item[2\SA] 4er-\tr (x-x-3-4)
        \edsc
      \item[3\mi] \conv{KCB} auf \tr/\ka-Basis
    \edsc
  \item[2\coe] 3-4-2-4
  \item[2\pik] 4-3-2-4
  \item[2\SA] beliebige 4333-Verteilung\footnote{4er-\ka ist nicht m"oglich, da mit ausgeglichener Hand und 4er-\ka nicht 1\tre er"offnet wird.}
    \bdsc
      \item[3\tre] \rel, Frage nach 4er-Farbe
        \bdsc
        \item[3\kar] \textbf{4er-\tr}
        \item[3\coe] 4er-\co
        \item[3\pik] 4er-\pi
        \edsc
    \edsc
  \item[3\tre] 3-3-2-5 \ra 3\kar: \conv{KCB}
  \item[3\kar] 4-4-2-3
  \edsc
\edsc

\minisec{Assfrage nach 2\SA-R"uckgebot}

\bdsc
  \item[1\tre{}\sep1\kar; 2\SA{}\sep{}3\of; ?]~

    Reizt der Antwortende nach \conv{Walsh}-\ka nun 3\of (verspricht ab 12
    Punkten), so ist meistens ein Schlemm m"oglich.  Der Er"offner
    beantwortet daher mit \ka-Fit sofort die Assfrage auf
    \ka-Basis (andernfalls Sign Off in \SA oder \ofa).

    \bdsc
      \item[3\coe] 12\pl FP, 5/4-Verteilung, Schlemm m"oglich
        \bdsc
          \item[3\pik] 1 oder 4 Keycards auf \ka-Basis
          \item[3\SA] \nat, kein Fit
          \item[4\tre] 0 oder 3 Keycards auf \ka-Basis
          \item[4\kar] 2 Keycards ohne \ka-Dame
          \item[4\coe] \co-Fit
          \item[4\pik] 2 Keycards mit \ka-Dame
        \edsc
      \item[3\pik] 12\pl FP, 5/4-Verteilung, Schlemm m"oglich
        \bdsc
          \item[3\SA] \nat, kein Fit
          \item[4\tre] 1 oder 4 Keycards auf \ka-Basis
          \item[4\kar] 0 oder 3 Keycards auf \ka-Basis
          \item[4\coe] 2 Keycards auf \ka-Basis (mit/ohne Dame)
          \item[4\pik] \pi-Fit
        \edsc
    \edsc
\edsc

\subsection{\conv{Wolff Sign Off} nach 2\SA-R"uckgebot} \label{wolff}

\conv{Wolff Sign Off} ist n"utzlich in Situationen, in denen der
Antwortende sehr schwach ist und seine erste Antwort lediglich den
Er"offner vor einem schlechten Treff- oder Karo-Kontrakt hat bewahren
sollen.

Der Antwortende kann das forcierte 3\kar-Gebot des Er"offners passen,
3\coe zum Spielen oder Ausbessern anbieten, oder ein beliebiges Gebot
zum Spielen abgeben.

\minisec{Beispiele}

Die Reizung beginnt jeweils mit [1\tre{}\sep1\pik;~2\SA{}].

\handwithdesc{Qxxxx}{Qxxxx}{xx}{x}{%
  Wir m"ochten entweder 3\coe oder 3\pik spielen.  Wir reizen daher
  [3\tre;~3\kar{}\sep3\coe;] zum Spielen in 3\coe oder Ausbessern nach
  3\pik.}

\handwithdesc{K10xx}{xx}{J10xxxx}{x}{%
  Wir m"ochten in jedem Fall 3\kar spielen.  Wir reizen
  [3\tre;~3\kar{}\sep{}pass].}

%
%%%%%%%%%%%%%%%%%%%%%%%%%%%%% Sequenzen nach 1T-1OF %%%%%%%%%%%%%%%%%%%%%%
%
\subsection{Bietsequenzen nach [1\tre{}\sep1\of{}]}

\bdsc
\item[1\tre{}\sep1\coe; ?]~
  \bdsc
  \item[1\pik] 12\pl FP, \nat

    Wiederholt der Antwortende seine Oberfarbe auf niedrigster Stufe
    nachdem der Er"offner zwei Farben gereizt hat, so zeigt dies eine
    einladende Hand mit 9-11 FP und 6er-Farbe.

    Wiederholt er seine Oberfarbe im Sprung, so zeigt dies eine
    partieforcierende Hand mit 6er-Farbe:
    \bdsc
    \item[2\coe] 9-11 FP, 6er-\co, \inv
    \item[3\coe] 6er-\co, \pf
    \edsc
  \item[1\SA] 12-14 FP \bal, kein 4er-\pi (\ra \ref{1sarebid})
  \edsc

\item[1\tre{}\sep1\pik; ?]~
  \bdsc
  \item[1\SA] 12-14 FP, \bal (\ra \ref{1sarebid})
  \item[2\tre] 12-16\bad{} FP, 5\pl{}er-Farbe (s.~u.)
  \item[2\SA] 18-19 FP, \bal; siehe \conv{Wolff} etc.
  \item[3\kar/\co] 4er-\pi, Single \ka/\co \conv{(\Index{Splinter})}
  \item[3\SA] 18-19 FP, stehendes 6er-\tr, K"urze in \pik, Deckung in den
    ungereizten Farben, 8-8$\frac{1}{2}$ Stiche
  \item[4\tre] 18\good{}\pl FP, 4er-\pi, 5\good{}\pl{}er-\tre \conv{(Fit-Sprung)}
  \item[4\kar/\co] 18\good{}\pl FP, 4er-\pi \conv{(Exclusion RCKB)}
  \edsc

\item[1\tre{}\sep1\pik; 2\tre{}\sep{}?] 12-16\bad{}, 5\pl{}er-Farbe
  \bdsc
  \item[2\kar] \conv{DFF} (\ra \ref{dff})
    \bdsc
    \item[2\coe] Frage nach \chstop (\conv{VFF}), kein 3er-\pi
    \item[2\pik] 3er-\pi, \mini f"ur 2\tre-R"uckgebot\\
      \ra~3\tre = Forcing mit \tre
    \item[2\SA] \cstop, kein 3er-\pi, \mini
    \item[3\pik] 3er-\pi, \maxi f"ur 2\tre-R"uckgebot
    \edsc
  \item[3\tre] \inv
  \edsc
\edsc

\subsection{Bietsequenzen nach \conv{Inverted}} \label{inverted}

Nach [1\tre{}\sep2\tre{};] bzw. [1\kar{}\sep2\kar{};] gibt es kein Revers.
Ziel ist meist, einen \sa-Kontrakt zu spielen, wenn alle Nebenfarben gestoppt
sind. Der Er"offner zeigt mit 2\SA eine ausgeglichene, passbare Hand.
3\tre bzw. 3\kar ist ebenfalls passbar.  Alle anderen Gebote des Er"offners
zeigen Werte (Stopper) und sind partieforcing, weitere Gebote es Antwortenden
zeigen ebenfalls Werte.

\bdsc
  \item[1\tre{}\sep2\tre; ?]~
    \bdsc
      \item[2\kar] 14\good{}\pl FP, \kstop
        \bdsc
        \item[2\coe] \cstop (h"ochstens \phstop)
          \bdsc
          \item[2\pik] fragt nach \phstop (siehe \conv{VFF})
          \edsc
        \item[2\pik] \pstop (h"ochstens \chstop)
        \item[2\SA] \stp in \co und \pi.
        \edsc
      \item[2\coe-2\pik] 14\good{}\pl FP, \stp in der Farbe
    \edsc
\edsc

[1\kar{}\sep2\tre] wird "ahnlich wie Inverted behandelt, mit dem R"uckgebot
zeigt der Er"offner seine Punktst"arke:

\bdsc
  \item[1\kar{}\sep2\tre; ?]~
    \bdsc
    \item[2\kar] 12-13 FP, \bal oder \nat (kann 3er-\ka sein)
    \item[2\SA] 14 FP, \bal
    \edsc
\edsc


\subsection{\label{zfregel}Er"offnungsregel f"ur Zweif"arber mit 6/5~\uf/\of}

\bdsc
%\setlength{\labelsep}{1ex}
\item[4\pl{} Verlierer:] \of er"offnen und \uf billig nachreizen
\item[3\good-4 Verlierer:] \uf er"offnen und anschlie"send \conv{Revers}
  reizen
\item[0-3\bad Verlierer:] Partieforcing (2\kar) er"offnen
\edsc

\newpage
%%%%%%%%%%%%%%%%%%%%%%%%%%%%%%%%%%%%%%%%%%%%%%%%%%%%%%%%%%%%%%%%%%%%%%%%%%%%%%
\section{Die 1\coe/\pi-Er"offnungen}

Die Fortsetzung nach 1\of-Er"offnungen folgt folgenden Prinzipien:
\begin{compactitem}
%\setlength{\itemsep}{0.5ex}
\item Schwache bis einladende H"ande mit 4\pl{}er-Anschluss werden durch
  \conv{Bergen-Hebungen} gezeigt.
\item Mit \pf und gutem Trumpfanschluss reizen wir
  \conv{\Index{Splinter}} oder \conv{Stenberg 2\SA}.
\item Die restlichen starken Varianten werden durch verz"ogertes
  Reizen der Trumpfunterst"utzung gezeigt (Farbwechsel).
\end{compactitem}

\subsection{Antworten auf 1\of-Er"offnung}

\bidins{%
  1\SA & 6-10\bad & kein Anschluss, keine weitere \ofa\\[1ex]
  2\tre & 10\good{}\pl & \textbf{2}\pl{}er-\tr{}\footnote{meist l"anger, im
    schlimmsten Fall 3-4-4-2-Verteilung nach 1\pik-Er"offnung, nach 1\coe
    mindestens 3er (3-3-4-3)}, selbstforcierend\\
  2\kar & 10\good{}\pl & 5\pl{}er-\ka\\[1ex]
  2\of & 6-10\bad & 3er-Anschluss\\[1ex]
  2\SA & 12\good{}\pl & 4\pl{}er-Anschluss, \pf (\conv{Stenberg})\\
       & 11-12 & mit gepasster Hand: Double-\ofa und 3er-\aof (\bal, \nat)\\[1ex]
  3\tre & 9\good{}-11 & 4\pl{}er-Anschluss \conv{(Bergen-Hebung)}\\
  3\kar & 7-9\bad{} & 4\pl{}er-Anschluss \conv{(Bergen-Hebung)}\\
  3\of & 0-6 & 4\pl{}er-Anschluss \conv{(Bergen-Hebung)}\\[1ex]
  3\aof & 11-14 & beliebiges Chicane \conv{(Splinter)}, s.~u.\\
  3\SA & 11-14 & Single in \aofa \conv{(Splinter)}, s.~u.\\
  4\tre/\ka & 11-14 & \tr/\ka-Single \conv{(Splinter)}\\[1ex]
  4\of && zum Spielen\\[1ex]
  4\aof && \conv{Exclusion KCB}\\[1ex]
  \multicolumn{3}{l}{Nach 1\coe-Er"offnung:}\\
  2\pik & 5-8 & 6\pl{}er-\pik (\conv{Weak Jump})
              \ra 2\SA: \conv{\Index{Ogust}}\footnote{\mini/\maxi entsprechend angepasst}\\[1ex]
  \multicolumn{3}{l}{Nach 1\pik-Er"offnung:}\\
  2\coe & 10\good{}\pl & 5\pl{}er-\coe
}

\notebox{%
\textbf{\conv{Splinter}-Gebote} zeigen immer mindestens einen guten
3er-Anschluss in der Trumpffarbe sowie eine K"urze.  Die Werte f"ur
ein Vollspiel werden versprochen, aber die Punktspanne ist nach oben
limitiert.  Mit st"arkeren H"anden sollte \conv{Stenberg}
gereizt werden.
}

\subsection{\conv{\Index{Splinter}}-Gebote nach 1\of-Er"offnung}
[1\of-3\SA;] zeigt ein Single in der anderen \ofa.
Chicane-\conv{Splinter} werden mit einem Sprung in die
andere \ofa auf 3er-Stufe gezeigt

\bdsc
\item[1\coe{}\sep3\pik] beliebiges Chicane
  \bdsc
  \item[3\SA] \rel, Frage nach dem Chicane
    \bdsc
    \item[4\tre] \tr-Chicane
    \item[4\kar] \ka-Chicane
    \item[4\coe] \pi-Chicane
    \edsc
  \edsc
\item[1\pik{}\sep3\coe] beliebiges Chicane
  \bdsc
  \item[3\pik] \rel, Frage nach dem Chicane
    \bdsc
    \item[3\SA] \co-Chicane
    \item[4\tre] \tr-Chicane
    \item[4\kar] \ka-Chicane
    \edsc
  \edsc
\edsc

\subsection{Bietsequenzen nach 1\of-Er"offnung}

\bdsc
\item[1\coe{}\sep1\pik; ?]~
  \bdsc
  \item[1\SA] 12-14 FP \bal
    \bdsc
    \item[2\tre] \conv{Relaistransfer} \ra 2\kar (\ra \ref{1sarebid})
    \item[2\kar] 4er-\pi, 5\pl\kar, zum Spielen
    \item[2\SA] \conv{Relaistransfer} \ra 3\tre (\ra \ref{1sarebid})
    \item[3\tre] \nat, forcing
      \bdsc
      \item[3\kar] Frage nach \hstp, zeigt \khstop, verneint 3er-\pi
        (siehe \conv{VFF})
      \item[3\coe] 2-5-3-3, verneint \khstop
      \item[3\pik] 3er-\pi, \maxi
      \item[3\SA] \kstop, verneint 3er-\pi
      \item[4\pik] 3er-\pi, \mini
      \edsc
    \edsc
  \item[2\tre/\ka] 12-18, 5/4\pl-Verteilung
    \bdsc
    \item[2\coe] Ausbessern, \coe-Double
    \item[3\tre/\ka] \inv
    \edsc
    \textbf{Nach \conv{1~"uber~1} ist die Hebung von Er"offners \emph{zweiter} Farbe
    auf die 3er-Stufe lediglich einladend.}
  \item[2\SA] 18-19 FP \bal
  \item[3\tre/\ka] 18\good{}\pl FP, 5\pl/4\pl \coe/\uf\\
    \ra~3\coe st"arker als 4\coe (\emph{Principle of Fast Arrival});
    \ra~3\pik \conv{Cuebid}, aber keine K"urze.

    \textbf{Ein \conv{Cuebid} in Partners erster Farbe zeigt \emph{nie} eine
    K"urze.}
  \edsc

\item[1\pik{}\sep2\tre; ?]~
\bdsc
\item[2\kar] 12-18 FP, 54\pl
  \bdsc
  \item[2\pik] \inv mit 3er-\pi
  \item[2\SA] \nat, \nf
  \item[3\tre] \nat, \nf
  \item[3\kar] \nat, \pf
    \bdsc
    \item[3\coe] Frage nach \hstp \conv{(VFF)}
    \item[3\pik] kein \chstop, verspricht kein 6er-\pi
    \item[3\SA] \cstop
    \edsc
  \item[3\pik] Schlemm-Interesse mit 3er-\pi
  \edsc
  \textbf{Nach \conv{2~"uber~1} ist die Hebung von Er"offners
    \emph{zweiter} Farbe auf die 3er-Stufe partieforcing.}
\item[2\coe]~
  \bdsc
  \item[2\SA] \nat
    \bdsc
    \item[3\kar] 5/5 in den \ofa, \pf
    \item[3\coe] 5/5 in den \ofa, \nf
    \edsc
  \item[3\coe] \nat, \pf
  \edsc
\item[2\pik] 12-14 FP, kann 5er-\pi sein!
  \bdsc
  \item[3\tre] 6er-\tr, \nf
  \item[3\kar/\co] Werte (siehe \conv{DFF}), \pf
  \item[3\pik] \inv, 3er-\pi
  \edsc
\item[2\SA] 18-19 FP \bal
\item[3\tre] 16\pl FP, 54\pl \pi{}+\tr
  \bdsc
  \item[3\kar] \kar-Werte, \ra~3\coe = Frage nach \hstp
  \item[3\coe] \coe-Werte
  \edsc
\edsc
\item[1\pik{}\sep2\coe; ?]~
  \bdsc
  \item[3\coe] 16\pl FP, 5/3\pl \pi{}+\co
  \item[4\coe] 12-14 FP, 4er-Anschluss (3er-Anschluss und \mini
    "uber 2\pik)
  \edsc
\edsc

\woreizung{%
1\pik & & 2\kar\\
3\tre & 5/4 in \pi{}+\tr, \pf & 3\kar & forcing\\
3\SA  & \cstop & 4\SA & quantitativ}

\subsection{Weiterreizung nach [1\of{}\sep2\SA{}] \conv{(Stenberg)}\label{stenberg}}

2\SA auf eine 1\of-Er"offnung verspricht 4er-Anschluss und
Vollspielwerte, oder besser.

Der Er"offner reizt daraufhin eine K"urze, falls vorhanden
(analog Splinter, ohne Sprung). Hat er
keine K"urze, so reizt er mit 11-13 4\tre, mit 14-15 3\SA und mit
16\pl 3\of (\conv{Principle of fast arrival}, je h"oher wir reizen,
desto \emph{schw"acher} sind wir).

Hat der Er"offner eine K"urze gezeigt, so fragt die n"achste Stufe (\rel) nach
Art der K"urze und nach Keycards.  Die erste Antwortstufe zeigt ein
Chicane, weiteres Relais ist \conv{KCB}.  Alle anderen Antwortstufen
zeigen ein Single und beantworten gleichzeitig \conv{KCB}.

Hat der Er"offner keine K"urze gezeigt, so ist die n"achste Stufe \conv{KCB}.

\bdsc
\item[1\of{}\sep2\SA; ?]~
  \bdsc
  \item[3\uf/\aof] K"urze in der gereizten Farbe \\
    \ra~\rel = \conv{Chicane-Assfrage}
    \bdsc
    \item[n"achste Stufe:] K"urze ist Chicane, \ra~\rel = \conv{KCB}
    \item[Rest:] Single, 14/30-Stufenantworten (\conv{KCB})
    \edsc
  \item[3\of] 16\pl, keine K"urze, \ra~\rel = \conv{KCB}
  \item[3\SA] 14-15, keine K"urze, \ra~4\tre = \conv{KCB}
  \item[4\tre] 11-13, keine K"urze, \ra~\rel = \conv{KCB}
  \edsc
\item[1\coe{}\sep2\SA; 3\tre{}\sep3\kar; ?]~

  \eo hat eine K"urze in \tr, \aw fragt nach Keycards
  \bdsc
  \item[3\coe] Chicane in \tre, \ra~3\pik = \conv{KCB}
  \item[3\pik] \tre-Single, eine oder vier Keycards
  \item[3\SA] \tre-Single, keine oder drei Keycards
  \item[4\tre] \tre-Single, zwei Keycards ohne \co-Dame
  \item[4\kar] \tre-Single, zwei Keycards mit \co-Dame
  \edsc
\edsc

\minisec{Beispielreizungen zu \conv{Stenberg}}

\exhand{432}{AB10852}{A3}{K2}%
{KB6}{K976}{K94}{A87}%
{
  1\coe & & 2\SA &\\
  4\tre & Minimum (11-13) & 4\coe
}

\exhand{4}{AB10852}{A1042}{K2}%
{876}{KD76}{KD3}{A87}%
{
  1\coe & & 2\SA &\\
  3\pik & \pi-K"urze (Single oder Chicane) & 3\SA & Frage\\
  4\pik & \pi-Single, zwei Keycards ohne Trumpf-Dame & 6\coe &
}

\exhand{KB1042}{DB753}{2}{K3}%
{AD53}{AK}{653}{A542}%
{
  1\pik & & 2\SA &\\
  3\kar & \ka-K"urze (Single oder Chicane) & 3\coe & Frage\\
  3\SA  & \ka-Single, eine oder vier Keycards & 6\pik &
}

\exhand{K752}{AD542}{-}{K742}%
{A8}{KB76}{D62}{ADB3}%
{
  1\coe & & 2\SA &\\
  3\kar & \ka-K"urze (Single oder Chicane) & 3\coe & Frage\\
  3\pik & \kar-Chicane & 3\SA & \conv{KCB}\\
  4\tre & eine oder vier Keycards & 4\kar & Frage nach \co-Dame\\
  4\pik & \co-Dame und \pi-K"onig vorhanden & 4\SA & Frage nach weiteren K"onigen\\
  5\tre & \tr-K"onig vorhanden & 7\coe\\
}

\newpage
%%%%%%%%%%%%%%%%%%%%%%%%%%%%%%%%%%%%%%%%%%%%%%%%%%%%%%%%%%%%%%%%%%%%%%%%%%%%%%
\section{Die 1\SA-Er"offnung}

Die Er"offnung 1\SA verspricht 15-17~FP und eine ausgeglichene
Verteilung.  5332-Verteilungen mit einer 5er-\ofa sind m"oglich.

\subsection{Antworten auf 1\SA-Er"offnung}
\bidins{%
2\tre & ab 0 & Stayman, verspricht 4er-\ofa\\
2\kar/\co & ab 0 & \xferto\coe/\pi, verspricht
5\pl{}er-Farbe\\
2\pik & ab 0 & \xferto\ufa, verspricht 6\pl{}er-\ufa oder 5/5\uf.\\
2\SA & 8-9 & \nat, \inv\\
3\,$x$ && 6\pl{}er-Farbe, \slamint{}\\
3\SA && zum Spielen\\
4\tre && \conv{\Index{Gerber}}-Assfrage (0/4, 1, 2, 3; \ra~\ref{gerber})\\
4\kar && 5/5\pl in den \ofa, kein \slamint \conv{(Wei"sberger)} \\
4\SA && quantitative Einladung zu 6\SA \\
5\SA && quantitative Einladung zu 6\SA/7\SA
}

\subsection{Reizung von \ofa-Zweif"arbern ("Ubersicht)}

Folgende Tabelle gibt eine "Ubersicht, wie beim Reizen von
\ofa-Zweif"arbern (5/4\pl-Verteilung) zu verfahren ist -- jeweils mit
schwachen, einladenden und starken H"anden.  Details siehe unten.

\begin{center}
\begin{tabular}[t]{|l|c|c|}
\hline
\textbf{St"arke} & \textbf{5/4} & \textbf{5/5}\\
\hline
\hline
schwach & Stayman & \conv{Transfer}\\
\hline
\inv & \multicolumn{2}{c|}{\conv{Transfer}}\\
\hline
\pf & Stayman, dann \conv{Smolen} & 4\kar\\
\hline
\pf (stark) & Stayman, dann \conv{Smolen} & \conv{Transfer}\\
\hline
Schlemm & \multicolumn{2}{c|}{Stayman, dann \conv{Smolen}}\\
\hline
\end{tabular}
\end{center}

\subsection{Schwache und einladende Stayman-Sequenzen}

Mit schwachen (0-7~FP) H"anden ist Stayman nur erlaubt, wenn
man jede Antwort des Er"offners (2\kar, 2\coe, 2\pik) passen kann -- also
mit 4-4-4-1, 4-4-5-0, 3-4-5-1 und 4-3-5-1 oder mit 5/4 in den \ofa.

\bdsc
\item[1\SA{}\sep2\tre; 2\kar{}\sep?]~
\bdsc
\item[2\coe] 4/4 oder 5/4 in \coe/\pi, mit 3/3~\ofa muss \eo passen.
\item[2\pik] 5/4~\pik/\co, \nf
\edsc
\edsc

Mit einladenden H"anden reizen wir eine 5/4-Verteilung in den \ofa
"uber \conv{Transfer}.

\subsection{Starke Stayman-Sequenzen}

\bdsc
\item[1\SA{}\sep2\tre; ?]~
\bdsc
\item[2\kar] keine 4er-\ofa
\bdsc
\item[3\coe] \pf{}\pl, \xferto\pi \conv{(\Index{Smolen} Transfer)};
  verspricht 4er-\co und 5er-\pi
  \begin{compactitem}
  \item falls 5er-\co: starkes \slamint{}
  \item mildes \slamint{} mit 5/5~\ofa wird "uber \conv{Transfer} gereizt
  \item ohne \slamint{} mit 5/5~\ofa wird [1\SA-4\kar;] gereizt
  \end{compactitem}
\item[3\pik] \pf{}\pl, \xferto\co \conv{(Smolen Transfer)};
  verspricht 5er-\pi und 4er-\co (siehe oben)
\edsc
\item[2\coe] 4er-\co (4er-\pi m"oglich)
  \bdsc
  \item[2\pik] \slamint{} in \co; keine K"urze
    \bdsc
    \item[2\SA] negativ
    \item[3\tre/\ka] \conv{Cuebid}
    \edsc
  \item[3\pik-4\kar] \conv{\Index{Splinter}}
  \item[4\SA] quantitativ mit 4er-\pi
  \edsc
\item[2\pik] 4er-\pi
  \bdsc
  \item[3\coe] \slamint{} in \pi; keine K"urze
    \bdsc
    \item[3\pik] positiv
    \item[3\SA] negativ
    \item[4\tre/\ka] \conv{Cuebid}
    \edsc
  \edsc
\edsc
\edsc

\subsection{Transfer-Sequenzen}

\bdsc
  \item[1\SA{}\sep2\kar] \xferto\co

  Ohne 4\pl{}er-\co f"uhrt der Er"offner den Transfer aus:
  \bdsc
    \item[2\coe] \xfer ausgef"uhrt
    \bdsc
      \item[2\pik] 5/4 \co/\pi, \inv
      \bdsc
        \item[2\SA] \mini, kein 3er-\co, kein 4er-\pi
        \bdsc
          \item[3\pik] 5/5 \co/\pi, zum Spielen
        \edsc
      \edsc
      \item[3\tre/\ka] \nat, \pf; ohne \slamint ist eine K"urze im
        Blatt
      \bdsc
        \item[3\kar] wenn \tr die \ufa war, dann \ka-Werte und keine
          \pi-Werte
        \bdsc
          \item[4\SA] quantitativ
          \bdsc
            \item[pass] \mini, Double-\co
            \item[5\coe] \mini mit \co-Fit
            \item[6\coe] \maxi mit \co-Fit
            \item[6\SA] \maxi, Double-\co
          \edsc
        \edsc
      \edsc
    \edsc
	\edsc
  Mit 4\pl{}er-\co bricht der Er"offner aus dem Transfer aus (\conv{Super Accept}).
  4\kar wiederholt dann den Transfer:
  \bdsc
    \item[2\pik] 4er-\co, \maxi, \pi-Double
    \item[2\SA] 4er-\co, \maxi, \ka-Double oder 4-3-3-3
    \item[3\tre] 4er-\co, \maxi, \tr-Double
    \item[3\coe] 4er-\co, \mini
  \edsc

  \item[1\SA{}\sep2\coe; 2\pik] \xfer ausgef"uhrt
  \bdsc
    \item[3\coe] 5/4 \pi/\co, \inv
    \item[4\coe] 5/5 \pi/\co, mildes \slamint
  \edsc
  \item[1\SA{}\sep2\pik] \xferto\ufa
  \bdsc
    \item[2\SA] Pr"aferenz f"ur \ka
    \bdsc
      \item[3\tre/\ka] schwache Variante
      \item[3\coe/\pi] 5/5\pl \tr/\ka, K"urze, \slamint
    \edsc
    \item[3\tre] Pr"aferenz f"ur \tr
  \edsc
\edsc

\subsection{Schlemm-Sequenzen}

\bdsc
  \item[1\SA{}\sep3\tre] \slamint in \tr
  \bdsc
    \item[3\kar/\co/\pi] Kontrolle in der Farbe, \maxi, \tr-Fit
    \item[3\SA] \mini, \tr-Double oder 3 kleine Karten
  \edsc
\edsc

\newpage
%%%%%%%%%%%%%%%%%%%%%%%%%%%%%%%%%%%%%%%%%%%%%%%%%%%%%%%%%%%%%%%%%%%%%%%%%%%%%%
\section{Die 2\SA-Er"offnung}

Die 2\SA-Er"offnung zeigt eine ausgeglichene Hand mit 20-21 FP, eine 5er-\ofa
kann enthalten sein.

\bdsc
\item[2\SA] 20-21 FP, \bal, 5er-\ofa m"oglich
  \bdsc
  \item[3\tre] Puppet-Stayman
  \item[3\kar] \xferto \co
  \item[3\coe] \xferto \pi
  \item[3\pik] \xferto \ufa
  \item[3\SA] \nat
  \item[4\tre] \Index{Gerber}-Assfrage 4/0, 1, 2, 3 (\ra \ref{gerber})
  \item[4\kar] 5/5 in \pi/\co, schwach
  \edsc
\edsc

\subsection{Transfer}

\begin{compactitem}
\item Die Annahme eines Transfers zeigt ein \emph{Double}.
\item 3\SA zeigt einen \emph{3er-Anschluss}.
\item Andere Gebote zeigen einen 3\good{}/4er-Anschluss und sind \emph{Cuebids}.
\end{compactitem}

\subsection{Puppet-Stayman}

Puppet-Stayman fragt nach 5er- und 4er-\ofa beim Er"offner. In den
Puppet-Sequenzen ist 4\SA die \conv{Gerber}-Assfrage (\ra \ref{gerber}),
solange kein Fit best"atigt wurde.

\bdsc
\item[2\SA{}\sep3\tre] Frage nach 5er- und 4er-\ofa
    (Alert: \emph{"`muss keine 4er-\of haben"'})
  \bdsc
  \item[3\kar] 4er-\ofa, keine 5er-\ofa \\
    Nach 3\kar reizt \aw die \emph{andere} \ofa ("ahnlich
    \conv{\Index{Smolen} Transfer}). 3\SA von \eo ist negativ (falsche \ofa),
    andere Gebote best"atigen den Fit.
    \bdsc
    \item[3\coe] 4er-\pi, 4er-\co m"oglich
      \bdsc
      \item[3\pik] \pi-Fit, danach Cuebids
      \item[3\SA] negativ, 4er-\co
        \bdsc
        \item[4\tre] Optional KCB
        \item[4\kar] Optional KCB
        \item[4\coe] \co-Fit, zum Spielen
        \item[4\pik] \co-Fit (!), KCB auf \co-Basis
        \item[4\SA] Gerber-Assfrage
        \edsc
      \edsc
    \item[3\pik] 4er-\co
      \bdsc
      \item[3\SA] negativ, 4er-\pi
        \bdsc
        \item[4\tre] Optional KCB
        \item[4\kar] Optional KCB
        \item[4\SA] Gerber-Assfrage
        \edsc
      \item[4\tre] \co-Fit, Cuebid
      \item[4\kar] \co-Fit, Cuebid
      \item[4\coe] \co-Fit, kein \ka- und \tr-Cuebid
      \edsc
    \item[3\SA] keine 4er-\ofa
    \item[4\tre] Optional KCB
    \item[4\kar] Optional KCB
    \item[4\SA] Gerber-Assfrage
    \edsc
  \item[3\coe] 5er-\co
    \bdsc
    \item[3\pik] \co-Fit (!), KCB auf \co-Basis
    \item[3\SA] zum Spielen
    \item[4\tre] Optional KCB
    \item[4\kar] Optional KCB
    \item[4\coe] zum Spielen
    \item[4\SA] Gerber-Assfrage
    \edsc
  \item[3\pik] 5er-\pi, weiter wie nach 3\coe
  \item[3\SA] keine 4er-, keine 5er-\ofa
  \edsc
\edsc

\newpage

%%%%%%%%%%%%%%%%%%%%%%%%%%%%%%%%%%%%%%%%%%%%%%%%%%%%%%%%%%%%%%%%%%%%%%%%%%%%%%
\section{Verhalten nach Zwischenreizung durch die
  Gegner\label{zwischenreizung}}

Nach allen \emph{pr"azisen} Er"offungen (1\SA, 2\SA, Weak Two, Sperransagen)
und einer Farbzwischenreizung des Gegners ist Kontra vom Partner \emph{immer}
Strafe.

\subsection{Allgemeines nach Farbzwischenreizung}
\begin{compactitem}
\item Das Reizen einer neuen Farbe auf der 1er-Stufe ist weiterhin
forcierend f"ur eine Runde.
%
\item Das Reizen einer neuen Farbe auf der 2er-Stufe ist nicht forcierend
falls die Zwischenreizung gest"ort hat, sonst forcierend:
\begin{description}
\item[1\coe{}\sep(1\pik){}\sep2\kar] ist forcierend da man auch
ohne die Zwischenreizung 2\kar gereizt h"atte.
\item[1\tre{}\sep(1\pik){}\sep2\coe] ist nicht forcierend da man ohne die
  Zwischenreizung 1\coe gereizt h"atte.
\end{description}
%
\item Eine neue Farbe auf der 3er-Stufe ist immer forcierend
  (z.~B. 1\pik{}\sep(2\kar){}\sep3\tre).
\item Nach gegnerischen \conv{Weak Jumps} ist das Reizen einer neuen Farbe immer
  forcierend.
\item \conv{Kontra} ist negativ bis 3\coe oder zeigt eine beliebige
  starke Hand f"ur die es in der augenblicklichen Situation kein Gebot
  gab.
\item Der \conv{"Uberruf} nach
  \begin{compactitem}
    \item \ufa-Er"offnung fragt nach Stopper,
    \item \ofa-Er"offnung zeigt eine partieforcierende Hand mit
      Anschluss in der \ofa.
    \end{compactitem}
\item \conv{Sprung-"Uberruf} oder "Uberruf auf 4er-Stufe ist \conv{Splinter}.
\item 1\SA und 3\SA sind nat"urlich (oder weiterhin konventionell).
\end{compactitem}

\subsection{Nach \uf-Er"offnung}
\begin{compactitem}
\item Es gilt weiterhin \conv{Inverted Minors}, solange alle dazu ben"otigten
	Gebote noch frei sind (z.B. 1\uf{}\sep(1\anybid)\sep2\SA).
\item Der \conv{"Uberruf} der gegnerischen Farbe fragt nach Stopper und ist
  Partieforcing (z.B. 1\uf{}\sep(1\anybid){}\sep2\anybid{} oder
  1\uf{}\sep(2\anybid)\sep3\anybid{}).
\end{compactitem}

Alle "ubrigen Gebote unterscheiden sich nicht von einer ungest"orten
Reizung (\conv{Inverted Minors}, \conv{Weak Jumps}, \conv{Fit Jumps}).

\minisec{Details}

\bdsc
\item[1\tre{}\sep(1\kar){}\sep?]~
\bdsc
\item[\kontra] beide \of \conv{(Negativkontra)}
\item[1\coe/\pi] mindestens 4er-L"ange
\item[2\tre/2\SA/3\tre] Inverted
\item[2\kar] \pf und Frage nach \ka-Stopper
\item[2\of/3\of] Weak Jump
\item[3\kar] Karo-\conv{\Index{Splinter}}
\edsc
\item[1\tre{}\sep(1\coe){}\sep?]~
\bdsc
\item[\kontra] genau 4er-\pi oder starke Hand mit \ka-L"ange
\item[1\pik] 5\pl{}er-\pi
\item[2\kar] \nf
\item[2\coe] \pf und Frage nach \co-Stopper
\item[3\coe] \co-\conv{Splinter}
\edsc
\item[1\tre{}\sep(1\pik){}\sep?]~
\bdsc
\item[\kontra] genau 4er-\co oder starke Hand mit \ka- oder \co-L"ange
\item[2\kar] \nf
\item[2\coe] 5\pl{}er-\co, \nf
\item[2\pik] \pf und Frage nach \pi-Stopper
\item[3\pik] \pi-\conv{Splinter}
\edsc
\edsc

\minisec{Beispiele}

\begin{description}
\item[1\kar{}\sep(\kontra){}\sep2\kar] zeigt 10\pl FP und 5\pl{}er
  \ka-Anschluss.
\item[1\tre{}\sep(1\pik){}\sep2\SA] zeigt 2-5 FP und 5\pl{}er
  \tr-Anschluss.
\item[1\kar{}\sep(2\tre){}\sep3\tre] ist partieforcierend und fragt nach
  \tr-Stopper.
\end{description}

Die Reizung nach 1\kar-Er"offnung ist analog.

\subsection{Nach \of-Er"offnung}
\begin{compactitem}
\item Alle \ofa-Hebungen sind schwach.  Falls \conv{Bergen-Hebungen}
  noch m"oglich sind (d.h. im Sprung), werden diese angewendet%
  \footnote{nach (2\tre) ist nur noch 3\kar m"oglich}.
\item 1\of{}\sep($x$){}\sep2\SA zeigt
  \begin{compactitem}
  \item wenn \conv{Bergen-Hebungen} noch m"oglich sind: eine einladende
    Hebung mit genau 3er-Anschluss,
  \item sonst eine einladende Hebung mit 3er- oder 4er-Anschluss.
  \end{compactitem}
\item 1\of{}\sep(\kontra){}\sep2\SA zeigt
  \begin{compactitem}
    \item eine einladende Hebung mit genau 3er-Anschluss oder
    \item eine partieforcierende Hebung.
  \end{compactitem}
Die anderen F"alle k"onnen mittels \conv{Bergen} gezeigt werden.
\item Der \conv{"Uberruf} der gegnerischen Farbe zeigt eine
  partieforcierende Hand mit Anschluss in der \ofa
  (1\of{}\sep(1\anybid){}\sep2\anybid{} oder 1\of{}\sep(2\anybid){}\sep3\anybid{}).
\end{compactitem}

\minisec{Details}

\bdsc
\item[1\coe{}\sep(1\pik){}\sep?]~
  \bdsc
  \item[\kontra] beide \ufa \conv{(Negativkontra)}
  \item[2\tre/\ka] 10\pl FP, 4\pl{}er-L"ange, forcierend
  \item[2\pik] \pf mit 3\pl{}er-Anschluss
  \item[2\SA] \inv mit \emph{genau} 3er-Anschluss
  \item[3\tre/\ka] weiterhin \conv{Bergen-Hebungen}
  \item[3\pik] beliebiger Single-\conv{Splinter} (wie in ungest"orter
    Reizung)
  \edsc
\item[1\coe{}\sep(2\kar){}\sep?]~
  \bdsc
  \item[\kontra] \conv{Negativ} oder stark mit \pik-L"ange
  \item[2\pik] \nf (denn das 1\pik-Gebot war nicht mehr frei)
  \item[2\SA] \inv mit 3er- oder 4er-Anschluss
  \item[3\tre] \nat, forcing
  \item[3\kar] \pf mit 3\pl{}er-Anschluss
  \edsc
\item[1\pik{}\sep(2\coe){}\sep?]~
  \bdsc
  \item[2\SA] \inv mit 3\pl{}er-Anschluss
  \item[3\coe] \pf mit 3\pl{}er-Anschluss
  \item[3\pik] schwache Hebung
  \item[3\SA] Chicane-Splinter (3\coe ist nicht mehr frei)
  \item[4\coe] \co-Splinter
  \edsc
\item[1\coe{}\sep(2\pik){}\sep?]~
  \bdsc
  \item[3\pik] \pf mit 3\pl{}er-Anschluss
  \item[3\SA] Chicane oder Single \pi (!)
  \item[4\pik] EKCB
  \edsc
\edsc

\minisec{Beispiele}

\begin{description}
\item[1\pik{}\sep(2\kar){}\sep2\pik] zeigt weiterhin 6-9 FP mit
  3er-Anschluss.
\item[1\coe{}\sep(1\pik){}\sep3\kar] ist weiterhin \conv{Bergen} und zeigt
  7-9 FP mit 4er-Anschluss.
\item[1\coe{}\sep(2\tre){}\sep3\kar] ist ebenfalls \conv{Bergen} und zeigt
  7-9 FP mit 4er-Anschluss. 9-11 FP wird "uber 2\SA gezeigt.
\item[1\pik{}\sep(X){}\sep2\SA] zeigt einen einladenden 3er-\pi-Anschluss oder
  eine Hand ab 12 FP mit 4er- oder besserem Anschluss (die w"are
  zu stark f"ur \conv{Bergen}).
\item[1\pik{}\sep(2\kar){}\sep2\SA] zeigt 3er- oder 4er-\pi-Anschluss und
  ist einladend (3\kar w"are partieforcierend; \conv{Bergen} kann
  nicht mehr gereizt werden um die einladende Hand mit 4er-\pi{}
  zu zeigen).
\item[1\coe{}\sep(1\pik){}\sep2\pik] zeigt eine partieforcierende Hand mit
  Coeur-Anschluss.
\end{description}

\subsection{Informationskontra vom Gegner}

Nach Info-\kontra vom Gegner gelten die gleichen Prinzipien.
\begin{compactitem}
\item Eine neue Farbe auf der ist 2er-Stufe nonforcing.
\item Weiterhin Inverted und Bergen.
\item 2\SA verspricht nach \ofa-Er"offnung eine mindestens einladende Hand mit
  3\pl{}er-Anschluss (\emph{kein} Stenberg).
\item Rekontra verspricht Punktemajorit"at und verneint einen Fit in der
  er"offneten Farbe.
\end{compactitem}

\subsection{Der Gegner reizt einen Zweif"arber \conv{(Unusual over
    Unusual)}}

Reizt der Gegner "uber unsere 1\anybid-Er"offnung einen Zweif"arber, bei
dem beide Farben bekannt sind, so verwenden wir die Konvention
\conv{Unusual over Unusual}. Das Reizen der niedrigeren bzw. h"oheren der
gegnerischen Farben steht indirekt f"ur die niedrigere bzw. h"ohere unserer Farben.
%
\begin{center}
\begin{tabular}[t]{|l|c|c|}
\hline
 & \textbf{Partnerfarbe} & \textbf{neue Farbe}\\
\hline
\hline
\textbf{direkt} & kompetitiv & stark \\
\hline
\textbf{indirekt} & \multicolumn{2}{c|}{einladend}\\
\hline
\end{tabular}
\end{center}

\begin{compactitem}
\item Indirekte Farbgebote und Hebungen sind \emph{einladend}.
\item Direktes Heben der Partner-Farbe ist schwach (man h"atte sonst 2\anybid
  gesagt).
\item Direktes Reizen der "`vierten"' Farbe ist stark (5\pl{}er-L"ange mit
  Partiewerten).
\item \conv{Kontra} zeigt Straf-Bereitschaft f"ur mindestens eine der
  beiden gegnerischen Farben.
\end{compactitem}

\minisec{Beispiele}

\bdsc
\item[1\coe{}\sep(2\SA{}*){}\sep?] (2\SA = beide \ufa)
\bdsc
\item[3\tre] \inv{}\pl mit \co-Anschluss
\item[3\kar] \inv mit 5\pl{}er-\pi
\item[3\coe] kompetitiv (ersetzt 2\co-Gebot)
\item[3\pik] \pf mit 5\pl{}er-\pi
\edsc

\item[1\pik{}\sep(2\SA{}*){}\sep?] (2\SA = beide \ufa)
\bdsc
\item[3\tre] \inv mit 5\pl{}er-\co
\item[3\kar] \inv{}\pl mit \pi-Anschluss
\item[3\coe] \pf mit 5\pl{}er-\co
\item[3\pik] kompetitiv (ersetzt 2\pik-Gebot)
\edsc

\item[1\coe{}\sep(2\coe{}*){}\sep?] (2\coe = \pi{}+\tr)
\bdsc
\item[2\pik] \inv{}\pl mit \co-Anschluss
\item[2\SA] \nat
\item[3\tre] \inv mit 5\pl{}er-\co
\item[3\kar] \pf mit 5\pl{}er-\ka
\item[3\coe] kompetitiv (ersetzt 2\coe-Gebot)
\edsc

\item[1\pik{}\sep(2\pik{}*){}\sep?] (2\pik = \co{}+\tr)
\bdsc
\item[2\SA] \nat
\item[3\tre] \inv mit 5\pl{}er-\ka
\item[3\kar] \pf mit 5\pl{}er-\ka
\item[3\coe] \inv\pl{} mit \pi-Anschluss
\item[3\pik] kompetitiv (ersetzt 2\pik-Gebot)
\edsc

\edsc

\subsection{Beide Gegner reizen eine Farbe}

Reizen beide Gegner eine Farbe, so ist der \conv{"Uberruf} einer der
Gegnerfarben \emph{nicht} die Frage nach Stopper sondern \emph{zeigt}
einen Stopper in der "uberrufenen Farbe:

\reizungmittext
{
  1\coe & 1\pik & 2\tre & 2\kar\\
  -- & -- & 2\pik{}\al{a} & --\\
  2\SA{}\al{b}
}
{ \smaller
  \al{a} zeigt \pstop \\
  \al{b} \nat, zeigt \kstop (3\kar w"are Frage nach \khstop)
}

\reizungmittext
{
  1\kar & 1\pik & \kontra & 2\tre\\
  2\pik{}\al{a} & -- & 2\SA{}\al{b}
}
{
  \al{a} zeigt \pstop \\
  \al{b} \nat, zeigt \tstop
}

\subsection{Nach 1\sa-Er"offnung}

\bdsc
\item[1\SA{}\sep(\kontra)] Gegner kontriert 1\SA
    \bdsc
    \item[\rekontra] \pupto 2\tre (schwach)
      \bdsc
      \item[pass] 5\pl{}er-\tr
      \item[2\kar] 4/4 \ofa (\ra~2\coe/\pi zum Spielen)
      \edsc
    \item[2\tre/\ka/\co] \xferto \ka/\co/\pi
    \item[pass] \pupto \rekontra (\bal, schwach oder stark)
      \bdsc
        \item[pass] stark, zum Spielen
        \item[Rest] schwach, zum Spielen
      \edsc
    \edsc

\item[1\SA{}\sep(2\anybid)] Gegner reizt eine Farbe \\
    \ra \conv{\Index{Lebensohl}} (\ref{lebensohl})

\item[1\SA{}\sep(p)\sep2\tre{}\sep(\kontra);] Gegner kontriert Stayman
    \bdsc
    \item[pass] verneint 4er-\ofa und 5er-\ufa
    \item[\rekontra] 5er-\tr
    \item[2\kar] 5er-\ka
    \item[2\of] 4er-\ofa
    \edsc
\item[1\SA{}\sep(p)\sep2\kar/\co{}\sep(\kontra);] Gegner kontriert \xferto \ofa
    \bdsc
    \item[pass] 2er-\ofa
    \item[2\coe/\pi] 3\pl{}er-\ofa
    \item[\rekontra] gutes 4er mit Werten in der Transferfarbe
    \edsc
\edsc

\newpage
%%%%%%%%%%%%%%%%%%%%%%%%%%%%%%%%%%%%%%%%%%%%%%%%%%%%%%%%%%%%%%%%%%%%%%%%%%%%%%
\section{Sperrer"offnungen auf 2er-Stufe}

\subsection{2\tre: Weak Two in \ka oder Benjamin}

Die 2\tre-Er"offnung zeigt entweder
\begin{compactitem}
\item ein \Index{\textbf{Weak Two}} in \ka,
\item ein \textbf{Semiforcing} in einer beliebigen Farbe
  (16\pl FP, 6\pl{}er-Farbe, 8 Spielstiche) oder
\item eine \textbf{ausgeglichene Hand} mit 22-23 oder 26-27~FP.
\end{compactitem}
Der Antwortende reizt in der Regel 2\kar als Relais, es sei denn dass
er gegen"uber einem Weak Two in \ka ein volles Spiel sieht, dann reizt
er 2\SA, was auch \ka-Fit impliziert. Neue Farben sind nat"urlich und forcing.

\bdsc
\item[2\tre{}\sep2\kar;] \rel
  \bdsc
  \item[2\coe] Semiforcing in \co
  \item[2\pik] Semiforcing in \pi
  \item[2\SA] 22-23 FP, \bal, 5er-\ofa m"oglich \\
    \ra weiter wie nach 2\SA-Er"offnung
  \item[3\tre] Semiforcing in \tr
  \item[3\kar] Semiforcing in \ka
  \item[3\SA] 26-27 FP, \bal, 5er-\ofa m"oglich \\
    \ra weiter wie nach 2\SA-Er"offnung
  \edsc
\edsc

\subsubsection{Ogust} \label{ogust}

Nach [2\tre{}\sep2\SA{}] benutzt der Er"offner die Ogust-Konvention, um sein Weak
Two n"aher zu beschreiben (Min Min Max Max). Danach kann der Antwortende mit der n"achsten Farbe
(unter Auslassung der Trumpffarbe und 3\SA, was zum Spielen ist) nach K"urze fragen, wobei eine \sa-Antwort des
Er"offners (meist 3\SA) eine K"urze in der Fragefarbe zeigt. Das Reizen der
Trumpffarbe verneint eine K"urze. \emph{Nach Beantworten der K"urzenfrage} ist
ist die n"achste Farbe (au"ser der Trumpffarbe auf Partiestufe) PKCA auf Basis
der Weak Two-Farbe.

\bdsc
\item[2\tre{}\sep2\SA;] Ogust\index{Ogust}
  \bdsc
  \item[3\tre] Weak Two in \ka, 6-8 FP, schlechte Farbe
    \bdsc
    \item[3\kar] zum Spielen
    \item[3\coe] K"urzenfrage
      \bdsc
      \item[3\pik] \pi-K"urze \ra 4\tre: PKCA
      \item[3\SA] \co-K"urze \ra 4\tre: PKCA
      \item[4\tre] \tr-K"urze \ra 4\kar: PKCA
      \item[4\kar] keine K"urze \ra 4\coe: PKCA
      \edsc
    \item[3\SA] zum Spielen
    \edsc
  \item[3\kar] 6-8 FP, gute Farbe (2 Topfiguren) \\
    weiter wie nach 3\tre
  \item[3\coe] 9-10 FP, schlechte Farbe
    \bdsc
    \item[3\pik] K"urzenfrage
      \bdsc
      \item[3\SA] \pi-K"urze \ra 4\tre: PKCA
      \item[4\tre] \tr-K"urze \ra 4\kar: PKCA
      \item[4\kar] keine K"urze \ra 4\coe: PKCA
      \item[4\coe] \co-K"urze \ra 4\pik: PKCA
      \edsc
    \item[3\SA] zum Spielen
    \edsc
  \item[3\pik] 9-10 FP, gute Farbe (2 Topfiguren)
    \bdsc
    \item[3\SA] zum Spielen
    \item[4\tre] K"urzenfrage (\emph{nicht} PKCA)
      \bdsc
      \item[4\SA] \tr-K"urze \ra 5\tre: PKCA\footnote{nicht besonders gl"ucklich}
      \edsc
    \edsc
  \item[3\SA] 3 Topfiguren: \suit{AKD}, weiter wie oben
  \edsc
\edsc

Alle anderen Gebote zeigen die starke Variante der Er"offnung.

\minisec{Zwischenreizung nach 2\SA-Relais}

Wenn der Gegner nach einem 2\SA-Relais, nach dem Stufenantworten des Er"offners
erforderlich sind, zwischenreizt, wird \conv{\Index{DOPI-ROPI}} angewendet:

\bdsc
\item[2\tre{}\sep2\SA;] ~
  \bdsc
  \item[(\kontra)] Gegner kontriert \ra \conv{ROPI}
    \bdsc
    \item[pass] erste Stufe
    \item[\rekontra] zweite Stufe
    \item[3\tre] dritte Stufe
    \item[\ldots]
    \edsc
  \item[(3\anybid)] Gegner reizt auf 3er-Stufe \ra \conv{DOPI}
    \bdsc
    \item[pass] erste Stufe
    \item[\kontra] zweite Stufe
    \item[3\,$y$] dritte Stufe
    \item[\ldots]
    \edsc
  \item[(4\anybid)] Gegner reizt auf 4er-Stufe
    \bdsc
    \item[pass] schlechtes Weak Two
    \item[\kontra] gutes Weak Two\footnote{vielleicht gibt es etwas Besseres, keine Ahnung}
    \edsc
  \edsc
\edsc

\subsection{2\kar: Weak Two in \co oder Benjamin}

Die 2\kar-Er"offnung zeigt entweder
\begin{compactitem}
\item ein \Index{\textbf{Weak Two}} in \co.
\item ein \textbf{Partieforcing} in einer beliebigen Farbe
  (18\pl FP, 6\pl{}er-Farbe, 9 Spielstiche) oder
\item eine \textbf{ausgeglichene Hand} mit 24-25 oder 28\pl~FP.
\end{compactitem}
Der Antwortende reizt in der Regel 2\coe als Relais, es sei denn dass
er gegen"uber einem Weak Two in \co ein volles Spiel sieht, dann reizt
er entweder 2\pik, was nat"urlich ist, oder 2\SA, was auch \co-Fit
impliziert.

\bdsc
\item[2\kar{}\sep2\coe;] \rel
  \bdsc
  \item[2\pik] \pf in \pi
  \item[2\SA] 24-25 FP, \bal, 5er-\ofa m"oglich \\
    \ra weiter wie nach 2\SA-Er"offnung
  \item[3\tre] \pf in \tr
  \item[3\kar] \pf in \ka
  \item[3\coe] \pf in \co
  \item[3\SA] 28\pl FP, \bal, 5er-\ofa m"oglich \\
    \ra weiter wie nach 2\SA-Er"offnung
  \edsc
\edsc

\minisec{Ogust}

Das Schema nach [2\kar{}\sep2\SA{}] funktioniert genau wie nach [2\tre{}\sep2\SA{}].

\bdsc
\item[2\coe{}\sep2\SA;] Ogust\index{Ogust}
  \bdsc
  \item[3\tre] Weak Two in \co, 6-8 FP, schlechte Farbe
    \bdsc
    \item[3\kar] K"urzenfrage
      \bdsc
      \item[3\coe] keine K"urze \ra 3\pik: PKCA
      \item[3\pik] \pi-K"urze \ra 4\tre: PKCA
      \item[3\SA] \ka-K"urze \ra 4\tre: PKCA
      \item[4\tre] \tr-K"urze \ra 4\kar: PKCA
      \edsc
    \item[3\coe] zum Spielen
    \item[3\SA] zum Spielen
    \edsc
  \item[3\kar] 6-8 FP, gute Farbe (2 Topfiguren)
    \bdsc
    \item[3\coe] zum Spielen
    \item[3\pik] K"urzenfrage
      \bdsc
      \item[3\SA] \pi-K"urze \ra 4\tre: PKCA
      \item[4\tre] \tr-K"urze \ra 4\kar: PKCA
      \item[4\kar] \ka-K"urze \ra 4\pik: PKCA
      \item[4\coe] keine K"urze \ra 4\pik: PKCA
      \edsc
    \edsc
  \item[3\coe] 9-10 FP, schlechte Farbe, weiter wie oben
  \item[3\pik] 9-10 FP, gute Farbe (2 Topfiguren)
    \bdsc
    \item[3\SA] zum Spielen
    \item[4\tre] K"urzenfrage (\emph{nicht} PKCA)
      \bdsc
      \item[4\SA] \tr-K"urze \ra 5\tre: PKCA
      \edsc
    \edsc
  \item[3\SA] 3 Topfiguren: \suit{AKD}, weiter wie oben
  \edsc
\edsc


\subsection{2\coe: Zweif"arber mit \co} \label{2coeur}

Die 2\coe-Er"offnung zeigt einen Zweif"arber mit einem 5er-\co und einer
weiteren 5er-Farbe im Bereich von 6-10 FP. In dritter Hand kann die Verteilung
auch 5/4 sein.

\bdsc
\item[2\coe] 5-10 FP, 5/5 \co/?
  \bdsc
  \item[2\pik] schwach, zum Spielen oder ausbessern
  \item[2\SA] Frage nach 2. Farbe und St"arke
  \item[3\coe] weitere Sperre
  \item[4\tre] PKCA auf \co-Basis (nur direkt)
  \edsc
\edsc

\minisec{Weiterreizung nach [2\coe{}\sep2\SA{}]}

Der Er"offner zeigt mit seinem R"uckgebot seine zweite Farbe und seine
Punktst"arke, nur bei einem \co/\tr-Zweif"arber kann er die Punktst"arke nicht
sofort mitteilen. Die n"achste freie Farbe fragt dann nach K"urze, bei \co/\tr
nach K"urze und Punktst"arke (Antworten "ahnlich der Ogust-Konvention: Min Min
Max Max).

F"ur die Beantwortung der K"urzenfrage gilt: niedrigste Stufe zeigt niedrigste
K"urze. Die n"achste Stufe des Antwortenden ist dann PKCA f"ur die niedrigere
Farbe, die "ubern"achste Stufe PKCA f"ur die h"ohere Farbe, nat"urlich unter
Auslassung von 3\SA und 4\coe. Beim \co/\pi-Zweif"arber wird mit den beiden
K"urzenfragen 4\tre und 4\kar gleich die Trumpffarbe festgelegt, um danach bei
PKCA eine Stufe zu sparen.

Im Folgenden zeigt \ra die PKCA-Gebote und deren Basis.

\bdsc
\item[2\coe{}\sep2\SA;] Frage nach St"arke und zweiter Farbe
  \bdsc
  \item[3\tre] zweite Farbe \tr, 6-10~FP
    \bdsc
    \item[3\kar] Frage nach St"arke und K"urze
      \bdsc
      \item[3\coe] 6-8 FP, \ka-K"urze \ra 3\pik: \tr, 4\tre: \co
      \item[3\pik] 6-8 FP, \pi-K"urze \ra 4\tre: \tr, 4\kar: \co
      \item[3\SA] 9-10 FP, \ka-K"urze \ra 4\tre: \tr, 4\kar: \co
      \item[4\tre] 9-10 FP, \pik-K"urze \ra 4\kar: \tr, 4\pik: \co
      \edsc
    \edsc
  \item[3\kar] zweite Farbe \ka, 6-8~FP
    \bdsc
    \item[3\coe] zum Spielen
    \item[3\pik] Frage nach K"urze
      \bdsc
      \item[3\SA] \tr-K"urze \ra 4\tre: \ka, 4\kar: \co
      \item[4\tre] \pi-K"urze \ra 4\kar: \ka, 4\pik: \co
      \edsc
    \edsc
  \item[3\coe] zweite Farbe \pi, 6-8~FP
    \bdsc
    \item[3\pik] zum Spielen (!)
    \item[4\tre] \co-Fit und Frage nach K"urze
      \bdsc
      \item[4\kar] \tr-K"urze \ra 4\pik: \co
      \item[4\coe] \ka-K"urze \ra 4\pik: \co
      \edsc
    \item[4\kar] \pi-Fit und Frage nach K"urze
      \bdsc
      \item[4\coe] \tr-K"urze \ra 4\SA: \pi
      \item[4\pik] \ka-K"urze \ra 4\SA: \pi
      \edsc
    \edsc
  \item[3\pik] zweite Farbe \pi, 9-10~FP; weiter wie nach 3\coe
  \item[3\SA] zweite Farbe \ka, 9-10~FP
    \bdsc
    \item[4\tre] Frage nach K"urze
      \bdsc
      \item[4\kar] \tr-K"urze \ra 4\pik: \ka, 4\SA: \co
      \item[4\coe] \pi-K"urze \ra 4\pik: \ka, 4\SA: \co \emph{(Vorsicht!)}
      \edsc
    \edsc
  \edsc
\edsc

\minisec{Beispiele}

\exhand{3}{DB865}{54}{K10975}
{9876}{AK}{AK}{AD86}{%
  2\coe && 2\SA \\
  3\tre & 5/5 \tr/\co & 3\kar & Frage nach St"arke und K"urze\\
  3\coe & \mini, \pi-K"urze & 4\tre & PKCA auf \tr-Basis\\
  4\kar & 1 Ass ohne \tr-Dame & 6\tre\\
}

\exhand{32}{KB874}{KD872}{5}
{AD54}{106}{A54}{ADB2}{%
  2\coe && 2\SA &\\
  3\SA  & 5/5 \co/\ka, \maxi & pass &\\
}

\exhand{KB986}{DB764}{54}{2}
{AD76}{2}{AKD106}{A54}{%
  2\coe & & 2\SA  & \\
  3\coe & 5/5 \co/\pi, \mini & 4\kar & \pi-Fit, K"urzenfrage \\
  4\coe & \tr-K"urze & 4\SA  & PKCA\\
  5\tre & 1 Ass ohne \pi-Dame & 6\pik
}

\subsection{2\pik: Weak Two in \pi}

Die 2\pik-Er"offnung zeigt ein \Index{Weak Two} in \pi, d.h. eine 6er-Farbe mit 6-10 FP
ohne 4er-\co (4er-\ufa m"oglich).

\bdsc
\item[2\pik] 6-10 FP, 6er-\pi
  \bdsc
\item[2\SA] 15/16\pl FP, \pi-Fit, Frage nach St"arke und Farbqualit"at
  \conv{(Ogust)}
  \item[3\pik] weitere Sperre
  \item[4\tre] PKCA auf \pi-Basis
  \item[4\pik] weitere Sperre oder konstruktiv
  \edsc
\edsc

Das Reizen einer neuen Farbe zeigt 6\pl (5 gute) Karten und ist
selbstverst"andlich forcing. Der Er"offner hebt mit 3er-Anschluss oder mit
Double-Anschluss und einer K"urze.

\minisec{Ogust}

Das Schema nach [2\pik{}\sep2\SA{}] funktioniert genau wie nach [2\tre{}\sep2\SA{}] und
[2\kar{}\sep2\SA{}].

\bdsc
\item[2\pik{}\sep2\SA;] Ogust\index{Ogust}
  \bdsc
  \item[3\tre] 6-8 FP, schlechte Farbe
    \bdsc
    \item[3\kar] K"urzenfrage
      \bdsc
      \item[3\coe] \co-K"urze \ra 4\tre: PKCA
      \item[3\pik] keine K"urze \ra 4\tre: PKCA
      \item[3\SA] \ka-K"urze \ra 4\tre: PKCA
      \item[4\tre] \tr-K"urze \ra 4\kar: PKCA
      \edsc
    \item[3\pik] zum Spielen
    \item[3\SA] zum Spielen
    \edsc
  \item[3\kar] 6-8 FP, gute Farbe (2 Topfiguren)
    \bdsc
    \item[3\coe] K"urzenfrage
      \bdsc
      \item[3\pik] keine -K"urze \ra 4\tre: PKCA
      \item[3\SA] \co-K"urze \ra 4\tre: PKCA
      \item[4\tre] \tr-K"urze \ra 4\kar: PKCA
      \item[4\kar] \ka-K"urze \ra 4\coe: PKCA
      \edsc
    \item[3\pik] zum Spielen
    \edsc
  \item[3\coe] 9-10 FP, schlechte Farbe
    \bdsc
    \item[3\pik] zum Spielen
    \item[3\SA] zum Spielen
    \item[4\tre] K"urzenfrage (\emph{nicht} PKCA)
      \bdsc
      \item[4\SA] \tr-K"urze \ra 5\tre: PKCA
      \edsc
    \edsc
  \item[3\pik] 9-10 FP, gute Farbe (2 Topfiguren), weiter wie oben
  \item[3\SA] 3 Topfiguren: \suit{AKD}, weiter wie oben
  \edsc
\edsc

\minisec{Beispiele}

\exhand{KD9874}{862}{B74}{2}
{A3}{A973}{A92}{A986}{%
  2\pik && 2\SA&\\
  3\kar & \mini, gute Farbe & 3\SA
}

\exhand{K109876}{862}{A74}{2}
{AD5}{AKD97}{2}{A873}{%
  2\pik & & 2\SA\\
  3\tre & \mini, schlechte Farbe & 3\kar & Frage nach K"urze\\
  4\tre & \tr-K"urze & 4\kar & PKCA\\
  5\tre & 2 Asse ohne \pi-Dame & 7\pik
}

\exhand{KD10874}{A62}{B74}{2}
{AB3}{KDB4}{AK2}{986}{%
  2\pik & & 2\SA\\
  3\pik & \maxi, gute Farbe & 4\tre & Frage nach K"urze\\
  4\SA  & \tr-K"urze & 5\tre & PKCA\\
  6\tre & 2 Asse mit \pi-Dame & 6\pik
}

\newpage

%%%%%%%%%%%%%%%%%%%%%%%%%%%%%%%%%%%%%%%%%%%%%%%%%%%%%%%%%%%%%%%%%%%%%%%%%%%%%%
\section{Sperrer"offnungen h"oherer Stufe}

\subsection{3 in Farbe}

Die Farber"offnung auf 3er-Stufe zeigt eine 7\pl{}er-L"ange und 6-10 FP.
In der Weiterreizung ist eine neue Farbe auf der 3er-Stufe nat"urlich und partieforcierend. Eine
neue Farbe auf der 4er-Stufe (au"ser 4\tre) best"atigt den Fit und ist Cuebid.
4\tre ist PKCA. 4\of auf eine 3\uf-Er"offnung ist zum Spielen. Hebung der
\ofa ins Vollspiel ist entweder konstruktiv oder weitere Sperre.

\minisec{Bietsequenzen nach 3\tre-Er"offnung}

\woreizung{
  3\tre & & 3\coe & \\
  3\pik & Werte auf dem Weg zu 3\SA oder vorverlegtes Cuebid & 3\SA & \aw will 3\SA spielen \\
  4\coe & 3\pik war vorverlegtes Cuebid
}

\woreizung{
  3\tre & & 3\coe & \\
  3\SA & kein \co-Fit, kein Single in Nebenfarbe \\
  4\tre & Cuebid, 3\pl{}er-\co \\
  4\coe & 2\pl{}er-\co
}

Nach anderen Sperransagen sind die Sequenzen entsprechend.

\subsection{3\SA: Gambling}\index{Gambling}

Die 3\SA-Er"offnung zeigt eine stehende 7er-\ufa ohne Werte in den
Nebenfarben (\conv{Gambling}).

\bdsc
\item[3\SA] \conv{Gambling}
  \bdsc
  \item[pass] Stopper in restlichen Farben und 2\pl{}er-Anschluss in Partners \ufa
  \item[4\tre] will 4\tre oder 4\kar spielen
  \item[4\kar] Frage nach K"urze
    \bdsc
    \item[4\coe] \co-K"urze
    \item[4\pik] \pi-K"urze
    \item[4\SA] keine K"urze (2-2-2-7 oder 2-2-7-2)
    \item[5\tre] \ka-K"urze (!)
    \item[5\kar] \tr-K"urze (!)
    \edsc
  \item[4\of] zum Spielen
  \item[5\tre] will 5\tre oder 5\kar spielen
  \item[5\kar] will 5\kar oder 6\tre spielen
  \item[6\tre] will 6\tre oder 6\kar spielen
  \edsc
\edsc

\subsection{4\uf: Namyats}\index{Namyats}

Die 4\uf-Er"offnung zeigt eine stehende 7er-\ofa (\tr{}\ra{}\co, \ka{}\ra{}\pi) mit
einer Nebenfarb-Kontrolle (K"onig/Ass). Partner reizt entweder 4 in der \ofa
oder \emph{Erstrunden}kontrollen, wenn er Schlemminteresse hat.

\newpage
%%%%%%%%%%%%%%%%%%%%%%%%%%%%%%%%%%%%%%%%%%%%%%%%%%%%%%%%%%%%%%%%%%%%%%%%%%%%%%
\section{Die Gegenreizung\label{gegenreizung}}

\subsection{Informationskontra}
\bdsc
\item[(1\anybid)\sep\kontra{}\sep{}(p)\sep{}?]~
  \bdsc
  \item[1\hspace{\cardskip}$y$] (ohne Sprung) 0-7 FP
  \item[2\anybid] ("Uberruf der Gegnerfarbe)
    \begin{compactitem}
      \item 11\pl FP, beliebige Verteilung oder
      \item beide \ofa, 8-10 FP
    \end{compactitem}
  \item[2\of] (einfacher Sprung) 4er-Farbe, 8-10 FP
  \item[3\of] (doppelter Sprung) 5er-Farbe, 8-10 FP
  \edsc
\edsc

Nach [(1\of){}\sep\kontra{}\sep(2\of)] ist 2\SA \Index{\conv{Lebensohl}}, ebenfalls
nach \Index{Weak Two}-Er"offnung: [(2\of){}\sep\kontra{}\sep(p)].

\subsection{Farbgegenreizung}

Antworten:
\begin{compactitem}
\item Alle Hebungen sind schwach.
\item Neue Farbe ist nicht forcierend (8-12 FP).
\item Neue Farbe im Sprung ist \conv{Fit-Sprung}.
\item Der "Uberruf zeigt starke H"ande:
  \begin{compactitem}
  \item eine mindestens einladende Hand mit Fit oder
  \item einen Einf"arber mit mehr als 12 FP oder
  \item eine starke \sa-Hand ohne Stopper.
  \end{compactitem}
\end{compactitem}

\subsection{Weak Jumps}

Spr"unge in die 2er-Stufe zeigen 6-10 FP und eine 6er-Farbe oder eine sehr gute
5er-Farbe. 2\SA vom \eo ist dann \conv{\Index{Ogust}} (\ra~\ref{ogust}).
Spr"unge in die 3er-Stufe zeigen eine 7\pl{}er-Farbe und eine punktschwache
Hand.

\subsection{Michaels Pr"azis}

\conv{Michaels Pr"azis} zeigt einen Zweif"arber in zwei der drei
nichtgereizten Farben.  Es \emph{muss} mindestens eine 5/5-Verteilung
vorhanden sein; die Werte sollen sich in den langen Farben befinden.
%
\bdsc
\item[(1\uf){}\sep2\kar] \co und \pi
\item[(1\uf){}\sep2\SA] \co und \aufa \conv{(Unusual Notrump)}
\item[(1\of){}\sep2\of] \aofa und \tr
\item[(1\of){}\sep2\SA] \tr und \ka \conv{(Unusual Notrump)}
\item[(1\of){}\sep3\tre] \aofa und \ka
\edsc

Nach \ofa-Er"offnung gibt es keinen schwachen Sprung in \tr! Eine
vorbereitende 1\tre-Er"offnung wird als nat"urlich behandelt.
Nach \ufa-Er"offnung wird der Zweif"arber mit \pi und \aufa nat"urlich gereizt.

\subsection{Die 1\SA-Gegenreizung}

Die 1\SA-Gegenreizung in zweiter Position zeigt 15-18 FP (in vierter Position
11-14 FP) und einen
Stopper in der er"offneten Farbe. Weiterreizung wie nach
1\SA-Er"offnung, d.h. Stayman und \conv{Transfers} wenn der
rechte Gegner passt, ansonsten \Index{\conv{Lebensohl}}.

Hat der Gegner eine \ofa er"offnet, so ist der Transfer \emph{in} diese \ofa
Stayman. 2\tre ist dann ebenfalls ein Transfer.

\subsection{Gegenreizung gegen 1\SA (\Index{Multi-Landy})}

\bdsc
\item[(1\SA){}\sep?] ~
 \bdsc
 \item[\kontra] abh"angig von der St"arke der Er"offnung:
   \bdsc
     \item[starker \sa] 5\pl{}er-\ufa und 4er \ofa
     \item[schwacher \sa] mindestens gleiche St"arke
   \edsc
 \item[2\tre] 5\pl/4\pl in \ofa
  \bdsc
  \item[2\kar] zeigt gleiche L"ange in den beiden \ofa; "Uberrufer soll
    seine l"angere \ofa reizen.
  \edsc
 \item[2\kar] Einf"arber in einer \ofa
 \item[2\of] 5\pl{}er \ofa und 4er \ufa
 \item[2\SA] 5/5\pl in \ufa
 \item[3\uf] Einf"arber in einer \ufa
 \edsc
\edsc

\subsection{Crash gegen starke 1\tre-Er"offnung}

\conv{Crash} zeigt einen Zweif"arber. Ziel ist neben der St"orung der
gegnerischen Reizung auch, Informationen f"ur das sp"atere Gegenspiel
zu "ubermitteln.  Die Zwischenreizung kann sehr schwach sein.  Mit
konstruktiven H"anden sollte \conv{Crash} nicht gereizt werden.

\bdsc
\item[(1\tre)] 16\pl FP, k"unstlich, beliebige Verteilung
\bdsc
\item[\kontra] \textbf{C}olour -- gleichfarbige (\tr/\pi oder \ka/\co)
\item[1\kar] \textbf{Ra}nk -- gleichrangige (\tr/\ka oder \co/\pi)
\item[1\of] \nat
\item[1\SA] \textbf{Sh}ape -- gleichf"ormige (\tr/\co oder \ka/\pi)
\item[2\uf] \nat
\edsc
\edsc

\newpage
%%%%%%%%%%%%%%%%%%%%%%%%%%%%%%%%%%%%%%%%%%%%%%%%%%%%%%%%%%%%%%%%%%%%%%%%%%%%%%
\section{Konventionen zum System}

%
%%%%%%%%%%%%%%%%%%%%%%%%%%%%% Sequenzen nach 1SA-Rebid (Puppet) %%%%%%%%%%%%%%%
%
\subsection{Relaistransfer nach 1\SA-R"uckgebot} \label{1sarebid}

Neben den nat"urlichen Antworten benutzen wir die beiden Puppet-Gebote 2\tre
und 2\SA, um m"oglichst viele Verteilungen zeigen zu k"onnen.

\minisec{Regeln}
\begin{compactitem}
\item schwache H"ande direkt (Ausnahme: \ufa-Canap\'e)
\item einladende H"ande "uber [2\tre; 2\kar{}] (Ausnahme:
  Karo-Canap\'e)
\item starke H"ande "uber [2\SA; 3\tre{}] oder direkt (mehrere Ausnahmen)
\end{compactitem}

Nicht exakt reizen lassen sich nur
\begin{compactitem}
\item mindestens einladende Zweif"arber mit 5er-\ofa und 4er-\ufa sowie
\item genau einladende Zweif"arber mit 4er-\ofa\ und 5er-\tr.
\end{compactitem}

\minisec{"Ubersicht}
Die Reizung beginnt jeweils [1\anybid{}\sep1\of; 1\SA{}\sep?].

\begin{minipage}{\columnwidth}
{\relsize{-2}%
\begin{tabularx}{\columnwidth}{|c|c|l|l|Y|}
\hline
\multicolumn{2}{|c|}{\textbf{Haltung}} & \multicolumn{3}{c|}{\textbf{St"arke}}\\
\hline
\emph{OF} & \emph{NF} & \emph{schwach} & \emph{einladend} & \emph{partieforcierend}\\
\hline
\hline
4\of  & \bal  & pass  & 2\tre{}\leadto2\SA & 3\SA\\
\hline
4\of  & 5\tre & 2\SA{}\leadto{}p &
                \emph{nicht zeigb.} &
                nach 1\tre-E"o.:\par 2\SA{}\leadto3\kar{}\\
4\of  & 5\kar & 2\tre{}\leadto{}p &
                2\kar{}\footnote{genau einladend, mit \pf \conv{Walsh} reizen} &
                nach 1\kar-E"o.:\par 2\SA{}\leadto3\kar{}\\
\hline
5\of  & --    & 2\of  &
                2\tre{}\leadto2\of{}\footnote{mindestens einladend} &
                2\tre{}\leadto3\SA\\
5\of  & 5\anybid & 2\of & 2\tre{}\leadto3\anybid & 3\anybid\\
5\pik & 4\coe & 2\coe & 2\tre{}\leadto2\coe & 2\SA{}\leadto3\coe\\
\hline
6\of  & --    & 2\of  & 2\tre{}\leadto3\of & 3\of{}\footnote{leichtes Schlemm-Interesse}\\
      & Single  &       &             & 4\anybid\\
      & Chicane  &       &             & 2\tre{}\leadto4\anybid\\
\hline
\end{tabularx}\\[1ex]
\centerline{\emph{OF = Oberfarbe des Antwortenden, NF = Nebenfarbe}}%
}%
\end{minipage}

\minisec{Bietsequenzen}
\bdsc
\item[1\tre{}\sep1\pik; 1\SA{}\sep?] ~

  Direkt zeigt man:
  \begin{compactitem}%
  \item[1] einladend, 5\pl{}er-\ka und 4er-\ofa
  \item[2] schwach, 5er-\pi, evtl. 4\pl{}er \co
  \item[3] \pf, 5/5
  \item[4] \slamint, 6er-\ofa
  \item[5] starker \ofa-Einf"arber mit Single
  \end{compactitem}

  \bdsc
  \item[2\tre] \pupto2\kar{} -- siehe unten
  \item[2\kar] 4er-\pi, 5\pl{}er-\ka, genau \inv (sonst \conv{Walsh}) (1)
  \item[2\coe] 5/4 \pi{}+\co, zum Spielen oder Ausbessern (2)
  \item[2\pik] 5er-\pi, schwach (2)
  \item[2\SA] \pupto3\tre{} -- siehe unten
  \item[3\anybid] 5/5 \pi{}+\any{}\footnote{beliebige Farbe au"ser \pi}, \pf (3)
  \item[3\pik] 6er-\pi, leichtes \slamint{} (4)
  \item[4\anybid] 6\pl{}er-\pi, Single \any \conv{(Autosplinter)} (5)
  \edsc

\item[1\tre{}\sep1\pik; 1\SA{}\sep2\tre; 2\kar{}\sep?]~

  \begin{compactitem}
  \item[1] schwach, 4/5\pl \ofa/\ka
  \item[2] mindestens \inv, 5er-\ofa, evtl. 4er-\ufa
  \item[3] genau \inv
    \begin{compactitem}
    \item[a] \bal, 4er-\ofa
    \item[b] 5/4-Zweif"arber mit beiden \ofa
    \item[c] 5/5-Zweif"arber
    \item[d] Einf"arber in einer \ofa
    \end{compactitem}
  \item[4] starker \ofa-Einf"arber mit Chicane
  \end{compactitem}

  \bdsc
  \item[pass] schwach (1)
  \item[2\coe] 5/4 \pi{}+\co, \inv (3b)
  \item[2\pik] 5er-\pi, mind. \inv (2)
  \item[2\SA] \bal, \inv, 4er-\pi (3a)
  \item[3\anybid] 5/5, \inv (3c)
  \item[3\pik] 6er-\pi, \inv (3d)
  \item[3\SA] \nat, 5er-\pi (2)
  \item[4\anybid] 6\pl{}er-\pi, Chicane \any \conv{(Autosplinter\index{Splinter})} (4)
  \edsc

\item[1\tre{}\sep1\pik; 1\SA{}\sep2\SA; 3\tre{}\sep?]~

  \begin{compactitem}
  \item[1] schwach, 4/5\pl \ofa/\tr
  \item[2] stark:
    \begin{compactitem}
    \item[a] 4/5 \ofa/\tr nach 1\tre-Er"offnung
    \item[b] 4/5 \ofa/\ka nach 1\kar-Er"offnung
    \item[c] 5/4 \pi/\co nach 1\pik-Antwort
    \end{compactitem}
  \end{compactitem}

  \bdsc
  \item[pass] schwach, 4/5\pl \pi{}+\tr (1)
  \item[3\kar] stark, 4/5 \pi{}+\tr (nach 1\tre-Er"offnung, 2a) \\
    stark, 4/5 \pi{}+\ka (nach 1\kar-Er"offnung, 2b)
  \item[3\coe] stark, 5/4 \pi{}+\co (2c)
  \edsc
\edsc

\subsection{Bietsequenzen nach Revers-Reizung}

\conv{Revers}-Reizungen zeigen 16\good{}\pl~FP und sind selbstforcierend f"ur den
Er"offner.

\bdsc
  \item[1\tre{}\sep1\pik; 2\kar{}\sep{}?]~

    \bdsc
    \item[2\coe] \conv{VFF}, sagt hier aber nichts "uber die St"arke
      aus, da 2\kar selbstforcierend war.  \emph{Verneint} in diesem Fall
      5er-\pi, da man \pi h"atte wiederholen k"onnen ohne
      dass der Er"offner passen darf.
      \bdsc
        \item[2\SA] \mini, \cstop, \nf
        \item[3\tre] \mini, \nf
        \item[3\kar] \maxi, \pf
        \item[3\SA] \maxi, \cstop
      \edsc
    \item[2\pik] 5er-Farbe, schwach oder stark
      \bdsc
        \item[2\SA] \mini, \cstop, kein 3er-\pi
        \item[3\tre] \mini, kein 3er-\pi
        \item[3\kar] \maxi, kein 3er-\pi
        \item[3\coe] \conv{VFF}, Frage nach \cstop
        \item[3\pik] \mini mit 3er-\pi
        \item[3\SA] \maxi, \cstop, kein 3er-\pi
      \edsc
    \item[2\SA] \pupto3\tre \conv{(\Index{Ingberman})}
      \bdsc
      \item[pass/3\kar] zum Spielen
      \item[3\SA] 8-9 FP, \nat
      \item[4\SA] 12\pl FP, quantitative Schlemmeinladung
      \edsc
    \item[3\uf] \nat, \pf
    \item[3\SA] 10-11 FP, \cstop
    \edsc
\edsc

\notebox{\textbf{Ingberman} (frz. \emph{Moderateur}): Nach einer Revers-Reizung
fordert 2\SA ein passbares 3\tre-Gebot an,
um auf der 3er-Stufe ein Abschlussgebot abgeben zu k"onnen
("ahnlich \conv{Lebensohl}).
Der Antwortende sollte mit Minimum (6-7) diese Konvention
anwenden.
Mit Zusatzst"arke (18\good{}\pl) muss der Partner das 3\tre{}-\rel
"uberspringen.
}

\subsection{Long Suit Trial Bids}

Nach [1\of{}\sep2\of;] ist 2\SA ein allgemeines Trial Bid was zu 3 oder 4 in
\ofa einl"adt. Der Er"offner muss nicht ausgeglichen sein, er hat lediglich keine
unterst"utzungbed"urftige Farbe. Der Antwortende soll mit Minimum Sign Off geben und mit
Maximum das Vollspiel ansagen. Mit massierten Werten in einer Farbe reizt der
Antwortende diese Farbe auf 3er-Stufe.

Bei St"orung durch die Gegner ersetzt die niedrigste freie Farbe, sofern es
noch eine gibt, das 2\SA-Trial Bid. Andernfalls ist \kontra das Trial Bid.

\minisec{Beispiele}
\begin{description}
\item[1\tre{}\sep1\coe;~1\pik{}\sep2\pik;~2\SA]~

  allgemeines Trial Bid
\item[1\tre{}\sep1\pik;~2\pik{}\sep2\SA]~

  allgemeines Trial Bid, auch als Schlemmvorbereiung nutzbar
\item[1\pik{}\sep(2\tre)\sep2\pik{}\sep(3\tre);~?]~
  \begin{description}
    \item[pass] 5er-\pi, Minimum
    \item[\kontra] Strafe
    \item[3\kar] allgemeines Trial Bid (die n"achste freie Stufe
      ersetzt 2\SA)
    \item[3\coe] normales Trial Bid
    \item[3\pik] kompetitiv
  \end{description}
\item[1\pik{}\sep(2\coe)\sep2\pik{}\sep(3\coe);~\kontra]~

  allgemeines Trial Bid (Competitive \kontra/Full Value \kontra;
  Gegnerfarbe direkt unter uns)
\end{description}

\subsection{Dritte Farbe Forcing (DFF)} \label{dff}

Nach Farbwiederholung des Er"offners ist die dritte, vom Antwortenden gereizte,
Farbe DFF. Nach \ofa-Er"offnung soll der Er"offner vorrangig 3er-Anschluss in
der anderen \ofa des Antwortenden zeigen, je nach St"arke billig oder im
Sprung. Ist die dritte Farbe eine \ofa, so zeigt der Er"offner ebenfalls
Anschluss. Die dritte Farbe zeigt auch Werte, in der Absicht einen \sa-Kontrakt
anzusteuern. Der Er"offner bietet \sa, wenn er die vierte Farbe stoppt.

Die dritte Farbe auf der 2er-Stufe gereizt zeigt eine mindestens einladende
Hand, auf der 3er-Stufe eine partieforcierende.

\minisec{Beispiel}

\woreizung{
  2\tre & Semiforcing & 2\kar & \rel \\
  2\coe & \nat & 2\tre & \nat \\
  3\coe & \nat & 3\pik & DFF, zeigt \pi-Werte
}

\subsection{Vierte Farbe Forcing (VFF)} \label{vff}

Die vierte Farbe ist fast immer k"unstlich (im Gegensatz zur dritten Farbe bei
\conv{DFF}) und zeigt L"ange und St"arke, die in der bisherigen Reizung noch
nicht bekannt ist. Der Partner soll Anschluss in der \ofa zeigen, \sa mit
Stopper in der vierten Farbe reizen oder seine Hand weiter beschreiben.

Man zeigt eine starke Hand mit Trumpunterst"utzung, wenn man die vierte
Farbe reizt und anschlie"send die Farbe des Partners hebt. Wird die vierte
Farbe im Sprung gereizt, ist dies nat"urlich und zeigt einen partieforcierenden
5/5-Zweif"arber.

Manchmal fragt die vierte Farbe auch nur nach Halbstopper, n"amlich dann, wenn
der Partner einen Vollstopper in der vierten Farbe bereits verneint hat.
Halbstopper sind: \suit{Kx}, \suit{Dxx}, \suit{Bxx}, \suit{10xxx}.

Die vierte Farbe auf der 2er-Stufe gereizt ist mindestens einladend, auf der
3er-Stufe partieforcierend.
Priorit"aten des Antwortenden:

\minisec{1. 3er-Anschluss in der \ofa des Partners zeigen}
\begin{description}
\item[1\coe{}\sep1\pik;~2\tre{}\sep2\kar;~?]~
  \begin{description}
    \item[2\pik] 3er-\pi, Minimum
    \item[3\pik] 3er-\pi mit Zusatzwerten
  \end{description}
\end{description}

\minisec{2. \sa mit Stopper in der vierten Farbe reizen}
\begin{description}
\item[1\coe{}\sep1\pik;~2\tre{}\sep2\kar;~?]~
  \begin{description}
    \item[2\SA] 12-14 FP, kein 3er-\pi, \ka-\stp
    \item[3\SA] 15-16 FP, kein 3er-\pi, \ka-\stp
    \item[4\SA] \nat{} (!), 17-18 FP, kein 3er-\pi, \ka-\stp
  \end{description}
\end{description}

\minisec{3. Verteilung zeigen}

\handwithdesc{KD94}{AB42}{2}{AKB2}{
Nach [1\tre{}\sep1\kar; 1\coe{}\sep1\pik;] darf der Er"offner mit
nicht in 4\pik springen, da 1\pik VFF war
und nicht nat"urlich gewesen sein muss (wenn 1\pik nat"urlich war,
zeigt dies eine Er"offnung wegen \conv{Walsh}). In dieser Situation
ist das richtige Gebot 3\coe!}

\minisec{Beispiele}

\woreizung{
  1\tre & & 1\kar \\
  2\tre & zeigt nach 1\kar 6er-Farbe & 2\pik & DFF, Werte in \pi \\
  3\tre & kein \co-\stp (sonst 2\SA) & 3\coe & fragt nach \co-\hstp
}

\woreizung{
  1\kar & & 1\coe \\
  2\kar & & 3\tre & DFF, \tr-Werte, \pf, fragt \co-Anschluss \\
  3\pik & VFF, kein 3er-\co, fragt nach \pi-\hstp
}

\woreizung{
  1\kar & & 2\tre \\
  3\kar & & 3\coe & DFF, \co-Werte, \pf, kein \pi-\stp (sonst 3\SA) \\
  3\pik & VFF, fragt nach \pi-\hstp
}

\woreizung{
  1\tre & & 2\tre & Inverted \\
  2\kar & 14\good{}\pl FP, \ka-Werte & 3\coe & \co-Werte, kein \pi-\stp (sonst 2/3\SA) \\
  2\pik & VFF, fragt nach \pi-\hstp
}

\woreizung{
  1\pik & & 2\tre \\
  2\coe & & 3\kar & VFF, fragt nach \ka-\stp \\
  3\coe & kein \ka-\stp, muss kein 5er-\co sein & 3\SA & zeigt \ka-\hstp \\
  pass & ebenfalls \ka-\hstp
}

Besonderheit:

\woreizung{
  1\pik & & 2\kar & \\
  2\coe & & 2\SA \\
  3\coe & \nf!, \ofa-Zweif"arber \\
  3\tre & VFF, \ofa-Zweif"arber
}

\subsection{Lebensohl} \label{lebensohl}

Wir spielen \Index{\conv{Lebensohl}} in folgenden Situationen:
%
\bdsc
\item[1\SA{}\sep{}(2\anybid)] nach \sa-Er"offnung des Partners und Farbgegenreizung der Gegner
 (\conv{Lebensohl} \emph{mit} Stopper)
\item[(2\anybid)\sep{}\kontra{}\sep{}(pass)]
 nach \Index{Weak Two}-Er"offnung des Gegners und Informationskontra des Partners
\item[(1\of)\sep{}\kontra{}\sep(2\of)]
 nach \ofa-Er"offnung des Gegners, Informationskontra des Partners und
 \ofa-Hebung durch den anderen Gegner (siehe auch \conv{Good Bad
   Notrump}, Seite~\pageref{goodbadnt}).
\edsc

"Ahnliche Situationen:
\bdsc
\item[(1\of)\sep{}pass\sep(2\of)\sep{}\kontra] \ra Scrambling 2\NT{}
\item[\dots\sep{}(2\anybid)] kompetitive Reizung \ra Good Bad Notrump
\edsc

\subsection{Scrambling 2\NT{} (Nebul"ose 2\SA)\label{scrambling2nt}}

Haben die Gegner eine Farbe er"offnet \emph{und gehoben} und der Partner
gibt ein Informationskontra, sowohl in der direkten als auch in der Pass
Out-Position, so ist 2\SA nicht echt, sondern bedeutet, dass man kein klares
Gebot hat.

\dealerW
S"ud \\
\handwithdesc{9754}{K52}{A64}{D54}{%
\begin{reizung}
  1\pik & pass & 2\pik & pass \\
  pass & \kontra & pass & 2\SA{}\al{a}
\end{reizung}}

Partner kann 1-5-4-3, 1-4-5-3 oder 1-4-3-5 verteilt sein,
wir wollen nicht im 3-3-Fit landen.

S"ud \\
\handwithdesc{B65}{D3}{K1053}{K763}{%
\begin{reizung}
  1\pik & pass & 2\pik & pass \\
  pass & \kontra & pass & 2\SA{}\al{a}
\end{reizung}}

Partner wird entweder seine 5er-Farbe reizen oder seine 4er-Farben von unten
nach oben.

\subsection{Good Bad Notrump (Kompetitive 2\SA)\label{goodbadnt}}

Hat der rechte Gegner ein Gebot auf der 2er-Stufe abgegeben, sind in
kompetitiven Situationen die 2\SA-Ansagen nicht nat"urlich, sondern zeigen den
Wunsch, in der 3er-Stufe zu spielen. Der Partner muss 3\tre bieten (siehe auch
\Index{\conv{Lebensohl}}), wonach der 2\SA-Reizer seine Farbe zeigt.

\dealerW
S"ud \\
\handwithdesc{32}{AK865}{5}{KB1063}{%
\begin{reizung}
  & & & 1\coe \\
  pass & 1\SA & 2\pik & 2\SA{}\al{a} \\
  pass & 3\tre & pass & pass
\end{reizung}}

Ein direktes 3\tre-Gebot von S"ud h"atte eine st"arkere Er"offnung gezeigt.

S"ud \\
\handwithdesc{K65}{A74}{D1093}{873}{%
\begin{reizung}
  1\coe & 2\kar & 2\coe & 3\kar{}\al{a}
\end{reizung}}

Hier direkt 3\kar, konstruktiv.

S"ud \\
\handwithdesc{865}{A74}{D1093}{873}{%
\begin{reizung}
  1\coe & 2\kar & 2\coe & 2\SA{}\al{a} \\
  pass & 3\tre & pass & 3\kar
\end{reizung}}

Mit dieser schwachen Hand zuerst 2\SA und sp"ater 3\kar.

S"ud \\
\handwithdesc{5}{AKD63}{AKB74}{52}{%
\begin{reizung}
  & & & 1\coe \\
  1\pik & pass & 2\pik & 3\kar{}\al{a}
\end{reizung}}

Stark genug, um 3\kar zu reizen. Mit \co{}\hspace{\cardskip}\suit{A9764}
und der gleichen \ka-Haltung w"urde man zuerst 2\SA reizen.

S"ud \\
\handwithdesc{63}{AKB8643}{K73}{9}{%
\begin{reizung}
  & & & 1\coe \\
  1\pik & 1\SA & 2\pik & 2\SA{}\al{a} \\
  pass & 3\tre & pass & 3\coe
\end{reizung}}

Wir wollen 3\coe spielen, ohne den Partner zu 4\coe einzuladen.

S"ud \\
\handwithdesc{A963}{A5}{762}{AB85}{%
\begin{reizung}
  & 1\kar & 1\coe & \kontra \\
  2\coe & 2\SA & pass & 3\coe{}\al{a}
\end{reizung}}

Nicht immer muss der Partner des 2\SA-Reizers 3\tre sagen, dann n"amlich, wenn
er eine st"arkere Hand hat und die Gefahr besteht, dass der 2\SA-Reizer 3\tre
passt.

\subsection{Sandwich \nt (1\SA/2\SA als Zweif"arber)}

Sandwich 1\NT benutzen wir in vierter Position nach Er"offnung, pass vom
Partner und einem Farbgebot auf der 1er-Stufe des rechten Gegners, um die
beiden nicht gereizten Farben zu zeigen.
Sandwich \nt zeigt eine Hand die zu schwach ist f"ur ein
Informationskontra, aber daf"ur etwas g"unstiger verteilt.

2\SA ist ebenfalls Sandwich \nt, nur mit noch besserer Verteilung
("ahnlich wie \conv{Unusual \nt{}} in zweiter Position).

Haben die Gegner eine Oberfarbe er"offnet und gehoben, dann zeigt 2\SA einen
\emph{unbestimmten} Zweif"arber.

Die Reizung (1\kar){}\sep pass{}\sep(1\pik){}\sep2\pik\ zeigt eine echte Farbe.

\dealerW
S"ud \\
\handwithdesc{DB108}{42}{53}{KD1094}{%
\begin{reizung}
  1\kar & pass & 1\coe & 1\SA \\
\end{reizung}}

S"ud \\
\handwithdesc{2}{AD105}{KB952}{643}{%
\begin{reizung}
  1\tre & pass & 1\pik & 1\SA \\
\end{reizung}}

S"ud \\
\handwithdesc{AB1087}{4}{73}{KD1093}{%
\begin{reizung}
  1\kar & pass & 1\coe & 2\SA \\
\end{reizung}}

\subsection{Kontras}

\subsubsection{Informationskontra}

Siehe \emph{Gegenreizung}, Seite \pageref{gegenreizung}.

\subsubsection{Negativkontra}

Siehe \emph{Verhalten nach Zwischenreizung durch die Gegner}, Seite
\pageref{zwischenreizung}.

\subsubsection{Responsive Double (Antwortkontra)}

Nach Informationskontra oder Farbgegenreizung vom Partner und
\emph{Hebung der Er"offnerfarbe} zeigt das Antwortkontra die nicht
gereizten Farben (siehe auch \conv{Scrambling 2\NT},
Seite~\pageref{scrambling2nt}).

\bdsc
\item[(1\kar){}\sep\kontra{}\sep(2\kar){}\sep\kontra] beide \ofa zu viert
\item[(1\coe){}\sep\kontra{}\sep(2\coe){}\sep\kontra] beide \ufa zu viert,
  kein 4er-\pi
\item[(1\coe){}\sep1\pik{}\sep(2\coe){}\sep\kontra] beide \ufa zu viert, kein
  3er-\pi
\item[(1\tre){}\sep1\coe{}\sep(2\pik){}\sep\kontra] \conv{\Index{Snapdragon Double}}, zeigt
  9/10\pl FP, 5er-\ka und Double-Figur in~\co
\edsc

\subsubsection{Competitive Double}

Das Competitive \kontra benutzen wir, um noch weitere Information bez"uglich der St"arke
einer Hand mitzuteilen.

\reizungmittext
{
  1\tre & 1\coe & 1\pik & 2\coe \\
  pass & pass & \kontra\al{a}
}
{
  Zusatzst"arke, ab etwa 10 FP
}

\reizungmittext
{
  1\pik & 2\coe & 2\pik & 3\coe \\
  \kontra\al{a}
}
{
  Full Value (siehe auch Trial Bids)
}

\reizungmittext
{
  1\pik & 2\tre & 2\pik & 3\tre \\
  \kontra\al{a}
}
{
  Strafe (siehe auch Trial Bids)
}

\reizungmittext
{
  1\pik & 2\coe & 2\tre & pass \\
  pass & \kontra\al{a}
}
{
  Wiederbelebung, zeigt gute Hand und Toleranz f"ur die nicht gereizten Farben
}

\subsubsection{Lightner-Kontra}

Das Kontra auf einen Endkontrakt -- sofern es sich um ungest"ort
gereizte Vollspiele oder Schlemms handelt -- verlangt normalerweise
die vom Dummy zuerst gereizte Farbe.  Nach [1\SA{}\sep3\SA;] verlangt
Kontra das Ausspiel der k"urzeren Oberfarbe.

\subsubsection{Trump Support Double}

Das Trumpfunterst"utzungs-Kontra benutzt der Er"offner, wenn der Partner eine Oberfarbe gereizt
und der n"achste Gegner gesprochen hat. Er zeigt mit \kontra bzw.
\rekontra 3er-Anschluss in der Oberfarbe des Partners:

\bdsc
\item[1\kar{}\sep(p)\sep1\pik{}\sep(2\tre);~?]~
	\bdsc
	  \item[2\pik] 4er-Anschluss
	  \item[\kontra] 3er-Anschluss
	\edsc
\item[1\kar{}\sep(p)\sep1\pik{}\sep(\kontra);~?]~
	\bdsc
	  \item[2\pik] 4er-Anschluss
	  \item[\rekontra] 3er-Anschluss
	\edsc
\edsc

\newpage
%%%%%%%%%%%%%%%%%%%%%%%%%%%%%%%%%%%%%%%%%%%%%%%%%%%%%%%%%%%%%%%%%%%%%%%%%%%%%%
\section{Schlemmkonventionen}

\subsection{Mixed Cuebids}

Kontrollgebote zeigen Erst- oder Zweitrundenkontrolle, also A/K/Single/Chicane.
Ein erstes Kontrollgebot auf der 5er-Stufe zeigt allerdings
Erstrundenkontrolle. Nicht K"urze in Partners erster Farbe als \emph{erstes
Kontrollgebot} reizen!

\exhand{2}{AD932}{AK92}{DB3}
{DB543}{B74}{DB3}{AK}{%
  1\coe & & 1\pik &\\
  3\kar & & 3\coe & st"arker als 4\coe\\
  4\kar & \ka-Kontrolle & 5\tre & keine \pi-Kontrolle, \tr-ERK\\
  5\kar & \pi-K"urze (sonst Sign Off), \ka-ERK & 6\tre
  & \tr-ERK und -ZRK\\
  6\coe & \pi-Single\\
}

\subsection{\Index{Gerber}-Assfrage} \label{gerber}

Ein direktes 4\tre-Gebot auf eine \sa-Er"offnung fragt den Er"offner nach Zahl
der Asse. Danach wird rollend nach platzierten K"onigen gefragt.

\bdsc
\item[1/2\SA{}\sep4\tre] Gerber-Assfrage
 \bdsc
 \item[4\kar/\co/\pi/\sa] 0/4, 1, 2, 3 Asse
  \bdsc
  \item[4\SA] zum Spielen
  \item[\rel] Frage nach platzierten K"onigen
  \edsc
 \edsc
\edsc

\subsection{Roman Keycard-Assfrage (KCB)}

4\SA auf Oberfarben-, 4\tre auf \tr- und 4\kar auf \ka-Basis. Antworten:
1/4, 3/0, 2 ohne Q, 2 mit Q. Anschlie"send rollend nach Trumpf-Dame und
gleichzeitig nach platzierten K"onigen. Hat man selbst die Trumpf-Dame, so ist
das "ubern"achste Gebot unter Auslassung der Trumpffarbe die Frage nach
platzierten K"onigen. Sind alle K"onige an Bord, kann man noch nach platzierten
Damen fragen.

Antwortschema: Hat der Antwortende die Trumpf-Dame nicht, geht er auf die
Trumpffarbe zur"uck. Hat er die Trumpf-Dame, aber keinen Nebenk"onig, so reizt
er 6 in der Trumpffarbe (das n"achste Gebot des Partners ist jetzt die Frage
nach platzierten Damen). Hat der die Trumpf-Dame und einen oder mehrere
Nebenfarbk"onige, so reizt er die niedrigste Farbe, in der er einen K"onig hat
und verneint damit gleichzeitig den K"onig in einer Farbe, die er h"atte
billiger reizen k"onnen. Das niedrigste \sa-Gebot zeigt den K"onig in der
Fragefarbe.

Man kann nun mit dem n"achsten Gebot nach weiteren K"onigen fragen, mit dem
"ubern"achsten nach platzierten Damen. Das Antwortschema bleibt sich gleich.

\exhand{A742}{76}{A2}{AKD52}
{KDB65}{AK}{KD3}{763}{%
  1\tre & & 1\pik \\
  4\tre & \pi-Fit, gute \tr-Farbe & 4\SA & KCB auf \pi-Basis\\
  5\kar & 3 oder 0 & 5\SA & platzierte K"onige?\\
  6\tre & \tr-K"onig vorhanden & 6\coe & platzierte Damen? (6\kar weitere K"onige?)\\
  7\tre & \tr-Dame vorhanden & 7\SA
}

\exhand{A2}{K754}{AKD43}{52}
{K532}{ADB}{7632}{A3}{%
  1\kar & & 1\pik &\\
  2\coe & & 4\kar & KCB\\
  4\pik & 3 oder 0 & 4\SA & \ka-Dame?\\
  5\coe & \ka-Dame und \co-K"onig, ohne \tr-K"onig & 7\ka &
  bei 3er-\pi und Single-\tr leider chancenlos
}

\exhand{AK2}{2}{D65}{AKD742}
{B6}{AB874}{AKB2}{B8}{%
  1\tre & & 1\coe\\
  3\tre & & 4\tre & KCB\\
  4\coe & 3 oder 0 & 4\pik & \tr-Dame?\\
  4\SA  & \tr-Dame und \pi-K"onig & 5\coe & platzierte Damen?\\
  6\kar & \ka-Dame & 7\SA
}

\exhand{AD1072}{K76}{B102}{D6}
{653}{A5}{AK98}{AK52}{%
  1\pik & & 2\tre\\
  2\pik & & 4\SA & KCB\\
  5\tre & 1 oder 4 & 5\kar & \pi-Dame?\\
  5\coe & \pi-Dame und \co-K"onig & 6\pik
}

\exhand{AK}{AD32}{65}{KD632}
{B432}{KB}{AB4}{AB109}{%
  1\tre & & 1\pik\\
  2\coe & & 4\tre & KCB\\
  4\coe & 3 oder 0 & 4\pik & \tr-Dame?\\
  4\SA  & \tr-Dame und \pi-K"onig & 5\coe & platzierte Damen?\\
  5\SA  & \co-Dame & 7\tre
}

\subsection{Preempt Keycard-Assfrage (PKCA)}

Diese Assfrage spielen wir nach preemptiven Er"offnungen
des Partners, also nach 2\of, 3\anybid und 4\of.
Nach 2er- und 3er-Starts ist das \emph{direkte} 4\tre-Gebot PKCA,
nach 4\of ist es 4\SA. Antworten:
%
\bdsc
\item[1. Stufe] 1 Ass ohne Trumpf-Dame
\item[2. Stufe] 0 Asse
\item[3. Stufe] 1 Ass mit Trumpf-Dame
\item[4. Stufe] 2 Asse ohne Trumpf-Dame
\item[5. Stufe] 2 Asse mit Trumpf-Dame
\edsc

\subsection{Exclusion Keycard-Assfrage (EKCB)}

Die Exclusion-Assfrage (Voidwood) wird durch einen ungew"ohnlichen Sprung
(meist in die 5er-Stufe) gestellt, der kein \Index{Splinter} sein kann:
%
\bdsc
\item[1\coe{}\sep1\pik; 2\coe{}\sep5\tre] Fragt nach Keycards unter
  Ausschluss der Treffs.
\item[1\tre{}\sep1\kar; 2\kar{}\sep4\coe] Fragt nach Keycards unter
  Ausschluss der Coeurs. Hier kein \conv{Splinter}, da wir den Fit schon
  haben.  \edsc

\subsection{\Index{DOPI-ROPI}}

\conv{DOPI} = ``double with zero, pass with one'', \conv{ROPI} =
``redouble with zero, pass with one''.

Nach der Assfrage und Zwischenreizung der Gegner (Farbe oder Kontra) zeigt pass
die erste Stufe und Kontra/Rekontra die zweite Stufe. Diese Konvention wenden
wir auch immer dann an, wenn Stufenantworten erforderlich sind und der Gegner
zwischengereizt hat.
%
\bdsc
  \item[pass] 1. Stufe (1 oder 4 Keycards)
  \item[\kontra/\rekontra] 2. Stufe (3 oder 0 Keycards)
\edsc

\subsection{Josephine}

Eine 5\SA-Reizung \emph{im Sprung} fordert den Partner auf, die Anzahl der
Topfiguren anzugeben: 0, 1, 2, 3.

\newpage
%%%%%%%%%%%%%%%%%%%%%%%%%%%%%%%%%%%%%%%%%%%%%%%%%%%%%%%%%%%%%%%%%%%%%%%%%%%%%%
\section{Ausspiele und Markierungen}

\subsection{Ausspiele gegen Farbkontrakte}

3./5.-h"ochste, h"ochste der Sequenz, hoch vom Double.
Von \cards{AK} wird \cards{K} ausgespielt.

Im weiteren Verlauf des Spiels zeigt eine kleine Karte eine
Figur, eine hohe hingegen zeigt keine Werte.

\subsection{Ausspiele gegen \sa-Kontrakte}

4.-h"ochste. Ausspiel der 10 und 9 verspricht 0 oder 2 h"ohere Karten. Von
\suit{KD10xx} wird der K"onig ausgespielt, der Partner soll den Buben deblockieren. In
Partners gereizter wird die 3./5.-h"ochste ausgespielt.

\subsection{Markierungen}

Markiert wird niedrig-hoch, d.h. eine kleine Karte ist positiv und zeigt somit
Interesse an der ausgespielten Farbe, eine hohe Karte ist negativ und zeigt
somit Desinteresse (\conv{UDCA}).

Spielt der Alleinspieler, so kann man L"angenmarken geben, wobei eine niedrige
Karte eine gerade L"ange und eine hohe Karte eine ungerade L"ange zeigt.

Erster freier Abwurf im \sa-Kontrakt ist Lavinthal. In Farbkontrakten wird
direkt markiert.
\raggedbottom

%%%%%%%%%%%%%%%%%%%%%%%%%%%%%%%%%%%%%%%%%%%%%%%%%%%%%%%%%%%%%%%%%%%%%%%%%%%%%%
\begin{appendix}
%\newpage
\section{Glossar}
\begin{flushleft}
\begin{tabularx}{\columnwidth}{lY}%
$n$\good{}, $n$\bad{} & $n$ gute/schlechte Punkte/Karten\\
$n$\pl & mindestens $n$ Punkte/Karten\\
\ra{}\anybid & Zielfarbe eines Transfers oder n"achstes Gebot auf unserer Seite\\
\ufa, \ofa & Unterfarbe(n), Oberfarbe(n)\\
\aufa, \aofa & andere \ufa, \ofa\\
1\anybid, 2\anybid, \ldots & beliebiges Gebot 1 in Farbe, 2 in Farbe usw.\\
(\any) & gegnerisches Gebot \\
5332 & beliebige 5332-Verteilung\\
5-3-3-2 & genau 5\pik 3\coe 3\kar 2\tre\\
\bal & ausgeglichen (4333, 4432, 5332)\\
\unbal & nicht ausgeglichen \\
\aw & Antwortender \\
\eo & Er"offner \\
\inv & einladend \\
\maxi & Maximum \\
\mini & Minimum \\
\nat & nat"urlich \\
\nf & nicht forcierend \\
\pf & Partieforcing \\
\pup & \conv{Puppet}-Gebot \\
\rel & Relais \\
\stp & Stopper (A, Kx, Dxx, Bxxx) \\
\hstp & Halbstopper (K, Dx, Bxx, 10xxx) \\
\slamint & Schlemminteresse \\
\xfer & \conv{Transfer}-Gebot \\
\end{tabularx}%
\end{flushleft}

%%%%%%%%%%%%%%%%%%%%%%%%%%%%%%%%%%%%%%%%%%%%%%%%%%%%%%%%%%%%%%%%%%%%%%%%%%%%%%
\section{"Anderungen am System}

Mai/Juni 2005:
\begin{compactitem}
\item Checkback durch 2\tre- und 2\SA-Transfer ersetzt (\ra \ref{1sarebid})
\item neue Struktur nach 2\coe-Er"offnung (\ra \ref{2coeur})
\item PKCA kann in Ogust-Sequenzen auch kleiner als 4\tre sein (\ra \ref{ogust})
\item nach \ufa-Er"offnung keine Fit Jumps mehr (sondern Weak Jumps) (\ra \ref{1treff})
\item nach Stenberg ist jetzt 4\tre statt 4\of die Minimum-Antwort (\ra \ref{stenberg})
\end{compactitem}

\subsection{Alte Konventionen}

Die folgenden Konventionen wurden aus dem System entfernt.

\subsection*{Checkback Stayman}

\emph{Checkback Stayman wurde durch den 2\tre/2\SA-Transfer (\ra~\ref{1sarebid})
ersetzt. (Mai 2005)}

\emph{Das vorher benutzte System mit Checkback-Stayman hatte eine L"ucke, wenn der Antwortende
eine starke Hand mit 4er-\ofa und 5er-\ufa hatte, die Er"offungsfarbe war.}

Der \aw benutzt diese Konvention mit einer 5er \ofa und einer mindestens
einladenden Hand, wenn der \eo 1\SA zur"uckgeboten hat; der \aw m"ochte
wissen, ob der \eo einen 3er-Anschluss hat.

\bdsc
\item[1\ufa{}\sep1\pik; 1\SA{}\sep2\tre] Checkback-Stayman
 \bdsc
 \item[2\kar] 12-13\bad{} FP, kein 3er-\pi, kein 4er-\co
  \bdsc
  \item[2\coe] 10/11 FP, 5/5 in \co und \pi, \nf
  \item[2\pik] 9-11 FP, 6er-\pi, \nf
  \item[2\SA] 11/12 FP, 5er-\pi, \nf
  \item[3\coe] 5/5 in \co und \pi, \slamint
  \edsc
 \item[2\coe] 12-13\bad{} FP, 4er-\co, 3er-\pi m"oglich
 \item[2\pik] 12-13\bad{} FP, 3er-\pi
 \item[2\SA]  13\good{}-14 FP, kein 3er-\pi, kein 4er-\co
 \item[3\ufa] 13\good{}-14 FP, 3er-\pi, gute \ufa
 \item[3\pik] 13\good{}-14 FP, 3er-\pi
 \edsc
\edsc

\subsection*{Weak Two, 2\coe}

\emph{Vor Mai 2005 war PKCA in den Ogust-Sequenzen immer mindestens 4\tre.}

Die K"urzenfrage nach 2\coe wurde wie folgt beantwortet:

(1) Wenn eine nat"urliche Antwort f"ur eine K"urze mit der n"achsten oder
"ubern"achsten Stufe m"oglich ist, zeigt die verbleibenden Stufe (1. oder 2.)
die verbleibende K"urze.

(2) Ist die n"achste oder "ubern"achste Stufe 3\SA\, zeigt dies K"urze in der
Fragefarbe. die verbleibende Stufe zeigt K"urze in der verbleibenden Farbe.

(3) Ansonsten gilt: niedrigste Stufe zeigt niedrigste K"urze.

Innerhalb einer Sequenz ist 4\pik immer PKCA auf \co-Basis, 4\tre und 4\kar
wenn m"oglich auf \tr- bzw. \ka-Basis.

\bdsc
\item[2\coe{}\sep2\SA;] Frage nach St"arke und zweiter Farbe
  \bdsc
  \item[3\tre] 6-10 FP, 5/5 \co/\tr
    \bdsc
    \item[3\kar] Frage nach St"arke und K"urze
      \bdsc
      \item[3\coe] 6-8 FP
        \bdsc
        \item[3\pik] Frage nach K"urze
          \bdsc
          \item[3\SA] \pi-K"urze (2) (\ra 4\tre = \tr-PKCA, 4\pik = \co-PKCA)
          \item[4\tre] \ka-K"urze (2) (\ra 4\kar = \tr-PKCA, 4\pik = \co-PKCA)
          \edsc
        \edsc
      \item[3\pik] 9-10 FP, \pi-K"urze (1)
      \item[3\SA] 9-10 FP, \ka-K"urze (2)
      \edsc
    \edsc
  \item[3\kar] 6-8 FP, 5/5 \co/\ka
    \bdsc
    \item[3\coe] zum Spielen
    \item[3\pik] Frage nach K"urze
      \bdsc
      \item[3\SA] \pi-K"urze (2), (\ra 4\kar = \ka-PKCA)
      \item[4\tr] \tr-K"urze (1), (\ra 4\pik = \co-PKCA)
      \edsc
    \edsc
  \item[3\coe] 6-8 FP, 5/5 \co/\pi
    \bdsc
    \item[3\pik] zum Spielen
    \item[4\tre] \co-Fit und Frage nach K"urze
      \bdsc
      \item[4\kar] \ka-K"urze (1)
      \item[4\coe] \tr-K"urze (1) (\ra 4\pik = \co-PKCA)
      \edsc
    \item[4\kar] \pi-Fit und Frage nach K"urze
      \bdsc
      \item[4\coe] \tr-K"urze (3)
      \item[4\pik] \ka-K"urze (3) (\ra 4\SA = \pi-PKCA)
      \edsc
    \edsc
  \item[3\pik] 9-10 FP, 5/5 \co/\pi, weiter wie nach 3\coe
  \item[3\SA] 9-10 FP, 5/5 \co/\ka
    \bdsc
    \item[4\tre] Frage nach K"urze
      \bdsc
      \item[4\kar] \tr-K"urze (3) (\ra 4\pik = \co-PKCA)
      \item[4\coe] \pi-K"urze (3) (\ra 4\SA = \ka-PKCA, Vorsicht!)
      \edsc
    \edsc
  \edsc
\edsc

%\printindex

\end{appendix}

\end{document}

