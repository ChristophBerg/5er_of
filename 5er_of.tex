\usepackage{babel}
\usepackage{fancyhdr}
\usepackage{tabularx}
\usepackage{calc}
\usepackage{suitsymbols}
\usepackage{makeidx}
\usepackage{xspace}

% Laengenparameter usw.
% L"angen
\setlength{\columnsep}{10mm}
\setlength{\columnseprule}{0.4pt}
%\setlength{\labelsep}{1.7ex}
\setlength{\itemsep}{0ex plus0.2ex}

\handwidth7.5em
\cardskip0.15ex
\parindent0mm
\parskip2ex plus1ex minus1ex
\renewcommand{\arraystretch}{1.07}

% compactitem
\usepackage{paralist}
\setlength{\plitemsep}{0.6ex}
\setlength{\pltopsep}{0.8ex}
\setlength{\plpartopsep}{0ex}
\defaultleftmargin{2.5ex}{2.3ex}{2.1ex}{1.9ex}


% Makros
% Farben ohne Space davor
\def\pi{\nobreak\Sp\xspace}
\def\co{\nobreak\He\xspace}
\def\ka{\nobreak\Di\xspace}
\def\tr{\nobreak\Cl\xspace}
\def\sa{\nobreak\textsf{S\kern-0.08emA}\xspace}
\def\nt{\nobreak\textsf{N\kern-0.08emT}\xspace}
% 2 Farben
\def\ofa{\nobreak\textsf{OF}\xspace}
\def\aofa{\nobreak\textsf{aOF}\xspace}
\def\ufa{\nobreak\textsf{UF}\xspace}
\def\aufa{\nobreak\textsf{aUF}\xspace}

% Farben mit Space davor
\def\pik{\nobreak\hspace{\cardskip}\Sp\xspace}
\def\coe{\nobreak\hspace{\cardskip}\He\xspace}
\def\kar{\nobreak\hspace{\cardskip}\Di\xspace}
\def\tre{\nobreak\hspace{\cardskip}\Cl\xspace}
\def\SA{\nobreak\hspace{\cardskip}\sa}
\def\NT{\nobreak\hspace{\cardskip}\nt}
% 2 Farben
\def\of{\nobreak\hspace{\cardskip}\textsf{OF}\xspace}
\def\aof{\nobreak\hspace{\cardskip}\textsf{aOF}\xspace}
\def\uf{\nobreak\hspace{\cardskip}\textsf{UF}\xspace}
\def\mi{\hspace{\cardskip}\Cl{}/\Di\xspace}
\def\ma{\hspace{\cardskip}\He{}/\Sp\xspace}

% diverse Makros
\def\good{$^+$\xspace}
\def\bad{$^-$\xspace}
\def\ra{$\rightarrow$\xspace}
\def\pl{$\uparrow$\xspace}
\def\any{$x$\xspace}
\def\anybid{\nobreak\hspace{\cardskip}\any}
\def\kontra{\textsf{X}\xspace}
\def\rekontra{\textsf{X\kern-0.08emX}\xspace}
\def\sep{\,--\,}
\newcommand{\conv}[1]{\emph{#1}}
\def\bal{\textsc{Ausg}\xspace}
\def\unbal{\textsc{Unausg}\xspace}
\def\nat{\textsc{Nat}\xspace}
\def\pf{\textsc{PF}\xspace}
\def\maxi{\textsc{Max}\xspace}
\def\mini{\textsc{Min}\xspace}
\def\inv{\textsc{Einl}\xspace}
\def\nf{\textsc{NF}\xspace}
\def\rel{\textsc{Rel}\xspace}
\def\stp{\textsc{Stop}\xspace}
\def\hstp{\textsc{Hstop}\xspace}
\def\cstop{\co-\stp}
\def\pstop{\pi-\stp}
\def\tstop{\tr-\stp}
\def\kstop{\ka-\stp}
\def\chstop{\co-\hstp}
\def\phstop{\pi-\hstp}
\def\thstop{\tr-\hstp}
\def\khstop{\ka-\hstp}
\def\aw{\textsc{Aw}\xspace}
\def\eo{\textsc{E\"o}\xspace}
\def\xfer{\textsc{Trf}\xspace}
\def\xferto{\xfer{}~\ra~}
\def\pup{\textsc{Pup}\xspace}
\def\pupto{\pup{}~\ra~}
\def\leadto{\nobreak\hspace{\cardskip}$\leadsto$\hspace{\cardskip}}
\def\slamint{\textsc{Schl-Int}\xspace}
\def\about{$\sim$\nobreak}

% Descriptions f"ur Bietb"aume
\def\bdsc{\begin{description}}
\def\edsc{\end{description}}

% neue Spalte f"ur tabularx
\newcolumntype{Y}{>{\raggedright\arraybackslash}X}

% Reizung mit 4 H"anden
\newcommand{\bidex}[1]{\bcbidding{\textsl{West}}{\textsl{Ost}}{#1}}

% Dreispaltige Tabelle f"ur "Ubersichten
\newcommand\bidins[1]%
{%
\begin{flushleft}
\begin{tabularx}{\columnwidth}{llY}%
#1
\end{tabularx}%
\end{flushleft}
}
% ... mit fester Breite
\newcommand\bidinsfixed[1]}%
#1
\end{tabularx}%
}

% Hand links, vertikale Linie, Text rechts
\newcommand{\handwithdesc}[5]{%
\handwidth6em
\vhand{#1}{#2}{#3}{#4}\hspace{1.5ex}\vrule\hspace{1ex}
\begin{minipage}[t]{\columnwidth-\handwidth-2.5ex-0.6pt}
#5
\end{minipage}
\smallskip
\handwidth7.5em
}

% W/O-Beispielhand mit Bietsequenz darunter
\newcommand{\exhand}[9]{%
\begin{minipage}{\columnwidth}
\westhand{#1}{#2}{#3}{#4}
\easthand{#5}{#6}{#7}{#8}
\centerline{\showEWgame}
\bigskip
{\smaller\begin{tabularx}{\columnwidth}{rYrY}
\multicolumn{2}{l}{West} & \multicolumn{2}{l}{Ost}\\
\hline
#9
\end{tabularx}}
\end{minipage}
}

% W/O-Bietsequenz
\newcommand{\woreizung}[1]{%
\begin{minipage}{\columnwidth}
{\smaller\begin{tabularx}{\columnwidth}{rYrY}
\multicolumn{2}{l}{West} & \multicolumn{2}{l}{Ost}\\
\hline
#1
\end{tabularx}}
\end{minipage}
}

\newenvironment{reizung}[1][t]%
{%
  \begin{tabular*}{\bidwidth}[#1]{@{}*{3}{l@{\extracolsep{0pt plus 1fil}}}l@{}}%
    West & Nord & Ost & S\"ud \\
    \hline
}
%    what's to be done at the end of the environment
{%
  \end{tabular*}%
}

% Reizung mit 4 H"anden
\newcommand{\reizungmittext}[3][\bidwidth]{%
\mbox{\smaller
\begin{minipage}{#1}
\begin{reizung}
#2
\end{reizung}
\end{minipage}%
\hspace{1.5ex}\vrule\hspace{1.5ex}%
\begin{minipage}{\columnwidth-#1-3ex-0.6pt}%
#3
\end{minipage}
}
}

% \newcommand\bidseq[1]%
% {%
% \begin{flushleft}
% \begin{tabularx}{\columnwidth}{llY}%
% #1
% \end{tabularx}%
% \end{flushleft}
% }

% Kasten mit Rahmen
\setlength{\fboxsep}{1.5ex}
\newcommand\notebox[1]%
{%
\fbox{\parbox{\columnwidth - 3ex}{#1}}%
}

% Index
\makeindex
\newcommand{\Index}[1]{#1\index{#1}}

% Pagestyle
\renewcommand{\sectionmark}[1]{\markleft{#1}}
\pagestyle{fancy}
\fancyhead[RE,LO]{\nouppercase{\emph{\leftmark}}}
\fancyhead[RO,LE]{\thepage}
\fancyfoot{}
\renewcommand{\headrulewidth}{0pt}



\begin{document}

\begin{center}
\textsf{{\relsize{5}\co{}~~5er-Oberfarben~~\pi\\[1ex]}
\tr{}~~Thomas Schmitt~~\ka\\[1ex]
\small \today}
\end{center}
\tableofcontents

%
%%%%%%%%%%%%%%%%%%%%%%%%%%%%%% Eroeffnungen %%%%%%%%%%%%%%%%%%%%%%%%%%%%%%
%
\newpage
\section{"Ubersicht der Er"offnungen}

\emph{Die letzte Spalte gibt ungef"ahre Wahrscheinlichkeiten f"ur die jeweilige
Er"offnung in 1. Hand an.}

\subsection*{1er-Stufe}
\bidinsfixed{%
1\mi & 12\pl	& 3\pl{}er-\mi & 17&9\\[1ex]
1\ma & 12\pl	& 5\pl{}er-\ma & 10&5\\[1ex]
1\SA & 15-17	& \bal, 5er-\ofa m"oglich & 4&9
}

\subsection*{2er-Stufe}
\bidinsfixed{%
2\tre	& 6-10	& Weak Two in \ka, oder\\
	& 16\pl	& 6er-Farbe, 8 Spielstiche, oder\\
	& 22-23	& \bal, oder\\
	& 26-27	& \bal & 2&2\\[1ex]
2\kar	& 6-10	& Weak Two in \co, oder\\
	& 18\pl	& 6er-Farbe, 9 Spielstiche, oder\\
	& 24-25	& \bal, oder\\
	& 28\pl	& \bal & 2&1\\[1ex]
2\coe	& 6-10	& Zweif"arber mit \co & 1&3\\[1ex]
2\pik	& 6-10	& Weak Two in \pi & 1&6\\[1ex]
2\SA	& 20-21	& \bal, 5er-\ofa m"oglich & 0&5
}

\subsection*{3er-Stufe}
\bidinsfixed{%
3\uf	& 5-10	& 7\pl{}er-\ufa (in 1. Hand in Nichtgefahr und 3. Hand 6er-\tr
m"oglich) & 0&8\\[1ex]
3\of	& 5-10	& 7\pl{}er-\ofa & 1&0\\[1ex]
3\SA	& 	& Gambling in 1./2. Hand, in 3./4. Hand zum Spielen & 0&1
}

\subsection*{4er-Stufe}
\bidinsfixed{%
4\mi	& 	& stehendes 7er-\ma mit einem Nebenwert\\[1ex]
4\of	&	& 7\good{}\pl{}er-\ofa, zum Spielen\\[1ex]
4\SA	&	& 6/5\pl\uf
}

%
%%%%%%%%%%%%%%%%%%%%%%%%%%%%%%% Grundsystem %%%%%%%%%%%%%%%%%%%%%%%%%%%%%%
%
\newpage
\section{Grundsystem}

\minisec{Er"offnung}

Er"offnet auf der 1er-Stufe werden alle H"ande ab 12 FP. Mit ausgeglichenen
H"anden und 15-17 FP wird immer 1\SA er"offnet (auch mit 5er-\ofa). Nach
Farber"offnung wird zun"achst auf der 1er-Stufe ein Fit gesucht. Neue Farben
auf der 1er-Stufe zeigen 4er-L"angen, ohne die bisher gereizten Farben zu
verl"angern. Neue Farben auf der 2er-Stufe zeigen 5\pl-4\pl-Zweif"arber in der
ersten und zweiten Farbe.

\minisec{1. Antwort}

Mit 6\pl FP zeigt der Antwortende seine niedrigste 4er-Farbe auf 1er-Stufe
oder reizt 1\SA mit 6-10\bad FP ohne reizbare 4er-Farbe (\conv{1/1}). Ab
10\good FP kann der Antwortende \conv{2/1} reizen.

Mit Fit in \ofa siehe \conv{Bergen-Hebungen}, \conv{Splinter},
2\SA-Partieforcing. Mit Fit in \ufa siehe \conv{Inverted}.

\minisec{R"uckgebot}

Mit unausgeglichenen H"anden zeigt der Er"offner nun seine 2. Farbe. Ist diese
rangniedriger, so zeigt er sie nach \conv{1/1} immer, falls rangh"oher, nur
wenn er 16\good\pl FP hat (\conv{Reverse}), ansonsten wiederholt er seine 1.
Farbe.

Mit ausgeglichenen H"anden zeigt der Er"offner nun seine St"arke, er reizt nach
1/1 mit 12-14 FP 1\SA, mit 18-19 FP 2\SA. Nach 2/1 wiederholt er seine \ofa mit
12-14 FP und reizt 2\SA mit 18-19 FP.

Mit Fit \dots

\minisec{2. Antwort}



%
%%%%%%%%%%%%%%%%%%%%%%%%%%%%% 1 UF-Eroeffnung %%%%%%%%%%%%%%%%%%%%%%%%%%%%
%
\newpage
\section{Die 1\tre/\ka-Er"offnungen}

Ohne 5\pl{}er-\ofa wird grunds"atzlich die l"angere \uf er"offnet.
Mit 3-3 in \ka/\tr wird 1\tre er"offnet, sonst bei gleicher L"ange 1\kar.

Siehe auch \textit{Er"offnungsregel f"ur Zweif"arber mit 6/5 \ufa/\ofa{}} auf
Seite \pageref{zfregel}.

\subsection{Antworten auf 1\tre-Er"offnung} \label{1treff}
\bidins{%
1\kar	& 6-7	& \bal ohne 4er-\ofa, 3-3-3-4 m"oglich, oder\\
	& 6\pl	& 5\pl{}er-\ka, oder\\
	& 12\pl	& 5\pl{}er-\ka und 4\pl{}er-\ofa \conv{(Walsh-\ka)}\\[1ex]
1\of	& 6\pl	& 4\pl{}er-\ofa\\[1ex]
1\SA	& 8-10	& \bal ohne 4er-\ofa\\[1ex]
2\tre	& 10\good{}\pl & 5\pl{}er-\tr \conv{(Inverted)}\\[1ex]
2\kar	& 2-5	& 5/5 \ofa\\[1ex]
2\of	& 5-8	& gute 6er-\ofa \conv{(Weak Jump)}\\
        &       & \ra 2\SA = \conv{\Index{Ogust}\footnote{\mini/\maxi
entsprechend angepasst}} \\[1ex]
2\SA    & 2-6   & 5\pl{}er-\tr \conv{(Inverted Spezial)}\\
3\tre	& 7-9	& 5\pl{}er-\tr \conv{(Inverted)}\\[1ex]
3\kar/\co/\pi & 5-8 & gutes 7\pl{}er-\ka/\co/\pi \conv{(Weak Jump)}\\[1ex]
3\SA	& 13-15	& zum Spielen\\[1ex]
4\tre	&	& \conv{KCB}\\[1ex]
4\of	&	& zum Spielen
}

\notebox{\textbf{Weiterreizung:}
Ein unn"otiger Sprung in einer neuen Farbe ist \conv{\Index{Splinter}}.
Ein unn"otiger Doppelsprung in einer neuen Farbe ist \conv{Exclusion KCB}.}

Nach 1\kar{}-Er"offnung sind die Antworten entsprechend
(Ausnahme: [1\kar{}\sep2\tre{}] \ra\ref{inverted}, S.~\pageref{inverted}).

%
%%%%%%%%%%%%%%%%%%%%%%%%%%%%% Walsh-Sequenzen %%%%%%%%%%%%%%%%%%%%%%%%%%%%
%
\subsection{Die \conv{Walsh}-Antwort auf 1\tre-Er"offnung}

Der Antwortende zeigt nach 1\tre-Er"offnung immer sofort eine
4er-\ofa, es sei denn
die Karos sind l"anger als die \ofa und der Antwortende ist bereit, bei der
n"achsten Gelegenheit die \ofa revers zu reizen (also ab 12 Punkten).

Der Er"offner bietet nach 1\kar-Antwort mit ausgeglichener Hand auch
dann 1\SA zur"uck, wenn er eine oder beide \ofa h"alt -- denn der
Antwortende hat entweder keine 4er-\ofa oder wird diese nun reizen,
was gleichzeitig ein Partieforcing etabliert.

Ein \ofa-R"uckgebot des Er"offners nach [1\tre-1\kar;] zeigt
immer eine unausgeglichene Hand mit langen Treffs.

Diese Gebote sollten folgenderma"sen alertiert werden:
\begin{description}
\item[1\tr-1\ka;] \emph{"`keine 4er-\ofa oder stark genug, um Revers zu
reizen"'}
\item[1\tr-1\ka;~1\SA] \emph{"`kann eine oder beide 4er-\ofa enthalten"'}
\item[1\tr-1\ka;~1\ofa] \emph{"`unausgeglichen"'}
\end{description}

\minisec{Bietsequenzen nach [1\tre{}\sep1\kar{}]}

\bdsc
\item[1\tre{}\sep1\kar; ?]~
  \bdsc
  \item[1\coe] 5\pl{}er-\tr und 4er-\co oder 4-4-1-4
  \item[1\pik] 5\pl{}er-\tr und 4er-\pi
  \item[1\SA] 12-14 FP, eine oder beide \of m"oglich
    \bdsc
      \item[2\tre] \pf/\slamint in \tre oder \kar;
            \textbf{Frage nach Verteilung} (s.~u.)
      \item[2\kar] schwach; zum Spielen
      \item[2\ma] \pf mit 5/4-Verteilung
      \item[2\SA] \nat
      \item[3\tre/\co/\pi] \conv{\Index{Autosplinter}} mit 6er-\ka
      \item[3\kar] \inv mit 6er-\ka
    \edsc
  \item[2\tre] 12-14 FP, 6er-\tr
    \bdsc
      \item[2\coe] \cstop, kein \pstop
        \bdsc
        \item[2\pik] Frage nach \phstop \conv{(VFF)}
        \edsc
      \item[2\pik] zeigt \pstop
        \bdsc
          \item[3\tre] verneint \cstop
	  \bdsc
	    \item[3\coe] Frage nach \chstop \conv{(VFF)}
	  \edsc
        \edsc
      \item[3\ma] \conv{Splinter} mit \tr-Anschluss
      \item[4\ma] \conv{EKCB} auf \tr-Basis
    \edsc
  \item[2\SA] 18-19 FP, eine oder beide \ofa m"oglich
    \bdsc
    \item[3\tre] \pupto3\kar, worauf man passen kann oder Sign Off gibt
    (\conv{Wolff Sign Off} \ra \ref{wolff})
    \item[3\kar] Forcing
    \item[3\of] 12\pl FP, 5/4-Verteilung, Schlemm m"oglich (s.~u.)
    \edsc
  \edsc
\edsc

\minisec{Verteilungsfrage nach 1\SA-R"uckgebot}

\bdsc
\item[1\tre{}\sep1\kar; 1\SA{}\sep2\tre; ?]~

  Hat der Antwortende Schlemm-Interesse in \ufa, so kann er die genaue
  Verteilung der ausgeglichenen Hand des Er"offners erfragen.
  Diese Sequenz ist sehr selten ($\sim$0,1\%).

  \bdsc
  \item[2\kar] 3er-\ka, aber nicht 4333
    \bdsc
      \item[2\coe] Frage nach Verteilung
        \bdsc
        \item[2\pik] 5er-\tr (x-x-3-5)
        \item[2\SA] 4er-\tr (x-x-3-4)
	  \bdsc
	  \item[3\mi] \conv{KCB} auf \tr/\ka-Basis
	  \edsc
        \edsc
    \edsc
  \item[2\coe] 3-4-2-4
  \item[2\pik] 4-3-2-4
  \item[2\SA] beliebige 4333-Verteilung\footnote{4er-\ka ist nicht m"oglich, da
mit ausgeglichener Hand und 4er-\ka nicht 1\tre er"offnet wird.}
    \bdsc
      \item[3\tre] Frage nach 4er-Farbe
        \bdsc
        \item[3\kar] \textbf{4er-\tr}
        \item[3\coe] 4er-\co
        \item[3\pik] 4er-\pi
        \edsc
    \edsc
  \item[3\tre] 3-3-2-5 \ra 3\kar: \conv{KCB}
  \item[3\kar] 4-4-2-3
  \edsc
\edsc

\minisec{Assfrage nach 2\SA-R"uckgebot und 3\of-Antwort}

\bdsc
  \item[1\tre{}\sep1\kar; 2\SA{}\sep{}3\of; ?]~

    Reizt der Antwortende nach \conv{Walsh}-\ka nun 3\of (verspricht ab 12
    Punkten), so ist meistens ein Schlemm m"oglich.  Der Er"offner
    beantwortet daher mit \ka-Fit sofort die Assfrage auf
    \ka-Basis.  Hat der Er"offner keine Unterst"utzung f"ur Karo, so best"atigt
    er den \ofa-Fit oder reizt 3\SA falls kein \ofa-Fit vorhanden ist.

  \bdsc
      \item[3\coe] 12\pl FP, 5/4-Verteilung, Schlemm m"oglich
        \bdsc
          \item[3\pik] 1 oder 4 Keycards auf \ka-Basis
          \item[3\SA] \nat, kein Fit
          \item[4\tre] 0 oder 3 Keycards auf \ka-Basis
          \item[4\kar] 2 Keycards ohne \ka-Dame
          \item[4\coe] \co-Fit
          \item[4\pik] 2 Keycards mit \ka-Dame
        \edsc
      \item[3\pik] 12\pl FP, 5/4-Verteilung, Schlemm m"oglich
        \bdsc
          \item[3\SA] \nat, kein Fit
          \item[4\tre] 1 oder 4 Keycards auf \ka-Basis
          \item[4\kar] 0 oder 3 Keycards auf \ka-Basis
          \item[4\coe] 2 Keycards auf \ka-Basis (mit/ohne Dame)
          \item[4\pik] \pi-Fit
        \edsc
    \edsc
\edsc

\subsection{\conv{Wolff Sign Off} nach 2\SA-R"uckgebot} \label{wolff}

\conv{Wolff Sign Off} fordert nach \ufa-Er"offnung und 2\SA-R"uckgebot mit
3\tre ein 3\kar-Relais an. Es ist n"utzlich in Situationen, in denen der
Antwortende sehr schwach ist und seine erste Antwort lediglich den
Er"offner vor einem schlechten Treff- oder Karo-Kontrakt hat bewahren
sollen. \cite{cuppaidge05}

Der Antwortende kann das forcierte 3\kar-Gebot des Er"offners passen,
3\coe zum Spielen oder Ausbessern anbieten, oder ein beliebiges Gebot
zum Spielen abgeben.

\minisec{Beispiele}

Die Reizung beginnt jeweils mit [1\tre{}\sep1\pik;~2\SA{}].

\handwithdesc{K10xx}{xx}{J10xxxx}{x}{%
  Wir m"ochten in jedem Fall 3\kar spielen.  Wir reizen
  [3\tre;~3\kar{}\sep{}pass].}

\handwithdesc{Qxxxx}{Qxxx}{xx}{xx}{%
  Wir m"ochten entweder 3\coe oder 3\pik spielen.  Wir reizen daher
  [3\tre;~3\kar{}\sep3\coe;] zum Spielen in 3\coe oder Ausbessern nach
  3\pik.}

%
%%%%%%%%%%%%%%%%%%%%%%%%%%%%% Sequenzen nach 1T-1OF %%%%%%%%%%%%%%%%%%%%%%
%
\subsection{Bietsequenzen nach [1\tre{}\sep1\of{}]}

\bdsc
\item[1\tre{}\sep1\coe; ?]~
  \bdsc
  \item[1\pik] 12\pl FP, \nat

    Wiederholt der Antwortende seine Oberfarbe auf niedrigster Stufe
    nachdem der Er"offner zwei Farben gereizt hat, so zeigt dies eine
    einladende Hand mit 9-11 FP und 6er-Farbe.

    Wiederholt er seine Oberfarbe im Sprung, so zeigt dies eine
    partieforcierende Hand mit 6er-Farbe:
    \bdsc
    \item[2\coe] 9-11 FP, 6er-\co, \inv
    \item[3\coe] 6er-\co, \pf
    \edsc
  \item[1\SA] 12-14 FP \bal, kein 4er-\pi (\ra \ref{1sarebid},
S.~\pageref{1sarebid})
  \edsc

\item[1\tre{}\sep1\pik; ?]~
  \bdsc
  \item[1\SA] 12-14 FP, \bal (\ra \ref{1sarebid}, S.~\pageref{1sarebid})
  \item[2\tre] 12-16\bad{} FP, 5\pl{}er-Farbe (s.~u.)
  \item[2\SA] 18-19 FP, \bal; siehe \conv{Wolff} etc.
  \item[3\kar/\co] 4er-\pi, Single \ka/\co \conv{(\Index{Minisplinter})}
  \item[3\SA] 18-19 FP, stehendes 6er-\tr, K"urze in \pik, Deckung in den
    ungereizten Farben, 8-8$\frac{1}{2}$ Stiche
  \item[4\tre] 18\good{}\pl FP, 4er-\pi, 5\good{}\pl{}er-\tre
\conv{(Fit-Sprung)}
  \item[4\kar/\co] 18\good{}\pl FP, 4er-\pi \conv{(Exclusion KCB)}
  \edsc

\item[1\tre{}\sep1\pik; 2\tre{}\sep{}?] 12-16\bad{} FP, 5\pl{}er-\tr
  \bdsc
  \item[2\kar] \conv{DFF} (\ra \ref{dff})
    \bdsc
    \item[2\coe] Frage nach \chstop (\conv{VFF}), kein 3er-\pi
    \item[2\pik] 3er-\pi, \mini f"ur 2\tre-R"uckgebot\\
      \bdsc
        \item[3\tre] Forcing mit \tre
      \edsc
    \item[2\SA] \cstop, kein 3er-\pi, \mini
    \item[3\pik] 3er-\pi, \maxi f"ur 2\tre-R"uckgebot
    \edsc
  \item[3\tre] \inv
  \edsc
\edsc

\subsection{Bietsequenzen nach \conv{Inverted}} \label{inverted}

Nach [1\tre{}\sep2\tre{};] bzw. [1\kar{}\sep2\kar{};] gibt es kein Revers.
Ziel ist meist einen \sa-Kontrakt zu spielen wenn alle Nebenfarben gestoppt
sind. Der Er"offner zeigt mit 2\SA eine ausgeglichene, passbare Hand.
3\tre bzw. 3\kar ist ebenfalls passbar.  Alle anderen Gebote des Er"offners
zeigen Werte (Stopper) und sind partieforcing, weitere Gebote des Antwortenden
zeigen ebenfalls Werte.

\bdsc
  \item[1\tre{}\sep2\tre; ?]~
    \bdsc
      \item[2\kar/\co/\pi] 14\good{}\pl FP, \ka/\co/\pi-\stp
        \bdsc
          \item[2\coe] \cstop (h"ochstens \phstop)
            \bdsc
              \item[2\pik] fragt nach \phstop (siehe \conv{VFF})
            \edsc
          \item[2\pik] \pstop (h"ochstens \chstop)
          \item[2\SA] \stp in \co und \pi
	\edsc
    \edsc
\edsc

[1\kar{}\sep2\tre{};] wird "ahnlich wie \conv{Inverted} behandelt, mit dem R"uckgebot
zeigt der Er"offner seine Punktst"arke:

\bdsc
  \item[1\kar{}\sep2\tre; ?]~
    \bdsc
    \item[2\kar] 12-13 FP, \bal oder \nat (kann 3er-\ka sein)
    \item[2\SA] 14 FP, \bal
    \edsc
\edsc


\subsection{\label{zfregel}Er"offnungsregel f"ur Zweif"arber mit 6-5~\uf/\of}

\begin{description}
%\setlength{\labelsep}{1ex}
\item[4\pl{} Verlierer:] \of er"offnen und \uf billig nachreizen
\item[3-4 Verlierer:] \uf er"offnen und anschlie"send \conv{Revers}
  reizen
\item[0-2 Verlierer:] Partieforcing (2\kar) er"offnen
\end{description}

\newpage
%%%%%%%%%%%%%%%%%%%%%%%%%%%%%%%%%%%%%%%%%%%%%%%%%%%%%%%%%%%%%%%%%%%%%%%%%%%%%%
\section{Die 1\coe/\pi-Er"offnungen}

Die Fortsetzung mit Anschluss nach 1\of-Er"offnung folgt folgenden Prinzipien:
\begin{itemize}
%\setlength{\itemsep}{0.5ex}
\item Schwache bis einladende H"ande mit 4\pl{}er-Anschluss werden durch
  \conv{Bergen-Hebungen} gezeigt.
\item Mit \pf und gutem Trumpfanschluss reizen wir
  \conv{\Index{Splinter}} oder partieforcierend 2\SA.
\item Die restlichen starken Varianten werden durch verz"ogertes
  Reizen der Trumpfunterst"utzung gezeigt (Farbwechsel).
\end{itemize}

\subsection{Antworten auf 1\of-Er"offnung}

\bidins{%
  1\SA & 6-10\bad & kein Anschluss, keine weitere \ofa\\[1ex]
  2\tre & 10\good{}\pl & \textbf{2}\pl{}er-\tr{}\footnote{meist l"anger, im
    schlimmsten Fall 3-4-4-2-Verteilung nach 1\pik-Er"offnung, nach 1\coe
    mindestens 3er (3-3-4-3)}, wenn kein 4\pl{}er-\tr, dann 3er-\ofa \\
  2\kar & 10\good{}\pl & 5\pl{}er-\ka\\[1ex]
  2\of & 6-10\bad & 3er-Anschluss\\[1ex]
  2\SA & 12\good{}\pl & 4\pl{}er-Anschluss, \pf\\
       & 11-12 & mit gepasster Hand: Double-\ofa und 3er-\aof (\bal,
\nat)\\[1ex]
  3\tre & 9\good{}-11 & 4\pl{}er-Anschluss \conv{(Bergen-Hebung)}\\
  3\kar & 7-9\bad{} & 4\pl{}er-Anschluss \conv{(Bergen-Hebung)}\\
  3\of & 0-6 & 4\pl{}er-Anschluss \conv{(Bergen-Hebung)}\\[1ex]
  3\aof & 11-14 & beliebiges Chicane \conv{(Splinter)}, s.~u.\\
  3\SA & 11-14 & Single in \aofa \conv{(Splinter)}, s.~u.\\
  4\tre/\ka & 11-14 & \tr/\ka-Single \conv{(Splinter)}\\[1ex]
  4\of && zum Spielen\\[1ex]
  4\aof && \conv{Exclusion KCB}\\[1ex]
  \multicolumn{3}{l}{Nach 1\coe-Er"offnung:}\\
  2\pik & 5-8 & 6\pl{}er-\pik (\conv{Weak Jump})
              \ra 2\SA: \conv{\Index{Ogust}}\footnote{\mini/\maxi entsprechend
angepasst}\\[1ex]
  \multicolumn{3}{l}{Nach 1\pik-Er"offnung:}\\
  2\coe & 10\good{}\pl & 5\pl{}er-\coe
}

\notebox{%
\textbf{\conv{Splinter}-Gebote} zeigen immer mindesten 4er-Anschluss in der
Trumpffarbe sowie eine K"urze.  Die Werte f"ur ein Vollspiel werden versprochen,
aber die Punktspanne ist nach oben limitiert.  Mit st"arkeren H"anden sollte
mittels 2\SA gehoben werden.
}

\subsection{\conv{\Index{Splinter}}-Gebote nach 1\of-Er"offnung\label{ofsplinter}}

[1\of{}\sep3\SA;] zeigt ein Single in der anderen \ofa. Chicane-\conv{Splinter}
werden mit einem Sprung in die andere \ofa auf 3er-Stufe gezeigt.

Nach St"orung durch die Gegner gelten Ausnahmen.  Siehe hierzu
Abschnitt~\ref{zwischenreizung} auf Seite~\pageref{zwischenreizung}.

\bdsc
\item[1\coe{}\sep3\pik] beliebiges Chicane
  \bdsc
  \item[3\SA] \rel, Frage nach dem Chicane
    \bdsc
    \item[4\tre] \tr-Chicane
    \item[4\kar] \ka-Chicane
    \item[4\coe] \pi-Chicane
    \edsc
  \edsc
\item[1\pik{}\sep3\coe] beliebiges Chicane
  \bdsc
  \item[3\pik] \rel, Frage nach dem Chicane
    \bdsc
    \item[3\SA] \co-Chicane
    \item[4\tre] \tr-Chicane
    \item[4\kar] \ka-Chicane
    \edsc
  \edsc
\edsc

\subsection{Bietsequenzen nach 1\of-Er"offnung}

\bdsc
\item[1\coe{}\sep1\pik; ?]~
  \bdsc
  \item[1\SA] 12-14 FP \bal
    \bdsc
    \item[2\tre] \pupto 2\kar (\conv{Relaistransfer} \ra \ref{1sarebid},
S.~\pageref{1sarebid})
    \item[2\kar] 4er-\pi, 5\pl\kar, genau \inv
    \item[2\SA] \pupto 3\tre (\conv{Relaistransfer} \ra \ref{1sarebid},
S.~\pageref{1sarebid})
    \item[3\tre] \nat, forcing, 5-5
      \bdsc
      \item[3\kar] Frage nach \khstop, verneint 3er-\pi
        (siehe \conv{VFF})
      \item[3\coe] 2-5-3-3, verneint \khstop
      \item[3\pik] 3er-\pi, \maxi
      \item[3\SA] \kstop, verneint 3er-\pi
      \item[4\pik] 3er-\pi, \mini
      \edsc
    \edsc
  \item[2\tre/\ka] 12-18, 5/4\pl-Verteilung
    \bdsc
    \item[2\coe] bessert aus (\co-Double)
    \item[3\tre/\ka] \inv
    \edsc
    \textbf{Nach \conv{1~"uber~1} ist die Hebung von Er"offners \emph{zweiter}
	    Farbe auf die 3er-Stufe lediglich einladend.}
  \item[2\SA] 18-19 FP \bal
  \item[3\tre/\ka] 18\good{}\pl FP, 5/4\pl \coe/\uf
    \bdsc
      \item[3\coe] st"arker als 4\coe (\emph{Principle of Fast Arrival})
      \bdsc
        \item[3\pik] \conv{Cuebid}, aber keine K"urze
        
	\textbf{Ein \conv{Cuebid} in Partners erster Farbe zeigt \emph{nie} eine
        K"urze.}
      \edsc
    \edsc
  \edsc

\item[1\pik{}\sep2\tre; ?]~
\bdsc
\item[2\kar] 12-18 FP, 5/4\pl
  \bdsc
  \item[2\pik] \inv mit 3er-\pi
  \item[2\SA] \nat, \nf
  \item[3\tre] \nat, \nf
  \item[3\kar] \nat, \pf

    \textbf{Nach \conv{2~"uber~1} ist die Hebung von Er"offners
	    \emph{zweiter} Farbe auf die 3er-Stufe partieforcing.}
    \bdsc
    \item[3\coe] Frage nach \chstop \conv{(VFF)}
    \item[3\pik] kein \chstop, verspricht kein 6er-\pi
    \item[3\SA] \cstop
    \edsc
  \item[3\pik] Schlemm-Interesse mit 3er-\pi
  \edsc
\item[2\coe]~
  \bdsc
  \item[2\SA] \nat
    \bdsc
    \item[3\kar] 5/5 in den \ofa, \pf \conv{(VFF)}\footnote{nach 2\kar-Antwort 3\tre}
    \item[3\coe] 5/5 in den \ofa, \nf
    \edsc
  \item[3\coe] \nat, \pf
  \edsc
\item[2\pik] 12-14 FP, kann 5er-\pi sein!
  \bdsc
  \item[3\tre] 6er-\tr, \nf
  \item[3\kar/\co] Werte (siehe \conv{DFF}), \pf
  \item[3\pik] \inv, 3er-\pi
  \edsc
\item[2\SA] 18-19 FP \bal
\item[3\tre] 16\pl FP, 5/4\pl \pi{}+\tr
  \bdsc
  \item[3\kar] \kar-Werte, \ra~3\coe = Frage nach \hstp
  \item[3\coe] \coe-Werte
  \edsc
\edsc
\item[1\pik{}\sep2\coe; ?]~
  \bdsc
  \item[3\coe] 16\pl FP, 5/3\pl \pi{}+\co
  \item[4\coe] 12-14 FP, 4er-Anschluss (3er-Anschluss und \mini
    "uber 2\pik)
  \edsc
\edsc

\woreizung{%
1\pik & & 2\kar\\
3\tre & 5/4 in \pi{}+\tr, \pf & 3\kar & forcing\\
3\SA  & \cstop & 4\SA & quantitativ}

\subsection{Weiterreizung nach [1\of{}\sep2\SA{}]\label{majorgf}}

Die 2\SA-Antwort auf eine 1\of-Er"offnung verspricht 4er-Anschluss (oder guten
3er) und Vollspielwerte, oder besser.

Der Er"offner teilt seine Hand in drei St"arkebereiche ein:
\begin{itemize}
\item[Minimum:] 11 gute bis 14 schlechte Punkte \ra 3\tre
\item[Medium:] 14 gute bis 16 Punkte \ra 3\coe-4\kar
\item[Maximum:] ab 17 Punkten \ra 3\kar
\end{itemize}

Mit einer Minimum-Hand reizt der Er"offner zun"achst 3\tre, mit einer Hand ab
17 Punkten zun"achst 3\kar{}.  Mit mittelstarken H"anden zeigt der Er"offner
sofort eine K"urze, sofern vorhanden.

Nach 3\tre und 3\kar kann der Antwortende den Er"offner nach K"urze fragen,
nach 3\tre auch selbst eine K"urze zeigen (starke Splinter).  K"urzen werden
genauso gezeigt wie als direkte Antwort auf 1\of{}-Er"offnung;
dies gilt auch f"ur die Frage nach der Chicane-Farbe (siehe
Abschnitt~\ref{ofsplinter} auf Seite~\pageref{ofsplinter}).

Danach reizt man mit Cuebids weiter; 4\SA ist KCB.

\bdsc
\item[1\of{}\sep2\SA; ?]~
	\bdsc
	\item[3\tre] \mini (11-14\bad FP), beliebige Verteilung
		\bdsc
		\item[3\kar] Frage nach K"urze
			\bdsc
			\item[3\of] keine K"urze
			\item[3\aof] beliebiges Chicane; \ra \rel fragt
			\item[3\SA] Single in \aofa
			\item[4\uf] Single in \ufa
			\edsc
		\item[3\aof] beliebiges Chicane; \ra \rel fragt
		\item[3\SA] Single in \aofa
		\item[4\uf] Single in \ufa
		\edsc
	\item[3\kar] \maxi (17\pl FP), beliebige Verteilung
		\bdsc
		\item[3\coe] Frage nach K"urze
			\bdsc
			\item[3\pik] beliebiges Chicane
				\bdsc
				\item[3\SA] Frage nach der Chicane-Farbe
					\bdsc
					\item[4\uf] Chicane in \ufa
					\item[4\coe] Chicane in \aofa
					\edsc
				\edsc
			\item[3\SA] Single in \aofa
			\item[4\uf] Single in \ufa
			\item[4\coe] keine K"urze, 17-18 FP
			\item[4\pik] keine K"urze, 19\pl FP
			\edsc
		\edsc
	\item[3\of] 14\good{}-16 FP, keine K"urze
	\item[3\aof] 14\good{}-16 FP, beliebiges Chicane; \ra \rel fragt
	\item[3\SA] 14\good{}-16 FP, Single in \aofa
	\item[4\uf] 14\good{}-16 FP, Single in \ufa
	\edsc
\edsc

\minisec{Beispielreizungen}

\exhand{432}{AB10852}{A3}{K2}%
{KB6}{K976}{K94}{A87}%
{
1\coe && 2\SA &\\
3\tre & \mini & 4\coe
}

\exhand{4}{AB10852}{A1042}{K2}%
{876}{KD76}{KD3}{A87}%
{
1\coe && 2\SA &\\
3\tre & Minimum & 3\kar & Relais\\
3\SA & \pi{}-Single & 4\tre & Cuebid\\
4\kar & Cuebid & 4\SA & RKCB\\
5\coe & 2 Keycards ohne \co{}-Dame & 6\coe
}

\exhand{KB1042}{DB753}{2}{K3}%
{AD53}{AK}{653}{A542}%
{
1\pik & & 2\SA &\\
3\tre & Minimum & 3\kar & Relais\\
4\kar & \ka{}-Single & 4\SA & RKCB\\
5\tre & 1 oder 4 Keycards & 6\pik
}

\exhand{K752}{AD542}{-}{K742}%
{A8}{KB76}{D62}{ADB3}%
{
1\coe & & 2\SA &\\
3\tre & Minimum & 3\kar\\
3\pik & unbestimmtes Chicane & 3\SA & Frage nach der Chicane-Farbe\\
4\kar & \ka{}-Chicane & 4\SA & RKCB\\
5\tre & 1 oder 4 Keycards & 5\kar & Frage nach der \co{}-Dame\\
5\pik & \co{}-Dame und \pi{}-K"onig vorhanden & 5\SA & Frage nach weiteren K"onigen\\
6\tre & \tr{}-K"onig vorhanden & 7\coe
}

\newpage
%%%%%%%%%%%%%%%%%%%%%%%%%%%%%%%%%%%%%%%%%%%%%%%%%%%%%%%%%%%%%%%%%%%%%%%%%%%%%%
\section{Die 1\SA-Er"offnung}

Die Er"offnung 1\SA verspricht 15-17~FP und eine ausgeglichene
Verteilung.  5332-Verteilungen mit einer 5er-\ofa sind m"oglich.

\subsection{Antworten auf 1\SA-Er"offnung}
\bidins{%
2\tre & ab 0 & Stayman, verspricht 4er-\ofa\\
2\kar/\co & ab 0 & \xferto\co/\pi, verspricht
5\pl{}er-Farbe\\
2\pik & ab 0 & \xferto\ufa, verspricht 6\pl{}er-\ufa oder 5/5 \ufa.\\
2\SA & 8-9 & \nat, \inv\\
3\,$x$ && 6\pl{}er-Farbe, \slamint{}\\
3\SA && zum Spielen\\
4\tre && \conv{\Index{Gerber}}-Assfrage (0/4, 1, 2, 3; \ra~\ref{gerber},
S.~\pageref{gerber})\\
4\kar && 5/5\pl in den \ofa, kein \slamint \\
4\SA && quantitative Einladung zu 6\SA \\
5\SA && quantitative Einladung zu 6\SA/7\SA
}

\subsection{Reizung von \ofa-Zweif"arbern ("Ubersicht)}

Folgende Tabelle gibt eine "Ubersicht, wie beim Reizen von
\ofa-Zweif"arbern (5/4\pl-Verteilung) zu verfahren ist -- jeweils mit
schwachen, einladenden und starken H"anden.  Details siehe unten.

\begin{center}
\begin{tabular}[t]{|l|c|c|}
\hline
\textbf{St"arke} & \textbf{5/4} & \textbf{5/5}\\
\hline
\hline
schwach & Stayman & \conv{Transfer}\\
\hline
\inv & \multicolumn{2}{c|}{\conv{Transfer}}\\
\hline
\pf & Stayman, dann \conv{Smolen} & 4\kar\\
\hline
\pf (stark) & Stayman, dann \conv{Smolen} & \conv{Transfer}\\
\hline
Schlemm & \multicolumn{2}{c|}{Stayman, dann \conv{Smolen}}\\
\hline
\end{tabular}
\end{center}

\subsection{Schwache und einladende Stayman-Sequenzen}

Mit schwachen (0-7~FP) H"anden ist Stayman nur erlaubt, wenn
man jede Antwort des Er"offners (2\kar, 2\coe, 2\pik) passen kann -- also
mit 4-4-4-1, 4-4-5-0, 3-4-5-1 und 4-3-5-1 oder mit 5/4 in den \ofa.

\bdsc
\item[1\SA{}\sep2\tre; 2\kar{}\sep?]~
\bdsc
\item[2\coe] 4/4 oder 5/4 in \coe/\pi, mit 3/3~\ofa muss \eo passen.
\item[2\pik] 5/4~\pik/\co, \nf
\edsc
\edsc

Mit einladenden H"anden reizen wir eine 5/4-Verteilung in den \ofa
"uber \conv{Transfer}.

\subsection{Starke Stayman-Sequenzen}

\bdsc
\item[1\SA{}\sep2\tre] \conv{Stayman}, zeigt 4er-\ofa, fragt nach 4er-\ofa
\bdsc
	\item[2\kar] keine 4er-\ofa
	\bdsc
		\item[3\kar] \inv, 4-5 \ofa/\ka
		\item[3\coe] \pf{}\pl, \xferto\pi \conv{(\Index{Smolen} Transfer)};
			verspricht 4er-\co und 5er-\pi
  			\begin{itemize}
				\item falls 5er-\co: starkes \slamint{}
				\item mildes \slamint{} mit 5/5~\ofa wird "uber \conv{Transfer} gereizt
				\item ohne \slamint{} mit 5/5~\ofa wird [1\SA-4\kar;] gereizt
			\end{itemize}
		\item[3\pik] \pf{}\pl, \xferto\co \conv{(Smolen Transfer)};
			verspricht 4er-\pi und 5er-\co (siehe oben)
	\edsc
	\item[2\coe] 4er-\co (4er-\pi m"oglich)
	\bdsc
		\item[2\pik] \slamint{} in \co; keine K"urze
		\bdsc
			\item[2\SA] negativ
			\item[3\tre/\ka] \conv{Cuebid}
		\edsc
		\item[3\pik-4\kar] \conv{\Index{Splinter}}
		\item[4\SA] quantitativ mit 4er-\pi
	\edsc
	\item[2\pik] 4er-\pi
	\bdsc
		\item[3\coe] \slamint{} in \pi; keine K"urze
		\bdsc
			\item[3\pik] positiv
			\item[3\SA] negativ
			\item[4\tre/\ka] \conv{Cuebid}
		\edsc
	\edsc
\edsc
\edsc

\subsection{Transfer-Sequenzen}

\paragraph{Transfer auf \ofa}

\bdsc
  \item[1\SA{}\sep2\kar] \xferto\co

  Ohne 4\pl{}er-\co f"uhrt der Er"offner den Transfer aus:
  \bdsc
    \item[2\coe] \xfer ausgef"uhrt
    \bdsc
      \item[2\pik] 5/4 \co/\pi, \inv
      \bdsc
        \item[2\SA] \mini, kein 3er-\co, kein 4er-\pi
        \bdsc
          \item[3\pik] 5/5 \co/\pi, zum Spielen
        \edsc
      \edsc
      \item[3\tre/\ka] \nat, \pf; ohne \slamint ist eine K"urze im
        Blatt
      \bdsc
        \item[3\kar] (nach 3\tre) \ka-Werte und keine \pi-Werte
        \bdsc
          \item[4\SA] quantitativ
          \bdsc
            \item[pass] \mini, Double-\co
            \item[5\coe] \mini mit \co-Fit
            \item[6\coe] \maxi mit \co-Fit
            \item[6\SA] \maxi, Double-\co
          \edsc
        \edsc
      \edsc
      \item[4\SA] quantitativ
    \edsc
  \edsc
  Mit 4\pl{}er-\co bricht der Er"offner aus dem Transfer aus (\conv{Super
  Accept}).
  3\kar wiederholt dann den Transfer:
  \bdsc
    \item[2\pik] 4er-\co, \maxi, \pi-Double
    \item[2\SA] 4er-\co, \maxi, \ka-Double oder 4-3-3-3
    \item[3\tre] 4er-\co, \maxi, \tr-Double
      \bdsc
      \item[3\kar] R"ucktransfer \ra \co
      \edsc
    \item[3\coe] 4er-\co, \mini
  \edsc

  \item[1\SA{}\sep2\coe; 2\pik{}] \xferto \pi ausgef"uhrt
  \bdsc
    \item[3\coe] 5/4 \pi/\co, \inv
    \item[4\coe] 5/5 \pi/\co, mildes \slamint
  \edsc
\edsc

\paragraph{Transfer auf \ufa}

\bdsc
  \item[1\SA{}\sep2\pik] \xferto\ufa
  \bdsc
    \item[2\SA] Pr"aferenz f"ur \ka
    \bdsc
      \item[3\tre/\ka] schwach, zum Spielen
      \item[3\coe/\pi] 5/5\pl \tr/\ka, \slamint in \ufa, \conv{Splinter}
    \edsc
    \item[3\tre] Pr"aferenz f"ur \tr
  \edsc
\edsc

\subsection{Schlemm-Sequenzen}

\bdsc
  \item[1\SA{}\sep3\tre] \slamint in \tr
  \bdsc
    \item[3\kar/\co/\pi] Kontrolle in der Farbe, \maxi, \tr-Fit
    \item[3\SA] negativ (\mini, \tr-Double, evtl. 3 kleine \tr)
  \edsc
\edsc

\newpage
%%%%%%%%%%%%%%%%%%%%%%%%%%%%%%%%%%%%%%%%%%%%%%%%%%%%%%%%%%%%%%%%%%%%%%%%%%%%%%
\section{Die 2\SA-Er"offnung}

Die 2\SA-Er"offnung zeigt eine ausgeglichene Hand mit 20-21 FP, eine 5er-\ofa
kann enthalten sein.

\bdsc
\item[2\SA] 20-21 FP, \bal, 5er-\ofa m"oglich
  \bdsc
  \item[3\tre] Puppet-Stayman
  \item[3\kar] \xferto \co
  \item[3\coe] \xferto \pi
  \item[3\pik] \xferto \ufa
  \item[3\SA] \nat
  \item[4\tre] \Index{Blackwood}-Assfrage 4/0, 1, 2, 3 (\conv{\Index{Gerber}}
    \ra \ref{gerber}, S.~\pageref{gerber})
  \item[4\kar] 5/5 in \pi/\co, schwach
  \edsc
\edsc

\subsection{Transfer}

\begin{itemize}
\item Die Annahme eines Transfers zeigt ein \emph{Double}.
\item 3\SA zeigt einen \emph{3er-Anschluss}.
\item Andere Gebote zeigen einen 3\good{}/4er-Anschluss und sind \emph{Cuebids}.
\end{itemize}

\subsection{Puppet-Stayman}

Puppet-Stayman fragt nach 5er- und 4er-\ofa beim Er"offner. In den
Puppet-Sequenzen ist 4\SA die \conv{Blackwood}-Assfrage solange kein Fit best"atigt
wurde.

\bdsc
\item[2\SA{}\sep3\tre] Frage nach 5er- und 4er-\ofa
    (Alert: \emph{"`verspricht keine 4er-\ofa{}\/"'})
  \bdsc
  \item[3\kar] 4er-\ofa, keine 5er-\ofa \\
    Nach 3\kar reizt der Antwortende die \emph{andere} \ofa ("ahnlich
    \conv{\Index{Smolen} Transfer}). 3\SA vom Er"offner ist negativ (falsche
\ofa),
    andere Gebote best"atigen den Fit.
    \bdsc
    \item[3\coe] 4er-\pi, 4er-\co m"oglich
      \bdsc
      \item[3\pik] \pi-Fit, danach Cuebids
      \item[3\SA] negativ, 4er-\co
        \bdsc
        \item[4\tre] Optional KCB
        \item[4\kar] Optional KCB
        \item[4\coe] \co-Fit, zum Spielen
        \item[4\pik] \co-Fit (!), KCB auf \co-Basis
        \item[4\SA] \conv{Blackwood}-Assfrage (\ra \ref{gerber}, S.~\pageref{gerber})
        \edsc
      \edsc
    \item[3\pik] 4er-\co
      \bdsc
      \item[3\SA] negativ, 4er-\pi
        \bdsc
        \item[4\tre] Optional KCB
        \item[4\kar] Optional KCB
        \item[4\SA] \conv{Blackwood}-Assfrage
        \edsc
      \item[4\tre] \co-Fit, Cuebid
      \item[4\kar] \co-Fit, Cuebid
      \item[4\coe] \co-Fit, kein \ka- und \tr-Cuebid
      \edsc
    \item[3\SA] keine 4er-\ofa
    \item[4\tre] Optional KCB
    \item[4\kar] Optional KCB
    \item[4\SA] \conv{Blackwood}-Assfrage
    \edsc
  \item[3\coe] 5er-\co
    \bdsc
    \item[3\pik] \co-Fit (!), KCB auf \co-Basis
    \item[3\SA] zum Spielen
    \item[4\tre] Optional KCB
    \item[4\kar] Optional KCB
    \item[4\coe] zum Spielen
    \item[4\SA] \conv{Blackwood}-Assfrage
    \edsc
  \item[3\pik] 5er-\pi, weiter wie nach 3\coe
  \item[3\SA] keine 4er-, keine 5er-\ofa
  \edsc
\edsc

\newpage

%%%%%%%%%%%%%%%%%%%%%%%%%%%%%%%%%%%%%%%%%%%%%%%%%%%%%%%%%%%%%%%%%%%%%%%%%%%%%%
\section{Verhalten nach Zwischenreizung durch die
  Gegner\label{zwischenreizung}}

\subsection{Kontra vom Partner}

Nach allen \emph{pr"azisen} Er"offungen (1\SA, 2\SA, Weak Two, Sperransagen)
und einer Farbzwischenreizung des Gegners ist Kontra vom Partner \emph{immer}
Strafe.

\subsection{Informationskontra vom Gegner}

Nach \conv{Informationskontra} vom Gegner gilt:
\begin{itemize}
\item Eine neue Farbe auf der 2er-Stufe ist immer nonforcing.
\item Weiterhin \conv{Inverted} und \conv{Bergen}.
\item 1\of{}\sep(\kontra){}\sep2\SA zeigt jetzt
  \begin{itemize}
    \item eine einladende Hebung mit genau 3er-Anschluss oder
    \item eine partieforcierende Hebung.
  \end{itemize}
\item \conv{Rekontra} verspricht Punktmajorit"at und verneint einen Fit in der
  er"offneten Farbe.
\end{itemize}

\subsection{Allgemeines nach Farbzwischenreizung}
\begin{itemize}
\item Das Reizen einer neuen Farbe auf der 1er-Stufe ist weiterhin
forcierend f"ur eine Runde.
%
\item Das Reizen einer neuen Farbe auf der 2er-Stufe ist nicht forcierend
falls die Zwischenreizung gest"ort hat, sonst forcierend:
\begin{description}
\item[1\coe{}\sep(1\pik){}\sep2\kar] ist forcierend da man auch
ohne die Zwischenreizung 2\kar gereizt h"atte.
\item[1\tre{}\sep(1\pik){}\sep2\coe] ist nicht forcierend da man ohne die
  Zwischenreizung 1\coe gereizt h"atte.
\end{description}
%
\item Eine neue Farbe auf der 3er-Stufe ist immer forcierend
  (z.~B. 1\pik{}\sep(2\kar){}\sep3\tre).
\item Nach gegnerischen \conv{Weak Jumps} ist das Reizen einer neuen Farbe immer
  forcierend.
\item \conv{Kontra} ist negativ bis 3\coe oder zeigt eine beliebige
  starke Hand, f"ur die es in der augenblicklichen Situation kein Gebot
  gab.
\item Der \conv{"Uberruf} nach
  \begin{itemize}
    \item \ufa-Er"offnung fragt nach Stopper,
    \item \ofa-Er"offnung zeigt eine partieforcierende Hand mit
      Anschluss in der \ofa.
    \end{itemize}
\item \conv{Sprung-"Uberruf} oder "Uberruf auf 4er-Stufe ist \conv{Splinter}.
\item 1\SA und 3\SA sind nat"urlich (oder weiterhin konventionell).
\end{itemize}

\subsection{Nach \uf-Er"offnung}
\begin{itemize}
\item Es gilt weiterhin \conv{Inverted Minors}, solange alle dazu ben"otigten
	Gebote noch frei sind (z.B. [1\uf{}\sep(1\anybid)\sep2\SA{}]).
\item Der \conv{"Uberruf} der gegnerischen Farbe fragt nach Stopper und ist
  Partieforcing (z.B. [1\uf{}\sep(1\anybid){}\sep2\anybid{}] oder
  [1\uf{}\sep(2\anybid)\sep3\anybid{}]).
\end{itemize}

Alle "ubrigen Gebote unterscheiden sich nicht von einer ungest"orten
Reizung (\conv{Inverted Minors}, \conv{Weak Jumps}).

\minisec{Details}

\bdsc
\item[1\tre{}\sep(1\kar){}\sep?]~
\bdsc
\item[\kontra] beide \of \conv{(Negativkontra)}
\item[1\coe/\pi] mindestens 4er-L"ange
\item[2\tre/2\SA/3\tre] Inverted
\item[2\kar] \pf und Frage nach \ka-Stopper
\item[2\of/3\of] Weak Jump
\item[3\kar] Karo-\conv{\Index{Splinter}}
\edsc
\item[1\tre{}\sep(1\coe){}\sep?]~
\bdsc
\item[\kontra] genau 4er-\pi oder starke Hand mit \ka-L"ange
\item[1\pik] 5\pl{}er-\pi
\item[2\kar] \nf
\item[2\coe] \pf und Frage nach \co-Stopper
\item[3\coe] \co-\conv{Splinter}
\edsc
\item[1\tre{}\sep(1\pik){}\sep?]~
\bdsc
\item[\kontra] genau 4er-\co oder starke Hand mit \ka- oder \co-L"ange
\item[2\kar] \nf
\item[2\coe] 5\pl{}er-\co, \nf
\item[2\pik] \pf und Frage nach \pi-Stopper
\item[3\pik] \pi-\conv{Splinter}
\edsc
\edsc

\minisec{Beispiele}

\begin{description}
\item[1\kar{}\sep(2\tre){}\sep2\kar] zeigt 10\pl FP und 5\pl{}er
  \ka-Anschluss.
\item[1\tre{}\sep(1\pik){}\sep2\SA] zeigt 2-6 FP und 5\pl{}er
  \tr-Anschluss.
\item[1\kar{}\sep(2\tre){}\sep3\tre] ist partieforcierend und fragt nach
  \tr-Stopper.
\end{description}

Die Reizung nach 1\kar-Er"offnung ist analog.

\subsection{Nach \of-Er"offnung}
\begin{itemize}
\item Alle \ofa-Hebungen sind schwach.  Falls \conv{Bergen-Hebungen}
  noch m"oglich sind (d.h. im Sprung), werden diese angewendet%
  \footnote{nach (2\tre) ist nur noch 3\kar m"oglich}.
\item 1\of{}\sep($x$){}\sep2\SA zeigt
  \begin{itemize}
  \item wenn \conv{Bergen-Hebungen} noch m"oglich sind: eine einladende
    Hebung mit genau 3er-Anschluss,
  \item sonst eine einladende Hebung mit 3er- oder 4er-Anschluss.
  \end{itemize}
Die anderen F"alle k"onnen mittels \conv{Bergen} gezeigt werden.
\item Der \conv{"Uberruf} der gegnerischen Farbe zeigt eine
  partieforcierende Hand mit Anschluss in der \ofa
  ([1\of{}\sep(1\anybid){}\sep2\anybid{}] oder
[1\of{}\sep(2\anybid){}\sep3\anybid{}]).
\end{itemize}

\minisec{Details}

\bdsc
\item[1\coe{}\sep(1\pik){}\sep?]~
  \bdsc
  \item[\kontra] beide \ufa \conv{(Negativkontra)}
  \item[2\tre/\ka] 10\pl FP, 4\pl{}er-L"ange, forcierend
  \item[2\pik] \pf mit 3\pl{}er-Anschluss
  \item[2\SA] \inv mit \emph{genau} 3er-Anschluss
  \item[3\tre/\ka] weiterhin \conv{Bergen-Hebungen}
  \item[3\pik] beliebiger Single-\conv{Splinter} (wie in ungest"orter
    Reizung)
  \edsc
\item[1\coe{}\sep(2\kar){}\sep?]~
  \bdsc
  \item[\kontra] \conv{Negativ} oder stark mit \pik-L"ange
  \item[2\pik] \nf (denn das 1\pik-Gebot war nicht mehr frei)
  \item[2\SA] \inv mit 3er- oder 4er-Anschluss
  \item[3\tre] \nat, forcing
  \item[3\kar] \pf mit 3\pl{}er-Anschluss
  \edsc
\item[1\pik{}\sep(2\coe){}\sep?]~
  \bdsc
  \item[2\SA] \inv mit 3\pl{}er-Anschluss
  \item[3\coe] \pf mit 3\pl{}er-Anschluss
  \item[3\pik] schwache Hebung
  \item[3\SA] Chicane-Splinter (3\coe ist nicht mehr frei)
  \item[4\coe] \co-Splinter
  \edsc
\item[1\coe{}\sep(2\pik){}\sep?]~
  \bdsc
  \item[3\pik] \pf mit 3\pl{}er-Anschluss
  \item[3\SA] Chicane oder Single \pi (!)
  \item[4\pik] EKCB
  \edsc
\edsc

\minisec{Beispiele}

\begin{description}
\item[1\pik{}\sep(2\kar){}\sep2\pik] zeigt weiterhin 6-9 FP mit
  3er-Anschluss.
\item[1\coe{}\sep(1\pik){}\sep3\kar] ist weiterhin \conv{Bergen} und zeigt
  7-9 FP mit 4er-Anschluss.
\item[1\coe{}\sep(2\tre){}\sep3\kar] ist ebenfalls \conv{Bergen} und zeigt
  7-9 FP mit 4er-Anschluss. 9-11 FP wird "uber 2\SA gezeigt.
\item[1\pik{}\sep(X){}\sep2\SA] zeigt einen einladenden 3er-\pi-Anschluss oder
  eine Hand ab 12 FP mit 4er- oder besserem Anschluss (die w"are
  zu stark f"ur \conv{Bergen}).
\item[1\pik{}\sep(2\kar){}\sep2\SA] zeigt 3er- oder 4er-\pi-Anschluss und
  ist einladend (3\kar w"are partieforcierend; \conv{Bergen} kann
  nicht mehr gereizt werden um die einladende Hand mit 4er-\pi{}
  zu zeigen).
\item[1\coe{}\sep(1\pik){}\sep2\pik] zeigt eine partieforcierende Hand mit
  Coeur-Anschluss.
\end{description}

\subsection{Der Gegner reizt einen Zweif"arber \conv{(Unusual over
    Unusual)}}

Reizt der Gegner "uber unsere 1\anybid-Er"offnung einen Zweif"arber bei
dem beide Farben bekannt sind, so verwenden wir die Konvention
\conv{Unusual over Unusual}. Das Reizen der niedrigeren bzw. h"oheren der
gegnerischen Farben steht indirekt f"ur die niedrigere bzw. h"ohere unserer
Farben.
%
\begin{center}
\begin{tabular}[t]{|l|c|c|}
\hline
 & \textbf{Partnerfarbe} & \textbf{neue Farbe}\\
\hline
\hline
\textbf{direkt} & kompetitiv & stark \\
\hline
\textbf{indirekt} & \multicolumn{2}{c|}{einladend}\\
\hline
\end{tabular}
\end{center}

\begin{itemize}
\item Indirekte Farbgebote und Hebungen sind \emph{einladend}.
\item Direktes Heben der Partner-Farbe ist schwach (man h"atte sonst 2\anybid
  gesagt).
\item Direktes Reizen der "`vierten"' Farbe ist stark (5\pl{}er-L"ange mit
  Partiewerten).
\item \conv{Kontra} zeigt Straf-Bereitschaft f"ur mindestens eine der
  beiden gegnerischen Farben.
\end{itemize}

\minisec{Beispiele}

\bdsc
\item[1\coe{}\sep(2\SA{}*){}\sep?] (2\SA = beide \ufa)
\bdsc
\item[3\tre] \inv{}\pl mit \co-Anschluss
\item[3\kar] \inv mit 5\pl{}er-\pi
\item[3\coe] kompetitiv (ersetzt 2\co-Gebot)
\item[3\pik] \pf mit 5\pl{}er-\pi
\edsc

\item[1\pik{}\sep(2\SA{}*){}\sep?] (2\SA = beide \ufa)
\bdsc
\item[3\tre] \inv mit 5\pl{}er-\co
\item[3\kar] \inv{}\pl mit \pi-Anschluss
\item[3\coe] \pf mit 5\pl{}er-\co
\item[3\pik] kompetitiv (ersetzt 2\pik-Gebot)
\edsc

\item[1\coe{}\sep(2\coe{}*){}\sep?] (2\coe = \pi{}+\tr)
\bdsc
\item[2\pik] \inv{}\pl mit \co-Anschluss
\item[2\SA] \nat
\item[3\tre] \inv mit 5\pl{}er-\co
\item[3\kar] \pf mit 5\pl{}er-\ka
\item[3\coe] kompetitiv (ersetzt 2\coe-Gebot)
\edsc

\item[1\pik{}\sep(2\pik{}*){}\sep?] (2\pik = \co{}+\tr)
\bdsc
\item[2\SA] \nat
\item[3\tre] \inv mit 5\pl{}er-\ka
\item[3\kar] \pf mit 5\pl{}er-\ka
\item[3\coe] \inv\pl{} mit \pi-Anschluss
\item[3\pik] kompetitiv (ersetzt 2\pik-Gebot)
\edsc

\edsc

\subsection{Beide Gegner reizen eine Farbe}

Reizen beide Gegner eine Farbe, so ist der \conv{"Uberruf} einer der
Gegnerfarben \emph{nicht} die Frage nach Stopper sondern \emph{zeigt}
einen Stopper in der "uberrufenen Farbe:

\reizungmittext
{
  1\coe & 1\pik & 2\tre & 2\kar\\
  -- & -- & 2\pik{}\al{a} & --\\
  2\SA{}\al{b}
}
{ \smaller
  \al{a} zeigt \pstop \\
  \al{b} \nat, zeigt \kstop (3\kar w"are Frage nach \khstop)
}

\reizungmittext
{
  1\kar & 1\pik & \kontra & 2\tre\\
  2\pik{}\al{a} & -- & 2\SA{}\al{b}
}
{
  \al{a} zeigt \pstop \\
  \al{b} \nat, zeigt \tstop
}

\subsection{Nach 1\sa-Er"offnung}

\bdsc
\item[1\SA{}\sep(\kontra)] (Gegner kontriert 1\SA)
    \bdsc
    \item[\rekontra] \pupto 2\tre (schwach)
      \bdsc
      \item[pass] 5\pl{}er-\tr
      \item[2\kar] 4/4 \ofa (\ra~2\coe/\pi zum Spielen)
      \edsc
    \item[2\tre/\ka/\co] \xferto \ka/\co/\pi
    \item[pass] \pupto \rekontra (\bal, schwach oder stark)
      \bdsc
        \item[pass] stark, zum Spielen
        \item[Rest] schwach, zum Spielen
      \edsc
    \edsc

\item[1\SA{}\sep(2\anybid)] (Gegner reizt eine Farbe)\\
    \ra \conv{\Index{Lebensohl}} (\ref{lebensohl}, S.~\pageref{lebensohl})

\item[1\SA{}\sep(p)\sep2\tre{}\sep(\kontra);] (Gegner kontriert Stayman)
    \bdsc
    \item[pass] verneint 4er-\ofa und 5er-\ufa
    \item[\rekontra] 5er-\tr
    \item[2\kar] 5er-\ka
    \item[2\of] 4er-\ofa
    \edsc
\item[1\SA{}\sep(p)\sep2\kar/\co{}\sep(\kontra);] (Gegner kontriert \xferto
\ofa)
    \bdsc
    \item[pass] 2er-\ofa
    \item[2\coe/\pi] 3\pl{}er-\ofa
    \item[\rekontra] gutes 4er mit Werten in der gereizten Farbe
    \edsc
\edsc

\newpage
%%%%%%%%%%%%%%%%%%%%%%%%%%%%%%%%%%%%%%%%%%%%%%%%%%%%%%%%%%%%%%%%%%%%%%%%%%%%%%
\section{Er"offnungen auf 2er-Stufe}

\subsection{Ogust} \label{ogust}

Mit Interesse am Vollspiel gegen"uber einer Weak-Two-Er"offnung oder gegen"uber
einem \conv{Weak Jump} kann der Antwortende mit 2\SA den Er"offner dazu
auffordern, sein Blatt weiter zu beschreiben (\conv{Ogust}-Konvention). Dieses
Gebot ist forcierend und impliziert Anschluss in der Weak-Two-Farbe.
\begin{enumerate}
\item Der Er"offner zeigt Punktst"arke und Farbqualit"at nach dem Muster
\mini{}-\mini{}-\maxi{}-\maxi:
	\begin{description}
	\item[1. Stufe] \mini, schlechte Farbe
	\item[2. Stufe] \mini, gute Farbe (2 von 3 Topfiguren)
	\item[3. Stufe] \maxi, schlechte Farbe
	\item[4. Stufe] \maxi, gute Farbe
	\item[5. Stufe] \maxi, sehr gute Farbe (3 Topfiguren)\footnote{Nach
	  dieser Antwort ist die Assfrage nat"urlich Unsinn.}
	\end{description}
\item Die n"achste freie Farbe (au"ser Trumpf) fragt nun nach K"urze:
	\begin{description}
	\item[Trumpffarbe] verneint eine K"urze
	\item[3\SA{}] zeigt K"urze in der Fragefarbe
	\end{description}
\item Nach Beantwortung der K"urzenfrage ist die n"achste freie Farbe die
\conv{Preempt-Assfrage} (siehe ~\ref{pkca}, S.~\pageref{pkca}).  Die Trumpffarbe
auf Partiestufe und 3\SA sind Gebote zum Spielen.
\end{enumerate}

In den einzelnen Abschnitten f"ur die Zweier-Er"offnungen sind jeweils Beispiele
zu dieser Konvention angegeben.

\minisec{Zwischenreizung nach 2\SA-Frage}
Bei St"orung durch die Gegner nach der 2\SA-Frage gilt
\conv{\Index{DOPI-ROPI}}:
\bdsc
\item[2\tre{}\sep2\SA;]~
  \bdsc
  \item[(\kontra)] Gegner kontriert \ra \conv{ROPI}
    \bdsc
    \item[pass] erste Stufe
    \item[\rekontra] zweite Stufe
    \item[3\tre] dritte Stufe
    \item[\ldots]
    \edsc
  \item[(3\anybid)] Gegner reizt auf 3er-Stufe \ra \conv{DOPI}
    \bdsc
    \item[pass] erste Stufe
    \item[\kontra] zweite Stufe
    \item[3\,$y$] dritte Stufe
    \item[\ldots]
    \edsc
  \item[(4\anybid)] Gegner reizt auf 4er-Stufe
    \bdsc
    \item[pass] schlechtes Weak Two
    \item[\kontra] gutes Weak Two\footnote{vielleicht gibt es etwas Besseres,
keine Ahnung}
    \edsc
  \edsc
\edsc

\subsection{2\tre: Weak Two in \ka oder Semiforcing}

Die 2\tre-Er"offnung zeigt entweder
\begin{itemize}
\item ein \Index{\textbf{Weak Two}} in Karo,
\item ein \textbf{Semiforcing} in einer beliebigen Farbe
  (16\pl FP, 6\pl{}er-Farbe, 8 Spielstiche) oder
\item eine \textbf{ausgeglichene Hand} mit 22-23 oder 26-27~FP.
\end{itemize}
Der Antwortende reizt in der Regel 2\kar als Relais, es sei denn, dass
er gegen"uber einem Weak Two in Karo ein volles Spiel sieht, dann reizt
er 2\SA, was auch Karo-Fit impliziert. Neue Farben sind nat"urlich und forcing.

\bdsc
\item[2\tre{}\sep2\kar;] \rel
  \bdsc
  \item[2\coe] Semiforcing in \co
  \item[2\pik] Semiforcing in \pi
  \item[2\SA] 22-23 FP, \bal, 5er-\ofa m"oglich \\
    \ra weiter wie nach 2\SA-Er"offnung
  \item[3\tre] Semiforcing in \tr
  \item[3\kar] Semiforcing in \ka
  \item[3\SA] 26-27 FP, \bal, 5er-\ofa m"oglich \\
    \ra weiter wie nach 2\SA-Er"offnung
  \edsc
\item[2\tre{}\sep2\SA;] \conv{(Ogust)}\index{Ogust}
  \bdsc
  \item[3\tre] Weak Two in \ka, 6-8 FP, schlechte Farbe
    \bdsc
    \item[3\kar] zum Spielen
    \item[3\coe] K"urzenfrage
      \bdsc
      \item[3\pik] \pi-K"urze \ra 4\tre: PKCA
      \item[3\SA] \co-K"urze \ra 4\tre: PKCA
      \item[4\tre] \tr-K"urze \ra 4\kar: PKCA
      \item[4\kar] keine K"urze \ra 4\coe: PKCA
      \edsc
    \item[3\SA] zum Spielen
    \edsc
  \item[3\kar] 6-8 FP, gute Farbe (2 Topfiguren) \\
    weiter wie nach 3\tre
  \item[3\coe] 9-10 FP, schlechte Farbe
    \bdsc
    \item[3\pik] K"urzenfrage
      \bdsc
      \item[3\SA] \pi-K"urze \ra 4\tre: PKCA
      \item[4\tre] \tr-K"urze \ra 4\kar: PKCA
      \item[4\kar] keine K"urze \ra 4\coe: PKCA
      \item[4\coe] \co-K"urze \ra 4\pik: PKCA
      \edsc
    \item[3\SA] zum Spielen
    \edsc
  \item[3\pik] 9-10 FP, gute Farbe (2 Topfiguren)
    \bdsc
    \item[3\SA] zum Spielen
    \item[4\tre] K"urzenfrage (\emph{nicht} PKCA)
      \bdsc
      \item[4\SA] \tr-K"urze \ra 5\tre: PKCA\footnote{nicht besonders
gl"ucklich}
      \edsc
    \edsc
  \item[3\SA] 3 Topfiguren: \suit{AKD}, weiter wie oben
  \edsc
\edsc

Alle anderen Gebote zeigen die starke Variante der Er"offnung.

\subsection{2\kar: Weak Two in \co oder Partieforcing}

Die 2\kar-Er"offnung zeigt entweder
\begin{itemize}
\item ein \Index{\textbf{Weak Two}} in \co.
\item ein \textbf{Partieforcing} in einer beliebigen Farbe
  (18\pl FP, 6\pl{}er-Farbe, 9 Spielstiche) oder
\item eine \textbf{ausgeglichene Hand} mit 24-25 oder 28\pl~FP.
\end{itemize}
Der Antwortende reizt in der Regel 2\coe als Relais, es sei denn dass
er gegen"uber einem Weak Two in Coeur ein volles Spiel sieht, dann reizt
er entweder 2\pik, was nat"urlich ist, oder 2\SA, was auch Coeur-Fit
impliziert.

\bdsc
\item[2\kar{}\sep2\coe;] \rel
  \bdsc
  \item[2\pik] \pf in \pi
  \item[2\SA] 24-25 FP, \bal, 5er-\ofa m"oglich \\
    \ra weiter wie nach 2\SA-Er"offnung
  \item[3\tre] \pf in \tr
  \item[3\kar] \pf in \ka
  \item[3\coe] \pf in \co
  \item[3\SA] 28\pl FP, \bal, 5er-\ofa m"oglich \\
    \ra weiter wie nach 2\SA-Er"offnung
  \edsc
\item[2\coe{}\sep2\SA;] \conv{(Ogust)}\index{Ogust}
  \bdsc
  \item[3\tre] Weak Two in \co, 6-8 FP, schlechte Farbe
    \bdsc
    \item[3\kar] K"urzenfrage
      \bdsc
      \item[3\coe] keine K"urze \ra 3\pik: PKCA
      \item[3\pik] \pi-K"urze \ra 4\tre: PKCA
      \item[3\SA] \ka-K"urze \ra 4\tre: PKCA
      \item[4\tre] \tr-K"urze \ra 4\kar: PKCA
      \edsc
    \item[3\coe] zum Spielen
    \item[3\SA] zum Spielen
    \edsc
  \item[3\kar] 6-8 FP, gute Farbe (2 Topfiguren)
    \bdsc
    \item[3\coe] zum Spielen
    \item[3\pik] K"urzenfrage
      \bdsc
      \item[3\SA] \pi-K"urze \ra 4\tre: PKCA
      \item[4\tre] \tr-K"urze \ra 4\kar: PKCA
      \item[4\kar] \ka-K"urze \ra 4\pik: PKCA
      \item[4\coe] keine K"urze \ra 4\pik: PKCA
      \edsc
    \edsc
  \item[3\coe] 9-10 FP, schlechte Farbe, weiter wie oben
  \item[3\pik] 9-10 FP, gute Farbe (2 Topfiguren)
    \bdsc
    \item[3\SA] zum Spielen
    \item[4\tre] K"urzenfrage (\emph{nicht} PKCA)
      \bdsc
      \item[4\SA] \tr-K"urze \ra 5\tre: PKCA
      \edsc
    \edsc
  \item[3\SA] 3 Topfiguren: \suit{AKD}, weiter wie oben
  \edsc
\edsc


\subsection{2\coe: Zweif"arber mit \co} \label{2coeur}

Die 2\coe-Er"offnung zeigt einen Zweif"arber mit einem 5er-\co und einer
weiteren 5er-Farbe im Bereich von 6-10 FP. In dritter Hand kann die Verteilung
auch 5/4 sein.

\bdsc
\item[2\coe] 5-10 FP, 5/5 \co/?
  \bdsc
  \item[2\pik] schwach, zum Spielen oder ausbessern
  \item[2\SA] Frage nach 2. Farbe und St"arke
  \item[3\coe] weitere Sperre
  \item[4\tre] PKCA auf \co-Basis (nur direkt)
  \edsc
\edsc

\minisec{Weiterreizung nach [2\coe{}\sep2\SA{}]}

Der Er"offner zeigt mit seinem R"uckgebot seine zweite Farbe und seine
Punktst"arke, nur bei einem \co/\tr-Zweif"arber kann er die Punktst"arke nicht
sofort mitteilen. Die n"achste freie Farbe fragt dann nach K"urze, bei \co/\tr
nach K"urze und Punktst"arke (Antworten "ahnlich der Ogust-Konvention:
\mini{}-\mini{}-\maxi{}-\maxi).

F"ur die Beantwortung der K"urzenfrage gilt: niedrigste Stufe zeigt niedrigste
K"urze. Die n"achste Stufe des Antwortenden ist dann PKCA f"ur die niedrigere
Farbe, die "ubern"achste Stufe PKCA f"ur die h"ohere Farbe, nat"urlich unter
Auslassung von 3\SA und 4\coe. Beim \co/\pi-Zweif"arber wird mit den beiden
K"urzenfragen 4\tre und 4\kar gleich die Trumpffarbe festgelegt, um danach bei
PKCA eine Stufe zu sparen.

Im Folgenden zeigt "`$\rightarrow$"' die PKCA-Gebote und die Trumpfbasis.

\bdsc
\item[2\coe{}\sep2\SA;] Frage nach St"arke und zweiter Farbe
  \bdsc
  \item[3\tre] zweite Farbe \tr, 6-10~FP
    \bdsc
    \item[3\kar] Frage nach St"arke und K"urze
      \bdsc
      \item[3\coe] 6-8\bad FP, \ka-K"urze \ra 3\pik: \tr, 4\tre: \co
      \item[3\pik] 6-8\bad FP, \pi-K"urze \ra 4\tre: \tr, 4\kar: \co
      \item[3\SA] 8\good-10 FP, \ka-K"urze \ra 4\tre: \tr, 4\kar: \co
      \item[4\tre] 8\good-10 FP, \pik-K"urze \ra 4\kar: \tr, 4\pik: \co
      \edsc
    \edsc
  \item[3\kar] zweite Farbe \ka, 6-8\bad~FP
    \bdsc
    \item[3\coe] zum Spielen
    \item[3\pik] Frage nach K"urze
      \bdsc
      \item[3\SA] \tr-K"urze \ra 4\tre: \ka, 4\kar: \co
      \item[4\tre] \pi-K"urze \ra 4\kar: \ka, 4\pik: \co
      \edsc
    \edsc
  \item[3\coe] zweite Farbe \pi, 6-8\bad~FP
    \bdsc
    \item[3\pik] zum Spielen (!)
    \item[4\tre] \co-Fit und Frage nach K"urze
      \bdsc
      \item[4\kar] \tr-K"urze \ra 4\pik: \co
      \item[4\coe] \ka-K"urze \ra 4\pik: \co
      \edsc
    \item[4\kar] \pi-Fit und Frage nach K"urze
      \bdsc
      \item[4\coe] \tr-K"urze \ra 4\SA: \pi
      \item[4\pik] \ka-K"urze \ra 4\SA: \pi
      \edsc
    \edsc
  \item[3\pik] zweite Farbe \pi, 8\good-10~FP; weiter wie nach 3\coe
  \item[3\SA] zweite Farbe \ka, 8\good-10~FP
    \bdsc
    \item[4\tre] Frage nach K"urze
      \bdsc
      \item[4\kar] \tr-K"urze \ra 4\pik: \ka, 4\SA: \co
      \item[4\coe] \pi-K"urze \ra 4\pik: \ka, 4\SA: \co \emph{(Vorsicht!)}
      \edsc
    \edsc
  \edsc
\edsc

\minisec{Beispiele}

\exhand{3}{DB865}{54}{K10975}
{9876}{AK}{AK}{AD86}{%
  2\coe && 2\SA \\
  3\tre & 5/5 \tr/\co & 3\kar & Frage nach St"arke und K"urze\\
  3\coe & \mini, \pi-K"urze & 3\pik & PKCA auf \tr-Basis\\
  3\SA & 1 Ass ohne \tr-Dame & 6\tre\\
}

\exhand{32}{KB874}{KD872}{5}
{AD54}{106}{A54}{ADB2}{%
  2\coe && 2\SA &\\
  3\SA  & 5/5 \co/\ka, \maxi & pass &\\
}

\exhand{KB986}{DB764}{54}{2}
{AD76}{2}{AKD106}{A54}{%
  2\coe & & 2\SA  & \\
  3\coe & 5/5 \co/\pi, \mini & 4\kar & \pi-Fit, K"urzenfrage \\
  4\coe & \tr-K"urze & 4\SA  & PKCA\\
  5\tre & 1 Ass ohne \pi-Dame & 6\pik
}

\subsection{2\pik: Weak Two in \pi}

Die 2\pik-Er"offnung zeigt ein \Index{Weak Two} in Pik, d.h. eine 6er-Farbe mit
6-10 FP ohne 4er-\co (4er-\ufa m"oglich).

\bdsc
\item[2\pik] 6-10 FP, 6er-\pi
  \bdsc
\item[2\SA] 15\good{}\pl FP, \pi-Fit, Frage nach St"arke und Farbqualit"at
  \conv{(Ogust)}
  \item[3\pik] weitere Sperre
  \item[4\tre] PKCA auf \pi-Basis
  \item[4\pik] weitere Sperre oder konstruktiv
  \edsc
\edsc

Das Reizen einer neuen Farbe zeigt 6\pl (5 gute) Karten und ist
selbstverst"andlich forcing. Der Er"offner hebt mit 3er-Anschluss oder mit
Double-Anschluss und einer K"urze.

\bdsc
\item[2\pik{}\sep2\SA;] \conv{(Ogust)}\index{Ogust}
  \bdsc
  \item[3\tre] 6-8 FP, schlechte Farbe
    \bdsc
    \item[3\kar] K"urzenfrage
      \bdsc
      \item[3\coe] \co-K"urze \ra 4\tre: PKCA
      \item[3\pik] keine K"urze \ra 4\tre: PKCA
      \item[3\SA] \ka-K"urze \ra 4\tre: PKCA
      \item[4\tre] \tr-K"urze \ra 4\kar: PKCA
      \edsc
    \item[3\pik] zum Spielen
    \item[3\SA] zum Spielen
    \edsc
  \item[3\kar] 6-8 FP, gute Farbe (2 Topfiguren)
    \bdsc
    \item[3\coe] K"urzenfrage
      \bdsc
      \item[3\pik] keine -K"urze \ra 4\tre: PKCA
      \item[3\SA] \co-K"urze \ra 4\tre: PKCA
      \item[4\tre] \tr-K"urze \ra 4\kar: PKCA
      \item[4\kar] \ka-K"urze \ra 4\coe: PKCA
      \edsc
    \item[3\pik] zum Spielen
    \edsc
  \item[3\coe] 9-10 FP, schlechte Farbe
    \bdsc
    \item[3\pik] zum Spielen
    \item[3\SA] zum Spielen
    \item[4\tre] K"urzenfrage (\emph{nicht} PKCA)
      \bdsc
      \item[4\SA] \tr-K"urze \ra 5\tre: PKCA
      \edsc
    \edsc
  \item[3\pik] 9-10 FP, gute Farbe (2 Topfiguren), weiter wie oben
  \item[3\SA] 3 Topfiguren: \suit{AKD}, weiter wie oben
  \edsc
\edsc

\minisec{Beispiele}

\exhand{KD9874}{862}{B74}{2}
{A3}{A973}{A92}{A986}{%
  2\pik && 2\SA&\\
  3\kar & \mini, gute Farbe & 3\SA
}

\exhand{K109876}{862}{A74}{2}
{AD5}{AKD97}{2}{A873}{%
  2\pik & & 2\SA\\
  3\tre & \mini, schlechte Farbe & 3\kar & Frage nach K"urze\\
  4\tre & \tr-K"urze & 4\kar & PKCA\\
  5\tre & 2 Asse ohne \pi-Dame & 7\pik
}

\exhand{KD10874}{A62}{B74}{2}
{AB3}{KDB4}{AK2}{986}{%
  2\pik & & 2\SA\\
  3\pik & \maxi, gute Farbe & 4\tre & Frage nach K"urze\\
  4\SA  & \tr-K"urze & 5\tre & PKCA\\
  6\tre & 2 Asse mit \pi-Dame & 6\pik
}

\newpage

%%%%%%%%%%%%%%%%%%%%%%%%%%%%%%%%%%%%%%%%%%%%%%%%%%%%%%%%%%%%%%%%%%%%%%%%%%%%%%
\section{Er"offnungen auf h"oherer Stufe}

\subsection{3 in Farbe}

Die Farber"offnung auf 3er-Stufe zeigt eine 7\pl{}er-L"ange und 6-10 FP.
In der Weiterreizung ist eine neue Farbe auf der 3er-Stufe nat"urlich und
partieforcierend. Eine
neue Farbe auf der 4er-Stufe (au"ser 4\tre) best"atigt den Fit und ist Cuebid.
4\tre ist PKCA. 4\of auf eine 3\uf-Er"offnung ist zum Spielen. Hebung der
\ofa ins Vollspiel ist entweder konstruktiv oder weitere Sperre.

\minisec{Bietsequenzen nach 3\tre-Er"offnung}

\woreizung{
  3\tre & & 3\coe & \\
  3\pik & Werte auf dem Weg zu 3\SA oder vorverlegtes Cuebid & 3\SA & \aw will
3\SA spielen \\
  4\coe & 3\pik war vorverlegtes Cuebid
}

\woreizung{
  3\tre & & 3\coe & \\
  3\SA & kein \co-Fit, kein Single in Nebenfarbe \\
  4\tre & Cuebid, 3\pl{}er-\co \\
  4\coe & 2\pl{}er-\co
}

Nach anderen Sperransagen sind die Sequenzen entsprechend.

\subsection{3\SA: Gambling}\index{Gambling}

Die 3\SA-Er"offnung zeigt eine stehende 7er-\ufa ohne Werte in den
Nebenfarben (\conv{Gambling}).

\bdsc
\item[3\SA] \conv{Gambling}
  \bdsc
  \item[pass] Stopper in restlichen Farben und 2\pl{}er-Anschluss in Partners
\ufa
  \item[4\tre] will 4\tre oder 4\kar spielen
  \item[4\kar] Frage nach K"urze
    \bdsc
    \item[4\coe] \co-K"urze
    \item[4\pik] \pi-K"urze
    \item[4\SA] keine K"urze (2-2-2-7 oder 2-2-7-2)
    \item[5\tre] \ka-K"urze (!)
    \item[5\kar] \tr-K"urze (!)
    \edsc
  \item[4\of] zum Spielen
  \item[5\tre] will 5\tre oder 5\kar spielen
  \item[5\kar] will 5\kar oder 6\tre spielen
  \item[6\tre] will 6\tre oder 6\kar spielen
  \edsc
\edsc

\subsection{4\uf: Namyats}\index{Namyats}

Die 4\uf-Er"offnung zeigt eine stehende 7er-\ofa (\tr{}\ra{}\co, \ka{}\ra{}\pi)
mit
einer Nebenfarb-Kontrolle (K"onig/Ass). Partner reizt entweder 4 in der \ofa
oder \emph{Erstrunden}kontrollen, wenn er Schlemminteresse hat.

\newpage
%%%%%%%%%%%%%%%%%%%%%%%%%%%%%%%%%%%%%%%%%%%%%%%%%%%%%%%%%%%%%%%%%%%%%%%%%%%%%%
\section{Die Gegenreizung\label{gegenreizung}}

\subsection{Informationskontra}
\bdsc
\item[(1\anybid)\sep\kontra{}\sep{}(p)\sep{}?]~
  \bdsc
  \item[1\hspace{\cardskip}$y$] (ohne Sprung) 0-7 FP
  \item[2\anybid{}] ("Uberruf der Gegnerfarbe)
    \begin{itemize}
      \item 11\pl FP, beliebige Verteilung oder
      \item beide \ofa, 8-10 FP
    \end{itemize}
  \item[2\of{}] (einfacher Sprung) 4er-Farbe, 8-10 FP
  \item[3\of{}] (doppelter Sprung) 5er-Farbe, 8-10 FP
  \edsc
\edsc

Nach [(1\of){}\sep\kontra{}\sep(2\of)] ist 2\SA \Index{\conv{Lebensohl}},
ebenfalls
nach \Index{Weak Two}-Er"offnung: [(2\of){}\sep\kontra{}\sep(p)].

\subsection{Farbgegenreizung}

Antworten:
\begin{itemize}
\item Alle Hebungen sind schwach.
\item Neue Farbe ist nicht forcierend (8-12 FP).
\item Neue Farbe im Sprung ist \conv{Fit-Sprung}.
\item Der "Uberruf zeigt starke H"ande:
  \begin{itemize}
  \item eine mindestens einladende Hand mit Fit oder
  \item einen Einf"arber mit mehr als 12 FP oder
  \item eine starke \sa-Hand ohne Stopper.
  \end{itemize}
\end{itemize}

\subsection{Weak Jumps}

Spr"unge in die 2er-Stufe zeigen 6-10 FP und eine 6er-Farbe oder eine sehr gute
5er-Farbe. 2\SA vom Er"offner ist dann \conv{\Index{Ogust}} (\ra~\ref{ogust},
S.~\pageref{ogust}).
Spr"unge in die 3er-Stufe zeigen eine 7\pl{}er-Farbe und eine punktschwache
Hand.

\subsection{Michaels Pr"azis}

\conv{Michaels Pr"azis} zeigt einen Zweif"arber in zwei der drei
nichtgereizten Farben.  Es \emph{muss} mindestens eine 5/5-Verteilung
vorhanden sein; die Werte sollen sich in den langen Farben befinden.
%
\bdsc
\item[(1\uf){}\sep2\kar] \co und \pi
\item[(1\uf){}\sep2\SA] \co und \aufa \conv{(Unusual Notrump)}
\item[(1\of){}\sep2\of] \aofa und \tr
\item[(1\of){}\sep2\SA] \tr und \ka \conv{(Unusual Notrump)}
\item[(1\of){}\sep3\tre] \aofa und \ka
\edsc

Nach \ofa-Er"offnung gibt es keinen schwachen Sprung in \tr!
Nach \ufa-Er"offnung wird der Zweif"arber mit \pi und \aufa nat"urlich gereizt.

\subsection{Die 1\SA-Gegenreizung}

Die 1\SA-Gegenreizung in zweiter Position zeigt 15-18 FP (in vierter Position
11-14 FP) und einen
Stopper in der er"offneten Farbe. Weiterreizung wie nach
1\SA-Er"offnung, d.h. Stayman und \conv{Transfers} wenn der
rechte Gegner passt, ansonsten \Index{\conv{Lebensohl}}.

Hat der Gegner eine \ofa er"offnet, so ist der Transfer \emph{in} diese \ofa
Stayman. 2\tre ist dann ebenfalls ein Transfer.

\subsection{Gegenreizung gegen 1\SA (\Index{Multi-Landy})}

\bdsc
\item[(1\SA){}\sep?] ~
 \bdsc
 \item[\kontra] abh"angig von der St"arke der Er"offnung:
   \bdsc
     \item[starker \sa] 5\pl{}er-\ufa und 4er \ofa
     \item[schwacher \sa] mindestens gleiche St"arke (oberes Ende der Punktspanne)
   \edsc
 \item[2\tre] 5\pl/4\pl in \ofa
  \bdsc
  \item[2\kar] zeigt gleiche L"ange in den beiden \ofa; "Uberrufer soll
    seine l"angere \ofa reizen.
  \edsc
 \item[2\kar] Einf"arber in einer \ofa
 \item[2\of] 5\pl{}er \ofa und 4er \ufa
 \item[2\SA] 5/5\pl in \ufa
 \item[3\uf] Einf"arber in einer \ufa
 \edsc
\edsc

\subsection{Mehrdeutige 2er-Er"offnungen}

Bei der Verteidigung gegen Er"offungen, die schwache und starke Varianten
enthalten, wird zun"achst von der/den schwachen Variante(n) ausgegangen.

\subsubsection*{Er"offnungen mit einer schwachen Variante}

Ist genau eine schwache Variante (z.~B. \conv{Weak Two}) enthalten und soll der
Antwortende diese Farbe als Relais reizen (wie bei unseren 2\tre- und
2\kar-Er"offnungen), dann zeigt Kontra L"ange und Werte in dieser Farbe. Das
Reizen der Farbe zeigt die Verteilung f"ur ein Informationskontra.

\bdsc
	\item[2\tre{}/\ka{}] (\conv{Weak Two} in \ka{}/\co{} oder starke Variante)
	\bdsc
		\item[\kontra{}] gute 5\pl{}er-L"ange in der "`er"offneten"' Farbe (Treff bzw. Karo)
		\item[2\kar{}/2\coe{}] Informationskontra gegen \conv{Weak-Two} in dieser Farbe
	\edsc
\edsc
Alle anderen Gebote sind nat"urlich.

\subsubsection*{Er"offnungen mit mehreren schwachen Varianten}

Sind mehrere schwache Varianten enthalten (z.~B. 2\kar \conv{Multi}), so zeigt
Kontra ein Informationskontra gegen die niedrigere Farbe. Andere Gebote sind
nat"urlich. Ein Informationskontra gegen die andere(n) Farbe(n) zeigt man
sp"ater durch Kontra, wenn das Relais zu uns durchgepasst wurde.
\cite{retzlaff04}

\subsection{Crash gegen starke 1\tre-Er"offnung}

\conv{Crash} zeigt einen Zweif"arber. Ziel ist neben der St"orung der
gegnerischen Reizung auch, Informationen f"ur das sp"atere Gegenspiel
zu "ubermitteln.  Die Zwischenreizung kann sehr schwach sein.  Mit
konstruktiven H"anden sollte \conv{Crash} nicht gereizt werden.

\bdsc
\item[(1\tre)] 16\pl FP, k"unstlich, beliebige Verteilung
\bdsc
\item[\kontra] \textbf{C}olour -- gleichfarbige (\tr/\pi oder \ka/\co)
\item[1\kar] \textbf{Ra}nk -- gleichrangige (\tr/\ka oder \co/\pi)
\item[1\of] \nat
\item[1\SA] \textbf{Sh}ape -- gleichf"ormige (\tr/\co oder \ka/\pi)
\item[2\uf] \nat
\edsc
\edsc

\newpage
%%%%%%%%%%%%%%%%%%%%%%%%%%%%%%%%%%%%%%%%%%%%%%%%%%%%%%%%%%%%%%%%%%%%%%%%%%%%%%
\section{Konventionen zum System}

%
%%%%%%%%%%%%%%%%%%%%%%%%%%%%% Sequenzen nach 1SA-Rebid (Puppet) %%%%%%%%%%%%%%%
%
\subsection{Relaistransfer nach 1\SA-R"uckgebot} \label{1sarebid}

Nach [1\anybid{}\sep1\of; 1\SA{}] verwenden wir neben den nat"urlichen Antworten
die beiden Puppet-Gebote 2\tre und 2\SA, um m"oglichst viele Verteilungen zeigen
zu k"onnen.

\minisec{Regeln}
\begin{itemize}
\item schwache H"ande direkt (Ausnahme: \ufa-Canap\'e)
\item einladende H"ande "uber [2\tre; 2\kar{}] (Ausnahme:
  Karo-Canap\'e)
\item starke H"ande "uber [2\SA; 3\tre{}] oder direkt (mehrere Ausnahmen)
\end{itemize}

Nicht exakt reizen lassen sich nur
\begin{itemize}
\item mindestens einladende Zweif"arber mit 5er-\ofa und 4er-\ufa sowie
\item genau einladende Zweif"arber mit 4er-\ofa\ und 5er-\tr.
\end{itemize}

\minisec{"Ubersicht}
Die Reizung beginnt jeweils [1\anybid{}\sep1\of; 1\SA{}\sep?].

\begin{minipage}{\columnwidth}
{\relsize{-2}%
\begin{tabularx}{\columnwidth}{|c|c|l|l|Y|}
\hline
\multicolumn{2}{|c|}{\textbf{Haltung}} & \multicolumn{3}{c|}{\textbf{St"arke}}\\
\hline
\emph{OF} & \emph{NF} & \emph{schwach} & \emph{einladend} &
\emph{partieforcierend}\\
\hline
\hline
4\of  & \bal  & pass  & 2\tre{}\leadto2\SA & 3\SA\\
\hline
4\of  & 5\tre & 2\SA{}\leadto{}p &
                \emph{nicht zeigb.} &
                nach 1\tre-E"o.: \mbox{2\SA{}\leadto3\kar{}}\\
4\of  & 5\kar & 2\tre{}\leadto{}p &
                2\kar{}\footnote{genau einladend, mit \pf \conv{Walsh} reizen} &
                nach 1\kar-E"o.: \mbox{2\SA{}\leadto3\kar{}}\\
\hline
5\of  & --    & 2\of  &
                2\tre{}\leadto2\of{}\footnote{mindestens einladend} &
                2\tre{}\leadto3\SA\\
5\of  & 5\anybid & 2\of & 2\tre{}\leadto3\anybid & 3\anybid\\
5\pik & 4\coe & 2\coe & 2\tre{}\leadto2\coe & 2\SA{}\leadto3\coe\\
\hline
6\of  & --    & 2\of  & 2\tre{}\leadto3\of & 3\of{}\footnote{leichtes
Schlemm-Interesse}\\
      & Single  &       &             & 4\anybid\\
      & Chicane  &       &             & 2\tre{}\leadto4\anybid\\
\hline
\end{tabularx}\\[1ex]
\centerline{\emph{OF = Oberfarbe des Antwortenden, NF = Nebenfarbe}}%
}%
\end{minipage}

\minisec{Bietsequenzen}
\bdsc
\item[1\tre{}\sep1\pik; 1\SA{}\sep?] ~

  Direkt zeigt man:
  \begin{itemize}%
  \item[1] einladend, 5\pl{}er-\ka und 4er-\ofa
  \item[2] schwach, 5er-\pi, evtl. 4\pl{}er-\co
  \item[3] \pf, 5/5
  \item[4] \slamint, 6er-\ofa
  \item[5] starker \ofa-Einf"arber mit Single
  \end{itemize}

  \bdsc
  \item[2\tre] \pupto2\kar{} (siehe unten)
  \item[2\kar] 4er-\pi, 5\pl{}er-\ka, genau \inv (sonst \conv{Walsh}) (1)
  \item[2\coe] 5/4 \pi{}+\co, zum Spielen oder Ausbessern (2)
  \item[2\pik] 5er-\pi, schwach (2)
  \item[2\SA] \pupto3\tre{} (siehe unten)
  \item[3\anybid] 5/5 \pi{}+\any{}\footnote{beliebige Farbe au"ser \pi}, \pf (3)
  \item[3\pik] 6er-\pi, leichtes \slamint{} (4)
  \item[4\anybid] 6\pl{}er-\pi, Single \any \conv{(Autosplinter)} (5)
  \edsc

\item[1\tre{}\sep1\pik; 1\SA{}\sep2\tre; 2\kar{}\sep?]~

  "Uber [2\tre{}; 2\kar{}] zeigt man:
  \begin{itemize}
  \item[1] schwach, 4/5\pl \ofa/\ka
  \item[2] mindestens \inv, 5er-\ofa, evtl. 4er-\ufa
  \item[3] genau \inv
    \begin{itemize}
    \item[a] \bal, 4er-\ofa
    \item[b] 5/4-Zweif"arber mit beiden \ofa
    \item[c] 5/5-Zweif"arber
    \item[d] Einf"arber in einer \ofa
    \end{itemize}
  \item[4] starker \ofa-Einf"arber mit Chicane
  \end{itemize}

  \bdsc
  \item[pass] schwach (1)
  \item[2\coe] 5/4 \pi{}+\co, \inv (3b)
  \item[2\pik] 5er-\pi, mind. \inv (2)
  \item[2\SA] \bal, \inv, 4er-\pi (3a)
  \item[3\anybid] 5/5, \inv (3c)
  \item[3\pik] 6er-\pi, \inv (3d)
  \item[3\SA] \nat, 5er-\pi (2)
  \item[4\anybid] 6\pl{}er-\pi, Chicane \any
\conv{(Autosplinter\index{Splinter})} (4)
  \edsc

\item[1\tre{}\sep1\pik; 1\SA{}\sep2\SA; 3\tre{}\sep?]~

  "Uber [2\SA{}; 3\tre{}] zeigt man:
  \begin{itemize}
  \item[1] schwach, 4/5\pl \ofa/\tr
  \item[2] stark:
    \begin{itemize}
    \item[a] 4/5 \ofa/\tr nach 1\tre-Er"offnung
    \item[b] 4/5 \ofa/\ka nach 1\kar-Er"offnung
    \item[c] 5/4 \pi/\co nach 1\pik-Antwort
    \end{itemize}
  \end{itemize}

  \bdsc
  \item[pass] schwach, 4/5\pl \pi{}+\tr (1)
  \item[3\kar] stark, 4/5 \pi{}+\tr (nach 1\tre-Er"offnung, 2a) \\
    stark, 4/5 \pi{}+\ka (nach 1\kar-Er"offnung, 2b)
  \item[3\coe] stark, 5/4 \pi{}+\co (2c)
  \edsc
\edsc

\subsection{Bietsequenzen nach Revers-Reizung}

\conv{Revers}-Reizungen zeigen 16\good{}\pl~FP und sind selbstforcierend f"ur
den Er"offner.

\bdsc
  \item[1\tre{}\sep1\pik; 2\kar{}\sep{}?]~

    \bdsc
    \item[2\coe] \conv{VFF}, sagt hier aber nichts "uber die St"arke
      aus, da 2\kar selbstforcierend war.  \emph{Verneint} in diesem Fall
      5er-\pi, da man \pi h"atte wiederholen k"onnen ohne
      dass der Er"offner passen darf.
      \bdsc
        \item[2\SA] \mini, \cstop, \nf
        \item[3\tre] \mini, \nf
        \item[3\kar] \maxi, \pf
        \item[3\SA] \maxi, \cstop
      \edsc
    \item[2\pik] 5er-Farbe, schwach oder stark
      \bdsc
        \item[2\SA] \mini, \cstop, kein 3er-\pi
        \item[3\tre] \mini, kein 3er-\pi
        \item[3\kar] \maxi, kein 3er-\pi
        \item[3\coe] \conv{VFF}, Frage nach \cstop
        \item[3\pik] \mini mit 3er-\pi
        \item[3\SA] \maxi, \cstop, kein 3er-\pi
      \edsc
    \item[2\SA] \pupto3\tre \conv{(\Index{Ingberman})}
      \bdsc
      \item[pass/3\kar] zum Spielen
      \item[3\SA] 8-9 FP, \nat
      \item[4\SA] 12\pl FP, quantitative Schlemmeinladung
      \edsc
    \item[3\uf] \nat, \pf
    \item[3\SA] 10-11 FP, \cstop
    \edsc
\edsc

\notebox{\textbf{Ingberman} (frz. \emph{Moderateur}): Nach einer Revers-Reizung
fordert 2\SA ein passbares 3\tre-Gebot an, um auf der 3er-Stufe ein
Abschlussgebot abgeben zu k"onnen ("ahnlich \conv{Lebensohl}). Der Antwortende
sollte mit Minimum (6-7) diese Konvention anwenden. Mit Zusatzst"arke
(18\good{}\pl) muss der Partner das 3\tre{}-\rel "uberspringen.}

\subsection{Long Suit Trial Bids}

Nach [1\of{}\sep2\of;] ist 2\SA ein allgemeines Trial Bid was zu 3 oder 4 in
\ofa einl"adt. Der Er"offner muss nicht ausgeglichen sein, er hat lediglich
keine unterst"utzungbed"urftige Farbe. Der Antwortende soll mit Minimum Sign Off
geben und mit Maximum das Vollspiel ansagen. Mit massierten Werten in einer
Farbe reizt der Antwortende diese Farbe auf 3er-Stufe.

Bei St"orung durch die Gegner ersetzt die niedrigste freie Farbe, sofern es
noch eine gibt, das 2\SA-Trial Bid. Andernfalls ist \kontra das Trial Bid.

\minisec{Beispiele}
\begin{description}
\item[1\tre{}\sep1\coe;~1\pik{}\sep2\pik;~2\SA]~

  allgemeines Trial Bid
\item[1\tre{}\sep1\pik;~2\pik{}\sep2\SA]~

  allgemeines Trial Bid, auch als Schlemmvorbereiung nutzbar
\item[1\pik{}\sep(2\tre)\sep2\pik{}\sep(3\tre);~?]~
  \begin{description}
    \item[pass] 5er-\pi, Minimum
    \item[\kontra] Strafe
    \item[3\kar] allgemeines Trial Bid (die n"achste freie Stufe
      ersetzt 2\SA)
    \item[3\coe] normales Trial Bid
    \item[3\pik] kompetitiv
  \end{description}
\item[1\pik{}\sep(2\coe)\sep2\pik{}\sep(3\coe);~\kontra]~

  \conv{Full Value Double}/\conv{Competitive Double}, ersetzt hier das
allgemeine Versuchsgebot da keine Farbe ausser der Trumpffarbe mehr frei ist;
3\pik w"are rein kompetitiv
\end{description}

\subsection{Dritte Farbe Forcing (DFF)} \label{dff}

Nach Farbwiederholung des Er"offners ist die dritte, vom Antwortenden gereizte,
Farbe DFF. Nach \ofa-Er"offnung soll der Er"offner vorrangig 3er-Anschluss in
der anderen \ofa des Antwortenden zeigen, je nach St"arke billig oder im
Sprung. Ist die dritte Farbe eine \ofa, so zeigt der Er"offner ebenfalls
Anschluss. Die dritte Farbe zeigt auch Werte, in der Absicht einen \sa-Kontrakt
anzusteuern. Der Er"offner bietet \sa, wenn er die vierte Farbe stoppt.

Die dritte Farbe auf der 2er-Stufe gereizt zeigt eine mindestens einladende
Hand, auf der 3er-Stufe eine partieforcierende.

\minisec{Beispiel}

\woreizung{
  2\tre & Semiforcing & 2\kar & \rel \\
  2\coe & \nat & 2\tre & \nat \\
  3\coe & \nat & 3\pik & DFF, zeigt \pi-Werte
}

\subsection{Vierte Farbe Forcing (VFF)} \label{vff}

Die vierte Farbe ist fast immer k"unstlich (im Gegensatz zur dritten Farbe bei
\conv{DFF}) und zeigt L"ange und St"arke, die in der bisherigen Reizung noch
nicht bekannt ist. Der Partner soll Anschluss in der \ofa zeigen, \sa mit
Stopper in der vierten Farbe reizen oder seine Hand weiter beschreiben.

Man zeigt eine starke Hand mit Trumpunterst"utzung, wenn man die vierte
Farbe reizt und anschlie"send die Farbe des Partners hebt. Wird die vierte
Farbe im Sprung gereizt, ist dies nat"urlich und zeigt einen partieforcierenden
5/5-Zweif"arber.

Manchmal fragt die vierte Farbe auch nur nach Halbstopper, n"amlich dann, wenn
der Partner einen Vollstopper in der vierten Farbe bereits verneint hat.
Halbstopper sind: \suit{Kx}, \suit{Dxx}, \suit{Bxx}, \suit{10xxx}.

Die vierte Farbe auf der 2er-Stufe gereizt ist mindestens einladend, auf der
3er-Stufe partieforcierend.
Priorit"aten des Antwortenden:

\minisec{1. 3er-Anschluss in der \ofa des Partners zeigen}
\begin{description}
\item[1\coe{}\sep1\pik;~2\tre{}\sep2\kar;~?]~
  \begin{description}
    \item[2\pik] 3er-\pi, Minimum
    \item[3\pik] 3er-\pi mit Zusatzwerten
  \end{description}
\end{description}

\minisec{2. \sa mit Stopper in der vierten Farbe reizen}
\begin{description}
\item[1\coe{}\sep1\pik;~2\tre{}\sep2\kar;~?]~
  \begin{description}
    \item[2\SA] 12-14 FP, kein 3er-\pi, \ka-\stp
    \item[3\SA] 15-16 FP, kein 3er-\pi, \ka-\stp
    \item[4\SA] \nat{} (!), 17-18 FP, kein 3er-\pi, \ka-\stp
  \end{description}
\end{description}

\minisec{3. Verteilung zeigen}

\handwithdesc{KD94}{AB42}{2}{AKB2}{
Nach [1\tre{}\sep1\kar; 1\coe{}\sep1\pik;] darf der Er"offner mit
nicht in 4\pik springen, da 1\pik VFF war
und nicht nat"urlich gewesen sein muss (wenn 1\pik nat"urlich war,
zeigt dies eine Er"offnung wegen \conv{Walsh}). In dieser Situation
ist das richtige Gebot 3\coe!}

\minisec{Beispiele}

\woreizung{
  1\tre & & 1\kar \\
  2\tre & zeigt nach 1\kar 6er-Farbe & 2\pik & DFF, Werte in \pi \\
  3\tre & kein \co-\stp (sonst 2\SA) & 3\coe & fragt nach \co-\hstp
}

\woreizung{
  1\kar & & 1\coe \\
  2\kar & & 3\tre & DFF, \tr-Werte, \pf, fragt \co-Anschluss \\
  3\pik & VFF, kein 3er-\co, fragt nach \pi-\hstp
}

\woreizung{
  1\kar & & 2\tre \\
  3\kar & & 3\coe & DFF, \co-Werte, \pf, kein \pi-\stp (sonst 3\SA) \\
  3\pik & VFF, fragt nach \pi-\hstp
}

\woreizung{
  1\tre & & 2\tre & Inverted \\
  2\kar & 14\good{}\pl FP, \ka-Werte & 2\coe & \co-Werte, kein \pi-\stp (sonst
2/3\SA) \\
  2\pik & VFF, fragt nach \pi-\hstp
}

\woreizung{
  1\pik & & 2\tre \\
  2\coe & & 3\kar & VFF, fragt nach \ka-\stp \\
  3\coe & kein \ka-\stp, muss kein 5er-\co sein & 3\SA & zeigt \ka-\hstp \\
  pass & ebenfalls \ka-\hstp
}

Besonderheit:

\woreizung{
  1\pik & & 2\kar & \\
  2\coe & & 2\SA \\
  3\coe & \nf!, \ofa-Zweif"arber \\
  3\tre & VFF, \ofa-Zweif"arber
}

\subsection{Lebensohl} \label{lebensohl}

Wir spielen \Index{\conv{Lebensohl}} in folgenden Situationen:
%
\bdsc
\item[1\SA{}\sep{}(2\anybid)] nach \sa-Er"offnung des Partners und
Farbgegenreizung der Gegner
 (\conv{Lebensohl} \emph{mit} Stopper)
\item[(2\anybid)\sep{}\kontra{}\sep{}(pass)]
 nach \Index{Weak Two}-Er"offnung des Gegners und Informationskontra des
Partners
\item[(1\of)\sep{}\kontra{}\sep(2\of)]
 nach \ofa-Er"offnung des Gegners, Informationskontra des Partners und
 \ofa-Hebung durch den anderen Gegner (siehe auch \conv{Good Bad
   Notrump}, Seite~\pageref{goodbadnt}).
\edsc

"Ahnliche Situationen:
\bdsc
\item[(1\of)\sep{}pass\sep(2\of)\sep{}\kontra] \ra Scrambling 2\NT{}
\item[\dots\sep{}(2\anybid)] kompetitive Reizung \ra Good Bad Notrump
\edsc

\subsection{Scrambling 2\NT{} (Nebul"ose 2\SA)\label{scrambling2nt}}

Haben die Gegner eine Farbe er"offnet \emph{und gehoben} und der Partner
gibt ein Informationskontra, sowohl in der direkten als auch in der Pass
Out-Position, so ist 2\SA nicht echt, sondern bedeutet, dass man kein klares
Gebot hat.

\dealerW
S"ud \\
\handwithdesc{9754}{K52}{A64}{D54}{%
\begin{reizung}
  1\pik & pass & 2\pik & pass \\
  pass & \kontra & pass & 2\SA{}\al{a}
\end{reizung}}

Partner kann 1-5-4-3, 1-4-5-3 oder 1-4-3-5 verteilt sein,
wir wollen nicht im 3-3-Fit landen.

S"ud \\
\handwithdesc{B65}{D3}{K1053}{K763}{%
\begin{reizung}
  1\pik & pass & 2\pik & pass \\
  pass & \kontra & pass & 2\SA{}\al{a}
\end{reizung}}

Partner wird entweder seine 5er-Farbe reizen oder seine 4er-Farben von unten
nach oben.

\subsection{Good Bad Notrump (Kompetitive 2\SA)\label{goodbadnt}}

Hat der rechte Gegner ein Gebot auf der 2er-Stufe abgegeben, sind in
kompetitiven Situationen die 2\SA-Ansagen nicht nat"urlich, sondern zeigen den
Wunsch, in der 3er-Stufe zu spielen. Der Partner muss 3\tre bieten (siehe auch
\Index{\conv{Lebensohl}}), wonach der 2\SA-Reizer seine Farbe zeigt.

\dealerW
S"ud \\
\handwithdesc{32}{AK865}{5}{KB1063}{%
\begin{reizung}
  & & & 1\coe \\
  pass & 1\SA & 2\pik & 2\SA{}\al{a} \\
  pass & 3\tre & pass & pass
\end{reizung}}

Ein direktes 3\tre-Gebot von S"ud h"atte eine st"arkere Er"offnung gezeigt.

S"ud \\
\handwithdesc{K65}{A74}{D1093}{873}{%
\begin{reizung}
  1\coe & 2\kar & 2\coe & 3\kar{}\al{a}
\end{reizung}}

Hier direkt 3\kar, konstruktiv.

S"ud \\
\handwithdesc{865}{A74}{D1093}{873}{%
\begin{reizung}
  1\coe & 2\kar & 2\coe & 2\SA{}\al{a} \\
  pass & 3\tre & pass & 3\kar
\end{reizung}}

Mit dieser schwachen Hand zuerst 2\SA und sp"ater 3\kar.

S"ud \\
\handwithdesc{5}{AKD63}{AKB74}{52}{%
\begin{reizung}
  & & & 1\coe \\
  1\pik & pass & 2\pik & 3\kar{}\al{a}
\end{reizung}}

Stark genug, um 3\kar zu reizen. Mit \co{}\hspace{\cardskip}\suit{A9764}
und der gleichen \ka-Haltung w"urde man zuerst 2\SA reizen.

S"ud \\
\handwithdesc{63}{AKB8643}{K73}{9}{%
\begin{reizung}
  & & & 1\coe \\
  1\pik & 1\SA & 2\pik & 2\SA{}\al{a} \\
  pass & 3\tre & pass & 3\coe
\end{reizung}}

Wir wollen 3\coe spielen, ohne den Partner zu 4\coe einzuladen.

S"ud \\
\handwithdesc{A963}{A5}{762}{AB85}{%
\begin{reizung}
  & 1\kar & 1\coe & \kontra \\
  2\coe & 2\SA & pass & 3\coe{}\al{a}
\end{reizung}}

Nicht immer muss der Partner des 2\SA-Reizers 3\tre sagen, dann n"amlich, wenn
er eine st"arkere Hand hat und die Gefahr besteht, dass der 2\SA-Reizer 3\tre
passt.

\subsection{Sandwich \nt (1\SA/2\SA als Zweif"arber)\label{sandwichnt}}

Sandwich 1\NT benutzen wir in vierter Position nach Er"offnung, pass vom
Partner und einem Farbgebot auf der 1er-Stufe des rechten Gegners, um die
beiden nicht gereizten Farben zu zeigen.
Sandwich \nt zeigt eine Hand die zu schwach ist f"ur ein
Informationskontra, aber daf"ur etwas g"unstiger verteilt.

2\SA ist ebenfalls Sandwich \nt, nur mit noch besserer Verteilung
("ahnlich wie \conv{Unusual \nt{}} in zweiter Position).

Haben die Gegner eine Oberfarbe er"offnet und gehoben, dann zeigt 2\SA einen
\emph{unbestimmten} Zweif"arber.

Die Reizung (1\kar){}\sep pass{}\sep(1\pik){}\sep2\pik\ zeigt eine echte Farbe.

\dealerW
S"ud \\
\handwithdesc{DB108}{42}{53}{KD1094}{%
\begin{reizung}
  1\kar & pass & 1\coe & 1\SA \\
\end{reizung}}

S"ud \\
\handwithdesc{2}{AD105}{KB952}{643}{%
\begin{reizung}
  1\tre & pass & 1\pik & 1\SA \\
\end{reizung}}

S"ud \\
\handwithdesc{AB1087}{4}{73}{KD1093}{%
\begin{reizung}
  1\kar & pass & 1\coe & 2\SA \\
\end{reizung}}

\subsection{Kontras}

\subsubsection{Informationskontra}

Siehe \emph{Gegenreizung}, Seite \pageref{gegenreizung}.

\subsubsection{Negativkontra}

Siehe \emph{Verhalten nach Zwischenreizung durch die Gegner}, Seite
\pageref{zwischenreizung}.

\subsubsection{Responsive Double (Antwortkontra)}

Nach Informationskontra oder Farbgegenreizung vom Partner und
\emph{Hebung der Er"offnerfarbe} zeigt das Antwortkontra die nicht
gereizten Farben (siehe auch \conv{Scrambling 2\NT},
Seite~\pageref{scrambling2nt}).

\bdsc
\item[(1\kar){}\sep\kontra{}\sep(2\kar){}\sep\kontra] beide \ofa zu viert
\item[(1\coe){}\sep\kontra{}\sep(2\coe){}\sep\kontra] beide \ufa zu viert,
  kein 4er-\pi
\item[(1\coe){}\sep1\pik{}\sep(2\coe){}\sep\kontra] beide \ufa zu viert, kein
  3er-\pi
\item[(1\tre){}\sep1\coe{}\sep(1\pik){}\sep\kontra] \conv{\Index{Snapdragon
Double}}, zeigt
  9/10\pl FP, 5er-\ka und Double-Figur in~\co
\edsc

\subsubsection{Competitive Double}

Das Competitive \kontra benutzen wir, um noch weitere Information bez"uglich der
St"arke
einer Hand mitzuteilen.

\reizungmittext
{
  1\tre & 1\coe & 1\pik & 2\coe \\
  pass & pass & \kontra\al{a}
}
{
  Zusatzst"arke, ab etwa 10 FP
}

\reizungmittext
{
  1\pik & 2\coe & 2\pik & 3\coe \\
  \kontra\al{a}
}
{
  Full Value (siehe auch Trial Bids)
}

\reizungmittext
{
  1\pik & 2\tre & 2\pik & 3\tre \\
  \kontra\al{a}
}
{
  Strafe (siehe auch Trial Bids)
}

\reizungmittext
{
  1\pik & 2\coe & 2\tre & pass \\
  pass & \kontra\al{a}
}
{
  Wiederbelebung, zeigt gute Hand und Toleranz f"ur die nicht gereizten Farben
}

\subsubsection{Lightner-Kontra}

Das Kontra auf einen Endkontrakt -- sofern es sich um ungest"ort
gereizte Vollspiele oder Schlemms handelt -- verlangt normalerweise
die vom Dummy zuerst gereizte Farbe.  Nach [1\SA{}\sep3\SA;] verlangt
Kontra das Ausspiel der k"urzeren Oberfarbe.

\subsubsection{Trump Support Double}

Das Trumpfunterst"utzungs-Kontra benutzt der Er"offner, wenn der Partner eine
Oberfarbe gereizt
und der n"achste Gegner gesprochen hat. Er zeigt mit \kontra bzw.
\rekontra 3er-Anschluss in der Oberfarbe des Partners:

\bdsc
\item[1\kar{}\sep(p)\sep1\pik{}\sep(2\tre);~?]~
	\bdsc
	  \item[2\pik] 4er-Anschluss
	  \item[\kontra] 3er-Anschluss
	\edsc
\item[1\kar{}\sep(p)\sep1\pik{}\sep(\kontra);~?]~
	\bdsc
	  \item[2\pik] 4er-Anschluss
	  \item[\rekontra] 3er-Anschluss
	\edsc
\edsc

\newpage
%%%%%%%%%%%%%%%%%%%%%%%%%%%%%%%%%%%%%%%%%%%%%%%%%%%%%%%%%%%%%%%%%%%%%%%%%%%%%%
\section{Schlemmkonventionen}

\subsection{Mixed Cuebids}

Kontrollgebote zeigen Erst- oder Zweitrundenkontrolle, also A/K/Single/Chicane.
Ein erstes Kontrollgebot auf der 5er-Stufe zeigt allerdings
Erstrundenkontrolle. Nicht K"urze in Partners erster Farbe als \emph{erstes
Kontrollgebot} reizen!

\exhand{2}{AD932}{AK92}{DB3}
{DB543}{B74}{DB3}{AK}{%
  1\coe & & 1\pik &\\
  3\kar & & 3\coe & st"arker als 4\coe\\
  4\kar & \ka-Kontrolle & 5\tre & keine \pi-Kontrolle, \tr-ERK\\
  5\kar & \pi-K"urze (sonst Sign Off), \ka-ERK & 6\tre
  & \tr-ERK und -ZRK\\
  6\coe & \pi-Single\\
}

\subsection{Blackwood-Assfrage (\Index{Gerber})} \label{gerber}

Das direkte 4\tre-Gebot auf die 1 oder 2\SA-Er"offnung fragt den Er"offner nach Zahl
der Asse. Danach wird rollend nach platzierten K"onigen gefragt.
Nach 2\SA-Er"offnung ist 4\SA solange Blackwood, bis ein Fit bestätigt wurde.

\bdsc
\item[1\SA/2\SA{}\sep4\tre] Gerber-Assfrage
 \bdsc
 \item[1./2./3./4. Stufe] 0/4, 1, 2, 3 Asse
  \bdsc
  \item[4\SA] zum Spielen
  \item[\rel] Frage nach platzierten K"onigen
  \edsc
 \edsc
\edsc

\subsection{Roman Keycard-Assfrage (KCB)}

4\SA auf Oberfarben-, 4\tre auf \tr- und 4\kar auf \ka-Basis. Antworten:
1/4, 3/0, 2 ohne Trumpf-Dame, 2 mit Trumpf-Dame. Anschlie"send rollend nach Trumpf-Dame und
gleichzeitig nach platzierten K"onigen. Hat man selbst die Trumpf-Dame, so ist
das "ubern"achste Gebot unter Auslassung der Trumpffarbe die Frage nach
platzierten K"onigen. Sind alle K"onige an Bord, kann man noch nach platzierten
Damen fragen.

Antwortschema: Hat der Antwortende die Trumpf-Dame nicht, geht er auf die
Trumpffarbe zur"uck. Hat er die Trumpf-Dame, aber keinen Nebenk"onig, so reizt
er 6 in der Trumpffarbe (das n"achste Gebot des Partners ist jetzt die Frage
nach platzierten Damen). Hat der die Trumpf-Dame und einen oder mehrere
Nebenfarbk"onige, so reizt er die niedrigste Farbe, in der er einen K"onig hat
und verneint damit gleichzeitig den K"onig in einer Farbe, die er h"atte
billiger reizen k"onnen. Das niedrigste \sa-Gebot zeigt den K"onig in der
Fragefarbe.

Man kann nun mit dem n"achsten Gebot nach weiteren K"onigen fragen, mit dem
"ubern"achsten nach platzierten Damen. Das Antwortschema bleibt sich gleich.

\exhand{A742}{76}{A2}{AKD52}
{KDB65}{AK}{KD3}{763}{%
  1\tre & & 1\pik \\
  4\tre & \pi-Fit, gute \tr-Farbe & 4\SA & KCB auf \pi-Basis\\
  5\kar & 3 oder 0 & 5\SA & platzierte K"onige?\\
  6\tre & \tr-K"onig vorhanden & 6\coe & platzierte Damen? (6\kar weitere
K"onige?)\\
  7\tre & \tr-Dame vorhanden & 7\SA
}

\exhand{A2}{K754}{AKD43}{52}
{K532}{ADB}{7632}{A3}{%
  1\kar & & 1\pik &\\
  2\coe & & 4\kar & KCB\\
  4\pik & 3 oder 0 & 4\SA & \ka-Dame?\\
  5\coe & \ka-Dame und \co-K"onig, ohne \tr-K"onig & 7\ka &
  bei 3er-\pi und Single-\tr leider chancenlos
}

\exhand{AK2}{2}{D65}{AKD742}
{B6}{AB874}{AKB2}{B8}{%
  1\tre & & 1\coe\\
  3\tre & & 4\tre & KCB\\
  4\coe & 3 oder 0 & 4\pik & \tr-Dame?\\
  4\SA  & \tr-Dame und \pi-K"onig & 5\coe & platzierte Damen?\\
  6\kar & \ka-Dame & 7\SA
}

\exhand{AD1072}{K76}{B102}{D6}
{653}{A5}{AK98}{AK52}{%
  1\pik & & 2\tre\\
  2\pik & & 4\SA & KCB\\
  5\tre & 1 oder 4 & 5\kar & \pi-Dame?\\
  5\coe & \pi-Dame und \co-K"onig & 6\pik
}

\exhand{AK}{AD32}{65}{KD632}
{B432}{KB}{AB4}{AB109}{%
  1\tre & & 1\pik\\
  2\coe & & 4\tre & KCB\\
  4\coe & 3 oder 0 & 4\pik & \tr-Dame?\\
  4\SA  & \tr-Dame und \pi-K"onig & 5\coe & platzierte Damen?\\
  5\SA  & \co-Dame & 7\tre
}

\subsection{Preempt Keycard-Assfrage (PKCA)}\label{pkca}

Diese Assfrage spielen wir nach preemptiven Er"offnungen
des Partners, also nach 2\of, 3\anybid und 4\of.
Nach 2er- und 3er-Starts ist das \emph{direkte} 4\tre-Gebot PKCA,
nach 4\of ist es 4\SA. Antworten:
%
\begin{description}
\item[1. Stufe] 1 Ass ohne Trumpf-Dame
\item[2. Stufe] 0 Asse
\item[3. Stufe] 1 Ass mit Trumpf-Dame
\item[4. Stufe] 2 Asse ohne Trumpf-Dame
\item[5. Stufe] 2 Asse mit Trumpf-Dame
\end{description}

\subsection{Exclusion Keycard-Assfrage (EKCB)}

Die Exclusion-Assfrage (Voidwood) wird durch einen ungew"ohnlichen Sprung
(meist in die 5er-Stufe) gestellt, der kein \conv{\Index{Splinter}} sein kann:
%
\bdsc
\item[1\coe{}\sep1\pik; 2\coe{}\sep5\tre] Fragt nach Keycards unter
  Ausschluss der Treffs.
\item[1\tre{}\sep1\kar; 2\kar{}\sep4\coe] Fragt nach Keycards unter
  Ausschluss der Coeurs (\conv{Splinter} w"are 3\coe).
\edsc

\subsection{\Index{DOPI-ROPI}}

\conv{DOPI} = ``double with zero, pass with one'', \conv{ROPI} =
``redouble with zero, pass with one''.

Nach der Assfrage und Zwischenreizung der Gegner (Farbe oder Kontra) zeigt pass
die erste Stufe und Kontra/Rekontra die zweite Stufe. Diese Konvention wenden
wir auch immer dann an, wenn Stufenantworten erforderlich sind und der Gegner
zwischengereizt hat.
%
\bdsc
  \item[pass] 1. Stufe (1 oder 4 Keycards)
  \item[\kontra/\rekontra] 2. Stufe (3 oder 0 Keycards)
\edsc

\subsection{Josephine}

Eine 5\SA-Reizung \emph{im Sprung} fordert den Partner auf, die Anzahl der
Topfiguren anzugeben: 0, 1, 2, 3.

\newpage
%%%%%%%%%%%%%%%%%%%%%%%%%%%%%%%%%%%%%%%%%%%%%%%%%%%%%%%%%%%%%%%%%%%%%%%%%%%%%%
\section{Ausspiele und Markierungen}

\subsection{Ausspiele gegen Farbkontrakte}

\begin{itemize}
\item 3./5.-h"ochste
\item h"ochste der Sequenz
\item hoch vom Double
\item von \cards{AK} sec wird \cards{K} ausgespielt
\end{itemize}

Im weiteren Verlauf des Spiels zeigt eine kleine Karte eine
Figur, eine hohe hingegen zeigt keine Werte.

\subsection{Ausspiele gegen \sa-Kontrakte}

\begin{itemize}
\item 4.-h"ochste Karte der l"angsten und besten Farbe
\item Ausspiel der 10 und 9 verspricht 0 oder 2 h"ohere Karten.
\item In Partners gereizter Farbe wird die 3./5.-h"ochste ausgespielt.
\item Das st"arkste Ausspiel ist der K"onig; bei Ausspiel von
	\begin{description}
	\item[\suit{KD10xx}] soll Partner den Buben deblockieren,
	\item[\suit{AKB10x}] soll Partner die Dame deblockieren.
	\end{description}
\end{itemize}

\subsection{Markierungen}

Markiert wird \emph{niedrig-hoch} (\conv{UDCA}; "`upside down count and attitude"').

Eine kleine Karte ist positiv und zeigt somit Interesse an der ausgespielten
Farbe, eine hohe Karte ist negativ und zeigt somit Desinteresse.

Spielt der Alleinspieler zu einem Stich, so kann man L"angenmarken geben, wobei eine niedrige
Karte eine gerade L"ange und eine hohe Karte eine ungerade L"ange zeigt.

Erster freier Abwurf im \sa-Kontrakt ist Lavinthal. In Farbkontrakten wird
direkt markiert.

Hat man Partners Farbe unterst"utzt dann verneint eine hohe Karte eine Figur,
eine kleine Karte zeigt eine Figur (\conv{Smith-Peter}-Markierung).

\raggedbottom

%%%%%%%%%%%%%%%%%%%%%%%%%%%%%%%%%%%%%%%%%%%%%%%%%%%%%%%%%%%%%%%%%%%%%%%%%%%%%%
\begin{appendix}
%\newpage
\section{Glossar}
\begin{flushleft}
\begin{tabularx}{\columnwidth}{lY}%
$n$\good{}, $n$\bad{} & $n$ gute/schlechte Punkte/Karten\\
$n$\pl & mindestens $n$ Punkte/Karten\\
\ufa, \ofa & Unterfarbe(n), Oberfarbe(n)\\
\aufa, \aofa & andere \ufa, \ofa\\
1\anybid, 2\anybid, \ldots & beliebiges Gebot 1 in Farbe, 2 in Farbe usw.\\
\ra{}\anybid & Zielfarbe eines Transfers oder n"achstes Gebot auf unserer Seite\\
(\any) & gegnerisches Gebot \\
5332 & beliebige 5332-Verteilung\\
5-3-3-2 & genau 5\pik 3\coe 3\kar 2\tre\\
\bal & ausgeglichen (4333, 4432, 5332)\\
\unbal & nicht ausgeglichen \\
\aw & Antwortender \\
\eo & Er"offner \\
\inv & einladend \\
FP & Figurenpunkte, bei Fit auch Verteilungspunkte \\
\maxi & Maximum \\
\mini & Minimum \\
\nat & nat"urlich \\
\nf & nicht forcierend \\
\pf & Partieforcing \\
\pup & \conv{Puppet}-Gebot \\
\rel & Relais \\
\stp & Stopper (A, Kx, Dxx, Bxxx) \\
\hstp & Halbstopper (K, Dx, Bxx, 10xxx) \\
\slamint & Schlemminteresse \\
\xfer & \conv{Transfer}-Gebot \\
\end{tabularx}%
\end{flushleft}

\section{Optionale Absprachen}

Die in diesem Abschnitt vorgestellten Konventionen sollten als optional
angesehen werden.  Sie k"onnen mit dem System frei kombiniert werden, ohne das
andere Teile des Systems abgewandelt werden m"ussen (soweit nicht im Rahmen der
Konvention erw"ahnt).

\subsection{Raptor SA-"Uberruf}

Diese Konvention ist als \conv{Raptor Notrump} und \conv{Polish Notrump}
bekannt. Sie ersetzt das nat"urliche 1\SA-Gebot in 2. Hand in der Gegenreizung wie folgt:
\cite{farebrother02}

\bdsc
	\item[(1\uf)\sep{}1\SA] 5\pl{}er-\aufa und genau 4er-\ofa
	\item[(1\of)\sep{}1\SA] 5\pl{}er-\ufa und genau 4er-\aofa
\edsc

Au"serdem verneint das Gebot eine Verteilung f"ur ein klassisches
Informationskontra. Die Punktspanne liegt bei 10-15 FP.

Alert: \emph{"`k"unstlich, nicht forcierend, 10-15 Punkte, 4er-Oberfarbe und
mindestens 5er-Unterfarbe, keine Verteilung f"ur ein Informationskontra"'} --
die Denomination der jeweils bekannten Farbe sollte explizit genannt werden.

Nicht gereizt wird \conv{Raptor} in folgenden Situationen:
\begin{itemize}
\item In der Balancing/Passout-Position ist 1\SA na\-t"ur\-lich, 11-14~FP.
\item In 4. Hand falls beide Gegner eine Farbe gereizt haben gilt \conv{Sandwich
Notrump}, siehe
\ref{sandwichnt}, S.~\pageref{sandwichnt}.
\end{itemize}
Eine angepasste Hand kann \conv{Raptor} reizen, mit entsprechend limitierter
Punkst"arke impliziert.

\minisec{Weiterreizung, allgemeine Prinzipien}

\begin{itemize}
	\item Gebote, die zum Spielen oder Ausbessern gedacht sein k"onnen, sind
dies auch.
	\item Die bereits bekannte Farbe ist immer zum Spielen.
	\item Cuebid und 2\SA sind starke Fragegebote.
\end{itemize}

\minisec{Details}
Nach 1\tre{}- bzw. 1\coe{}-Er"offnung wird folgendermassen weitergereizt falls
der rechte Gegner nicht spricht (nach 1\kar{}- bzw. 1\pik{}-Er"offnung ist die
Weiterreizung komplett analog):

\bdsc
	\item[(1\tre)\sep{}1\SA{}\sep{}(p)\sep{}?]~
		
		10-15 FP, 5\pl{}er-\ka, genau 4er-\ofa
		\bdsc
			\item[2\tre] \inv{}\pl mit \ka
				\bdsc
					\item[2\kar] schwach; zum Spielen
					\item[2\coe/\pi] Stopper in der Farbe
					\item[2\SA] Treff-Stopper
				\edsc
			\item[2\kar] zum Spielen; entweder "`geringstes "Ubel"'
oder konstruktive Hebung wie nach nat"urlicher Gegenreizung
			\item[2\coe] zum Spielen oder Ausbessern in 2\pik
			\item[2\pik] zum Spielen oder Ausbessern nach folgendem
Schema:
				\bdsc
					\item[Pass] Pik ist die \ofa
					\item[2\nt] \maxi, Coeur ist die \ofa
					\item[3\kar] 6\pl{}er-Karo, \mini, Coeur
ist die \ofa
					\item[3\coe] 5er-Karo, \mini, Coeur ist
die \ofa
				\edsc
			\item[2\SA] fragt nach der \ofa mit Interesse am
Vollspiel
				\bdsc
					\item[3\tre] \maxi mit Coeur
					\item[3\kar] \maxi mit Pik
					\item[3\of] \nat, \mini
				\edsc
			\item[3\kar/4\kar/5\kar] sperrend
			\item[3\of/4\of] zum spielen oder Ausbessern; sperrend
			\item[3\SA] zum Spielen
			
		\edsc
	\item[(1\coe)\sep{}1\SA{}\sep{}(p)\sep{}?]~

		10-15 FP, 5\pl{}er-\ufa, genau 4er-\pi
		\bdsc
			\item[2\tre] zum Spielen oder Ausbessern
			\item[2\kar] zum Spielen oder Ausbessern nach folgendem
Schema:
				\bdsc
					\item[2\SA] \maxi, Treff ist die \ufa
					\item[3\tre] \mini, Treff ist die \ufa
				\edsc
			\item[2\coe] \inv{}\pl mit Pik
				\bdsc
					\item[2\SA] \maxi, zeigt Treff
					\item[3\uf] \mini, \nat
					\item[3\coe] \maxi, zeigt Karo
					\item[4\pik] \maxi, nimmt Einladung an
				\edsc
			\item[2\pik] zum Spielen; entweder "`geringstes "Ubel"'
oder konstruktive Hebung wie nach nat"urlicher Gegenreizung
			\item[2\NT] starkes Frage-Relais
				\bdsc
					\item[3\uf] \mini, \nat
					\item[3\coe] \maxi mit Treff
					\item[3\pik] \maxi mit Karo
					\item[3\SA] stehende 6er-\ufa (wie nach
\conv{Ogust})
				\edsc
			\item[3\uf/4\uf/5\tre] zum Spielen oder Ausbessern;
sperrend
			\item[3\pik] sperrend (normalerweise 5er-\pi)
			\item[3\SA] zum Spielen
			\item[4\pik] zum Spielen; kann sperrend oder konstruktiv
sein
		\edsc
\edsc

\minisec{St"orung durch die Gegner}
Der rechte Gegner kontriert oder reizt auf der Zweierstufe:
\bdsc
	\item[(1\anybid{})\sep{}1\SA{}\sep{}(\kontra{})\sep{}?]~
		\bdsc
			\item[Pass] Partner soll seine \ufa reizen
			\item[\rekontra] wie nach nat"urlicher
\ofa-Gegenreizung: Punktmajorit"at ohne (echten) Fit
			\item[Rest] wie oben
		\edsc
	\item[(1\anybid{})\sep{}1\SA{}\sep{}(2\anybid{})\sep{}?]~
	
		(Der rechte Gegner hebt die Farbe des Er"offners.)
		\bdsc
			\item[\kontra] Partner soll seine unbekannte Farbe
zeigen
			\item[Rest] nat"urlich
		\edsc
	\item[(1\anybid{})\sep{}1\SA{}\sep{}(2\hspace{\cardskip}$y$)\sep{}?]~
	
		(Der rechte Gegner reizt eine neue Farbe auf Zweierstufe.)
		\bdsc
			\item[\kontra] Partner soll passen, falls $y$ seine
unbekannte Farbe ist (Strafvorschlag).
			\item[Rest] nat"urlich
		\edsc
\edsc
Reizt der rechte Gegner unsere bekannte Farbe ("Uberruf), so setzen wir die
Reizung fort wie nach einer nat"urlichen Sequenz der Art
[(1\coe{})\sep1\pik{}\sep(2\pik)].

%%%%%%%%%%%%%%%%%%%%%%%%%%%%%%%%%%%%%%%%%%%%%%%%%%%%%%%%%%%%%%%%%%%%%%%%%%%%%%
\section{"Anderungen am System}

Gegen"uber fr"uheren Versionen des Systems wurde ge"andert:

\begin{itemize}
\item Checkback durch 2\tre- und 2\SA-Transfer ersetzt (\ra \ref{1sarebid})
\item neue Struktur nach 2\coe-Er"offnung (\ra \ref{2coeur})
\item PKCA kann in Ogust-Sequenzen auch kleiner als 4\tre sein (\ra \ref{ogust})
\item nach \ufa-Er"offnung keine Fit Jumps mehr (sondern Weak Jumps) (\ra
\ref{1treff})
\item Schema nach partieforcierender \ofa{}-Hebung ge"andert (\ra \ref{majorgf}, S.~\pageref{majorgf})
\end{itemize}

\subsection{Alte Konventionen}

Die folgenden Konventionen wurden aus dem System entfernt.

\subsection*{Checkback Stayman}

\emph{Checkback Stayman wurde durch den 2\tre/2\SA-Transfer (\ra~\ref{1sarebid})
ersetzt. (Mai 2005)}

\emph{Das vorher benutzte System mit Checkback-Stayman hatte eine L"ucke, wenn
der Antwortende eine starke Hand mit 4er-\ofa und 5er-\ufa hatte, die Er"offungsfarbe war.}

Der \aw benutzt diese Konvention mit einer 5er \ofa und einer mindestens
einladenden Hand, wenn der \eo 1\SA zur"uckgeboten hat; der \aw m"ochte
wissen, ob der \eo einen 3er-Anschluss hat.

\bdsc
\item[1\ufa{}\sep1\pik; 1\SA{}\sep2\tre] Checkback-Stayman
 \bdsc
 \item[2\kar] 12-13\bad{} FP, kein 3er-\pi, kein 4er-\co
  \bdsc
  \item[2\coe] 10/11 FP, 5/5 in \co und \pi, \nf
  \item[2\pik] 9-11 FP, 6er-\pi, \nf
  \item[2\SA] 11/12 FP, 5er-\pi, \nf
  \item[3\coe] 5/5 in \co und \pi, \slamint
  \edsc
 \item[2\coe] 12-13\bad{} FP, 4er-\co, 3er-\pi m"oglich
 \item[2\pik] 12-13\bad{} FP, 3er-\pi
 \item[2\SA]  13\good{}-14 FP, kein 3er-\pi, kein 4er-\co
 \item[3\ufa] 13\good{}-14 FP, 3er-\pi, gute \ufa
 \item[3\pik] 13\good{}-14 FP, 3er-\pi
 \edsc
\edsc

%\printindex

%\section{Literatur}

\bibliographystyle{alpha}
\bibliography{literatur}

\end{appendix}

\end{document}

